% !TeX program = xelatex ?me -synctex=0 -interaction=nonstopmode -aux-directory=../tex_aux -output-directory=./release
% !TeX program = xelatex

\documentclass[12pt]{article}

\usepackage{lineno,changepage,lipsum}
\usepackage[colorlinks=true,urlcolor=blue]{hyperref}
\usepackage{fontspec}[ Path =../../../ ]
\usepackage{xeCJK}
\usepackage{tabularx}
\usepackage{graphicx}
\setCJKfamilyfont{chanto}{AOZORAMINCHOREGULAR_0.TTF}%
\setCJKfamilyfont{tegaki}{Mushin.otf}%
\usepackage[CJK,overlap]{ruby}
\usepackage{hhline}
\usepackage{multirow,array,amssymb}
\usepackage[croatian]{babel}
\usepackage{soul}
\usepackage[usenames, dvipsnames]{color}
\usepackage{wrapfig,booktabs}
\usepackage{calc}
\renewcommand{\rubysep}{0.1ex}
\renewcommand{\rubysize}{0.75}
\usepackage[margin=50pt]{geometry}
\usepackage{hyperref}
\modulolinenumbers[2]

\date{\today}

\usepackage{fancyhdr}
\pagestyle{fancy}
\fancyhf{}
\fancyhead[LE,RO]{\thepage}
\makeatletter
\fancyhead[RE,LO]{rev. \@date 誠}
\makeatother

\usepackage{pifont}
\newcommand{\cmark}{\ding{51}}%
\newcommand{\xmark}{\ding{55}}%

\newcommand{\dosl}{{\normalfont dosl. }}%
\newcommand{\rem}[1]{{\normalfont #1 }}%

\definecolor{faded}{RGB}{100, 100, 100}

\renewcommand{\arraystretch}{1.2}

%\ruby{}{}
%$($\href{URL}{text}$)$

\newcommand{\furigana}[2]{\ruby{#1}{#2}}
\newcommand{\tegaki}[1]{
	\CJKfamily{tegaki}\CJKnospace
	#1
	\CJKfamily{chanto}\CJKnospace
}

\newcommand{\dai}[1]{
	\vspace{20pt}
	\large
	\noindent\textbf{#1}
	\normalsize
	\vspace{20pt}
}

\newcommand{\fukudai}[1]{
	\vspace{10pt}
	\noindent\textbf{#1}
	\vspace{10pt}
}

\newenvironment{bunshou}{
	\vspace{10pt}
	\begin{adjustwidth}{1cm}{3cm}
	\begin{linenumbers}
}{
	\end{linenumbers}
	\end{adjustwidth}
}

\newenvironment{reibun}[1][]{
	\vspace{10pt}
	#1
	
	\begin{tabular}{l l}
}{
	\end{tabular}
	\vspace{10pt}
}
\newcommand{\rei}[2]{
	#1&\textit{#2}\\
}
\newcommand{\reinagai}[2]{
	\multicolumn{2}{l}{#1}\\
	\multicolumn{2}{l}{\hspace{10pt}\textit{#2}}\\
}

\newenvironment{mondai}[1]{
	\vspace{10pt}
	\noindent #1
	
	\begin{enumerate}
		\itemsep-5pt
	}{
	\end{enumerate}
}

\newenvironment{hyou}{
	\begin{itemize}
		\itemsep-5pt
	}{
	\end{itemize}
	\vspace{10pt}
}

\newcommand{\juuyou}[2][20pt]{
	\vspace{5pt}
		\noindent\hspace{#1}\parbox[c]{\textwidth-#1-#1}{\centering\textit{#2}}
	\vspace{5pt}
}

\newcommand{\ten}{
	\vspace{5pt}
	\noindent\hspace{-10pt}$\bullet$
}

\CJKfamily{chanto}\CJKnospace

\frenchspacing

\newenvironment{dictentry}[1]{
	\begin{tabular}{p{2cm} p{3cm} p{10cm}}
		#1 &&\\
}{
	\end{tabular}
	\vspace{20pt}
}

\newcommand{\example}[3]{
	\hspace*{\fill}#1 & #2 & #3\\
}

\author{ロボット君}
\begin{document}
	\dai{Lekcija 013}

\begin{dictentry}{右}
\example{右}{みぎ}{desno}
\end{dictentry}

\begin{dictentry}{左}
\example{左}{ひだり}{lijevo}
\end{dictentry}

\begin{dictentry}{中}
\example{中}{なか}{sredina, unutar}
\example{中々}{なか.なか}{podosta}
\example{中央}{ちゅう.おう}{centar, centralno}
\end{dictentry}

\begin{dictentry}{下}
\example{下}{した}{dolje, ispod}
\example{下さい}{くだ.さい}{molim lijepo}
\example{下げる}{さ.げる}{spustiti (nešto)}
\example{下手}{へた <specijalno čitanje (mislim, trebalo bi možda provjeriti na nekom japanskom sajtu jel 熟字訓 ili nepravilno čitanje)>}{loš(a) u nečemu}
\example{地下}{ち.か}{podzemno}
\end{dictentry}

\begin{dictentry}{後}
\example{後で}{あと.で}{poslije, nakon (vremenski)}
\example{後ろ}{うし.ろ}{iza (fizički), straga}
\example{午後}{ご.ご}{poslijepodne}
\example{後輩}{こう.はい}{mlađi u nečemu (student, djelatnik, itd.)}
\end{dictentry}

\begin{dictentry}{外}
\example{外}{そと}{vani}
\example{外国人}{がい.こく.じん}{stranac}
\example{外す}{はず.す}{odvojiti, odspojiti, odstraniti}
\end{dictentry}

\begin{dictentry}{歩}
\example{歩く}{ある.く}{hodati}
\example{歩道}{ほ.どう}{nogostup}
\example{散歩}{さん.ぽ}{šetnja}
\end{dictentry}

\begin{dictentry}{家}
\example{家}{いえ}{kuća, prebivalište}
\example{家}{うち}{kuća, dom}
\example{家族}{か.ぞく}{obitelj}
\end{dictentry}

\begin{dictentry}{走}
\example{走る}{はし.る}{trčati}
\example{走行}{そう.こう}{putovanje, kretanje (vozila)}
\end{dictentry}

\begin{dictentry}{学}
\example{学生}{がく.せい}{učenik, student}
\example{文学}{ぶん.がく}{literatura}
\end{dictentry}

\begin{dictentry}{校}
\example{学校}{がっ.こう}{škola}
\end{dictentry}

\begin{dictentry}{店}
\example{店}{みせ}{dućan, lokal}
\example{店員}{てん.いん}{zaposlenik dućana ili lokala}
\end{dictentry}

\begin{dictentry}{近}
\example{近い}{ちか.い}{blizu}
\example{近所}{きん.じょ}{susjedstvo}
\end{dictentry}

\begin{dictentry}{達}
\example{友達}{とも.だち}{prijatelj(i)}
\end{dictentry}

\end{document}