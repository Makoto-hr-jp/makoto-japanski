% !TeX program = xelatex ?me -synctex=0 -interaction=nonstopmode -aux-directory=../tex_aux -output-directory=./release
% !TeX program = xelatex

\documentclass[12pt]{article}

\usepackage{lineno,changepage,lipsum}
\usepackage[colorlinks=true,urlcolor=blue]{hyperref}
\usepackage{fontspec}[ Path =../../../ ]
\usepackage{xeCJK}
\usepackage{tabularx}
\usepackage{graphicx}
\setCJKfamilyfont{chanto}{AOZORAMINCHOREGULAR_0.TTF}%
\setCJKfamilyfont{tegaki}{Mushin.otf}%
\usepackage[CJK,overlap]{ruby}
\usepackage{hhline}
\usepackage{multirow,array,amssymb}
\usepackage[croatian]{babel}
\usepackage{soul}
\usepackage[usenames, dvipsnames]{color}
\usepackage{wrapfig,booktabs}
\usepackage{calc}
\renewcommand{\rubysep}{0.1ex}
\renewcommand{\rubysize}{0.75}
\usepackage[margin=50pt]{geometry}
\usepackage{hyperref}
\modulolinenumbers[2]

\date{\today}

\usepackage{fancyhdr}
\pagestyle{fancy}
\fancyhf{}
\fancyhead[LE,RO]{\thepage}
\makeatletter
\fancyhead[RE,LO]{rev. \@date 誠}
\makeatother

\usepackage{pifont}
\newcommand{\cmark}{\ding{51}}%
\newcommand{\xmark}{\ding{55}}%

\newcommand{\dosl}{{\normalfont dosl. }}%
\newcommand{\rem}[1]{{\normalfont #1 }}%

\definecolor{faded}{RGB}{100, 100, 100}

\renewcommand{\arraystretch}{1.2}

%\ruby{}{}
%$($\href{URL}{text}$)$

\newcommand{\furigana}[2]{\ruby{#1}{#2}}
\newcommand{\tegaki}[1]{
	\CJKfamily{tegaki}\CJKnospace
	#1
	\CJKfamily{chanto}\CJKnospace
}

\newcommand{\dai}[1]{
	\vspace{20pt}
	\large
	\noindent\textbf{#1}
	\normalsize
	\vspace{20pt}
}

\newcommand{\fukudai}[1]{
	\vspace{10pt}
	\noindent\textbf{#1}
	\vspace{10pt}
}

\newenvironment{bunshou}{
	\vspace{10pt}
	\begin{adjustwidth}{1cm}{3cm}
	\begin{linenumbers}
}{
	\end{linenumbers}
	\end{adjustwidth}
}

\newenvironment{reibun}[1][]{
	\vspace{10pt}
	#1
	
	\begin{tabular}{l l}
}{
	\end{tabular}
	\vspace{10pt}
}
\newcommand{\rei}[2]{
	#1&\textit{#2}\\
}
\newcommand{\reinagai}[2]{
	\multicolumn{2}{l}{#1}\\
	\multicolumn{2}{l}{\hspace{10pt}\textit{#2}}\\
}

\newenvironment{mondai}[1]{
	\vspace{10pt}
	\noindent #1
	
	\begin{enumerate}
		\itemsep-5pt
	}{
	\end{enumerate}
}

\newenvironment{hyou}{
	\begin{itemize}
		\itemsep-5pt
	}{
	\end{itemize}
	\vspace{10pt}
}

\newcommand{\juuyou}[2][20pt]{
	\vspace{5pt}
		\noindent\hspace{#1}\parbox[c]{\textwidth-#1-#1}{\centering\textit{#2}}
	\vspace{5pt}
}

\newcommand{\ten}{
	\vspace{5pt}
	\noindent\hspace{-10pt}$\bullet$
}

\CJKfamily{chanto}\CJKnospace

\frenchspacing

\newenvironment{dictentry}[1]{
	\begin{tabular}{p{2cm} p{3cm} p{10cm}}
		#1 &&\\
}{
	\end{tabular}
	\vspace{20pt}
}

\newcommand{\example}[3]{
	\hspace*{\fill}#1 & #2 & #3\\
}

\author{ロボット君}
\begin{document}
	\dai{Lekcija 002}

\begin{dictentry}{点}
\example{点}{てん}{točka, bod, ocjena}
\end{dictentry}

\begin{dictentry}{半}
\example{半}{はん}{pola, polovica}
\end{dictentry}

\begin{dictentry}{音}
\example{音}{おと}{zvuk}
\example{音楽}{おん.がく}{glazba}
\end{dictentry}

\begin{dictentry}{先}
\example{先}{さき}{prethodno}
\example{先生}{せん.せい}{učitelj}
\end{dictentry}

\begin{dictentry}{生}
\example{生}{なま}{sirovo, uživo (npr. TV prijenos)}
\example{人生}{じん.せい}{ljudski život}
\example{生きる}{い.きる}{živjeti}
\example{生まれ}{う.まれ}{rodno mjesto}
\end{dictentry}

\begin{dictentry}{今}
\example{今}{いま}{sada}
\example{今日}{きょう}{danas}
\example{今日は}{こん.にち.は}{dobar dan}
\end{dictentry}

\begin{dictentry}{日}
\example{日}{ひ}{dan, sunce}
\example{日曜日}{にち.よう.び}{nedjelja}
\example{先日}{せん.じつ}{nekidan}
\end{dictentry}

\begin{dictentry}{猫}
\example{猫}{ねこ}{mačka}
\end{dictentry}

\begin{dictentry}{犬}
\example{犬}{いぬ}{pas}
\end{dictentry}

\begin{dictentry}{森}
\example{森}{もり}{šuma}
\end{dictentry}

\begin{dictentry}{石}
\example{石}{いし}{kamen}
\end{dictentry}

\begin{dictentry}{風}
\example{風}{かぜ}{vjetar}
\example{\textasciitilde 風}{\textasciitilde ふう}{stil}
\end{dictentry}

\begin{dictentry}{光}
\example{光}{ひかり}{svjetlo}
\end{dictentry}

\begin{dictentry}{月}
\example{月}{つき}{mjesec}
\example{三月}{さん.がつ}{ožujak}
\example{今月}{こん.げつ}{ovaj mjesec}
\end{dictentry}

\begin{dictentry}{空}
\example{空}{そら}{nebo}
\example{空気}{くう.き}{zrak, atmosfera}
\end{dictentry}

\begin{dictentry}{天}
\example{天気}{てん.き}{vrijeme (u meteorološkom smislu)}
\end{dictentry}

\begin{dictentry}{川}
\example{川}{かわ}{rijeka}
\end{dictentry}

\end{document}