% !TeX program = xelatex ?me -synctex=0 -interaction=nonstopmode -aux-directory=../tex_aux -output-directory=./release
% !TeX program = xelatex

\documentclass[12pt]{article}

\usepackage{lineno,changepage,lipsum}
\usepackage[colorlinks=true,urlcolor=blue]{hyperref}
\usepackage{fontspec}
\usepackage{xeCJK}
\usepackage{tabularx}
\setCJKfamilyfont{chanto}{AozoraMinchoRegular.ttf}
\setCJKfamilyfont{tegaki}{Mushin.otf}
\usepackage[CJK,overlap]{ruby}
\usepackage{hhline}
\usepackage{multirow,array,amssymb}
\usepackage[croatian]{babel}
\usepackage{soul}
\usepackage[usenames, dvipsnames]{color}
\usepackage{wrapfig,booktabs}
\renewcommand{\rubysep}{0.1ex}
\renewcommand{\rubysize}{0.75}
\usepackage[margin=50pt]{geometry}
\modulolinenumbers[2]

\usepackage{pifont}
\newcommand{\cmark}{\ding{51}}%
\newcommand{\xmark}{\ding{55}}%

\definecolor{faded}{RGB}{100, 100, 100}

\renewcommand{\arraystretch}{1.2}

%\ruby{}{}
%$($\href{URL}{text}$)$

\newcommand{\furigana}[2]{\ruby{#1}{#2}}
\newcommand{\tegaki}[1]{
	\CJKfamily{tegaki}\CJKnospace
	#1
	\CJKfamily{chanto}\CJKnospace
}

\newcommand{\dai}[1]{
	\vspace{20pt}
	\large
	\noindent\textbf{#1}
	\normalsize
	\vspace{20pt}
}

\newcommand{\fukudai}[1]{
	\vspace{10pt}
	\noindent\textbf{#1}
	\vspace{10pt}
}

\newenvironment{bunshou}{
	\vspace{10pt}
	\begin{adjustwidth}{1cm}{3cm}
	\begin{linenumbers}
}{
	\end{linenumbers}
	\end{adjustwidth}
}

\newenvironment{reibun}{
	\vspace{10pt}
	\begin{tabular}{l l}
}{
	\end{tabular}
	\vspace{10pt}
}
\newcommand{\rei}[2]{
	#1&\textit{#2}\\
}
\newcommand{\reinagai}[2]{
	\multicolumn{2}{l}{#1}\\
	\multicolumn{2}{l}{\hspace{10pt}\textit{#2}}\\
}

\newenvironment{mondai}[1]{
	\vspace{10pt}
	#1
	
	\begin{enumerate}
		\itemsep-5pt
	}{
	\end{enumerate}
	\vspace{10pt}
}

\newenvironment{hyou}{
	\begin{itemize}
		\itemsep-5pt
	}{
	\end{itemize}
	\vspace{10pt}
}

\date{\today}

\CJKfamily{chanto}\CJKnospace

\newenvironment{dictentry}[1]{
	\begin{tabular}{p{2cm} p{3cm} p{10cm}}
		#1 &&\\
}{
	\end{tabular}
	\vspace{20pt}
}

\newcommand{\example}[3]{
	\hspace*{\fill}#1 & #2 & #3\\
}

\author{ロボット君}
\begin{document}
	\dai{Lekcija 022}

\begin{dictentry}{足}
\example{足}{あし}{stopalo, noga}
\example{足りる}{た.りる}{dostajati, biti dovoljno}
\example{不足}{ふ.そく}{nedovoljno}
\end{dictentry}

\begin{dictentry}{速}
\example{速い}{はや.い}{brzo}
\example{速度}{そく.ど}{brzina}
\end{dictentry}

\begin{dictentry}{見}
\example{見る}{み.る}{vidjeti}
\example{発見}{はっ.けん}{otkriće}
\end{dictentry}

\begin{dictentry}{玉}
\example{玉}{たま}{lopta, kugla}
\end{dictentry}

\begin{dictentry}{朝}
\example{朝}{あさ}{jutro}
\example{今朝}{け.さ \footnotemark[1]}{ovo jutro}
\end{dictentry}

\begin{dictentry}{冬}
\example{冬}{ふゆ}{zima (godišnje doba)}
\end{dictentry}

\begin{dictentry}{東}
\example{東}{ひがし}{istok}
\example{東北}{とう.ほく}{sjeveroistok, Tōhoku regija}
\end{dictentry}

\begin{dictentry}{京}
\example{京と}{きょう.と}{Kyōto}
\example{東京}{とう.きょう}{Tōkyō}
\end{dictentry}

\begin{dictentry}{笑}
\example{笑う}{わら.う}{smijati se}
\example{笑顔}{え.がお}{osmijeh}
\end{dictentry}

\begin{dictentry}{顔}
\example{顔}{かお}{lice}
\example{顔面}{がん.めん}{lice}
\end{dictentry}

\footnotetext[1]{posebno čitanje}

\end{document}