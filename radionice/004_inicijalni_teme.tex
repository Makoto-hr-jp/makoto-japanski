% !TeX document-id = {301e5d09-5946-4337-a5d9-05a6fc72814a}
% !TeX program = xelatex ?me -synctex=0 -interaction=nonstopmode -aux-directory=../tex_aux -output-directory=./release
% !TeX program = xelatex

\documentclass[12pt]{article}

\usepackage{lineno,changepage,lipsum}
\usepackage[colorlinks=true,urlcolor=blue]{hyperref}
\usepackage{fontspec}[ Path =../../../ ]
\usepackage{xeCJK}
\usepackage{tabularx}
\usepackage{graphicx}
\setCJKfamilyfont{chanto}{AOZORAMINCHOREGULAR_0.TTF}%
\setCJKfamilyfont{tegaki}{Mushin.otf}%
\usepackage[CJK,overlap]{ruby}
\usepackage{hhline}
\usepackage{multirow,array,amssymb}
\usepackage[croatian]{babel}
\usepackage{soul}
\usepackage[usenames, dvipsnames]{color}
\usepackage{wrapfig,booktabs}
\usepackage{calc}
\renewcommand{\rubysep}{0.1ex}
\renewcommand{\rubysize}{0.75}
\usepackage[margin=50pt]{geometry}
\usepackage{hyperref}
\modulolinenumbers[2]

\date{\today}

\usepackage{fancyhdr}
\pagestyle{fancy}
\fancyhf{}
\fancyhead[LE,RO]{\thepage}
\makeatletter
\fancyhead[RE,LO]{rev. \@date 誠}
\makeatother

\usepackage{pifont}
\newcommand{\cmark}{\ding{51}}%
\newcommand{\xmark}{\ding{55}}%

\newcommand{\dosl}{{\normalfont dosl. }}%
\newcommand{\rem}[1]{{\normalfont #1 }}%

\definecolor{faded}{RGB}{100, 100, 100}

\renewcommand{\arraystretch}{1.2}

%\ruby{}{}
%$($\href{URL}{text}$)$

\newcommand{\furigana}[2]{\ruby{#1}{#2}}
\newcommand{\tegaki}[1]{
	\CJKfamily{tegaki}\CJKnospace
	#1
	\CJKfamily{chanto}\CJKnospace
}

\newcommand{\dai}[1]{
	\vspace{20pt}
	\large
	\noindent\textbf{#1}
	\normalsize
	\vspace{20pt}
}

\newcommand{\fukudai}[1]{
	\vspace{10pt}
	\noindent\textbf{#1}
	\vspace{10pt}
}

\newenvironment{bunshou}{
	\vspace{10pt}
	\begin{adjustwidth}{1cm}{3cm}
	\begin{linenumbers}
}{
	\end{linenumbers}
	\end{adjustwidth}
}

\newenvironment{reibun}[1][]{
	\vspace{10pt}
	#1
	
	\begin{tabular}{l l}
}{
	\end{tabular}
	\vspace{10pt}
}
\newcommand{\rei}[2]{
	#1&\textit{#2}\\
}
\newcommand{\reinagai}[2]{
	\multicolumn{2}{l}{#1}\\
	\multicolumn{2}{l}{\hspace{10pt}\textit{#2}}\\
}

\newenvironment{mondai}[1]{
	\vspace{10pt}
	\noindent #1
	
	\begin{enumerate}
		\itemsep-5pt
	}{
	\end{enumerate}
}

\newenvironment{hyou}{
	\begin{itemize}
		\itemsep-5pt
	}{
	\end{itemize}
	\vspace{10pt}
}

\newcommand{\juuyou}[2][20pt]{
	\vspace{5pt}
		\noindent\hspace{#1}\parbox[c]{\textwidth-#1-#1}{\centering\textit{#2}}
	\vspace{5pt}
}

\newcommand{\ten}{
	\vspace{5pt}
	\noindent\hspace{-10pt}$\bullet$
}

\CJKfamily{chanto}\CJKnospace

\frenchspacing
\author{Tomislav Mamić}
\begin{document}
	\dai{Teme za inicijalni ispit}
	
	Bitno je da zadaci idu navedenim redoslijedom tako da ispit ide od osnovnih stvari prema težima. Zadatke bih stavio da imaju 3\textasciitilde5 stvari koje treba riješiti/odabrati/nadopuniti tako da ispit ne bude predug.
	
	\begin{hyou}
		\item čestica の i pokazne zamjenice
		\item pravilno oslovljavanje sebe i drugih
		\item oblici spojnog glagola (i pristojni i kolokvijalni)
		\item pridjevi - pr. i neg.
		\begin{hyou}
			\item kao predikat
			\item kao opis
		\end{hyou}
		\item glagoli - pr. i neg, barem po jedan 一段, 五段 i 不規則
	\end{hyou}
	Nadalje ubacivati jednostavnije ideograme. Za mjesto, vrijeme i količinu i barem po jednu upitnu rečenicu.
	\begin{hyou}
		\item lokacija, smjer (に,で,へ,から,まで)
		\item vrijeme (に,から,まで)
		\item količina - brojači za ljude, sate, životinje itd, stari つ brojevi
		\item 連用形
		\begin{hyou}
			\item ます
			\item たい, やすい, にくい
			\item すぎる, はじめる
		\end{hyou}
		\item て
		\begin{hyou}
			\item nizanje rečenica
			\item nizanje pridjeva
			\item いる, みる, あげる, くれる, もらう
		\end{hyou}
	\end{hyou}
	Ovdje završava gradivo početnih radionica.
	\begin{hyou}
		\item spojevi rečenica
		\begin{hyou}
			\item でも, そして, だから, それから, それとも
			\item けど, ので, のに
			\item から kao razlog
			\item \textasciitilde て+から i \textasciitilde い+ながら
		\end{hyou}
	\end{hyou}
	\begin{hyou}
		\item slaganje opisom (連体形)
		\begin{hyou}
			\item upotreba こと i (も)の
			\item ことができる, ことにする / なる
			\item zav. reč. vremena i količine s とき, とたん, ほど, ぐらい
		\end{hyou}
		\item 意志形
		\begin{hyou}
			\item +と思う
			\item +とする
		\end{hyou}
		\item 可能形
		\item 命令形
		\begin{hyou}
			\item pravi
			\item い+なさい
			\item 辞書形+な
			\item て oblik sam kao imperativ
		\end{hyou}
		\item 受動態
		\item 使役動詞
	\end{hyou}
Ovdje završava gradivo srednjih radionica.
\end{document}