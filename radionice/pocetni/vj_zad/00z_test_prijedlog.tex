% !TeX program = xelatex
% !TeX program = xelatex

\documentclass[12pt]{article}

\usepackage{lineno,changepage,lipsum}
\usepackage[colorlinks=true,urlcolor=blue]{hyperref}
\usepackage{fontspec}[ Path =../../../ ]
\usepackage{xeCJK}
\usepackage{tabularx}
\usepackage{graphicx}
\setCJKfamilyfont{chanto}{AOZORAMINCHOREGULAR_0.TTF}%
\setCJKfamilyfont{tegaki}{Mushin.otf}%
\usepackage[CJK,overlap]{ruby}
\usepackage{hhline}
\usepackage{multirow,array,amssymb}
\usepackage[croatian]{babel}
\usepackage{soul}
\usepackage[usenames, dvipsnames]{color}
\usepackage{wrapfig,booktabs}
\usepackage{calc}
\renewcommand{\rubysep}{0.1ex}
\renewcommand{\rubysize}{0.75}
\usepackage[margin=50pt]{geometry}
\usepackage{hyperref}
\modulolinenumbers[2]

\date{\today}

\usepackage{fancyhdr}
\pagestyle{fancy}
\fancyhf{}
\fancyhead[LE,RO]{\thepage}
\makeatletter
\fancyhead[RE,LO]{rev. \@date 誠}
\makeatother

\usepackage{pifont}
\newcommand{\cmark}{\ding{51}}%
\newcommand{\xmark}{\ding{55}}%

\newcommand{\dosl}{{\normalfont dosl. }}%
\newcommand{\rem}[1]{{\normalfont #1 }}%

\definecolor{faded}{RGB}{100, 100, 100}

\renewcommand{\arraystretch}{1.2}

%\ruby{}{}
%$($\href{URL}{text}$)$

\newcommand{\furigana}[2]{\ruby{#1}{#2}}
\newcommand{\tegaki}[1]{
	\CJKfamily{tegaki}\CJKnospace
	#1
	\CJKfamily{chanto}\CJKnospace
}

\newcommand{\dai}[1]{
	\vspace{20pt}
	\large
	\noindent\textbf{#1}
	\normalsize
	\vspace{20pt}
}

\newcommand{\fukudai}[1]{
	\vspace{10pt}
	\noindent\textbf{#1}
	\vspace{10pt}
}

\newenvironment{bunshou}{
	\vspace{10pt}
	\begin{adjustwidth}{1cm}{3cm}
	\begin{linenumbers}
}{
	\end{linenumbers}
	\end{adjustwidth}
}

\newenvironment{reibun}[1][]{
	\vspace{10pt}
	#1
	
	\begin{tabular}{l l}
}{
	\end{tabular}
	\vspace{10pt}
}
\newcommand{\rei}[2]{
	#1&\textit{#2}\\
}
\newcommand{\reinagai}[2]{
	\multicolumn{2}{l}{#1}\\
	\multicolumn{2}{l}{\hspace{10pt}\textit{#2}}\\
}

\newenvironment{mondai}[1]{
	\vspace{10pt}
	\noindent #1
	
	\begin{enumerate}
		\itemsep-5pt
	}{
	\end{enumerate}
}

\newenvironment{hyou}{
	\begin{itemize}
		\itemsep-5pt
	}{
	\end{itemize}
	\vspace{10pt}
}

\newcommand{\juuyou}[2][20pt]{
	\vspace{5pt}
		\noindent\hspace{#1}\parbox[c]{\textwidth-#1-#1}{\centering\textit{#2}}
	\vspace{5pt}
}

\newcommand{\ten}{
	\vspace{5pt}
	\noindent\hspace{-10pt}$\bullet$
}

\CJKfamily{chanto}\CJKnospace

\frenchspacing

\author{Tomislav Mamić, Željka Ludošan}	% dodajte se ovdje

\begin{document}
	
	\dai{Ponavljanje - て oblik i priložne oznake}
	
% sviđa mi se kombo かわいい i いい, daje ljudima prilike da krivo zaključe
	\begin{mondai}{Spojite dvije rečenice u jednu istim redoslijedom koristeći てoblik.
		\begin{reibun}[Primjer:]
		\rei{あかい。 あたらしい。 }{$\rightarrow$ あかくて あたらしい。} % izbačen primjer jer rješava 4/5 zadataka
		\end{reibun} }

		\item あおい。 ながい。
		\item かわいい。 あたたかい。
		\item きれいな。 みじかい。 % izbačeno な, znat će oni to
		\item せいが たかい。 こわい。% zamijenjeno せい sa せ jer je češće u tom konkretnom izrazu i tako smo dosad spominjali
		\item あたまが いい。 かみが ながい。 
		\vspace{-15pt} % ovako možete regulirati vertikalni razmak među stvarima ako vam se ne sviđa raspored
	\end{mondai}

	\begin{mondai}{Upišite redoslijed riječi u rečenici tako da rečenica bude gramatički ispravna. 
		\begin{reibun}[Primjer:]
		\rei{あした\_\_ \_\_ \_\_おきる。 $\rightarrow$ あした  \underbar{  2  }  \underbar{  3  }  \underbar{  1  } おきる。}{\vspace{10pt}}
		\rei{1.に 2.は3.6じ}{}
		\end{reibun} }

		\item あの\_\_ \_\_ \_\_ください。
		\vspace{10pt}
		\newline 1.まって 2.しま 3.で
		\vspace{20pt}
		\item へや\_\_ \_\_ \_\_だ。 % ispravljen kraj rečenice (い pridjevi kolokvijalno završavaju reč, nikad s だ), stavljeno ず umjesto づ
		\vspace{10pt}
		\newline 1.は 2.さむい 3.しづかで
		\vspace{20pt}
		\item たけしさん\_\_ \_\_ \_\_ \_\_ \_\_いった。 % ovaj zad mi nije baš jasan, što je trebao biti rezultat i značenje?
		\vspace{10pt}
		\newline 1.に 2.いえ3.でて4.は 5.かいに
	\end{mondai}

	\vspace{10pt}
	\fukudai{Kratka priča} % maknuo podnaslov, izbacio format teksta. Svrha onog formata je da olakša čitanje dugih tekstova jer formatira na način na koji bi bio pisan tekst u knjizi, samo vodoravno i dodaje brojeve redova. Nije baš zgodno za kućice itd.
	
	\begin{mondai}{Popunite kućice sljedećim elementima. Svaki element može biti korišten samo jednom. }

		\item 
	\begin{tabular}{|l|l|l|}
		\hline
		よく&いる&とても\\\hline
		いて&まいにち&きょう\\\hline % zamijenio きょう s いま - reći "danas je veljača" nije baš zgodno jer je "danas" neki "dan"
	\end{tabular}

	\end{mondai}

	\begin{bunshou}
		\framebox{ \begin{minipage}{0.4in}\hfill\vspace{0.3in}\end{minipage} }\vspace{10pt}は 2がつ だ。 こんげつは 私の ねこが 
		\framebox{ \begin{minipage}{0.4in}\hfill\vspace{0.3in}\end{minipage} }\vspace{10pt}
		 にゃにゃしている。 ねこの なまえは ミミ だ。 ミミは % dodane tilde iza svakog にゃ, している promijenjeno u いっている
		\framebox{ \begin{minipage}{0.4in}\hfill\vspace{0.3in}\end{minipage} }\vspace{10pt} ソファーに すわて % dodano っ u すわる
		\framebox{ \begin{minipage}{0.4in}\hfill\vspace{0.3in}\end{minipage} }\vspace{10pt} そらを みて
		\framebox{ \begin{minipage}{0.4in}\hfill\vspace{0.3in}\end{minipage} }\vspace{10pt}。 ミミは 
		\framebox{ \begin{minipage}{0.4in}\hfill\vspace{0.3in}\end{minipage} }\vspace{10pt} うれしい ねこ だ。% zamijenjeno うれしい sa しあわせな jer je to prirodnije reći za nekog drugog u ovakvoj situaciji
	\end{bunshou}
	\end{document}