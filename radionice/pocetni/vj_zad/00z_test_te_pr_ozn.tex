% !TeX program = xelatex
% !TeX program = xelatex

\documentclass[12pt]{article}

\usepackage{lineno,changepage,lipsum}
\usepackage[colorlinks=true,urlcolor=blue]{hyperref}
\usepackage{fontspec}
\usepackage{xeCJK}
\usepackage{tabularx}
\setCJKfamilyfont{chanto}{AozoraMinchoRegular.ttf}
\setCJKfamilyfont{tegaki}{Mushin.otf}
\usepackage[CJK,overlap]{ruby}
\usepackage{hhline}
\usepackage{multirow,array,amssymb}
\usepackage[croatian]{babel}
\usepackage{soul}
\usepackage[usenames, dvipsnames]{color}
\usepackage{wrapfig,booktabs}
\renewcommand{\rubysep}{0.1ex}
\renewcommand{\rubysize}{0.75}
\usepackage[margin=50pt]{geometry}
\modulolinenumbers[2]

\usepackage{pifont}
\newcommand{\cmark}{\ding{51}}%
\newcommand{\xmark}{\ding{55}}%

\definecolor{faded}{RGB}{100, 100, 100}

\renewcommand{\arraystretch}{1.2}

%\ruby{}{}
%$($\href{URL}{text}$)$

\newcommand{\furigana}[2]{\ruby{#1}{#2}}
\newcommand{\tegaki}[1]{
	\CJKfamily{tegaki}\CJKnospace
	#1
	\CJKfamily{chanto}\CJKnospace
}

\newcommand{\dai}[1]{
	\vspace{20pt}
	\large
	\noindent\textbf{#1}
	\normalsize
	\vspace{20pt}
}

\newcommand{\fukudai}[1]{
	\vspace{10pt}
	\noindent\textbf{#1}
	\vspace{10pt}
}

\newenvironment{bunshou}{
	\vspace{10pt}
	\begin{adjustwidth}{1cm}{3cm}
	\begin{linenumbers}
}{
	\end{linenumbers}
	\end{adjustwidth}
}

\newenvironment{reibun}{
	\vspace{10pt}
	\begin{tabular}{l l}
}{
	\end{tabular}
	\vspace{10pt}
}
\newcommand{\rei}[2]{
	#1&\textit{#2}\\
}
\newcommand{\reinagai}[2]{
	\multicolumn{2}{l}{#1}\\
	\multicolumn{2}{l}{\hspace{10pt}\textit{#2}}\\
}

\newenvironment{mondai}[1]{
	\vspace{10pt}
	#1
	
	\begin{enumerate}
		\itemsep-5pt
	}{
	\end{enumerate}
	\vspace{10pt}
}

\newenvironment{hyou}{
	\begin{itemize}
		\itemsep-5pt
	}{
	\end{itemize}
	\vspace{10pt}
}

\date{\today}

\CJKfamily{chanto}\CJKnospace

\author{Tomislav Mamić, Željka Ludošan}	% dodajte se ovdje

\begin{document}
	
	\dai{Ponavljanje - て oblik i priložne oznake}
	
	\begin{mondai}{Spojite dvije rečenice u jednu istim redoslijedom koristeći てoblik.}
		\item あおい。 ながい。
		\item かわいい。 あたたかい。
		\item きれい。 みじかい。
		\item せが たかい。 こわい。
		\item あたまが いい。 かみが ながい。
	\end{mondai}

	\begin{mondai}{Upišite redoslijed riječi u rečenici tako da rečenica bude gramatički ispravna.
			
			\begin{reibun}[Primjer:]
				\rei{あした\_\_ \_\_ \_\_おきる。 $\rightarrow$ あした  \underbar{  2  }  \underbar{  3  }  \underbar{  1  } おきる。}{\vspace{10pt}}
				\rei{1.に 2.は3.6じ}{}
		\end{reibun} }
		
		\item あの\_\_ \_\_ \_\_ください。
		\vspace{10pt}
		\newline 1.まって 2.しま 3.で
		\vspace{20pt}
		\item へや\_\_ \_\_ \_\_。
		\vspace{10pt}
		\newline 1.は 2.さむい 3.しずかで
		\vspace{20pt}
		\item たけしさん\_\_ \_\_ \_\_ \_\_ \_\_でかけた。
		\vspace{10pt}
		\newline 1.かいもの 2.いえを3.でて4.は 5.に
	\end{mondai}

	\begin{mondai}{Popunite kućice sljedećim elementima. Svaki element može biti korišten samo jednom.
		
		\item\hspace{30pt}
		\begin{tabular}{|l|l|l|}
			\hline
			よく&いる&とても\\\hline
			いて&まいにち&いま\\\hline
		\end{tabular}
	}
	\item\framebox{ \begin{minipage}{0.4in}\hfill\vspace{0.3in}\end{minipage} }\vspace{10pt} は 2がつ だ。
	\item こんげつは 私の ねこが \framebox{ \begin{minipage}{0.4in}\hfill\vspace{0.3in}\end{minipage} }\vspace{10pt} にゃ\textasciitilde にゃ\textasciitilde いっている。 ねこの なまえは ミミ だ。
	\item ミミは \framebox{\begin{minipage}{0.4in}\hfill\vspace{0.3in}\end{minipage}}\vspace{10pt} ソファーに すわって
	\framebox{\begin{minipage}{0.4in}\hfill\vspace{0.3in}\end{minipage}}\vspace{10pt} そらを みて
	\framebox{\begin{minipage}{0.4in}\hfill\vspace{0.3in}\end{minipage}}\vspace{10pt} 。
	\item ミミは \framebox{\begin{minipage}{0.4in}\hfill\vspace{0.3in}\end{minipage}}\vspace{10pt} しあわせな ねこ だ。
		
	\end{mondai}

\end{document}