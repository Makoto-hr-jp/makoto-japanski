% !TeX document-id = {bb99a8fd-d8b6-458e-8710-48a7470c9983}
% !TeX program = xelatex ?me -synctex=0 -interaction=nonstopmode -aux-directory=../tex_aux -output-directory=./release

% !TeX program = xelatex

\documentclass[12pt]{article}

\usepackage{lineno,changepage,lipsum}
\usepackage[colorlinks=true,urlcolor=blue]{hyperref}
\usepackage{fontspec}
\usepackage{xeCJK}
\usepackage{tabularx}
\setCJKfamilyfont{chanto}{AozoraMinchoRegular.ttf}
\setCJKfamilyfont{tegaki}{Mushin.otf}
\usepackage[CJK,overlap]{ruby}
\usepackage{hhline}
\usepackage{multirow,array,amssymb}
\usepackage[croatian]{babel}
\usepackage{soul}
\usepackage[usenames, dvipsnames]{color}
\usepackage{wrapfig,booktabs}
\renewcommand{\rubysep}{0.1ex}
\renewcommand{\rubysize}{0.75}
\usepackage[margin=50pt]{geometry}
\modulolinenumbers[2]

\usepackage{pifont}
\newcommand{\cmark}{\ding{51}}%
\newcommand{\xmark}{\ding{55}}%

\definecolor{faded}{RGB}{100, 100, 100}

\renewcommand{\arraystretch}{1.2}

%\ruby{}{}
%$($\href{URL}{text}$)$

\newcommand{\furigana}[2]{\ruby{#1}{#2}}
\newcommand{\tegaki}[1]{
	\CJKfamily{tegaki}\CJKnospace
	#1
	\CJKfamily{chanto}\CJKnospace
}

\newcommand{\dai}[1]{
	\vspace{20pt}
	\large
	\noindent\textbf{#1}
	\normalsize
	\vspace{20pt}
}

\newcommand{\fukudai}[1]{
	\vspace{10pt}
	\noindent\textbf{#1}
	\vspace{10pt}
}

\newenvironment{bunshou}{
	\vspace{10pt}
	\begin{adjustwidth}{1cm}{3cm}
	\begin{linenumbers}
}{
	\end{linenumbers}
	\end{adjustwidth}
}

\newenvironment{reibun}{
	\vspace{10pt}
	\begin{tabular}{l l}
}{
	\end{tabular}
	\vspace{10pt}
}
\newcommand{\rei}[2]{
	#1&\textit{#2}\\
}
\newcommand{\reinagai}[2]{
	\multicolumn{2}{l}{#1}\\
	\multicolumn{2}{l}{\hspace{10pt}\textit{#2}}\\
}

\newenvironment{mondai}[1]{
	\vspace{10pt}
	#1
	
	\begin{enumerate}
		\itemsep-5pt
	}{
	\end{enumerate}
	\vspace{10pt}
}

\newenvironment{hyou}{
	\begin{itemize}
		\itemsep-5pt
	}{
	\end{itemize}
	\vspace{10pt}
}

\date{\today}

\CJKfamily{chanto}\CJKnospace

\author{Tomislav Mamić}

\begin{document}
	\dai{Ponavljanje - て oblik i osnovne pril. oznake}
	
	\fukudai{Kad?}
	
	\begin{mondai}{Prevedi na japanski:}
		\item Sinoć sam kasno došao kući, a jutros sam rano ustao.
		\item Film traje od 19:45 do 21:30.
		\item Od jutra sjedi i gleda u zid.
		\item Takeši još nije pospremio sobu.
		\item Njegova mama to već zna.
	\end{mondai}

	\fukudai{Gdje?}
	
	\begin{mondai}{Prevedi na japanski:}
		\item U kutiji pod stolom spavao je crni mačić.
		\item Pošta je lijevo od stanice vlaka.
		\item Pred dućanom je bio pas kojeg smo jučer vidjeli pod borom u svom vrtu.
		\item Izišao je iz ormara i potrčao prema vratima.
		\item Od mog prozora do drveta u vrtu ima 10ak metara.
	\end{mondai}

	\fukudai{Koliko?}
	
	\begin{mondai}{Prevedi na japanski:}
		\item Pojeo sam tri jabuke i dvije kruške.
		\item Koliko mačaka imaš?
		\item U koliko sati i gdje ćemo se naći?
		\item Kyousuke je za valentinovo dobio 14 čokolada.
		\item Jučer sam u Kyotu kupio četiri knjige.
	\end{mondai}

	\fukudai{Još malo て oblika}
	
	\begin{mondai}{Prevedi na hrvatski:}
		\item なっとうを たべてみて、いがいと おいしかった。
		\item これから まいにち へやを そうじ してください。
		\item わたしの にわから はなを とらないで。
		\item きのう、きょねん なくした ゆびわを さがしている ひとに あった。*
	\end{mondai}
\end{document}