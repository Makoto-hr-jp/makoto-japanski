% !TeX document-id = {aa1964cd-e4d1-4c8c-8134-6010424df1b4}
% !TeX program = xelatex ?me -synctex=0 -interaction=nonstopmode -aux-directory=../../tex_aux -output-directory=./release
% !TeX program = xelatex

\documentclass[12pt]{article}

\usepackage{lineno,changepage,lipsum}
\usepackage[colorlinks=true,urlcolor=blue]{hyperref}
\usepackage{fontspec}[ Path =../../../ ]
\usepackage{xeCJK}
\usepackage{tabularx}
\usepackage{graphicx}
\setCJKfamilyfont{chanto}{AOZORAMINCHOREGULAR_0.TTF}%
\setCJKfamilyfont{tegaki}{Mushin.otf}%
\usepackage[CJK,overlap]{ruby}
\usepackage{hhline}
\usepackage{multirow,array,amssymb}
\usepackage[croatian]{babel}
\usepackage{soul}
\usepackage[usenames, dvipsnames]{color}
\usepackage{wrapfig,booktabs}
\usepackage{calc}
\renewcommand{\rubysep}{0.1ex}
\renewcommand{\rubysize}{0.75}
\usepackage[margin=50pt]{geometry}
\usepackage{hyperref}
\modulolinenumbers[2]

\date{\today}

\usepackage{fancyhdr}
\pagestyle{fancy}
\fancyhf{}
\fancyhead[LE,RO]{\thepage}
\makeatletter
\fancyhead[RE,LO]{rev. \@date 誠}
\makeatother

\usepackage{pifont}
\newcommand{\cmark}{\ding{51}}%
\newcommand{\xmark}{\ding{55}}%

\newcommand{\dosl}{{\normalfont dosl. }}%
\newcommand{\rem}[1]{{\normalfont #1 }}%

\definecolor{faded}{RGB}{100, 100, 100}

\renewcommand{\arraystretch}{1.2}

%\ruby{}{}
%$($\href{URL}{text}$)$

\newcommand{\furigana}[2]{\ruby{#1}{#2}}
\newcommand{\tegaki}[1]{
	\CJKfamily{tegaki}\CJKnospace
	#1
	\CJKfamily{chanto}\CJKnospace
}

\newcommand{\dai}[1]{
	\vspace{20pt}
	\large
	\noindent\textbf{#1}
	\normalsize
	\vspace{20pt}
}

\newcommand{\fukudai}[1]{
	\vspace{10pt}
	\noindent\textbf{#1}
	\vspace{10pt}
}

\newenvironment{bunshou}{
	\vspace{10pt}
	\begin{adjustwidth}{1cm}{3cm}
	\begin{linenumbers}
}{
	\end{linenumbers}
	\end{adjustwidth}
}

\newenvironment{reibun}[1][]{
	\vspace{10pt}
	#1
	
	\begin{tabular}{l l}
}{
	\end{tabular}
	\vspace{10pt}
}
\newcommand{\rei}[2]{
	#1&\textit{#2}\\
}
\newcommand{\reinagai}[2]{
	\multicolumn{2}{l}{#1}\\
	\multicolumn{2}{l}{\hspace{10pt}\textit{#2}}\\
}

\newenvironment{mondai}[1]{
	\vspace{10pt}
	\noindent #1
	
	\begin{enumerate}
		\itemsep-5pt
	}{
	\end{enumerate}
}

\newenvironment{hyou}{
	\begin{itemize}
		\itemsep-5pt
	}{
	\end{itemize}
	\vspace{10pt}
}

\newcommand{\juuyou}[2][20pt]{
	\vspace{5pt}
		\noindent\hspace{#1}\parbox[c]{\textwidth-#1-#1}{\centering\textit{#2}}
	\vspace{5pt}
}

\newcommand{\ten}{
	\vspace{5pt}
	\noindent\hspace{-10pt}$\bullet$
}

\CJKfamily{chanto}\CJKnospace

\frenchspacing
\author{Goran Zovak, Tomislav Mamić}
\begin{document}
	\dai{Kratka provjera znanja 1}
	
	\begin{mondai}{1. Odaberite ispravan zapis:}
		\item ka.ra.o.ke\\ % カラオケ
		\begin{tabular}{l l l l l}
			a) からをけ & b) からあけ & c) からおげ & d) かちおけ & e) からおけ\\
		\end{tabular}
		\item ho.n.mo.no\\ % 本物
		\begin{tabular}{l l l l l}
			a) ほのもん & b) ぽんもの & c) はんもの & d) ほんもの & e) ぼのもの\\
		\end{tabular}
		\item kyo.u\\ % 今日
		\begin{tabular}{l l l l l}
			a) きょう & b) きよう & c) きょお & d) きよぉ & e) きょおぅ\\
		\end{tabular}
		\item shu.s.sha\\ % 出社
		\begin{tabular}{l l l l l}
			a) すっしゃ & b) しゆしゃ & c) すゅしっや & d) しゅっしゃ & e) しゅしゃ\\
		\end{tabular}
		\item to.u.kyo.u\\ % 東京
		\begin{tabular}{l l l l l}
			a) ときお & b) とうきょ & c) とおきょお & d) とうきょう & e) とおきゅ\\
		\end{tabular}
		\item ta.shi.ro.ji.ma\\ % 田代島
		\begin{tabular}{l l l l l}
			a) なしろじま & b) だしろじま & c) たじろしま & d) なしるじま & e) たしろじま\\
		\end{tabular}
		\item bo.u.so.u.zo.ku\\ % 暴走族
		\begin{tabular}{l l l l l}
			a) ぼそうぞく & b) ぽぞうそく & c) ほうそおそぐ & d) ぼうそうぞく & e) ぼうぞうそく\\
		\end{tabular}
		\item kyu.u.shu.u\\ % 九州
		\begin{tabular}{l l l l l}
			a) きゅしゅー & b) きょうしょう & c) きゅうしゅう & d) きょっしゅう & e) きゅうしょう\\
		\end{tabular}
		\item ju.n.bi\\ % 準備
		\begin{tabular}{l l l l l}
			a) じゅんび & b) しゅんひ & c) じゅうんび & d) しゅーんひ & e) じゅんぴ\\
		\end{tabular}
		\item nu.su.bi.to\\ % 盗人
		\begin{tabular}{l l l l l}
			a) めすひと & b) ぬしびと & c) めずぴと & d) ぬすびと & e) めしぴと\\
		\end{tabular}
		\item ne.s.shi.n\\ % 熱心
		\begin{tabular}{l l l l l}
			a) わっしん & b) れつじん & c) ねっしん & d) わつじん & e) れっしん\\
		\end{tabular}
	\end{mondai}

	\begin{mondai}{2. Zapišite čitanje sljedećih riječi:}
		\item ふくろう
		\item ちゃいろ
		\item かきごおり
		\item はっぴょう
		\item ばっちり
		\item ばんごうふだ
		\item ぎゅうにゅう
		\item おもわず
		\item しゅっしん
		\item しょうがっこう
		\item こんばんは
		\item きをつける
	\end{mondai}

	\pagebreak
	\begin{mondai}{3. Zapišite hiraganom:}
		\item mo.no.ga.ta.ri
		\item chi.ka.te.tsu
		\item ryu.u.ga.ku.se.i
		\item nyu.u.jo.u
		\item kya.k.ka
		\item shu.u.de.n
		\item sho.ku.do.u
		\item do.ryo.ku
		\item do.u.ryo.u
		\item i.p.po.n.gi
		\item mo.ji.do.o.ri
		\item so.tsu.gyo.u.shi.ki
	\end{mondai}

	\begin{mondai}{4. Prevedite:}
		\item \textit{ono}
		\item \textit{ta mačka}
		\item \textit{onaj pas}
		\item \textit{hrana ove ptice}
		\item \textit{ovaj pas i njegov prijatelj}
		\item \textit{Takeši i moj mlađi brat}
	\end{mondai}

	\begin{mondai}{5. Povežite riječi česticom の kako bi dobili traženo značenje:}
		\item \textit{učiteljeva knjiga} - (せんせい, ほん)
		\item \textit{ime brata} - (おとうと, なまえ)
		\item \textit{svačije pravo} - (みんな, けん)
		\item \textit{pitanje vremena} - (もんだい, じかん)
		\item \textit{mačka prijateljeve sestre} - (ねこ, ともだち, いもうと)
		\item \textit{njezin uobičajen dan} - (かのじょ, ふつう, ひ)
		\item \textit{njegov stvarni razlog} - (かれ, ほんとう, りゆう)
	\end{mondai}

	\begin{mondai}{6. Prikladno oslovite ljude uzevši u obzir situaciju u zagradama:}
		\item \textit{gospodin Miyagi}
		\item \textit{učitelj Tanaka}
		\item \textit{njegov otac}
		\item \textit{tvoj djed} (nekome mlađem od sebe)
		\item \textit{vaša žena} (kolegi koji se preziva Matsuyama)
		\item \textit{Hanako, Takeshi i Keiichi} (nabrajajući ljude koji će biti na zabavi)
		\item \textit{brat moje majke}
		\item \textit{draga Keiko i Hiroshi} (djed o svojoj unučadi)
	\end{mondai}

\end{document}
