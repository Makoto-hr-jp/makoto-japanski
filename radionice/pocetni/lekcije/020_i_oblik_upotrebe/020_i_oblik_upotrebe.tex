% !TeX document-id = {882a7644-d3fa-41ce-8935-90b47948086e}
% !TeX program = xelatex ?me -synctex=0 -interaction=nonstopmode -aux-directory=../tex_aux -output-directory=./release
% !TeX program = xelatex

\documentclass[12pt]{article}

\usepackage{lineno,changepage,lipsum}
\usepackage[colorlinks=true,urlcolor=blue]{hyperref}
\usepackage{fontspec}
\usepackage{xeCJK}
\usepackage{tabularx}
\setCJKfamilyfont{chanto}{AozoraMinchoRegular.ttf}
\setCJKfamilyfont{tegaki}{Mushin.otf}
\usepackage[CJK,overlap]{ruby}
\usepackage{hhline}
\usepackage{multirow,array,amssymb}
\usepackage[croatian]{babel}
\usepackage{soul}
\usepackage[usenames, dvipsnames]{color}
\usepackage{wrapfig,booktabs}
\renewcommand{\rubysep}{0.1ex}
\renewcommand{\rubysize}{0.75}
\usepackage[margin=50pt]{geometry}
\modulolinenumbers[2]

\usepackage{pifont}
\newcommand{\cmark}{\ding{51}}%
\newcommand{\xmark}{\ding{55}}%

\definecolor{faded}{RGB}{100, 100, 100}

\renewcommand{\arraystretch}{1.2}

%\ruby{}{}
%$($\href{URL}{text}$)$

\newcommand{\furigana}[2]{\ruby{#1}{#2}}
\newcommand{\tegaki}[1]{
	\CJKfamily{tegaki}\CJKnospace
	#1
	\CJKfamily{chanto}\CJKnospace
}

\newcommand{\dai}[1]{
	\vspace{20pt}
	\large
	\noindent\textbf{#1}
	\normalsize
	\vspace{20pt}
}

\newcommand{\fukudai}[1]{
	\vspace{10pt}
	\noindent\textbf{#1}
	\vspace{10pt}
}

\newenvironment{bunshou}{
	\vspace{10pt}
	\begin{adjustwidth}{1cm}{3cm}
	\begin{linenumbers}
}{
	\end{linenumbers}
	\end{adjustwidth}
}

\newenvironment{reibun}{
	\vspace{10pt}
	\begin{tabular}{l l}
}{
	\end{tabular}
	\vspace{10pt}
}
\newcommand{\rei}[2]{
	#1&\textit{#2}\\
}
\newcommand{\reinagai}[2]{
	\multicolumn{2}{l}{#1}\\
	\multicolumn{2}{l}{\hspace{10pt}\textit{#2}}\\
}

\newenvironment{mondai}[1]{
	\vspace{10pt}
	#1
	
	\begin{enumerate}
		\itemsep-5pt
	}{
	\end{enumerate}
	\vspace{10pt}
}

\newenvironment{hyou}{
	\begin{itemize}
		\itemsep-5pt
	}{
	\end{itemize}
	\vspace{10pt}
}

\date{\today}

\CJKfamily{chanto}\CJKnospace
\author{Tomislav Mamić}
\begin{document}
	\dai{Upotrebe い oblika I}
	
	Prisjetimo se - い oblik koristimo uglavnom za spajanje glavnog i pomoćnih glagola ili pridjeva koji dopunjuju osnovno značenje glagola. U nastavku ćemo vidjeti nekoliko čestih primjera upotrebe i ugrubo ih objasniti.
	
	\fukudai{Imperativ s なさい}
	
	Slično kao što na て oblik možemo dodati ください da bismo napravili molbu koja je zapravo blagi imperativ, na い oblik možemo dodati なさい što ga pretvara u \textit{strogu} (ali nikako \textit{grubu}) naredbu. Ovako će se izražavati iskusni prema manje iskusnima, na primjer roditelji prema djeci. Zadaci na ispitima također često mogu biti napisani ovim imperativom.
	
	\begin{reibun}
		\rei{にんじんを \furigana{食}{た}べなさい。}{Jedi mrkve.}
		\rei{\furigana{外}{そと}に 出なさい。}{Izađi van.}
		\rei{\furigana{次}{つぎ}の\furigana{文}{ぶん}を クロアチア\furigana{語}{ご}に \furigana{訳}{やく}しなさい。}{Prevedi sljedeću rečenicu na hrvatski.}
	\end{reibun}

	Valja zapamtiti kako se korištenjem ovog imperativa stavljamo hijerarhijski iznad onog kome je upućen. Tako je recimo u redu obratiti se profesoru koristeći て + ください, ali nije pristojno učiniti isto s い + なさい.
	
	\fukudai{Izražavanje želje s たい}
	
	Dodamo li na い oblik glagola pom. pridjev たい, dobivamo značenje \textit{želim raditi} <glagol>. S ovim oblikom postoje dvije stvari na koje valja obratiti pažnju:
	\begin{hyou}
		\item Čestica koja označava objekt glavnog glagola može biti i を i が bez razlike u značenju. Povijesno, ispravno je u ovakvim rečenicama objekt označiti sa が kao kod pridjeva すき i きらい koje smo ranije naučili, ali u zadnje se vrijeme često čuje i (po starom govoru krivo) を.
		\item Ovaj oblik je subjektivan - izražava našu želju i nećemo ga nikad koristiti kad govorimo o željama drugih.
	\end{hyou}

	\begin{reibun}
		\rei{アイスクリームが食べたい。}{Želim jesti sladoled. \cmark}
		\rei{アイスクリームを食べたい。}{Želim jesti sladoled. \cmark}
		\rei{たけしくんはアイスクリームが食べたい。}{Takeši želi jesti sladoled. \xmark}
	\end{reibun}

	Kad je predikat u ovom obliku, uvijek možemo pretpostaviti subjekt わたし.
	
	\fukudai{Lagano i teško s やすい i にくい}
	
	Dodavanjem navedenih pridjeva glagolu možemo izraziti da je neka radnja lagana ili teška.
	
	\begin{reibun}
		\rei{このかいだんは のぼりにくい。}{Po ovim stepenicama se teško penjati.}
		\rei{すしは食べやすい。}{Sushi je lako jesti.}
		\rei{たけしくんの もじは よみにくい。}{Takešijev rukopis je teško čitati.}
		\rei{たけしくんの もじは よみやすくない。}{Takešijev rukopis nije lagano čitati.}
		\rei{先生は よみにくい もじが きらいです。}{Učitelj ne voli rukopis koji je teško čitati.}
	\end{reibun}

	U primjerima iznad dobro je obratiti pažnju na to da spajanjem glagola s pom. pridjevima dobivamo \textbf{pridjev}. U zadnjem primjeru vidimo da tako nastali pridjev možemo koristiti i u opisnom obliku (よみにくいもじ) što se u hrvatskom raspakira u cijelu opisnu rečenicu.
	
	\fukudai{Doći i ići raditi s いく i くる}
	
	U hrvatskom jeziku svakodnevno koristimo izraze oblika \textit{idem jesti} ili \textit{došao sam pogledati}. U japanskom istu ideju možemo izraziti tako da na い oblik glagola dodamo česticu に i to prikvačimo na glagol 行く ili 来る. Bitno je ne zaboraviti da ove glagole ne lijepimo direktno na glavni, već im dodajemo informaciju (\textit{što idem?}) česticom に.
	
	\begin{reibun}
		\rei{えいがを 見に行く。}{Idem gledati film.}
		\rei{あした、花子さんが あそびに来る。}{Sutra će Hanako doći u posjetu\footnotemark[1].}
		\rei{たけしくんは ぎゅうにゅうを かいに行った。}{Takeši je otišao kupiti mlijeko.}
		\rei{WiFiを なおしに来た。}{Došao sam popraviti WiFi.}
	\end{reibun}

	\footnotetext[1]{あそぶ ne znači samo igrati se već zabavljati se općenito i zato se koristi i za neformalno druženje. Rečenica ne implicira nužno da je Hanako dijete i da će se doći igrati.}
	
	\fukudai{Početak i kraj radnje s はじめる i おわる}
	
	Dodavanjem spomenutih glagola dobivamo značenje kakvo u hrv. imamo u izrazima \textit{počeo sam gledati} ili \textit{završio sam s jelom}.
	
	\begin{reibun}
		\rei{日本語を べんきょうしはじめた。}{Počeo sam učiti japanski.}
		\rei{すずきさんは ほそいみちを あるきはじめました。}{Suzuki je krenuo hodati uskim putem.}
		\rei{さけを のみおわって、かえった。}{Dovršio sam (piti) sake i vratio se kući.}
	\end{reibun}

	\fukudai{Vježba}
	
	\begin{mondai}{Rečenice u nastavku prevedite na hrvatski.}
		\item そのとき、たけしくんは かえりたいと おもいはじめた。
		\item 日本語の どうしは わかりにくい ですか?\\(どうし - \textit{glagol})
		\item まえの本を よみおわって、こんどは よみやすい本を かりに いきます。\\(こんど - \textit{ovaj put} ili u kontekstu \textit{drugi put})
		\item 先生が たけしくんに 「あとで しょくいんしつに きなさい」と いいました。\\(しょくいんしつ - zbornica)
		\item まいしゅう ここに すしを たべに くるよ。\\(まい\textasciitilde~\textit{svaki} \textasciitilde~za vrijeme, npr. にち, しゅう, つき, とし, あさ, ばん...)
		\item すずきさんは きょねん ひっこして、あそびに こなく なった。\\(ひっこす - \textit{preseliti se})
		\item *日本から もどった ともだちの はなしを きいて わたしも いきたく なりました。
	\end{mondai}
\end{document}