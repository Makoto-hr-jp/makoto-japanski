% !TeX document-id = {9c857013-c43a-4cad-9d1a-c416c3979d37}
% !TeX program = xelatex ?me -synctex=0 -interaction=nonstopmode -aux-directory=../tex_aux -output-directory=./release
% !TeX program = xelatex

\documentclass[12pt]{article}

\usepackage{lineno,changepage,lipsum}
\usepackage[colorlinks=true,urlcolor=blue]{hyperref}
\usepackage{fontspec}
\usepackage{xeCJK}
\usepackage{tabularx}
\setCJKfamilyfont{chanto}{AozoraMinchoRegular.ttf}
\setCJKfamilyfont{tegaki}{Mushin.otf}
\usepackage[CJK,overlap]{ruby}
\usepackage{hhline}
\usepackage{multirow,array,amssymb}
\usepackage[croatian]{babel}
\usepackage{soul}
\usepackage[usenames, dvipsnames]{color}
\usepackage{wrapfig,booktabs}
\renewcommand{\rubysep}{0.1ex}
\renewcommand{\rubysize}{0.75}
\usepackage[margin=50pt]{geometry}
\modulolinenumbers[2]

\usepackage{pifont}
\newcommand{\cmark}{\ding{51}}%
\newcommand{\xmark}{\ding{55}}%

\definecolor{faded}{RGB}{100, 100, 100}

\renewcommand{\arraystretch}{1.2}

%\ruby{}{}
%$($\href{URL}{text}$)$

\newcommand{\furigana}[2]{\ruby{#1}{#2}}
\newcommand{\tegaki}[1]{
	\CJKfamily{tegaki}\CJKnospace
	#1
	\CJKfamily{chanto}\CJKnospace
}

\newcommand{\dai}[1]{
	\vspace{20pt}
	\large
	\noindent\textbf{#1}
	\normalsize
	\vspace{20pt}
}

\newcommand{\fukudai}[1]{
	\vspace{10pt}
	\noindent\textbf{#1}
	\vspace{10pt}
}

\newenvironment{bunshou}{
	\vspace{10pt}
	\begin{adjustwidth}{1cm}{3cm}
	\begin{linenumbers}
}{
	\end{linenumbers}
	\end{adjustwidth}
}

\newenvironment{reibun}{
	\vspace{10pt}
	\begin{tabular}{l l}
}{
	\end{tabular}
	\vspace{10pt}
}
\newcommand{\rei}[2]{
	#1&\textit{#2}\\
}
\newcommand{\reinagai}[2]{
	\multicolumn{2}{l}{#1}\\
	\multicolumn{2}{l}{\hspace{10pt}\textit{#2}}\\
}

\newenvironment{mondai}[1]{
	\vspace{10pt}
	#1
	
	\begin{enumerate}
		\itemsep-5pt
	}{
	\end{enumerate}
	\vspace{10pt}
}

\newenvironment{hyou}{
	\begin{itemize}
		\itemsep-5pt
	}{
	\end{itemize}
	\vspace{10pt}
}

\date{\today}

\CJKfamily{chanto}\CJKnospace
\author{Tomislav Mamić}
\begin{document}
	\dai{Domaća zadaća - upotrebe い oblika}
	
	\noindent Prevedite sljedeće rečenice. Pokušajte ih rastaviti na dijelove i napisati / skicirati njihov međusobni odnos. Zadaci su varijacije na oblik rečenica iz vježbe za pripadni listić! (5. zad. mi se učinio lagan pa sam ga preskočio)
	
	\begin{mondai}{Zad. 1}
		\item すしが たべたい。
		\item 花子さんは すしが たべたいと いった。
		\item そのとき、花子さんは すしが たべたいと いった。
		\item そのとき、花子さんは すしが たべたいと おもった。
		\item そのとき、花子さんは すしが たべたいと おもいはじめた。
	\end{mondai}

	\begin{mondai}{Zad. 2}
		\item たけしくんは すうがくが わからない。
		\item すうがくは わかりにくい。
		\item たけしくんは すうがくが わかりにくいと 先生に いいました。
	\end{mondai}

	\begin{mondai}{Zad. 3}
		\item まえの 本を よんだ。
		\item まえの 本を よみおわった。
		\item こんどは よみやすい本を かります。
		\item こんどは よみやすい本を かりに いきます。
		\item まえの 本を よみおわって、こんどは よみやすい本を かりに いきます。
	\end{mondai}

	\begin{mondai}{Zad. 4}
		\item あとで しょくいんしつに きなさい。
		\item 先生は 「あとで しょくいんしつ に きなさい」と たけしくんに いいました。
		\item 先生は きょうしつを でました。
		\item けさ、先生は「あとで しょくいんしつに きなさい」と たけしくんに いって きょうしつを でた。
	\end{mondai}

	\begin{mondai}{Zad. 6}
		\item すずきさんが あそびに くる。
		\item すずきさんが まいにち あそびに きています。
		\item きょねん、すずきさんは ひっこした。
		\item きょねんまで すずきさんが まいにち あそびに きていました。
		\item ひっこして、 もう あそびに こない。
		\item すずきさんは あそびに こなく なった。
	\end{mondai}

	\begin{mondai}{Zad. 7}
		\item ともだちの はなしを きいた。
		\item わたしも いきたい。
		\item ともだちの はなしを きいて、 わたしも いきたいと おもいました。
		\item ともだちが 日本から もどった。
		\item 日本から もどった ともだちが いる。
		\item 日本から もどった ともだちの はなしを きいて、 わたしも いきたいと おもいはじめた。
	\end{mondai}
\end{document}