% !TeX document-id = {814c7464-6208-43ac-ab6a-20a93446ba13}
% !TeX program = xelatex ?me -synctex=0 -interaction=nonstopmode -aux-directory=../tex_aux -output-directory=./release
% !TeX program = xelatex

\documentclass[12pt]{article}

\usepackage{lineno,changepage,lipsum}
\usepackage[colorlinks=true,urlcolor=blue]{hyperref}
\usepackage{fontspec}
\usepackage{xeCJK}
\usepackage{tabularx}
\setCJKfamilyfont{chanto}{AozoraMinchoRegular.ttf}
\setCJKfamilyfont{tegaki}{Mushin.otf}
\usepackage[CJK,overlap]{ruby}
\usepackage{hhline}
\usepackage{multirow,array,amssymb}
\usepackage[croatian]{babel}
\usepackage{soul}
\usepackage[usenames, dvipsnames]{color}
\usepackage{wrapfig,booktabs}
\renewcommand{\rubysep}{0.1ex}
\renewcommand{\rubysize}{0.75}
\usepackage[margin=50pt]{geometry}
\modulolinenumbers[2]

\usepackage{pifont}
\newcommand{\cmark}{\ding{51}}%
\newcommand{\xmark}{\ding{55}}%

\definecolor{faded}{RGB}{100, 100, 100}

\renewcommand{\arraystretch}{1.2}

%\ruby{}{}
%$($\href{URL}{text}$)$

\newcommand{\furigana}[2]{\ruby{#1}{#2}}
\newcommand{\tegaki}[1]{
	\CJKfamily{tegaki}\CJKnospace
	#1
	\CJKfamily{chanto}\CJKnospace
}

\newcommand{\dai}[1]{
	\vspace{20pt}
	\large
	\noindent\textbf{#1}
	\normalsize
	\vspace{20pt}
}

\newcommand{\fukudai}[1]{
	\vspace{10pt}
	\noindent\textbf{#1}
	\vspace{10pt}
}

\newenvironment{bunshou}{
	\vspace{10pt}
	\begin{adjustwidth}{1cm}{3cm}
	\begin{linenumbers}
}{
	\end{linenumbers}
	\end{adjustwidth}
}

\newenvironment{reibun}{
	\vspace{10pt}
	\begin{tabular}{l l}
}{
	\end{tabular}
	\vspace{10pt}
}
\newcommand{\rei}[2]{
	#1&\textit{#2}\\
}
\newcommand{\reinagai}[2]{
	\multicolumn{2}{l}{#1}\\
	\multicolumn{2}{l}{\hspace{10pt}\textit{#2}}\\
}

\newenvironment{mondai}[1]{
	\vspace{10pt}
	#1
	
	\begin{enumerate}
		\itemsep-5pt
	}{
	\end{enumerate}
	\vspace{10pt}
}

\newenvironment{hyou}{
	\begin{itemize}
		\itemsep-5pt
	}{
	\end{itemize}
	\vspace{10pt}
}

\date{\today}

\CJKfamily{chanto}\CJKnospace
\author{Tomislav Mamić}
\begin{document}
	\dai{Priložne oznake vremena}
	
	\fukudai{Priložne imenice}
	
	U japanskom postoji klasa imenica koje se u određenim situacijama ponašaju kao prilozi. Takva upotreba ima dalekosežne posljedice za način na koji se u japanskom slažu kompliciranije rečenice. Neke od ovih priložnih imenica se koriste gotovo isključivo za formiranje priložnih oznaka i same za sebe gotovo da i ne znače ništa (iako su imenice!).
	
	Najjednostavnija podskupina priložnih imenica su vremenske. Sve vremenske imenice imaju samostalno značenje i relativno jednostavnu upotrebu.
	
	\fukudai{Vremenske imenice}
	
	\begin{tabular}{l l l l l l l l}
		ゆうべ & \textit{sinoć} & けさ & \textit{jutros} & こんばん & \textit{večeras} & こんや & \textit{noćas}\\
 		いま & \textit{sada} & あさ & \textit{jutro} & ひる & \textit{podne} & よる/ばん & \textit{noć}/\textit{večer}\\
	\end{tabular}

	\vspace{10pt}
	Kad ih koristimo kao priložnu oznaku vremena - kad želimo da u rečenici odgovaraju na pitanje \textit{kada} - ove imenice ćemo uglavnom staviti na početak rečenice i uz njih \textbf{ne moramo staviti česticu} (ali možemo ako nam zatreba):
	
	\begin{reibun}
		\rei{ゆうべ そとで ねた。}{Sinoć sam spavao vani.}
		\rei{けさ りんごを たべた。}{Jutros sam pojeo jabuku.}
	\end{reibun}

	Imenice za dane, tjedne, mjesece i godine uredno su posložene:
	
	\begin{table}[h]
		\centering
		\begin{tabular}{l r r r r}\toprule[2pt]
			& dan & tjedan & mjesec & godina \\
			\midrule
			pretprošli 		& おととい & せんせんしゅう & せんせんげつ & おととし \\
			prošli 			& きのう & せんしゅう & せんげつ & きょねん \\
			trenutni 		& きょう & こんしゅう & こんげつ & ことし \\
			sljedeći 		& あした & らいしゅう & らいげつ & らいねん \\
			preksljedeći 	& あさって & さらいしゅう & さらいげつ & さらいねん \\
			\bottomrule[2pt]
		\end{tabular}
	\end{table}
	
	Osim što ih možemo koristiti kao vremenske priloge u hrvatskom, možemo ih koristiti i kao obične imenice u japanskom:
	
	\begin{reibun}
		\rei{\furigana{昨日}{きのう}の ばんごはんは おいしかった。}{Jučerašnja večera je bila fina.}
		\reinagai{ゆうべを おもいだした。}{Prisjetio sam se prošle večeri. \rem{jer u hr. nije ispravno reći}Prisjetio sam se \rem{(čega)}sinoć.}
	\end{reibun}

	\newpage
	\fukudai{Čestice から i まで s vremenom}
	
	Ranije smo vidjeli da čestice から i まで mogu označavati prostorni raspon (od \textasciitilde\ do). Na potpuno isti način možemo ih koristiti i za označavanje vremenskog raspona:
	
	\begin{reibun}
		\rei{昨日から あさってまで}{od jučer do preksutra}
		\rei{\furigana{今日}{きょう}から}{od danas}
		\rei{\furigana{来年}{らいねん}まで}{do sljedeće godine}
	\end{reibun}

	\fukudai{Vježba - Prevedite sljedeće rečenice na japanski}
	
	\begin{mondai}{Lv. 1}
		\item \textit{Sutra ću ići u školu.}
		\item \textit{Jučer sam vidio prijatelja.}
		\item \textit{Noćas idem u Japan.}
		\item \textit{Jutros su se djeca igrala.}
	\end{mondai}

	\begin{mondai}{Lv. 2}
		\item \textit{Sutra ću s Takešijem ići u školu.}
		\item \textit{Jučer sam u gradu vidio prijatelja.}
		\item \textit{Noćas avionom idem u Japan.}
		\item \textit{Jutros su se djeca igrala u parku.}
	\end{mondai}

	\begin{mondai}{Lv. 3}
		\item \textit{Sutra ću s Takešijem ići u novu školu.}
		\item \textit{Jučer sam u kafiću pokraj stanice vidio prijatelja.}
		\item \textit{Sljedeće godine ću avionom ići u Japan.}
		\item \textit{Jutros su se djeca loptom igrala u parku.}
	\end{mondai}

	\begin{mondai}{Lv. 4*}
		\item \textit{Škola u koju ću sutra ići s Takešijem je daleko.}
		\item \textit{Prijatelj kojeg sam vidio jučer u kafiću pokraj stanice mi je poslao pismo.}
		\item \textit{Razgovarao sam s prijateljem koji će sljedeće godine avionom ići u Japan.}
		\item \textit{Djeca koja su se jutros loptom igrala u parku sad su u školi.}
	\end{mondai}
\end{document}