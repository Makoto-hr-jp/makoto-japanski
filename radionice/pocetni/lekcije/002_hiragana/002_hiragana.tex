% !TeX document-id = {87c116f6-681a-4b3d-9fc9-28ee9ca2f882}
% !TeX program = xelatex ?me -synctex=0 -interaction=nonstopmode -aux-directory=../../tex_aux -output-directory=./release
% !TeX program = xelatex

\documentclass[12pt]{article}

\usepackage{lineno,changepage,lipsum}
\usepackage[colorlinks=true,urlcolor=blue]{hyperref}
\usepackage{fontspec}
\usepackage{xeCJK}
\usepackage{tabularx}
\setCJKfamilyfont{chanto}{AozoraMinchoRegular.ttf}
\setCJKfamilyfont{tegaki}{Mushin.otf}
\usepackage[CJK,overlap]{ruby}
\usepackage{hhline}
\usepackage{multirow,array,amssymb}
\usepackage[croatian]{babel}
\usepackage{soul}
\usepackage[usenames, dvipsnames]{color}
\usepackage{wrapfig,booktabs}
\renewcommand{\rubysep}{0.1ex}
\renewcommand{\rubysize}{0.75}
\usepackage[margin=50pt]{geometry}
\modulolinenumbers[2]

\usepackage{pifont}
\newcommand{\cmark}{\ding{51}}%
\newcommand{\xmark}{\ding{55}}%

\definecolor{faded}{RGB}{100, 100, 100}

\renewcommand{\arraystretch}{1.2}

%\ruby{}{}
%$($\href{URL}{text}$)$

\newcommand{\furigana}[2]{\ruby{#1}{#2}}
\newcommand{\tegaki}[1]{
	\CJKfamily{tegaki}\CJKnospace
	#1
	\CJKfamily{chanto}\CJKnospace
}

\newcommand{\dai}[1]{
	\vspace{20pt}
	\large
	\noindent\textbf{#1}
	\normalsize
	\vspace{20pt}
}

\newcommand{\fukudai}[1]{
	\vspace{10pt}
	\noindent\textbf{#1}
	\vspace{10pt}
}

\newenvironment{bunshou}{
	\vspace{10pt}
	\begin{adjustwidth}{1cm}{3cm}
	\begin{linenumbers}
}{
	\end{linenumbers}
	\end{adjustwidth}
}

\newenvironment{reibun}{
	\vspace{10pt}
	\begin{tabular}{l l}
}{
	\end{tabular}
	\vspace{10pt}
}
\newcommand{\rei}[2]{
	#1&\textit{#2}\\
}
\newcommand{\reinagai}[2]{
	\multicolumn{2}{l}{#1}\\
	\multicolumn{2}{l}{\hspace{10pt}\textit{#2}}\\
}

\newenvironment{mondai}[1]{
	\vspace{10pt}
	#1
	
	\begin{enumerate}
		\itemsep-5pt
	}{
	\end{enumerate}
	\vspace{10pt}
}

\newenvironment{hyou}{
	\begin{itemize}
		\itemsep-5pt
	}{
	\end{itemize}
	\vspace{10pt}
}

\date{\today}

\CJKfamily{chanto}\CJKnospace
\author{Tomislav Mamić}
\begin{document}
	\dai{Glasovni sustav japanskog - hiragana}
	
	\fukudai{Organizacija glasova}
	
	U japanskom jeziku glasovi dolaze u \textit{m\={o}rama} (jap. はく - \textit{otkucaj}). M\={o}ra je izgovorna jedinica slična slogu, no za razliku od slogova, m\={o}rama se duljina nikad ne mijenja. Iako su m\={o}re uvijek jednake duljine, ovisno o riječi i situaciji ćemo im mijenjati visinu tona (npr. kraj upitne rečenice vs. izjavne).
	
	U japanskom jeziku m\={o}re dolaze u skupu jednog suglasnika i jednog samoglasnika ili samo jednog samoglasnika. More mogu nositi naglaske baš kao što u hrvatskom jeziku jedan slog sadrži jedan naglasak. Zbog same prirode jezika i zbog gore navedenih razloga, u japanskom jeziku je nemoguće napisti riječi koje sadrže više uzastopnih suglasnika u svom sastavu kao na primjer riječ "stol". U riječi "stol" je jedino moguće napisati "to" jer je jedini dio riječi koji sadrži suglasnik i samoglasnik jedan za drugim.
	
	Postoji samo jedna "iznimka" u japanskom jeziku, a to je hiragana ん (\textit{n}, \textit{ŋ}, \textit{m}) koji nije niti samoglasnik niti suglasnik zbog svoje prirode izgovora, i koji nikad ne može započeti riječ.
	O detaljima kasnije.
	
	\footnotetext[1]{od lat. nasus - \textit{nos}}
	
	\fukudai{Zapis glasova}
	
	Japanski jezik ne zapisuje pojedinačne glasove nego m\={o}re. U tablici ispod nalazi se 46 osnovnih znakova pisma hiragana. M\={o}re u istom stupcu počinju istim suglasnikom, one u istom redu završavaju istim samoglasnikom. Jedina iznimka ovome je ranije spomenuti ん.
	
	\setlength{\tabcolsep}{10pt}
	\vspace{10pt}
	\begin{tabular}{|r|c|c|c|c|c|c|c|c|c|c|c|}
		\hline
		&\textasciitilde&k&s&t&n&h&m&y&r&w&\textasciitilde\\
		\hline
		a&あ&か&さ&た&な&は&ま&や&ら&わ&ん\\
		i&い&き&$^a$し&$^b$ち&に&ひ&み&&り&&\\
		u&う&く&す&$^c$つ&ぬ&$^d$ふ&む&ゆ&る&&\\
		e&え&け&せ&て&ね&へ&め&&れ&&\\
		o&お&こ&そ&と&の&ほ&も&よ&ろ&を&\\
		\hline
	\end{tabular}
	
	\fukudai{Izgovor glasova}
	
	\noindent\ten あ・い・う・え・お
	
	Samoglasnici se, izuzev \textit{u}, izgovaraju gotovo kao i u hrvatskom. Glas \textit{u} je u japanskom znatno plići (jezik je blizu nepcu) i "spljošten" (usne su u prirodnom položaju, ne zaokružene) u odnosu na hrvatski. Glasovi \textit{e} i \textit{i} su također nešto plići (japanski glasovi su općenito takvi, ali razlike su najuočljivije na \textit{u} i \textit{e}).
	
	\vspace{5pt}
	\noindent\ten か・き・く・け・こ
	
	Suglasnik \textit{k} je identičan hrvatskom, zajedno sa svojim zvučnim parom \textit{g}. Zvučnost se označava dijakritičkim znakom \textit{tenten} (dosl. točka-točka ili zarez-zarez, službeno 濁点・だくてん) kao u primjerima u nastavku:
	
	\begin{tabular}{l l l l}
		か&\textit{ka}&が&\textit{ga}\\
		く&\textit{ku}&ぐ&\textit{gu}\\
		こ&\textit{ko}&ご&\textit{go}\\
	\end{tabular}

	\vspace{5pt}
	\noindent\ten さ・し・す・せ・そ
	
	Iako je glas \textit{s} isti kao u hrvatskom, zbog sitne razlike u izgovoru glasa \textit{i}, u m\={o}ri \textit{si} dolazi do palatalizacije. Nastali glas se u eng. zapisu zapisuje kao \textit{shi}, a izgovara se kao mekša verzija hrvatskog š. Ova promjena utječe na izgovor m\={o}re $^a$し.
	
	Kao i u hrvatskom, zvučni par \textit{s} je glas \textit{z} koji ispred \textit{i} zbog palatalizacije prelazi u \textit{đ}. Ovo utječe na izgovor じ. U eng. zapisu, budući da nemaju slovo \textit{đ}, pišu \textit{ji}. U japanskom glas \textit{z} ima blagu primjesu glasa \textit{d}.
	
	\vspace{5pt}
	\noindent\ten た・ち・つ・て・と
	
	Glas \textit{t} ima svoj zvučni par \textit{d} kao i u hrvatskom. Oba se glasa ispred \textit{i} palataliziraju u \textit{ć} i \textit{đ}, što utječe na izgovor znakova $^b$ち i ぢ. Valja uočiti kako je izgovor znakova じ i ぢ identičan. Izuzev rijetkih slučajeva glasovnih promjena u kojima drugi od dva uzastopna ち prelazi u ぢ, \textbf{uvijek koristimo じ}.
	
	Zbog velike razlike u izgovoru glasa \textit{u}, \textit{t} se ispred \textit{u} izgovara kao hrvatski glas \textit{c}. Ovo utječe na izgovor znaka $^c$つ. Zvučna verzija prelazi iz \textit{c} u \textit{dz} koji se zbog primjese \textit{d} u jap. glasu \textit{z} ne razlikuje od \textit{z}. Zbog toga je izgovor glasova ず i づ identičan. Kao i u slučaju じ i ぢ, izuzev iznad opisanih glasovnih promjena \textbf{uvijek pišemo ず}.
	
	\vspace{5pt}
	\noindent\ten な・に・ぬ・ね・の、 ま・み・む・め・も
	
	Glasovi \textit{n} i \textit{m} ne razlikuju se od hrvatskih.
	
	\vspace{5pt}
	\noindent\ten ら・り・る・れ・ろ
	
	Glas \textit{r} znatno se razlikuje od hrvatskog. U hrvatskom \textit{r}, vrh jezika je usmjeren prema gore što mu omogućuje da "vibrira". Ovo omogućuje korištenje hrvatskog \textit{r} kao jezgre sloga (npr. \textit{prst}). U japnskom \textit{r}, vrh jezika je gotovo paralelan s nepcem što ga čini sličnim glasovima \textit{t}, \textit{d} i \textit{l}. S obzirom da u japanskom glas \textit{l} ne postoji, mnogi japanci ne razlikuju \textit{r} i \textit{l}.
	
	\vspace{5pt}
	\noindent\ten は・ひ・ふ・へ・ほ
	
	Ispred \textit{u}, dolazi do pomaka glasa \textit{h} prema naprijed što ga po zvuku čini sličnim glasu \textit{f}. U eng. zapisu će zbog toga znak ふ biti pisan kao \textit{fu}, međutim hrvatski (a i engleski) glas \textit{f} \textbf{ne odgovara} japanskom izgovoru. U hrvatskom je glas \textit{f} zubnousneni (gornji zubi dodiruju gornju usnu), ali u japanskom ostaje potpuno otvoren pa je po zvuku negdje \textbf{između hrvatskih \textit{f} i \textit{h}}.
	
	U japanskom se glasovi \textit{h}, \textit{b} i \textit{p} po zvučnosti grupiraju u jedan skup, gdje se \textit{h} smatra bezvučnom, \textit{p} poluzvučnom\footnotemark[2], a \textit{b} zvučnom varijantom. Zvučnost se označava kao i kod ostalih znakova, a za poluzvučnost (glas \textit{p}) koristi se oznaka \textit{maru} (dosl. \textit{kružić}, službeno 半濁点・はんだくてん).
	
	\footnotetext[3]{Budući da u hrvatskom nemamo \textit{poluzvučne} glasove, ovo je doslovni prijevod japanskog 半濁音(はんだくおん).}
	
	\vspace{5pt}
	\noindent\ten や・ゆ・よ
	
	Glas \textit{j} u ovim znakovima identičan je hrvatskom. Treba paziti pri čitanju latiničnog zapisa jer se u eng. notaciji zvuk hrvatskog \textit{j} zapisuje znakom \textit{y}, a \textit{j} iz eng. notacije je hrvatski zvuk \textit{đ}.
	
	\vspace{5pt}
	\noindent\ten わ・を
	
	Ovi zvukovi predstavljaju dvoglase \textit{wa} i \textit{wo}. U modernom japanskom, dvoglas \textit{wo} se više ne koristi (izgovara se kao \textit{o}), ali je znak を zadržan kao gramatička oznaka. U hrvatskom se suglasnik \textit{w} ne pojavljuje, ali izgovor je između glasova \textit{u} i \textit{o}.
	
	\fukudai{Dvoglasi iz \textit{i}}
	
	Osim kao samostalni znakovi, や, ゆ i よ se mogu pojaviti kao "mali znakovi" iza drugih znakova hiragane koji završavaju samoglasnikom \textit{i}. U tom slučaju dolazi do dvoglasa\footnotemark[3]\footnotetext[2]{Dva uzastopna samoglasnika izgovorena tako da se prvi "prelije" u drugi.} u m\={o}rama u kojima nema glasovnih promjena. U m\={o}rama し, じ i ち umjesto dvoglasa dolazi do zamjene samoglasnika \textit{i} samoglasnikom \textit{a}, \textit{u} ili \textit{o}. Pogledajmo primjere ispod:
	
	\begin{tabular}{l l l | l l l}
		jap.&hr.&eng.&jap.&hr.&eng.\\
		きゃ&\textit{kja}&\textit{kya}&しょ&\textit{\'{s}o}&\textit{sho}\\
		りょ&\textit{rjo}&\textit{ryo}&じゅ&\textit{đu}&\textit{ju}\\
		にゃ&\textit{nja}&\textit{nya}&ちゃ&\textit{ća}&\textit{cha}\\
	\end{tabular}

	\fukudai{Glotalna\footnotemark[4]\footnotetext[4]{lat. \textit{glotis} - \textit{glasnice}} stanka っ}
	
	Mali znak っ (jap. 促音・そくおん) označava pauzu - prekid toka zraka kroz glasnice jezikom. U eng. zapisu piše se kao ponovljeni suglasnik (npr. がっこう$\rightarrow$\textit{gakkou}) pa ga nekad zovu i \textit{double consonant}. Međutim, taj naziv je potpuno besmislen u kontekstu japanskog jezika i treba ga izbjegavati. Pojavljuje se ili kao dio riječi ili kao rezultat glasovne promjene.
	
	\fukudai{Dugi samoglasnici}
	
	U japanskom je česta pojava produljenja samoglasnika. Dugi samoglasnici nastaju kad se iza more koja završava nekim samoglasnikom pojavi još jedna mora koja sadrži samo taj isti samoglasnik, ili iznimno u sljedećim situacijama:
	
	\vspace{5pt}
	\begin{tabular}{l l}
		\textit{ei}$\rightarrow$\textit{ee} & せんせい (\textit{učitelj}) $\rightarrow$ \textit{sens\={e}}\\
		\textit{ou}$\rightarrow$\textit{oo} & おうさま (\textit{kralj}) $\rightarrow$ \textit{\={o}sama}\\
	\end{tabular}

	\vspace{5pt}
	Tako ćemo na primjer riječ きょう (\textit{danas}) pročitati kao \textit{kjo.o}, koristeći točku da odvojimo m\={o}re.
	
	\fukudai{Vježba}
	
	\begin{tabular}{p{200pt} p{200pt}}
		\begin{mondai}{Pročitajte sljedeće riječi:}
			\item ねこ \textit{mačka}
			\item いぬ \textit{pas}
			\item しゅくだい \textit{domaća zadaća}
			\item せんぱい (nema dosl. prijevoda)
			\item がっこう \textit{škola}
			\item とうきょう \textit{Tokio}
			\item きょうと \textit{Kyoto}
		\end{mondai}
		&
		\begin{mondai}{Zapišite sljedeće riječi hiraganom:}
			\item \textit{to.ri} - \textit{ptica}
			\item \textit{mo.ri} - \textit{šuma}
			\item \textit{shu.mi} - \textit{hobi}
			\item \textit{shi.n.pa.i} - \textit{briga/zabrinutost}
			\item \textit{a.sa.t.te} - \textit{preksutra}
			\item \textit{O.o.sa.ka}
			\item \textit{To.u.ka.ma.chi}
		\end{mondai}\\
	\end{tabular}
	
	Bonus bodovi: とくしゅそうたいせいりろん
	
\end{document}