% !TeX document-id = {a362ff7d-0fb6-4f4a-b3ca-773bacc27ad4}
% !TeX program = xelatex ?me -synctex=0 -interaction=nonstopmode -aux-directory=../tex_aux -output-directory=./release
% !TeX program = xelatex

\documentclass[12pt]{article}

\usepackage{lineno,changepage,lipsum}
\usepackage[colorlinks=true,urlcolor=blue]{hyperref}
\usepackage{fontspec}
\usepackage{xeCJK}
\usepackage{tabularx}
\setCJKfamilyfont{chanto}{AozoraMinchoRegular.ttf}
\setCJKfamilyfont{tegaki}{Mushin.otf}
\usepackage[CJK,overlap]{ruby}
\usepackage{hhline}
\usepackage{multirow,array,amssymb}
\usepackage[croatian]{babel}
\usepackage{soul}
\usepackage[usenames, dvipsnames]{color}
\usepackage{wrapfig,booktabs}
\renewcommand{\rubysep}{0.1ex}
\renewcommand{\rubysize}{0.75}
\usepackage[margin=50pt]{geometry}
\modulolinenumbers[2]

\usepackage{pifont}
\newcommand{\cmark}{\ding{51}}%
\newcommand{\xmark}{\ding{55}}%

\definecolor{faded}{RGB}{100, 100, 100}

\renewcommand{\arraystretch}{1.2}

%\ruby{}{}
%$($\href{URL}{text}$)$

\newcommand{\furigana}[2]{\ruby{#1}{#2}}
\newcommand{\tegaki}[1]{
	\CJKfamily{tegaki}\CJKnospace
	#1
	\CJKfamily{chanto}\CJKnospace
}

\newcommand{\dai}[1]{
	\vspace{20pt}
	\large
	\noindent\textbf{#1}
	\normalsize
	\vspace{20pt}
}

\newcommand{\fukudai}[1]{
	\vspace{10pt}
	\noindent\textbf{#1}
	\vspace{10pt}
}

\newenvironment{bunshou}{
	\vspace{10pt}
	\begin{adjustwidth}{1cm}{3cm}
	\begin{linenumbers}
}{
	\end{linenumbers}
	\end{adjustwidth}
}

\newenvironment{reibun}{
	\vspace{10pt}
	\begin{tabular}{l l}
}{
	\end{tabular}
	\vspace{10pt}
}
\newcommand{\rei}[2]{
	#1&\textit{#2}\\
}
\newcommand{\reinagai}[2]{
	\multicolumn{2}{l}{#1}\\
	\multicolumn{2}{l}{\hspace{10pt}\textit{#2}}\\
}

\newenvironment{mondai}[1]{
	\vspace{10pt}
	#1
	
	\begin{enumerate}
		\itemsep-5pt
	}{
	\end{enumerate}
	\vspace{10pt}
}

\newenvironment{hyou}{
	\begin{itemize}
		\itemsep-5pt
	}{
	\end{itemize}
	\vspace{10pt}
}

\date{\today}

\CJKfamily{chanto}\CJKnospace
\usepackage{xcolor}
\author{Tomislav Mamić}
\begin{document}
	\dai{Količina II}
	
	Prošli put smo naučili kako se čitaju brojevi i promotrili što se događa kad na njih dodajemo brojače kako bismo izrazili količinu. U nastavku ćemo količinu pokušati smjestiti u rečenicu, učeći pritom još nekoliko korisnih brojača.
	
	\fukudai{Korisni brojači}
	
	Ispod se nalazi tablica često korištenih brojača koji se pojavljuju u ovoj lekciji.
	
	\vspace{5pt}
	\begin{table}[h]
		\centering
		\begin{tabular}{r l l l l l l}\toprule[2pt]
			rimski & 分(ふん) & 秒(びょう) & 枚(まい) & 本(ほん) & 台(だい) & 匹(ひき) \\
			\midrule
			1	& \colorbox{blue!10}{いっ.ぷん} & いち.びょう & いち.まい & \colorbox{blue!10}{いっ.ぽん} & いち.だい & \colorbox{blue!10}{いっ.ぴき} \\
			2	& に.ふん & に.びょう & に.まい & に.ほん & に.だい & に.ひき \\
			3	& \colorbox{blue!10}{さん.ぷん} & さん.びょう & さん.まい & \colorbox{blue!10}{さん.ぼん} & さん.だい & \colorbox{blue!10}{さん.びき} \\
			4	& \colorbox{blue!10}{よん.ぷん} & よん.びょう & よん.まい & よん.ほん & よん.だい & よん.ひき \\
			5	& ご.ふん & ご.びょう & ご.まい & ご.ほん & ご.だい & ご.ひき \\
			6	& \colorbox{blue!10}{ろっ.ぷん} & ろく.びょう & ろく.まい & \colorbox{blue!10}{ろっ.ぽん} & ろく.だい & \colorbox{blue!10}{ろっ.ぴき} \\
			7	& なな.ふん & なな.びょう & なな.まい & なな.ほん & なな.だい & なな.ひき \\
			8	& \colorbox{blue!10}{はっ.ぷん} & はち.びょう & はち.まい & \colorbox{blue!10}{はっ.ぽん} & はち.だい & \colorbox{blue!10}{はっ.ぴき} \\
			9	& きゅう.ふん & きゅう.びょう & きゅう.まい & きゅう.ほん & きゅう.だい & きゅう.ひき \\
			10	& \colorbox{blue!10}{じゅっ.ぷん} & じゅう.びょう & じゅう.まい & \colorbox{blue!10}{じゅっ.ぽん} & じゅう.だい & \colorbox{blue!10}{じゅっ.ぴき} \\
			\bottomrule
		\end{tabular}
	\end{table}
	
	\fukudai{Pridruživanje količine imenici česticom の}
	
	Ovo je univerzalno pravilo i može se koristiti u svim situacijama. Ako želimo reći koliko neke imenice ima, to možemo učiniti po receptu:
	
	\juuyou{<količina> の <imenica>}
	
	Prednost ovog oblika je to što s desne strane izgleda kao imenica pa ga možemo smjestiti u rečenicu po svim pravilima koja smo dosad naučili. Pogledajmo neke primjere:
	
	\begin{reibun}
		\rei{6匹の ねこを みた。}{Vidio sam šest mačaka.}
		\rei{はこの なかに 6匹の こねこが いた。}{U kutiji je bilo šest mačića.}
		\rei{3人の ともだちに あった。}{Sreo sam tri prijatelja.}
		\rei{3人の ともだちと はなした。}{Pričao sam s troje prijatelja.}
		\rei{たけしくんは 2本の フォークで にくを さした。}{Takeši je napiknuo meso dvjema vilicama.}
	\end{reibun}

	Uočimo i da ovakav izraz ima dobro definirano značenje izvan konteksta rečenice (6匹のねこ uvijek znamo protumačiti kao \textit{šest mačaka}) što u narednim primjerima neće biti slučaj.
	
	\newpage
	\fukudai{Količina s česticama が i を}
	
	Kad je imenica kojoj želimo dodati podatak o količini u ulozi subjekta ili objekta, smijemo količinu izreći na nešto kraći način:
	
	\juuyou{<imenica> が <količina>}
	
	\juuyou{<imenica> を <količina>}
	
	Prednost ovakvog izraza je nešto kraća rečenica.
	
	\fukudai{Količina s bilo kojom česticom}
	
\end{document}