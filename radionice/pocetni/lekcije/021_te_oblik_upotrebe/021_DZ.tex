% !TeX document-id = {f83e8e1f-7723-4a91-830a-5f147b90220a}
% !TeX program = xelatex ?me -synctex=0 -interaction=nonstopmode -aux-directory=../tex_aux -output-directory=./release
% !TeX program = xelatex

\documentclass[12pt]{article}

\usepackage{lineno,changepage,lipsum}
\usepackage[colorlinks=true,urlcolor=blue]{hyperref}
\usepackage{fontspec}
\usepackage{xeCJK}
\usepackage{tabularx}
\setCJKfamilyfont{chanto}{AozoraMinchoRegular.ttf}
\setCJKfamilyfont{tegaki}{Mushin.otf}
\usepackage[CJK,overlap]{ruby}
\usepackage{hhline}
\usepackage{multirow,array,amssymb}
\usepackage[croatian]{babel}
\usepackage{soul}
\usepackage[usenames, dvipsnames]{color}
\usepackage{wrapfig,booktabs}
\renewcommand{\rubysep}{0.1ex}
\renewcommand{\rubysize}{0.75}
\usepackage[margin=50pt]{geometry}
\modulolinenumbers[2]

\usepackage{pifont}
\newcommand{\cmark}{\ding{51}}%
\newcommand{\xmark}{\ding{55}}%

\definecolor{faded}{RGB}{100, 100, 100}

\renewcommand{\arraystretch}{1.2}

%\ruby{}{}
%$($\href{URL}{text}$)$

\newcommand{\furigana}[2]{\ruby{#1}{#2}}
\newcommand{\tegaki}[1]{
	\CJKfamily{tegaki}\CJKnospace
	#1
	\CJKfamily{chanto}\CJKnospace
}

\newcommand{\dai}[1]{
	\vspace{20pt}
	\large
	\noindent\textbf{#1}
	\normalsize
	\vspace{20pt}
}

\newcommand{\fukudai}[1]{
	\vspace{10pt}
	\noindent\textbf{#1}
	\vspace{10pt}
}

\newenvironment{bunshou}{
	\vspace{10pt}
	\begin{adjustwidth}{1cm}{3cm}
	\begin{linenumbers}
}{
	\end{linenumbers}
	\end{adjustwidth}
}

\newenvironment{reibun}{
	\vspace{10pt}
	\begin{tabular}{l l}
}{
	\end{tabular}
	\vspace{10pt}
}
\newcommand{\rei}[2]{
	#1&\textit{#2}\\
}
\newcommand{\reinagai}[2]{
	\multicolumn{2}{l}{#1}\\
	\multicolumn{2}{l}{\hspace{10pt}\textit{#2}}\\
}

\newenvironment{mondai}[1]{
	\vspace{10pt}
	#1
	
	\begin{enumerate}
		\itemsep-5pt
	}{
	\end{enumerate}
	\vspace{10pt}
}

\newenvironment{hyou}{
	\begin{itemize}
		\itemsep-5pt
	}{
	\end{itemize}
	\vspace{10pt}
}

\date{\today}

\CJKfamily{chanto}\CJKnospace
\author{Tomislav Mamić}
\begin{document}
	\dai{Domaća zadaća - Upotrebe て oblika I}
	
	\begin{mondai}{Lv. 1.a}
		\item すうがくは わかりにくい。
		\item たけしくんと \furigana{話}{はな}してみる。
		\item あめを かってあげた。 (あめ - \textit{slatkiš})
	\end{mondai}

	\begin{mondai}{Lv. 1.b}
		\item たけしくんは すうがくが わかりにくい。
		\item たけしくんに 話してみる。
		\item あめを かってくれた。
	\end{mondai}

	\begin{mondai}{Lv. 1.c}
		\item 花子ちゃんは すうがくを べんきょうしておいた。
		\item たけしくんと 話してみてください。
		\item もう あめを かってあげません。
	\end{mondai}

	\begin{mondai}{Lv. 2}
		\item たけしくんは はなこちゃんに すうがくを おしえてもらった。
		\item たけしくんと 話してみてくださいと おにいさんが \furigana{言}{い}いました。
		\item きんじょの おねえさん\footnotemark[1]に あめを かってもらった。 (きんじょ - \textit{susjedstvo})
	\end{mondai}

	\begin{mondai}{Lv. 3}
		\item *たけしくんは べんきょうしておいた はなこちゃんに すうがくを おしえてもらった。
		\item おにいさんは たけしくんと 話してみてくださいと 言って、わたしは たけしくんと もう\\話したくないと こたえました。 (こたえる - \textit{odgovoriti})
		\item きんじょの おねえさんが いつも あめを かってくれます。
	\end{mondai}

	\begin{mondai}{Lv. 4}
		\item *べんきょうしておいた はなこちゃんは すうがくが わからない たけしくんに すうがくを おしえてあげた。
		\item Lv. 3 je već dovoljno dugačak :)
		\item いつも あめを かってくれる きんじょの おねえさんは らいねん アメリカに\\りゅうがくしにいく。 (りゅうがく - \textit{studij u inozemstvu})
	\end{mondai}

	\footnotetext[1]{おねえさん se koristi i za mlade djevojke koje nam nisu doslovno sestre. Analogno, starije žene možemo zvati おばさん.}
\end{document}