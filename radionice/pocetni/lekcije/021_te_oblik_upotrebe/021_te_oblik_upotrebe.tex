% !TeX document-id = {4c5d9ae9-1fdf-4e5b-bfea-8e54ad650dbb}
% !TeX program = xelatex ?me -synctex=0 -interaction=nonstopmode -aux-directory=../tex_aux -output-directory=./release
% !TeX program = xelatex

\documentclass[12pt]{article}

\usepackage{lineno,changepage,lipsum}
\usepackage[colorlinks=true,urlcolor=blue]{hyperref}
\usepackage{fontspec}
\usepackage{xeCJK}
\usepackage{tabularx}
\setCJKfamilyfont{chanto}{AozoraMinchoRegular.ttf}
\setCJKfamilyfont{tegaki}{Mushin.otf}
\usepackage[CJK,overlap]{ruby}
\usepackage{hhline}
\usepackage{multirow,array,amssymb}
\usepackage[croatian]{babel}
\usepackage{soul}
\usepackage[usenames, dvipsnames]{color}
\usepackage{wrapfig,booktabs}
\renewcommand{\rubysep}{0.1ex}
\renewcommand{\rubysize}{0.75}
\usepackage[margin=50pt]{geometry}
\modulolinenumbers[2]

\usepackage{pifont}
\newcommand{\cmark}{\ding{51}}%
\newcommand{\xmark}{\ding{55}}%

\definecolor{faded}{RGB}{100, 100, 100}

\renewcommand{\arraystretch}{1.2}

%\ruby{}{}
%$($\href{URL}{text}$)$

\newcommand{\furigana}[2]{\ruby{#1}{#2}}
\newcommand{\tegaki}[1]{
	\CJKfamily{tegaki}\CJKnospace
	#1
	\CJKfamily{chanto}\CJKnospace
}

\newcommand{\dai}[1]{
	\vspace{20pt}
	\large
	\noindent\textbf{#1}
	\normalsize
	\vspace{20pt}
}

\newcommand{\fukudai}[1]{
	\vspace{10pt}
	\noindent\textbf{#1}
	\vspace{10pt}
}

\newenvironment{bunshou}{
	\vspace{10pt}
	\begin{adjustwidth}{1cm}{3cm}
	\begin{linenumbers}
}{
	\end{linenumbers}
	\end{adjustwidth}
}

\newenvironment{reibun}{
	\vspace{10pt}
	\begin{tabular}{l l}
}{
	\end{tabular}
	\vspace{10pt}
}
\newcommand{\rei}[2]{
	#1&\textit{#2}\\
}
\newcommand{\reinagai}[2]{
	\multicolumn{2}{l}{#1}\\
	\multicolumn{2}{l}{\hspace{10pt}\textit{#2}}\\
}

\newenvironment{mondai}[1]{
	\vspace{10pt}
	#1
	
	\begin{enumerate}
		\itemsep-5pt
	}{
	\end{enumerate}
	\vspace{10pt}
}

\newenvironment{hyou}{
	\begin{itemize}
		\itemsep-5pt
	}{
	\end{itemize}
	\vspace{10pt}
}

\date{\today}

\CJKfamily{chanto}\CJKnospace
\author{Tomislav Mamić}
\begin{document}
	\dai{Upotrebe て oblika I}
	
	U nastavku su opisani neki česti pomoćni glagoli koji se koriste s て oblikom. Osim što se na njega lijepe pomoćni glagoli, て oblik često sudjeluje i u raznim drugim gramatičkim izrazima.
	
	\fukudai{Pokus, pokušaj s \textasciitilde みる}
	
	U hrvatskom jeziku nema ekvivalentnog izraza - koristimo ga kad ne znamo kakav će biti rezultat neke radnje, ali ćemo je svejedno napraviti i \textbf{vidjeti}. Na prirodni hrvatski to u kontekstu možemo prevesti na razne načine.
	
	\begin{reibun}
		\rei{なっとうを たべてみた。}{Probao sam (jesti) natt\={o}.}
		\rei{おかあさんと はなしてみます。}{Pokušat ću pričati s majkom.}
		\rei{いちど 日本に いってみたい。}{Htio bih jednom otići u Japan.}
	\end{reibun}

	Pogledamo li rečenice iznad, možemo uočiti jednu vrlo bitnu stvar. Radnja glavnog glagola uopće nije upitna - ne radi se o tome da ćemo pokušati pa možda ne uspijemo - nesigurnost je u posljedicama naše radnje.
	
	\fukudai{Priprema za budućnost s \textasciitilde おく}
	
	Još jedan od izraza kojima u hrvatskom nemamo dobar par - ovako kažemo da radnju glavnog glagola obavljamo u pripremi za nešto što će se dogoditi u budućnosti. Od punog prijevoda na hrvatski u konkretnim situacijama najčešće odustanemo.
	
	\begin{reibun}
		\rei{たべておいて。}{Jedi \textnormal{(sada dok možeš)}.}
		\rei{ゆうべ ごみを 出しておいた。}{Sinoć sam iznio smeće \textnormal{(unaprijed da kasnije ne moram)}.}
		\rei{しようほうほうを よんでおいた。}{Pročitao sam upute \textnormal{(jer će mi dobro doći)}.}
	\end{reibun}

	Implicirano značenje je jako ovisno o kontekstu i puno ga je važnije shvatiti nego prevesti.
	
	\fukudai{Učiniti nešto za nekoga s \textasciitilde くれる i \textasciitilde あげる}
	
	Značenje ovog izraza vrlo je slično osnovnom značenju pomoćnih glagola koje koristimo, ali se primjenjuje na radnju. Hoćemo li koristiti くれる ili あげる, kao i u osnovnom značenju, ovisit će o tome tko nešto radi za koga. Kad netko tko nam je (u društvenom smislu) bliže čini nešto za nekog tko je izvan našeg zajedničkog kruga, koristit ćemo あげる. Obratno, kad netko izvan našeg kruga čini nešto za nekoga tko nam je bliže, reći ćemo くれる.
	
	\begin{reibun}
		\reinagai{たけしくんに すうがくを おしえてあげた。}{Pokazao sam Takešiju matematiku.}
		\reinagai{はなこちゃんが (わたしに) すうがくを おしえてくれた。}{Hanako mi je pokazala matematiku.}
		\reinagai{あにの へやを そうじ してあげた。}{Počistio sam sobu starijem bratu.}
		\reinagai{(かわりに) あにが わたしの しゅくだいを かいてくれた。}{(Zauzvrat) je on meni napisao zadaću.}
	\end{reibun}

	Iako općenito subjekt i meta mogu biti drugi ljudi, većinom ćemo あげる koristiti kad mi nešto radimo za nekoga, a くれる kad netko drugi nešto radi za nas. U pravilu je dobro izbjegavati あげる jer u krivoj situaciji može zvučati arogantno.
	
	\fukudai{Dobiti nekoga da nešto učini s \textasciitilde もらう}
	
	Ovaj ćemo glagol koristiti najčešće kad subjekt nešto za metu učini na zahtjev ili molbu. U odnosu na prethodna dva glagola, ovdje je perspektiva promijenjena - govorimo iz perspektive onoga za koga je radnja učinjena.
	
	\begin{reibun}
		\reinagai{たけしくんが (はなこちゃんに) すうがくを おしえてもらった。}{Hanako je Takešiju pokazala matematiku. \textnormal{(On je to zamolio.)}}
		\reinagai{(わたしは) はなこちゃんに すうがくを おしえてもらった。}{Hanako mi je pokazala matematiku. \textnormal{(Ja sam je zamolio.)}}
		\reinagai{あにの へやを そうじ してあげた。}{Počistio sam sobu starijem bratu. \textnormal{(Ovdje もらう nije prirodno!)}}
		\reinagai{(かわりに) わたしは あにに しゅくだいを かいてもらった。}{(Zauzvrat) sam tražio brata da mi napiše zadaću.}
	\end{reibun}

	U pravilu もらう nećemo koristiti tako da mi budemo meta (čestica に) jer je u značenje glagola ugrađena zahvalnost subjekta meti pa je neprirodno implicirati da je nama netko zahvalan. Ovo je dio šire filozofije jezika u kojoj se pokušava izbjeći govor o tuđim osjećajima.
	
	\fukudai{Vježba}
	
	\begin{mondai}{Lv. 1}
		\item のんでみた。
		\item そうじ しておいた。
		\item よんであげた。
		\item おしえてくれた。
		\item かってもらった。
	\end{mondai}

	\begin{mondai}{Lv. 2}
		\item このおちゃを のんでみましたか。
		\item いえを そうじ しておくと きめた。
		\item 先生は 子供たちに えほんを よんであげた。
		\item ともだちが わたしに ぶつりがくを おしえてくれた。
		\item かのじょに ぎゅうにゅうを かってもらった。
	\end{mondai}

	\begin{mondai}{Lv. 3}
		\item このおちゃを のんでみたいですか。
		\item いえを そうじ しておくと きめて せんざいを かいに いった。
		\item 先生は まいにち 子供たちに えほんを よんであげていました。
		\item ぶつりがくを おしえてくれて ありがとうと ともだちに いいました。
		\item *かのじょに かいわすれた ぎゅうにゅうを かいにいってもらった。
	\end{mondai}
\end{document}