% !TeX document-id = {5632978e-5791-4deb-8aa6-27d3fb53c997}
% !TeX program = xelatex ?me -synctex=0 -interaction=nonstopmode -aux-directory=../tex_aux -output-directory=./release
% !TeX program = xelatex

\documentclass[12pt]{article}

\usepackage{lineno,changepage,lipsum}
\usepackage[colorlinks=true,urlcolor=blue]{hyperref}
\usepackage{fontspec}
\usepackage{xeCJK}
\usepackage{tabularx}
\setCJKfamilyfont{chanto}{AozoraMinchoRegular.ttf}
\setCJKfamilyfont{tegaki}{Mushin.otf}
\usepackage[CJK,overlap]{ruby}
\usepackage{hhline}
\usepackage{multirow,array,amssymb}
\usepackage[croatian]{babel}
\usepackage{soul}
\usepackage[usenames, dvipsnames]{color}
\usepackage{wrapfig,booktabs}
\renewcommand{\rubysep}{0.1ex}
\renewcommand{\rubysize}{0.75}
\usepackage[margin=50pt]{geometry}
\modulolinenumbers[2]

\usepackage{pifont}
\newcommand{\cmark}{\ding{51}}%
\newcommand{\xmark}{\ding{55}}%

\definecolor{faded}{RGB}{100, 100, 100}

\renewcommand{\arraystretch}{1.2}

%\ruby{}{}
%$($\href{URL}{text}$)$

\newcommand{\furigana}[2]{\ruby{#1}{#2}}
\newcommand{\tegaki}[1]{
	\CJKfamily{tegaki}\CJKnospace
	#1
	\CJKfamily{chanto}\CJKnospace
}

\newcommand{\dai}[1]{
	\vspace{20pt}
	\large
	\noindent\textbf{#1}
	\normalsize
	\vspace{20pt}
}

\newcommand{\fukudai}[1]{
	\vspace{10pt}
	\noindent\textbf{#1}
	\vspace{10pt}
}

\newenvironment{bunshou}{
	\vspace{10pt}
	\begin{adjustwidth}{1cm}{3cm}
	\begin{linenumbers}
}{
	\end{linenumbers}
	\end{adjustwidth}
}

\newenvironment{reibun}{
	\vspace{10pt}
	\begin{tabular}{l l}
}{
	\end{tabular}
	\vspace{10pt}
}
\newcommand{\rei}[2]{
	#1&\textit{#2}\\
}
\newcommand{\reinagai}[2]{
	\multicolumn{2}{l}{#1}\\
	\multicolumn{2}{l}{\hspace{10pt}\textit{#2}}\\
}

\newenvironment{mondai}[1]{
	\vspace{10pt}
	#1
	
	\begin{enumerate}
		\itemsep-5pt
	}{
	\end{enumerate}
	\vspace{10pt}
}

\newenvironment{hyou}{
	\begin{itemize}
		\itemsep-5pt
	}{
	\end{itemize}
	\vspace{10pt}
}

\date{\today}

\CJKfamily{chanto}\CJKnospace
\author{Tomislav Mamić}
\begin{document}
	\dai{Glagoli II}
	
	\fukudai{Nepravilni glagoli}
	
	Glagole kojima se na nepredvidiv način osim gramatičkog repa mijenja i korijen (prvi dio riječi) smatramo nepravilnima. U japanskom postoji jako malo takvih glagola. Najčešća četiri dana su u tablici ispod, a ostale nećemo sresti još jako dugo.
	
	\begin{table}[h]
		\centering
		\begin{tabular}{l l l l l}\toprule[2pt]
			značenje & poz. neprošlost & poz. prošlost & neg. neprošlost & neg. prošlost\\
			\midrule
			\textit{doći} & くる & きた & こない & こなかった\\
			\textit{raditi}, \textit{činiti} & する & した & しない & しなかった\\
			\textit{ići} & いく & いった & いかない & いかなかった\\
			\textit{biti} (za neživo) & ある & あった & ない & なかった\\
			\bottomrule[2pt]
		\end{tabular}
	\end{table}

	Usporedimo li predzadnji i zadnji stupac, uočit ćemo da i za nepravilne glagole vrijedi pravilna tvorba prošlosti negacije (ない $\rightarrow$ なかった). Od četiri iznad navedena glagola, いく je zapravo uljez - izuzev prošlosti i て oblika (kojeg ćemo naučiti nešto kasnije), ponaša se kao ごだん glagol.
	
	\fukudai{する glagoli}
	
	Naučivši glagol する moguće je izreći puno više od njegovog osnovnog značenja. U japanskom postoji potkategorija imenica koje u kombinaciji s nekim oblikom glagola する funkcioniraju kao glagoli.
	
	\begin{reibun}
		\rei{べんきょう}{učenje}
		\rei{たけし\footnotemark[1]くんは べんきょうを しない。}{Takeši neće učiti.}
		\rei{そうじ}{čišćenje}
		\rei{たかぎ\footnotemark[2]さんは へやを そうじした。}{G. Takagi je očistio sobu.}
		\rei{せんたく}{pranje odjeće}
		\rei{すずき\footnotemark[2]さんは せんたくを しなかった。}{G. Suzuki nije oprala rublje.}
	\end{reibun}
	\footnotetext[1]{Muško ime.}
	
	Kod する glagola postoji zanimljiva pojava u korištenju čestice を. S obzirom na to kako se tvore, smisleno je reći npr. そうじを する (dosl. \textit{raditi čišćenje}). Međutim, kad želimo reći da smo očistili \textit{nešto}, čestica を nam treba da bismo označili to \textit{nešto}, pa je smisleno reći \textit{nešto}を そうじする. Neki する glagoli koriste se pretežno bez čestice を, neki je gotovo uvijek imaju, a postoje i glagoli gdje su oba slučaja jedako česta. Ovo je jedna od stvari koje nema smisla eksplicitno pokušavati učiti - najbolje je pogledati puno primjera i "dobiti osjećaj" za prirodnu upotrebu.
	
	\fukudai{Čestica mete i lokacije に}
	
	Ova čestica pojavljuje se okvirno u dvije različite upotrebe - kao oznaka lokacije za glagole stanja ili kao cilj/meta glagola radnje. Problem u upotrebi najčešće proizlazi iz razlika u shvaćanju glagola stanja u odnosu na hrvatski jezik - iako se filozofija donekle poklapa, postoje glagoli koji su u hrvatskom aktivni, a u japanskom se izražavaju glagolom stanja i obratno. Da stvar bude zabavnija, jedan dio glagola za lokaciju može koristiti i česticu に i で (lokacija za aktivne glagole). Zbog toga je u početku najbolje glagole učiti zajedno s česticama koje uz njih idu. Većina glagola bit će smislena u odnosu na hrvatski, ali oslonimo li se potpuno na hrvatski "zdravi razum", jedan dio glagola bit će nam jako neintuitivan.
	
	\fukudai{Lokacija glagola stanja}
	
	Dva prava glagola stanja koje smo dosad naučili su glagoli bivanja いる i ある. Njihova lokacija uvijek je označena s に.
	
	\begin{reibun}
		\rei{いえに ねこが いる。}{U kući je mačka.}
		\rei{はこに りんごが あった。}{U kutiji su bile jabuke.}
	\end{reibun}

	\fukudai{Meta aktivnih glagola}
	
	U ovom slučaju, \textit{meta} se može široko shvatiti. Za glagole kretanja, meta je krajnje odredište (hr. lokativ):
	
	\begin{reibun}
		\rei{にほんに いく。}{Idem u Japan.}
		\rei{ともだちの いえに くる。}{Doći ću kod prijatelja. \normalfont (dosl. \textit{Doći ću do prijateljeve kuće.}\normalfont)}
	\end{reibun}

	Za glagole u kojima se objekt prenosi, meta je ili polazište ili odredište objekta (hr. dativ):
	
	\begin{reibun}
		\rei{ともだちに ほんを あげた。}{Dao sam knjigu prijatelju.}
		\rei{せんせいは たけしくんに ほんを あげた。}{Učitelj je Takešiju dao knjigu.}
		\reinagai{せんせいに ほんを かりた。}{Posudio sam knjigu od učitelja. (\normalfont ne \textit{Učitelju sam posudio knjigu}!)\footnotemark[3]}
	\end{reibun}
	
	\footnotetext[2]{Prezime.}
	\footnotetext[3]{U hrvatskom koristimo jedan glagol za oba smjera posuđivanja (\textit{posuditi nekome} i \textit{posuditi od nekoga}), ali u japanskom su to dva odvojena glagola (かす i かりる).}

	Neki glagoli mogu promijeniti značenje ili konotaciju ovisno o tome koju česticu lokacije koristimo:
	
	\begin{reibun}
		\rei{ゆかに ねた。}{Legao sam na pod. \normalfont (\textit{pod} je meta)}
		\rei{ゆかで ねた。}{Zaspao sam na podu. \normalfont (\textit{pod} je mjesto radnje)}
	\end{reibun}

	\fukudai{Vježba}
	
	\begin{mondai}{Prevedi na hrvatski:}
		\item すずきさんは がっこうに いかなかった。
		\item かれの ともだちの ねこは いえに きた。
		\item にほんごを べんきょうする。
		\item かばんに かぎが ない。
		\item やまだ\footnotemark[2]さんは ともだちに おかねを かりた。
		\item やまださんの ともだちは かれに おかねを あげた。
	\end{mondai}

	
\end{document}
