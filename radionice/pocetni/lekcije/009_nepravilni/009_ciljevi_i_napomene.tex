% !TeX document-id = {c5b1505a-f30f-4093-80bc-284ee6feff4a}
% !TeX program = xelatex ?me -synctex=0 -interaction=nonstopmode -aux-directory=../tex_aux -output-directory=./release
% !TeX program = xelatex

\documentclass[12pt]{article}

\usepackage{lineno,changepage,lipsum}
\usepackage[colorlinks=true,urlcolor=blue]{hyperref}
\usepackage{fontspec}
\usepackage{xeCJK}
\usepackage{tabularx}
\setCJKfamilyfont{chanto}{AozoraMinchoRegular.ttf}
\setCJKfamilyfont{tegaki}{Mushin.otf}
\usepackage[CJK,overlap]{ruby}
\usepackage{hhline}
\usepackage{multirow,array,amssymb}
\usepackage[croatian]{babel}
\usepackage{soul}
\usepackage[usenames, dvipsnames]{color}
\usepackage{wrapfig,booktabs}
\renewcommand{\rubysep}{0.1ex}
\renewcommand{\rubysize}{0.75}
\usepackage[margin=50pt]{geometry}
\modulolinenumbers[2]

\usepackage{pifont}
\newcommand{\cmark}{\ding{51}}%
\newcommand{\xmark}{\ding{55}}%

\definecolor{faded}{RGB}{100, 100, 100}

\renewcommand{\arraystretch}{1.2}

%\ruby{}{}
%$($\href{URL}{text}$)$

\newcommand{\furigana}[2]{\ruby{#1}{#2}}
\newcommand{\tegaki}[1]{
	\CJKfamily{tegaki}\CJKnospace
	#1
	\CJKfamily{chanto}\CJKnospace
}

\newcommand{\dai}[1]{
	\vspace{20pt}
	\large
	\noindent\textbf{#1}
	\normalsize
	\vspace{20pt}
}

\newcommand{\fukudai}[1]{
	\vspace{10pt}
	\noindent\textbf{#1}
	\vspace{10pt}
}

\newenvironment{bunshou}{
	\vspace{10pt}
	\begin{adjustwidth}{1cm}{3cm}
	\begin{linenumbers}
}{
	\end{linenumbers}
	\end{adjustwidth}
}

\newenvironment{reibun}{
	\vspace{10pt}
	\begin{tabular}{l l}
}{
	\end{tabular}
	\vspace{10pt}
}
\newcommand{\rei}[2]{
	#1&\textit{#2}\\
}
\newcommand{\reinagai}[2]{
	\multicolumn{2}{l}{#1}\\
	\multicolumn{2}{l}{\hspace{10pt}\textit{#2}}\\
}

\newenvironment{mondai}[1]{
	\vspace{10pt}
	#1
	
	\begin{enumerate}
		\itemsep-5pt
	}{
	\end{enumerate}
	\vspace{10pt}
}

\newenvironment{hyou}{
	\begin{itemize}
		\itemsep-5pt
	}{
	\end{itemize}
	\vspace{10pt}
}

\date{\today}

\CJKfamily{chanto}\CJKnospace
\author{Tomislav Mamić}
\begin{document}
	\dai{Ciljevi i napome - nepravilni glagoli}
	
	\fukudai{Ciljevi}
	
	\begin{hyou}
		\item objasniti zašto su nepravilni glagoli nepravilni (promjena korijena)
		\item naučiti 4 osnovna nepravilna glagola
		\item する glagoli
		\item čestica に kao lokacija glagola stanja (いる/ある) i meta aktivnog glagola (いく/くる/あげる)
	\end{hyou}

	\fukudai{Napomene}
	
	\begin{hyou}
		\item いく je nepravilan na samo dva mjesta (prošlost, て), a inače se ponaša kao ごだん く glagol
		\item ispričati da postoji još glagola koji mjestimično odstupaju od savršene pravilnosti (npr. stari pristojni glagoli u い obliku nemaju る $\rightarrow$ り nego る $\rightarrow$ い), ali ne ulaziti duboko u priču
		\item ispričati priču o を i suru glagolima (npr. それを選択する vs. 選択をする), ali ne ulaziti u detalje
		\item objasniti da glagoli stanja nisu uvijek isti kao u hrvatskom i da je zato ponekad teško pogoditi je li に prikladno za lokaciju, a i da postoje situacije gdje je i jedno i drugo ispravno
		
		(npr. に寝る, で寝る).
	\end{hyou}

	\fukudai{Dodatne pričice}
	
	Glavni izvor nepravilnosti u jeziku je jednostavnost izgovora, a protuteža tome je razumljivost govora. Krajnje stanje svakog jezika je optimum u kojem je izgovor maksimalno pojednostavljen bez pojave nerazrješivih dvoznačnosti. U usporedbi s hrvatskim, japanski je dosta bliže tom optimumu iz raznih razloga, a jedan od glavnih je to što je glasovni sustav japanskog po dizajnu puno jednostavniji za izgovor pa je unos nepravilnosti u jezik manji.
	
	Najbolji primjeri toga su glasovne promjene jednačenja po zvučnosti i mjestu tvorbe. U hrvatskom jeziku, zbog nizova suglasnika ove promjene su velikim dijelom neizbježne (pokušajte izgovoriti \textit{podpis} umjesto \textit{potpis}), ali u japanskom se susjedni suglasnici ne pojavljuju pa je primjena glasovnih promjena puno slobodnija (pokušajte izgovoriti みずくも umjesto みずぐも).
	
	Vrlo često je moguće pratiti razvoj nepravilnosti kroz povijest jezika. Jedan zgodan primjer toga je glagol ある čija je moderna negacija početkom 20. st. skraćena u ない. Dok je nastavak negacije bio ん/ぬ, bilo je uobičajeno reći あらん što se još i danas zna čuti u okamenjenim izrazima. Kao kolokvijalni govor, još i danas se može čuti わからん umjesto わからない, a nešto rjeđe i kao pravilnost za ostale glagole, iako takav govor zvuči arhaično. U modernom Kansai dijalektu, još se uvijek vidi izvorna negacija kao あらへん (=ない).
\end{document}