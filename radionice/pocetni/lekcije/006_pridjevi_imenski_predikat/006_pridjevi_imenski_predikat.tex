% !TeX document-id = {048a8486-122a-4c5e-ab1d-890bfad423dd}
% !TeX program = xelatex ?me -synctex=0 -interaction=nonstopmode -aux-directory=../../tex_aux -output-directory=./release
% !TeX program = xelatex

\documentclass[12pt]{article}

\usepackage{lineno,changepage,lipsum}
\usepackage[colorlinks=true,urlcolor=blue]{hyperref}
\usepackage{fontspec}
\usepackage{xeCJK}
\usepackage{tabularx}
\setCJKfamilyfont{chanto}{AozoraMinchoRegular.ttf}
\setCJKfamilyfont{tegaki}{Mushin.otf}
\usepackage[CJK,overlap]{ruby}
\usepackage{hhline}
\usepackage{multirow,array,amssymb}
\usepackage[croatian]{babel}
\usepackage{soul}
\usepackage[usenames, dvipsnames]{color}
\usepackage{wrapfig,booktabs}
\renewcommand{\rubysep}{0.1ex}
\renewcommand{\rubysize}{0.75}
\usepackage[margin=50pt]{geometry}
\modulolinenumbers[2]

\usepackage{pifont}
\newcommand{\cmark}{\ding{51}}%
\newcommand{\xmark}{\ding{55}}%

\definecolor{faded}{RGB}{100, 100, 100}

\renewcommand{\arraystretch}{1.2}

%\ruby{}{}
%$($\href{URL}{text}$)$

\newcommand{\furigana}[2]{\ruby{#1}{#2}}
\newcommand{\tegaki}[1]{
	\CJKfamily{tegaki}\CJKnospace
	#1
	\CJKfamily{chanto}\CJKnospace
}

\newcommand{\dai}[1]{
	\vspace{20pt}
	\large
	\noindent\textbf{#1}
	\normalsize
	\vspace{20pt}
}

\newcommand{\fukudai}[1]{
	\vspace{10pt}
	\noindent\textbf{#1}
	\vspace{10pt}
}

\newenvironment{bunshou}{
	\vspace{10pt}
	\begin{adjustwidth}{1cm}{3cm}
	\begin{linenumbers}
}{
	\end{linenumbers}
	\end{adjustwidth}
}

\newenvironment{reibun}{
	\vspace{10pt}
	\begin{tabular}{l l}
}{
	\end{tabular}
	\vspace{10pt}
}
\newcommand{\rei}[2]{
	#1&\textit{#2}\\
}
\newcommand{\reinagai}[2]{
	\multicolumn{2}{l}{#1}\\
	\multicolumn{2}{l}{\hspace{10pt}\textit{#2}}\\
}

\newenvironment{mondai}[1]{
	\vspace{10pt}
	#1
	
	\begin{enumerate}
		\itemsep-5pt
	}{
	\end{enumerate}
	\vspace{10pt}
}

\newenvironment{hyou}{
	\begin{itemize}
		\itemsep-5pt
	}{
	\end{itemize}
	\vspace{10pt}
}

\date{\today}

\CJKfamily{chanto}\CJKnospace
\author{Tomislav Mamić, Željka Ludošan}

\begin{document}
	\dai{Pridjevi kao imenski predikat}
	
	Pridjevi u japanskom su vrste riječi koje opisuju imenice i dijelimo ih prema nastavcima na: い、な i の pridjeve. U japanskom je moguće da jedna riječ ulazi u više vrsta riječi, pa su tako な i の pridjevi ujedno i imenice.
	
\underline{U rječničkom obliku}, い pridjevi uvijek završavaju na い, dok な i の pridjevi mogu završavati na bilo koji znak.

\fukudai{い pridjevi}


Naučimo neke korisne い pridjeve vezane uz boje:

	\vspace{10pt}
	\begin{tabular}{|l|l|}
		\hline
		しろい&bijelo\\\hline
		くろい&crno\\\hline
		くらい&tamno\\\hline
		あかるい&svijetlo\\\hline
		あおい&plavo, plavo-zeleno, zeleno(svjetlo na semaforu), nezrelo\\\hline
		あかい&crveno\\\hline
		きいろい&žuto\\\hline
	\end{tabular}
	
\vspace{10pt}
Neki korisni い pridjevi vezani uz oblike:
	
	\vspace{10pt}
	\begin{tabular}{|l|l|l|l|}
		\hline
		おおきい&veliko&せまい&usko\\\hline
		ちいさい&malo&ひろい&široko\\\hline
		まるい&okruglo&ふかい&duboko\\\hline
		しかくい&četvrtasto&あさい&plitko\\\hline
	\end{tabular}
	
\vspace{10pt}
Korisni pridjevi koji su slični u hrvatskom, ali bitno različiti:
	
	\vspace{10pt}
	\begin{tabular}{|l|l|}
		\hline
		むずかしい&teško (potrebno je mnogo truda)\\\hline
		おもい&teško (ima veliku težinu)\\\hline
		ふとい&debelo\\\hline
		あつい&debelo\\\hline
		ほそい&tanko\\\hline
		うすい&tanko, slabo, polu-prozirno\\\hline
	\end{tabular}
	
	\vspace{10pt}
	ふとい i ほそい se koriste za većinu stvari, uglavnom cilindrične objekte i ljude, dok se あつい i うすい koriste za plosnate stvari i stvari koje su naslagane jedno na drugo.

	
	\vspace{10pt}
	I šećer na kraju:
	
	\vspace{10pt}
	\begin{tabular}{|l|l|}
		\hline
		かわいい&slatko (za osobe, čupave kućne ljubimce i ostale stvari od kojih možete dobiti infarkt)\\\hline
		あまい&slatko (okus), naivno\\\hline
		おもしろい&zanimljivo, zabavno\\\hline
		すごい&super, odlično (za osobe, situacije, kada se nečemu divite), grozno\\\hline
		うるさい&(nešto što mi) ide na živce, glasno\\\hline
	\end{tabular}
	\vspace{10pt}
	
Imenski predikat je onaj predikat koji sadrži pravi glagol već pridjev ili imenicu i spojni glagol.

	\begin{reibun}
		\rei{その ねこ は \underline{くろい}。}{Ta mačka je crna.}
		\rei{にほんご の べんきょう は \underline{むずかしい}。}{Učenje japanskog je teško.}
		\rei{デスクトップ ピーシー は \underline{おもい}。}{Desktop PC je težak.}
	\end{reibun}
	
	
	\vspace{10pt}
	\begin{tabular}{|l||l|l|}
		\hline
		\multicolumn{3}{c}{Konjugacije い pridjeva}\\\hline
		 &Pozitivno&Negativno\\\hline
		Neprošlost&-い&-くない\\\hline
		Prošlost&-かった&-くなかった\\\hline
	\end{tabular}

	\vspace{10pt}

	\begin{reibun}
		\rei{ほん が \underline{あつくなかった}。}{Knjiga nije bila debela.}
		\rei{\underline{ひくい} やま は \underline{むずかしくない}。}{Niska planina nije teška(za svladati).}
		\rei{ライブ が \underline{すごかった}。}{Live je bio odličan.(Live koncert)}
	\end{reibun}
	
\fukudai{な pridjevi}

な pridjevi su različiti po tome što se drugačije konjugiraju.
Zovemo ih tako jer im se mora pridodati な kada se nalaze ispred imenice u neprošlom pozitivnom vremenu.

Zanimljiva stvar kod njih je to što se u svim ostalim slučajevima konjugiraju kao imenice, što zapravo i jesu. To znači da se, kada rečenicu završavamo な pridjevom, umjesto nastavka な dodaje spojni glagol kao i kod imenica.\footnotemark[1] Time se sam pridjev pretvara u imenski predikat.

\footnotetext[1]{Iako, u kolokvijalnom jeziku, だ se na kraju ponekad zna izostaviti.}

	\begin{reibun}
		\rei{きれい \underline{な} そら}{lijepo nebo}
		\rei{そら は きれい \underline{だ}。}{Nebo je lijepo.}
	\end{reibun}

	\vspace{10pt}
	\begin{tabular}{|l|l|}
		\hline
		すき&drago, svidljivo\\\hline
		きらい&mrsko\\\hline
		きれい&lijepo, čisto\\\hline
		しずか&tiho\\\hline
		ふくざつ&komplicirano\\\hline
		かんたん&jednostavno, nekomplicirano\\\hline
		めんどう&problematično\\\hline
	\end{tabular}
\newpage
	
		\vspace{10pt}
	\begin{tabular}{|l||l|l|}
		\hline
		\multicolumn{3}{c}{Konjugacije な pridjeva}\\\hline
		 &Pozitivno&Negativno\\\hline
		Neprošlost&-な/-だ&-じゃない\\\hline
		Prošlost&-だった&-じゃなかった\\\hline
	\end{tabular}


	\begin{reibun}
		\rei{\underline{かんたんな} もんだい だ。}{Problem je jednostavan.}
		\rei{へや は \underline{きれい じゃなかった}。}{Soba nije bila čista.}
		\rei{\underline{まるい} め が \underline{すき だ}。}{Sviđaju mi se okrugle oči.}
	\end{reibun}
	

\fukudai{の pridjevi}

 の pridjevi su zapravo imenice koje opisuju druge imenice tako da ih se povezuje česticom の. Konjugiraju se kao obične imenice.
 
 	\vspace{10pt}
	\begin{tabular}{|l|l|}
		\hline
		オレンジ&naranča, narančasto\\\hline
		みどり&zelenilo, zeleno\\\hline
		みどりいろ&zelena boja, zelene boje\\\hline
		むらさき&ljubičasta, ljubičasto\\\hline
		ピンク&roza, rozo\\\hline
		はいいろ&siva, boja pepela\\\hline
	\end{tabular}
	
	\vspace{10pt}
	\begin{tabular}{|l||l|l|}
		\hline
		\multicolumn{3}{c}{Konjugacije の pridjeva}\\\hline
		 &Pozitivno&Negativno\\\hline
		Neprošlost&-の/-だ&-じゃない\\\hline
		Prošlost&-だった&-じゃなかった\\\hline
	\end{tabular}
	
	\begin{reibun}
	\rei{もうふ は \underline{むらさき じゃなかった}。}{Deka nije bila ljubičasta.}
	\rei{\underline{みどりいろ だった。}}{Bila je zelene boje.}
	\rei{\underline{ピンクの} はな が \underline{すき じゃない}。}{Ne volim rozo cvijeće.}
	\end{reibun}
	
\fukudai{Pristojni oblici}


な i の pridjevi, kada se konjugiraju u pristojnom obliku, konjugiraju se isto kao imenice, dok い pridjevi imaju svoja pravila. U negativnom obliku mogu se koristiti -です i -ます oblik, s time da je -ます oblik formalniji od -です.

	\vspace{10pt}
	\begin{tabular}{|l||l|l|}
		\hline
		\multicolumn{3}{c}{Konjugacije な i の pridjeva u pristojnom obliku}\\\hline
		 &Pozitivno&Negativno\\\hline
		Neprošlost&-です&-じゃないです/ -じゃありません\\\hline
		Prošlost&-でした&-じゃなかったです/ -じゃありませんでした\\\hline
	\end{tabular}
	
		\vspace{10pt}
	\begin{tabular}{|l||l|l|}
		\hline
		\multicolumn{3}{c}{Konjugacije い pridjeva u pristojnom obliku}\\\hline
		 &Pozitivno&Negativno\\\hline
		Neprošlost&-です&-くないです/ -くありません\\\hline
		Prošlost&-かったです&-くなかったです/ -くありませんでした\\\hline
	\end{tabular}
	
\vspace{10pt}


Ako želite da zvuči još formalnije, možete じゃ pretvoriti u では(čita se kao でわ).

\fukudai{Opisne zamjenice}

Opisne zamjenice su zamjenice koje opisuju prvu imenicu ispred koje se nalaze.


Par čestih opisnih zamjenica:

 	\vspace{10pt}
	\begin{tabular}{|l|l|}
		\hline
		こんな&ovakvo\\\hline
		そんな&takvo\\\hline
		あんな&onakvo\\\hline
	\end{tabular}
\vspace{10pt}

	\begin{reibun}
	\rei{\underline{そんな} やさい が すき じゃない。}{Ne volim takvo povrće.}
	\end{reibun}

	
	\normalsize \textbf{Primjeri za vježbu}
	
	\begin{mondai}{Pridjevi}
		\item あかるい そら
		\item そんな いろ
		\item かわいくない おんなのこ
		\item かれ は めんどう だった。
		\item くろい デスクトップ ピーシー は オレンジ じゃない。
		\item オレンジ は デスクトップ ピーシー じゃない。
		\item おばさん の おおきい ねこ が おもくなかった。
	\end{mondai}

	



\end{document}
