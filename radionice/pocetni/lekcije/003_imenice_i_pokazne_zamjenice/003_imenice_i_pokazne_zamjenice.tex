% !TeX document-id = {048a8486-122a-4c5e-ab1d-890bfad423dd}
% !TeX program = xelatex ?me -synctex=0 -interaction=nonstopmode -aux-directory=../../tex_aux -output-directory=./release
% !TeX program = xelatex

\documentclass[12pt]{article}

\usepackage{lineno,changepage,lipsum}
\usepackage[colorlinks=true,urlcolor=blue]{hyperref}
\usepackage{fontspec}
\usepackage{xeCJK}
\usepackage{tabularx}
\setCJKfamilyfont{chanto}{AozoraMinchoRegular.ttf}
\setCJKfamilyfont{tegaki}{Mushin.otf}
\usepackage[CJK,overlap]{ruby}
\usepackage{hhline}
\usepackage{multirow,array,amssymb}
\usepackage[croatian]{babel}
\usepackage{soul}
\usepackage[usenames, dvipsnames]{color}
\usepackage{wrapfig,booktabs}
\renewcommand{\rubysep}{0.1ex}
\renewcommand{\rubysize}{0.75}
\usepackage[margin=50pt]{geometry}
\modulolinenumbers[2]

\usepackage{pifont}
\newcommand{\cmark}{\ding{51}}%
\newcommand{\xmark}{\ding{55}}%

\definecolor{faded}{RGB}{100, 100, 100}

\renewcommand{\arraystretch}{1.2}

%\ruby{}{}
%$($\href{URL}{text}$)$

\newcommand{\furigana}[2]{\ruby{#1}{#2}}
\newcommand{\tegaki}[1]{
	\CJKfamily{tegaki}\CJKnospace
	#1
	\CJKfamily{chanto}\CJKnospace
}

\newcommand{\dai}[1]{
	\vspace{20pt}
	\large
	\noindent\textbf{#1}
	\normalsize
	\vspace{20pt}
}

\newcommand{\fukudai}[1]{
	\vspace{10pt}
	\noindent\textbf{#1}
	\vspace{10pt}
}

\newenvironment{bunshou}{
	\vspace{10pt}
	\begin{adjustwidth}{1cm}{3cm}
	\begin{linenumbers}
}{
	\end{linenumbers}
	\end{adjustwidth}
}

\newenvironment{reibun}{
	\vspace{10pt}
	\begin{tabular}{l l}
}{
	\end{tabular}
	\vspace{10pt}
}
\newcommand{\rei}[2]{
	#1&\textit{#2}\\
}
\newcommand{\reinagai}[2]{
	\multicolumn{2}{l}{#1}\\
	\multicolumn{2}{l}{\hspace{10pt}\textit{#2}}\\
}

\newenvironment{mondai}[1]{
	\vspace{10pt}
	#1
	
	\begin{enumerate}
		\itemsep-5pt
	}{
	\end{enumerate}
	\vspace{10pt}
}

\newenvironment{hyou}{
	\begin{itemize}
		\itemsep-5pt
	}{
	\end{itemize}
	\vspace{10pt}
}

\date{\today}

\CJKfamily{chanto}\CJKnospace
\author{Tomislav Mamić, Željka Ludošan}

\begin{document}
	\dai{Imenice i pokazne zamjenice}
	
	\fukudai{Imenice}
	
	Imenice u japanskom su nepromjenjive riječi bez roda, broja i padeža kojima imenujemo bića, stvari, pojave, pojmove, pa čak i radnje. Imenice se mogu opisivati ili koristiti da se njima opisuju druge imenice, a na sebe mogu ljepiti razne nastavke. Zamjenice u japanskom se gramatički ponašaju kao imenice.

Naučimo neke česte imenice:
\vspace{10pt}

	\begin{tabular}{|l|l|}
		\hline
		ねこ&mačka\\\hline
	\end{tabular}
	\vspace{10pt}
	
	Sad kad smo to rješili, možemo na ostale manje bitne:
	
	\vspace{10pt}
	\begin{tabular}{|l|l|l||l||l||l|}
		\hline
		しょくぶつ&biljke&どうぶつ&životinje&くだもの&voće\\\hline
		き&drvo&とり&ptica&りんご&jabuka\\\hline
		かわ&rijeka&にわとり&kokoš&はな&cvijet\\\hline
		こうえん&park&いぬ&pas&なし&kruška\\\hline
		は&list&うし&krava&うめ&šljiva\\\hline
		つち&zemlja&むし&kukac&いちご&jagoda\\\hline
		いえ&kuća&いろ&boja&ぶどう&grožđe\\\hline
		にわ&dvorište&うま&konj&すいか&lubenica\\\hline
	\end{tabular}

	
	\vspace{10pt}
	I za kraj, najčešća zamjenica:
	
	\vspace{10pt}
	\begin{tabular}{|l|l|}
		\hline
		わたし&ja\\\hline
	\end{tabular}
	\vspace{10pt}

	
	\fukudai{Pokazne zamjenice - これ、それ、あれ\footnotemark[1]}

	Pokazne zamjenice これ、それ、あれ ponašaju se kao imenice i koriste se za ukazivanje na neku stvar, osobu, događaj ili čak apstraktan pojam. U japanskom jeziku pokazne zamjenice veoma ovise o tome koliko je stvar na koju se ukazuje udaljena od govornika.
	
	\vspace{10pt}
	\begin{tabular}{|l|l|l|}
		\hline
		これ&ovo&ukazuje na nešto što je relativno blizu nama\\\hline
		それ&to&ukazuje na nešto što je blizu sugovorniku, a udaljeno od nas\\\hline
		あれ&ono&ukazuje na nešto što je udaljeno i od nas i od sugovornika\\\hline
	\end{tabular}
	\vspace{10pt}
	
	Zamislimo da u ruci držimo jabuku. Zatim nam dođe netko i pita nas što je to.
	Tada možemo reći \textit{\underline{これ} は りんご です。 Ovo je jabuka}. Ako tu istu jabuku dodamo toj drugoj osobi tada možemo reći:	\textit{ qundeline{それ} は りんご です。 To je jabuka.} Ako smo ukrali tu jabuku i ne želimo da nas ulove možemo baciti jabuku u susjedovo dvorište i reći: \textit{\underline{あれ} は りんご です。Ono je jabuka.}
	
	\newpage
	
		\fukudai{Posvojna čestica - の}
	
	Čestica の može se koristiti u razne svrhe. Kao posvojnu česticu koristimo je za opisivanje imenica i ukazivanje na pripadnost.

	
	\juuyou[10pt]{<riječ kojom opisujemo>の<riječ koja se opisuje>}
	
	\begin{reibun}
		\rei{はな の いろ}{Boja cvijeta.}
	\end{reibun}
	
	Imenica \textit{boja} opisuje se imenicom \textit{cvijet}, pa tako はな の いろ postaje \textit{boja od cvijeta}.
	
	\begin{reibun}
		\rei{かわ の むし}{Riječni kukac.}
		\rei{き の は\footnotemark[2]}{Lišće drveća.}
	\end{reibun}
		
		
	Osim imenica mogu se koristiti i pokazne zamjenice, ali tada one mijenjaju oblik u この、 その、 あの.\footnotemark[1]

	\vspace{10pt}
	\begin{tabular}{|l|l|l|}
		\hline
		これの&→&この\\\hline
		それの&→&その\\\hline
		あれの&→&あの\\\hline
	\end{tabular}
	\vspace{10pt}
	
	\begin{reibun}
		\rei{この ぶどう}{Ovo grožđe.}
		\rei{あの とり}{Ona ptica.}
		\rei{たなか さん の いえ}{Tanakina kuća.}
	\end{reibun}
		

	Osim općih imenica mogu se koristiti i osobne zamjenice, kao i vlastita imena.
	
	\begin{reibun}
		\rei{わたし の ねこ}{Moja mačka.}
		\rei{たなか さん の ほん}{Tanakina knjiga.}
	\end{reibun}
	
	Također, mogu se koristiti i vremenske oznake.
	
	\begin{reibun}
		\rei{きょう の てんき}{Vrijeme danas.}
		\rei{らいねん の たんじょうび}{Rođendan sljedeće godine.}
	\end{reibun}
	
	Posvojna zamjenica の može se koristiti za spajanje više međusobno povezanih opisa za redom.
	
	\begin{reibun}
		\rei{この うま の すいか。}{Lubenica od ovog konja.}
	\end{reibun}
	
	Ovdje se nalaze dva opisa.  \textit{この うま Ovaj konj} i \textit{うま の すいか Lubenica od konja}.
	
	\begin{reibun}
		\rei{たなか さん の ほん の おもさ}{Težina Tanakine knjige.}
		\rei{この いえ の たかさ}{Visina ove kuće.}
	\end{reibun}


		\footnotetext[1]{Iako se これ、それ、あれ i この、その、あの mogu pisati sa kanji znakovima u pravilu se pišu u hiragani.}
	\footnotetext[2]{\textit{Lišće drveća} se zapravo češće čita こ の は}
	
	\vspace{100pt}
	\normalsize \textbf{Primjeri za vježbu}
	
	\begin{mondai}{Lv. 1}
		\item たなか さん の ねこ
		\item この ほん
		\item はな の いろ
		\item その うま
		\item あの すいか
	\end{mondai}
	
	\begin{mondai}{Lv. 2}
		\item は の むし
		\item かわ の つち
		\item こうえん の しょくぶつ
		\item にわ の どうぶつ
	\end{mondai}
	
	\begin{mondai}{Lv. 3}
		\item にわ の にわとり の とり
		\item この たなか さん の りんご の き
	\end{mondai}
	



\end{document}