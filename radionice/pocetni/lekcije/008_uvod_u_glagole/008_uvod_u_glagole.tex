% !TeX document-id = {852ff5fb-83c4-488d-a82c-10ef1d6436d7}
% !TeX program = xelatex ?me -synctex=0 -interaction=nonstopmode -aux-directory=../tex_aux -output-directory=./release
% !TeX program = xelatex

\documentclass[12pt]{article}

\usepackage{lineno,changepage,lipsum}
\usepackage[colorlinks=true,urlcolor=blue]{hyperref}
\usepackage{fontspec}
\usepackage{xeCJK}
\usepackage{tabularx}
\setCJKfamilyfont{chanto}{AozoraMinchoRegular.ttf}
\setCJKfamilyfont{tegaki}{Mushin.otf}
\usepackage[CJK,overlap]{ruby}
\usepackage{hhline}
\usepackage{multirow,array,amssymb}
\usepackage[croatian]{babel}
\usepackage{soul}
\usepackage[usenames, dvipsnames]{color}
\usepackage{wrapfig,booktabs}
\renewcommand{\rubysep}{0.1ex}
\renewcommand{\rubysize}{0.75}
\usepackage[margin=50pt]{geometry}
\modulolinenumbers[2]

\usepackage{pifont}
\newcommand{\cmark}{\ding{51}}%
\newcommand{\xmark}{\ding{55}}%

\definecolor{faded}{RGB}{100, 100, 100}

\renewcommand{\arraystretch}{1.2}

%\ruby{}{}
%$($\href{URL}{text}$)$

\newcommand{\furigana}[2]{\ruby{#1}{#2}}
\newcommand{\tegaki}[1]{
	\CJKfamily{tegaki}\CJKnospace
	#1
	\CJKfamily{chanto}\CJKnospace
}

\newcommand{\dai}[1]{
	\vspace{20pt}
	\large
	\noindent\textbf{#1}
	\normalsize
	\vspace{20pt}
}

\newcommand{\fukudai}[1]{
	\vspace{10pt}
	\noindent\textbf{#1}
	\vspace{10pt}
}

\newenvironment{bunshou}{
	\vspace{10pt}
	\begin{adjustwidth}{1cm}{3cm}
	\begin{linenumbers}
}{
	\end{linenumbers}
	\end{adjustwidth}
}

\newenvironment{reibun}{
	\vspace{10pt}
	\begin{tabular}{l l}
}{
	\end{tabular}
	\vspace{10pt}
}
\newcommand{\rei}[2]{
	#1&\textit{#2}\\
}
\newcommand{\reinagai}[2]{
	\multicolumn{2}{l}{#1}\\
	\multicolumn{2}{l}{\hspace{10pt}\textit{#2}}\\
}

\newenvironment{mondai}[1]{
	\vspace{10pt}
	#1
	
	\begin{enumerate}
		\itemsep-5pt
	}{
	\end{enumerate}
	\vspace{10pt}
}

\newenvironment{hyou}{
	\begin{itemize}
		\itemsep-5pt
	}{
	\end{itemize}
	\vspace{10pt}
}

\date{\today}

\CJKfamily{chanto}\CJKnospace
\author{Tomislav Mamić}
\begin{document}
	\dai{Glagoli I}
	
	\fukudai{Teorija}
	
	Glagoli su, uz い pridjeve, jedina druga promjenjiva vrsta riječi u japanskom. Mijenjaju se dodavanjem nastavaka (ごび - dosl. \textit{rep riječi}). Dodani nastavci mogu na razne načine utjecati na značenje glagola, mijenjajući oblik, vrijeme, pristojnost ili čak vrstu riječi.
	
	Osnovni oblik glagola zove se \textbf{rječnički oblik} (じしょけい), jer su glagoli tako pisani u rječniku, a karakteristika mu je da svi glagoli završavaju glasom \textit{u}. Postoji više slogova hiragane koji završavaju tim glasom, ali ne pojavljuju se svi na kraju glagola. Oni koji mogu doći na kraju glagola su く, ぐ, す, ぬ, む, ぶ, う, つ i る.
	
	Prema načinu na koji se mijenjaju, glagoli su podijeljeni u sljedeće tri skupine:
	
	\begin{table}[h]
		\centering
		\begin{tabular}{l r r r}\toprule[2pt]
			skupina & količina & učestalost & gnjavaža\\
			\midrule
			nepravilni & 4 & $\approx 25$\% & $\star\star\star$\\
			いちだん & $\approx 33$\% & $\approx 25$\% & $\star$\\
			ごだん & $\approx 66$\% & $\approx 50$\% & $\star\:\star$\\
			\bottomrule[2pt]
		\end{tabular}
	\end{table}

	U repu glagola sadržano je jako puno informacija pa je stoga važno dobro poznavati sve moguće nastavke i oblike. Za početak, naučit ćemo dvije osnovne promjene - prošlost i negaciju.
	
	\fukudai{Prošlost i... neprošlost}
	
	U japanskom jeziku postoje samo dva eksplicitna glagolska vremena - prošlost i neprošlost. U neprošlom obliku sadržana su značenja budućnosti i prošlosti, a uvijek su jasna iz konteksta. Kako bismo vidjeli da je to moguće, pogledajmo sličnu situaciju u hrvatskom:
	
	\vspace{5pt}
	\hspace{10pt} \textit{Jedem jabuku.} - \textit{jedem} je nesvršeni oblik glagola u prezentu - pretpostavljamo da se radnja događa upravo sada
	
	\vspace{5pt}
	\hspace{10pt} \textit{Sljedeće godine idem u Japan.} - \textit{idem} je nesvršeni oblik u prezentu kao i gore, ali iz konteksta znamo da se radi o budućnosti
	
	\vspace{5pt}
	Kao informacija na repu glagola, vrijeme je \textbf{uvijek na zadnjem mjestu}. Neprošlost poznajemo po zadnjem glasu \textit{u}, a prošlost po た ili だ. Ranije spomenuti rječnički oblik zapravo je kolokvijalni (kratki) neprošli oblik glagola.
	
	\fukudai{Negacija}
	
	Kao kod い pridjeva, negaciju prepoznajemo po nastavku ない. Kao i ranije, negacija je u repu na predzadnjem mjestu, i može se kombinirati s informacijom o vremenu. Iako u praksi zbog mutne granice između glagola i い pridjeva ova informacija uopće nije bitna, zgodno je primijetiti da se negirani glagol zapravo ponaša kao い pridjev.
	
	\newpage
	\fukudai{いちだん glagoli}
	
	Što se tiče količine informacija koju treba zapamtiti, ovo je najlakša skupina glagola. Nužan (ali ne dovoljan) uvjet da bi glagol pripadao ovoj skupini je da završava na glasove -\textit{iru} ili -\textit{eru}. Problem u prepoznavanju ovih glagola je u tome što u ごだん skupini postoje glagoli koji završavaju na る, a ispred imaju glasove \textit{i} ili \textit{e} pa je u općenitom slučaju za takve nemoguće odrediti kojoj skupini pripadaju. U praksi postoje i istozvučni glagoli u obje skupine (npr. きる kao いちだん - \textit{nositi odjeću}, kao ごだん - \textit{prerezati} ili かえる kao いちだん - \textit{promijeniti}, kao ごだん - \textit{vratiti se}), ali ovi primjeri su relativno rijetki pa ih je lako zapamtiti.
	
	\begin{table}[h]
		\centering
		\begin{tabular}{l l l}\toprule[2pt]
			& neprošlost & prošlost\\
			\midrule
			poz. & \textasciitilde る & \textasciitilde た\\
			neg. & \textasciitilde ない & \textasciitilde なかった\\
			\bottomrule[2pt]
		\end{tabular}
	\end{table}

	Tablica iznad pokazuje osnovne kombinacije vremena i negacije za いちだん glagole. Ako smo dobro proučili い pridjeve, uočit ćemo da je drugi red identičan onome što već znamo o njima. Pogledajmo neke primjere i njihova značenja.
	
	\begin{reibun}
		\rei{みる}{vidjet/pogledat ću}
		\rei{みた}{vidio/pogledao sam}
		\rei{みない}{neću vidjeti/pogledati}
		\rei{みなかった}{nisam vidio/pogledao}
	\end{reibun}
	
	Uočimo sljedeće:
	\begin{hyou}
		\item Glagol u japanskom ne nosi nikakvu informaciju o subjektu - pretpostavili smo da je subjekt \textit{ja} i muškog roda. Ovakve su pretpostavke u japanskom svakodnevne - bitno je znati tko izriče rečenicu.
		\item みる prevodimo i kao \textit{vidjeti} i kao \textit{pogledati}. Za većinu glagola ne postoji prijevod "jedan za jedan" - situacije u kojima se koriste različite su u odnosu na hrvatski pa nije dovoljno samo naučiti riječ već i kontekst u kojem se prirodno koristi.
		\item Prijevod glagola je po vidu svršen.\footnotemark[1] Ovo je pravilnost za sve glagole u japanskom osim glagola stanja za koje razlikovanje po vidu nema smisla (npr. \textit{postojim} - \textit{postojavam}???). Zasad ćemo ovo samo primijetiti i zaboraviti jer se radi o težoj temi.
		\item Neprošlost je protumačena kao budućnost. Osim za glagole stanja, ovo je gotovo uvijek tako.
	\end{hyou}
	
	\footnotetext[1]{Vid (svršenost) glagola je oblik koji ukazuje na to je li radnja u tijeku ili završena. Neki primjeri: \textit{gledati} - \textit{pogledati}, \textit{jesti} - \textit{pojesti}, \textit{stajati} - \textit{zastati}.}
	
	Naučimo neke korisne glagole:
	
	\vspace{10pt}
	\begin{tabular}{l l l l l l}
		(を) みる & \textit{vidjeti} & (が) みえる & \textit{vidjeti se} & (が) いる & \textit{biti} (za živa bića)\\
		(を) でる & \textit{izići} & (を) あける & \textit{otvoriti} & (を) しめる & \textit{zatvoriti}\\
		(が) ねる & \textit{leći, spavati} & (が) おきる & \textit{ustati} & あげる & \textit{dati}, \textit{podići} i još 25 značenja\\
	\end{tabular}

	\newpage
	\fukudai{Čestica direktnog objekta を}
	
	Iako se čita kao お, iz povijesnih i praktičnih razlika, piše se znakom を čiji se izgovor u modernom japanskom ne pojavljuje. Kao i ostale čestice, dolazi na kraju imenica ili imeničkih izraza, a označava objekt glagola. U većini slučajeva odgovara akuzativu u hrvatskom jeziku. Jedna lako uočiva razlika je da se koristi za sredstvo po kojem ili kroz koje se vrši kretnja (npr. いえを でる - \textit{izići iz kuće}, dosl. \textit{izići kuću}).
	
	\fukudai{Primjeri}
	
	\begin{reibun}
		\rei{りんごを たべる。}{Pojesti jabuku. / Pojest ću jabuku.}
		\rei{とりを みた。}{Vidio sam pticu.}
		\rei{ほんを あげない。}{Ne dati knjigu. / Neću dati knjigu.}
		\rei{いえを でなかった。}{Nisam izišao iz kuće.}
		\rei{すずき\footnotemark[2]さんは まどを あけた。}{G. Suzuki je otvorila prozor.}
	\end{reibun}

	\fukudai{Vježba} - prevedite sljedeće rečenice na hrvatski:
	
	\begin{mondai}{Lv. 1}
		\item みた。
		\item たべない。
		\item あげる。
		\item あけなかった。
	\end{mondai}
	
	\vspace{-10pt}
	\begin{mondai}{Lv. 2}
		\item ねこを みた。
		\item たけし\footnotemark[3]くんは たべない。
		\item りんごを あげる。
		\item \underline{はこ}を あけなかった。
	\end{mondai}

	\vspace{-10pt}
	\begin{mondai}{Lv. 3}
		\item こどもたちは ねこを みた。
		\item たけしくんは にんじんを たべない。
		\item りんごと みかんを あげる。
		\item かれは はこを あけなかった。
	\end{mondai}

	\vspace{-10pt}
	\begin{mondai}{Lv. 4}
		\item ちいさい こどもたちは くろい ねこを みた。
		\item たけしくんは きらいな にんじんを たべない。
		\item あかい りんごと おいしい みかんを あげる。
		\item かれは あの ふるい はこを あけなかった。
	\end{mondai}
	
	\footnotetext[2]{すずき je prezime i ne govori nam radi li se o muškoj ili ženskoj osobi.}
	\footnotetext[3]{たけし je muško ime pa zbog načina na koji je oslovljen znamo da se radi od dječaku.}
\end{document}