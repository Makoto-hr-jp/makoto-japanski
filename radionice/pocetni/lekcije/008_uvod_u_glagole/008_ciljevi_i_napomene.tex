% !TeX document-id = {8ddcb487-7e97-4de5-83ad-9d29d9b4393e}
% !TeX program = xelatex ?me -synctex=0 -interaction=nonstopmode -aux-directory=../tex_aux -output-directory=./release
% !TeX program = xelatex

\documentclass[12pt]{article}

\usepackage{lineno,changepage,lipsum}
\usepackage[colorlinks=true,urlcolor=blue]{hyperref}
\usepackage{fontspec}
\usepackage{xeCJK}
\usepackage{tabularx}
\setCJKfamilyfont{chanto}{AozoraMinchoRegular.ttf}
\setCJKfamilyfont{tegaki}{Mushin.otf}
\usepackage[CJK,overlap]{ruby}
\usepackage{hhline}
\usepackage{multirow,array,amssymb}
\usepackage[croatian]{babel}
\usepackage{soul}
\usepackage[usenames, dvipsnames]{color}
\usepackage{wrapfig,booktabs}
\renewcommand{\rubysep}{0.1ex}
\renewcommand{\rubysize}{0.75}
\usepackage[margin=50pt]{geometry}
\modulolinenumbers[2]

\usepackage{pifont}
\newcommand{\cmark}{\ding{51}}%
\newcommand{\xmark}{\ding{55}}%

\definecolor{faded}{RGB}{100, 100, 100}

\renewcommand{\arraystretch}{1.2}

%\ruby{}{}
%$($\href{URL}{text}$)$

\newcommand{\furigana}[2]{\ruby{#1}{#2}}
\newcommand{\tegaki}[1]{
	\CJKfamily{tegaki}\CJKnospace
	#1
	\CJKfamily{chanto}\CJKnospace
}

\newcommand{\dai}[1]{
	\vspace{20pt}
	\large
	\noindent\textbf{#1}
	\normalsize
	\vspace{20pt}
}

\newcommand{\fukudai}[1]{
	\vspace{10pt}
	\noindent\textbf{#1}
	\vspace{10pt}
}

\newenvironment{bunshou}{
	\vspace{10pt}
	\begin{adjustwidth}{1cm}{3cm}
	\begin{linenumbers}
}{
	\end{linenumbers}
	\end{adjustwidth}
}

\newenvironment{reibun}{
	\vspace{10pt}
	\begin{tabular}{l l}
}{
	\end{tabular}
	\vspace{10pt}
}
\newcommand{\rei}[2]{
	#1&\textit{#2}\\
}
\newcommand{\reinagai}[2]{
	\multicolumn{2}{l}{#1}\\
	\multicolumn{2}{l}{\hspace{10pt}\textit{#2}}\\
}

\newenvironment{mondai}[1]{
	\vspace{10pt}
	#1
	
	\begin{enumerate}
		\itemsep-5pt
	}{
	\end{enumerate}
	\vspace{10pt}
}

\newenvironment{hyou}{
	\begin{itemize}
		\itemsep-5pt
	}{
	\end{itemize}
	\vspace{10pt}
}

\date{\today}

\CJKfamily{chanto}\CJKnospace
\author{Tomislav Mamić}
\begin{document}
	\dai{Ciljevi i napomene - uvod u glagole}
	
	\fukudai{Ciljevi}
	
	\vspace{-10pt}
	\begin{hyou}
		\item uloga glagola kao predikata
		\item informacije sadržane u glagolu
		\vspace{-5pt}
		\begin{hyou}
			\item vrijeme
			\item pristojnost (društveni odnosi govornika)
			\item raspoloženje (govornikov stav prema temi)
		\end{hyou}
		\vspace{-10pt}
		\item informacije kojih u japanskim glagolima \textbf{nema}
		\vspace{-5pt}
		\begin{hyou}
			\item lice
			\item množina
		\end{hyou}
		\vspace{-10pt}
		\item skupine glagola
		\item fokus na kolokvijalne glagole i zašto
		\item 4 osnovna oblika za 一段
		\item čestica を i prijelaznost glagola
	\end{hyou}

	\fukudai{Napomene}

	Iz praktičnih razloga, čestica を se piše znakom を, iako se čita kao お. Danas većina govornika japanskog uopće ne razlikuje zvukove を i お, izuzev ponekog dijalekta, glas を je nestao iz modernog japanskog jezika. Razlog takvom pisanju je dodavanje glasa お kao izraza poštovanja prema imenicama. U situacijama gdje takvoj imenici prethodi imenica označena česticom を, bilo bi zbunjujuće odrediti ulogu znaka お, npr それを父さんに任せる ili それ、お父さんに任せる. Pišući お umjesto を, gubi se suptilna razlika u odnosu govornika prema ocu. Pjevačima i glumcima preporučuje se izgovor glasa を, ponekad čak i umjesto お radi naglašavanja riječi, slično kao što im se preporučuje da ん uvijek izgovaraju kao \textit{m}.
	
	U hrvatskom jeziku postoje tri skupine glagola po prijelaznosti - prijelazni, neprijelazni i povratni. Ovu treću skupinu karakterizira korištenje povratne zamjenice \textit{se} kao objekta. Tehnički, svi povratni glagoli su zapravo prijelazni (npr. \textit{oprati se}/\textit{oprati nešto}), ali se dobar dio njih koristi isključivo kao povratni glagol (npr. \textit{spotaknuti se}). Ovo je zapravo vrlo elegantno rješenje u odnosu na japanski.
	
	U japanskom jeziku, glagoli mogu biti samo prijelazni ili neprijelazni, a značenje povratnih glagola nose neprijelazne varijante prijelaznih glagola. To dodaje na složenosti nekih glagola jer moramo zapamtiti parove koji nisu uvijek morfološki pravilni (npr. (を)見る - (が)見える vs. (を)変える - (が)変わる).

\end{document}