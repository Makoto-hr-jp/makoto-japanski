% !TeX document-id = {003fd66c-0c18-4281-97ec-15e213c5c40d}
% !TeX program = xelatex ?me -synctex=0 -interaction=nonstopmode -aux-directory=../tex_aux -output-directory=./release
% !TeX program = xelatex

\documentclass[12pt]{article}

\usepackage{lineno,changepage,lipsum}
\usepackage[colorlinks=true,urlcolor=blue]{hyperref}
\usepackage{fontspec}
\usepackage{xeCJK}
\usepackage{tabularx}
\setCJKfamilyfont{chanto}{AozoraMinchoRegular.ttf}
\setCJKfamilyfont{tegaki}{Mushin.otf}
\usepackage[CJK,overlap]{ruby}
\usepackage{hhline}
\usepackage{multirow,array,amssymb}
\usepackage[croatian]{babel}
\usepackage{soul}
\usepackage[usenames, dvipsnames]{color}
\usepackage{wrapfig,booktabs}
\renewcommand{\rubysep}{0.1ex}
\renewcommand{\rubysize}{0.75}
\usepackage[margin=50pt]{geometry}
\modulolinenumbers[2]

\usepackage{pifont}
\newcommand{\cmark}{\ding{51}}%
\newcommand{\xmark}{\ding{55}}%

\definecolor{faded}{RGB}{100, 100, 100}

\renewcommand{\arraystretch}{1.2}

%\ruby{}{}
%$($\href{URL}{text}$)$

\newcommand{\furigana}[2]{\ruby{#1}{#2}}
\newcommand{\tegaki}[1]{
	\CJKfamily{tegaki}\CJKnospace
	#1
	\CJKfamily{chanto}\CJKnospace
}

\newcommand{\dai}[1]{
	\vspace{20pt}
	\large
	\noindent\textbf{#1}
	\normalsize
	\vspace{20pt}
}

\newcommand{\fukudai}[1]{
	\vspace{10pt}
	\noindent\textbf{#1}
	\vspace{10pt}
}

\newenvironment{bunshou}{
	\vspace{10pt}
	\begin{adjustwidth}{1cm}{3cm}
	\begin{linenumbers}
}{
	\end{linenumbers}
	\end{adjustwidth}
}

\newenvironment{reibun}{
	\vspace{10pt}
	\begin{tabular}{l l}
}{
	\end{tabular}
	\vspace{10pt}
}
\newcommand{\rei}[2]{
	#1&\textit{#2}\\
}
\newcommand{\reinagai}[2]{
	\multicolumn{2}{l}{#1}\\
	\multicolumn{2}{l}{\hspace{10pt}\textit{#2}}\\
}

\newenvironment{mondai}[1]{
	\vspace{10pt}
	#1
	
	\begin{enumerate}
		\itemsep-5pt
	}{
	\end{enumerate}
	\vspace{10pt}
}

\newenvironment{hyou}{
	\begin{itemize}
		\itemsep-5pt
	}{
	\end{itemize}
	\vspace{10pt}
}

\date{\today}

\CJKfamily{chanto}\CJKnospace
\author{Tomislav Mamić}
\begin{document}
	\dai{Ciljevi i napomene - pravi prilozi}
	
	\fukudai{Ciljevi}
	
	\begin{hyou}
		\item pravi prilozi i njihovo mjesto u rečenici
		\item jako naglasiti da se prilozi odnose na predikat, a pridjevi na imenice i im. fraze
		\item tvorba priloga od pridjeva i veza između njih
		\begin{hyou}
			\item u hrvatskom se prilozi od pridjeva dobivaju stavljanjem pridjeva u 3. lice jednine srednjeg roda pa ih je teže razlikovati - koristimo se pitanjima \textit{kako?} i \textit{kakvo?}
			\item u japanskom je to puno lakše jer je pretvorba eksplicitna
		\end{hyou}
		\item s obzirom da je ovo gramatički lagana lekcija, treba dodati više vokabulara
		\item ubaciti i kanji 小大白子女男父母
	\end{hyou}

	\fukudai{Napomene}
	
	\begin{hyou}
		\item S obzirom da će se na satu koristiti pridjevi, ubaciti 小大白 u listić je zgodno, ali 子女男父母 vjerojatno bolje pašu u DZ.
		\item Nakon 3 tjedna glagola, ovaj sat bi trebao biti opuštajuć.
	\end{hyou}
\end{document}