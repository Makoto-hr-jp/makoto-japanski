% !TeX document-id = {50324916-363d-47a9-b81f-5167dc67e8aa}
% !TeX program = xelatex ?me -synctex=0 -interaction=nonstopmode -aux-directory=../tex_aux -output-directory=./release
% !TeX program = xelatex

\documentclass[12pt]{article}

\usepackage{lineno,changepage,lipsum}
\usepackage[colorlinks=true,urlcolor=blue]{hyperref}
\usepackage{fontspec}
\usepackage{xeCJK}
\usepackage{tabularx}
\setCJKfamilyfont{chanto}{AozoraMinchoRegular.ttf}
\setCJKfamilyfont{tegaki}{Mushin.otf}
\usepackage[CJK,overlap]{ruby}
\usepackage{hhline}
\usepackage{multirow,array,amssymb}
\usepackage[croatian]{babel}
\usepackage{soul}
\usepackage[usenames, dvipsnames]{color}
\usepackage{wrapfig,booktabs}
\renewcommand{\rubysep}{0.1ex}
\renewcommand{\rubysize}{0.75}
\usepackage[margin=50pt]{geometry}
\modulolinenumbers[2]

\usepackage{pifont}
\newcommand{\cmark}{\ding{51}}%
\newcommand{\xmark}{\ding{55}}%

\definecolor{faded}{RGB}{100, 100, 100}

\renewcommand{\arraystretch}{1.2}

%\ruby{}{}
%$($\href{URL}{text}$)$

\newcommand{\furigana}[2]{\ruby{#1}{#2}}
\newcommand{\tegaki}[1]{
	\CJKfamily{tegaki}\CJKnospace
	#1
	\CJKfamily{chanto}\CJKnospace
}

\newcommand{\dai}[1]{
	\vspace{20pt}
	\large
	\noindent\textbf{#1}
	\normalsize
	\vspace{20pt}
}

\newcommand{\fukudai}[1]{
	\vspace{10pt}
	\noindent\textbf{#1}
	\vspace{10pt}
}

\newenvironment{bunshou}{
	\vspace{10pt}
	\begin{adjustwidth}{1cm}{3cm}
	\begin{linenumbers}
}{
	\end{linenumbers}
	\end{adjustwidth}
}

\newenvironment{reibun}{
	\vspace{10pt}
	\begin{tabular}{l l}
}{
	\end{tabular}
	\vspace{10pt}
}
\newcommand{\rei}[2]{
	#1&\textit{#2}\\
}
\newcommand{\reinagai}[2]{
	\multicolumn{2}{l}{#1}\\
	\multicolumn{2}{l}{\hspace{10pt}\textit{#2}}\\
}

\newenvironment{mondai}[1]{
	\vspace{10pt}
	#1
	
	\begin{enumerate}
		\itemsep-5pt
	}{
	\end{enumerate}
	\vspace{10pt}
}

\newenvironment{hyou}{
	\begin{itemize}
		\itemsep-5pt
	}{
	\end{itemize}
	\vspace{10pt}
}

\date{\today}

\CJKfamily{chanto}\CJKnospace
\author{Tomislav Mamić}
\begin{document}
	\dai{Prilozi}
	
	Analogno pridjevima koji opisuju \textbf{kakva} je imenica, prilozi su riječi koje opisuju \textbf{kako} se odvija radnja predikata. Međutim, za razliku od pridjeva koji odgovaraju samo na pitanja o tome kakva je imenica, prilozi o radnji mogu reći nešto više. Ugrubo ih možemo podijeliti na vremenske (\textit{kada}?), mjesne (\textit{gdje}?), načinske (\textit{kako}?) i količinske (\textit{koliko}?).
	
	U japanskom osim priloga postoji i posebna kategorija imenica (priložne imenice) koje mogu obavljati sličnu funkciju. O priložnim imenicama ćemo naučiti sljedećih tjedana.
	
	\fukudai{Pravi prilozi u japanskom}
	
	S obzirom da su povezani direktno s predikatom, mjesto priloga u rečenici nije pretjerano bitno. Običaj ih je staviti ili na početak rečenice ili odmah ispred predikata.
	
	\vspace{10pt}
	\begin{tabular}{l l l l}
		もう&\textit{već} (uz poz. glagol)&まだ&\textit{još uvijek}\\
		たくさん&\textit{puno}&すこし&\textit{malo}\\
		ちょっと&\textit{malo}, \textit{nakratko}&また&\textit{opet}, \textit{ponovno}\\
		ゆっくり&\textit{polako}, \textit{opušteno}&とても&\textit{jako}, \textit{potpuno}, \textit{skroz}\\
	\end{tabular}

	\fukudai{Prilozi od pridjeva}
	
	U hrvatskom, pridjev u 3. licu jednine srednjeg roda možemo koristiti kao prilog. U japanskom, pridjevi se u priloge pretvaraju vrlo jednostavnom zamjenom nastavka.
	
	\begin{table}[h]
		\centering
		\begin{tabular}{l l}\toprule[2pt]
			pridjev&prilog\\
			\midrule
			い&く\\
			いい\footnotemark[1]&よく\\
			な&に\\
			の&に\\
			\bottomrule[2pt]
		\end{tabular}
	\end{table}

	\footnotetext[1]{Sjetimo se da je いい nepravilan pridjev.}

	Pogledajmo neke primjere:
	
	\begin{reibun}
		\rei{これは いい ところだ。}{Ovo je dobro mjesto. \rem{(\textit{kakvo mjesto?})}}
		\rei{よく たべた。}{Dobro sam se najeo. \rem{(\textit{kako sam se najeo?})}}
		\rei{たけしくんの へやは きれいだ。}{Takešijeva soba je čista. \rem{(\textit{kakva soba?})}}
		\rei{たけしくんは へやを きれいに そうじした。}{Takeši je lijepo očistio sobu. \rem{(\textit{kako je očistio?})}}
	\end{reibun}

	\fukudai{Prilog + する}
	
	Kad koristimo prilog s glagolom する, značenje koje dobivamo je \textit{učiniti nešto nekakvim}, \textit{učiniti da bude nekako}. U hrvatskom za ovo značenje vrlo često postoji glagol koji ga spretnije izražava.
	
	\begin{reibun}
		\rei{きれいにする。}{Uljepšati. \dosl Učiniti lijepim.}
		\rei{にわを \furigana{大}{おお}きく した。}{Povećao sam vrt. \dosl Učinio sam vrt većim.}
	\end{reibun}

	\fukudai{Prilog + なる}

	Osnovno značenje glagola なる je \textit{postati}. U kombinaciji s prilogom, dobivamo značenje \textit{postati nekakvo}. U hrvatskom, glagoli koje bismo koristili za prijevod kombinacije sa する moći će se koristiti i ovdje, ali će biti povratni (objekt će im biti zamjenica \textit{se}). Ovaj ćemo izraz koristiti kad želimo da je nešto samo od sebe postalo nekakvo.

	\begin{reibun}
		\rei{きれいに なる。}{Uljepšati se. \dosl Postati lijep.}
		\rei{にわは 大きく なった。}{Vrt se povećao. \dosl Vrt je postao veći.}
	\end{reibun}

	\fukudai{Vježba}

	\begin{mondai}{Prevedite na hrvatski:}
		\item たけしくんは まだ いえに いる。
		\item すずきさんは もう いえを でた。
		\item \underline{いるか}は さかなを たくさん たべる。
		\item にほんごが すこし \underline{わかる}。
		\item また くる。
		\item ゆっくり \underline{え}を みる。
		\item かのじょは とても きれいな ひとだ。
		\vspace{5pt}
		\item さつきちゃんの けしゴムは \furigana{小}{ちい}さく なった。
		\item へやの かべを \furigana{白}{しろ}く ぬった。
		\vspace{10pt}
		\item 日は みじかく なった。
		\item やまは きれいに なる。
		\item てんきは さむく なった。
		\item まつの 木が たかく なる。
		\item \underline{となり}の パンやさん\footnotemark[2]は パンを やすく した。
		\item さつきちゃんは かみを みじかく しない。
		\item すずきさんは おとうとの コーヒーを あまく した。
		\item こどもたちは \underline{ぜんぜん} しずかに しなかった。
		\item *かみを みじかく した さつきちゃんは かわいかった。
	\end{mondai}

	\footnotetext[2]{Često se dućan ili obrt koji se nečim bavi zove kao i ono čime se bavi s nastavkom や, npr. ほんや, パンや, ラーメンや. Radnike je onda običaj zvati prema mjestu gdje rade s nastavkom さん, npr. ほんやさん, ぱんやさん, ラーメンやさん itd.}
\end{document}