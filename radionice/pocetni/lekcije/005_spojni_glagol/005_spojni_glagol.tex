% !TeX document-id = {4c6afbdb-e428-439d-8137-c9df7aed7793}
% !TeX program = xelatex ?me -synctex=0 -interaction=nonstopmode -aux-directory=../tex_aux -output-directory=./release
% !TeX program = xelatex

\documentclass[12pt]{article}

\usepackage{lineno,changepage,lipsum}
\usepackage[colorlinks=true,urlcolor=blue]{hyperref}
\usepackage{fontspec}[ Path =../../../ ]
\usepackage{xeCJK}
\usepackage{tabularx}
\usepackage{graphicx}
\setCJKfamilyfont{chanto}{AOZORAMINCHOREGULAR_0.TTF}%
\setCJKfamilyfont{tegaki}{Mushin.otf}%
\usepackage[CJK,overlap]{ruby}
\usepackage{hhline}
\usepackage{multirow,array,amssymb}
\usepackage[croatian]{babel}
\usepackage{soul}
\usepackage[usenames, dvipsnames]{color}
\usepackage{wrapfig,booktabs}
\usepackage{calc}
\renewcommand{\rubysep}{0.1ex}
\renewcommand{\rubysize}{0.75}
\usepackage[margin=50pt]{geometry}
\usepackage{hyperref}
\modulolinenumbers[2]

\date{\today}

\usepackage{fancyhdr}
\pagestyle{fancy}
\fancyhf{}
\fancyhead[LE,RO]{\thepage}
\makeatletter
\fancyhead[RE,LO]{rev. \@date 誠}
\makeatother

\usepackage{pifont}
\newcommand{\cmark}{\ding{51}}%
\newcommand{\xmark}{\ding{55}}%

\newcommand{\dosl}{{\normalfont dosl. }}%
\newcommand{\rem}[1]{{\normalfont #1 }}%

\definecolor{faded}{RGB}{100, 100, 100}

\renewcommand{\arraystretch}{1.2}

%\ruby{}{}
%$($\href{URL}{text}$)$

\newcommand{\furigana}[2]{\ruby{#1}{#2}}
\newcommand{\tegaki}[1]{
	\CJKfamily{tegaki}\CJKnospace
	#1
	\CJKfamily{chanto}\CJKnospace
}

\newcommand{\dai}[1]{
	\vspace{20pt}
	\large
	\noindent\textbf{#1}
	\normalsize
	\vspace{20pt}
}

\newcommand{\fukudai}[1]{
	\vspace{10pt}
	\noindent\textbf{#1}
	\vspace{10pt}
}

\newenvironment{bunshou}{
	\vspace{10pt}
	\begin{adjustwidth}{1cm}{3cm}
	\begin{linenumbers}
}{
	\end{linenumbers}
	\end{adjustwidth}
}

\newenvironment{reibun}[1][]{
	\vspace{10pt}
	#1
	
	\begin{tabular}{l l}
}{
	\end{tabular}
	\vspace{10pt}
}
\newcommand{\rei}[2]{
	#1&\textit{#2}\\
}
\newcommand{\reinagai}[2]{
	\multicolumn{2}{l}{#1}\\
	\multicolumn{2}{l}{\hspace{10pt}\textit{#2}}\\
}

\newenvironment{mondai}[1]{
	\vspace{10pt}
	\noindent #1
	
	\begin{enumerate}
		\itemsep-5pt
	}{
	\end{enumerate}
}

\newenvironment{hyou}{
	\begin{itemize}
		\itemsep-5pt
	}{
	\end{itemize}
	\vspace{10pt}
}

\newcommand{\juuyou}[2][20pt]{
	\vspace{5pt}
		\noindent\hspace{#1}\parbox[c]{\textwidth-#1-#1}{\centering\textit{#2}}
	\vspace{5pt}
}

\newcommand{\ten}{
	\vspace{5pt}
	\noindent\hspace{-10pt}$\bullet$
}

\CJKfamily{chanto}\CJKnospace

\frenchspacing
\author{Tomislav Mamić}
\begin{document}
	\dai{Spojni glagol}
	
	\fukudai{Teorija}

	Spojni glagol nam omogućuje da izrazimo identitet (npr. \textit{Ono \underline{je} mačka.}), opis (npr. \textit{Nebo \underline{je bilo} plavo.}) ili pripadnost skupini (npr. \textit{Rajčice \underline{su} zapravo voće!}). U japanskom spojni glagol obavlja iste zadaće, ali se, za razliku od hrvatskog glagola \textit{biti}, koristi \textbf{samo kao spojni glagol} dok \textit{biti} ima i drugo značenje - \textit{postojati, biti prisutan}. Zbog toga ćemo morati paziti kad je glagol \textit{biti} spojni glagol, a kad ima svoje pravo značenje.
	
	U jeziku općenito, spojni glagol je onaj glagol koji omogućuje stvaranje imenskih predikata. U matematici, to je znak $=$ ili $\subseteq$. U hrvatskom jeziku, to je glagol \textit{biti}.
	
	\fukudai{Prošlost, neprošlost i negacija}
	
	Za razliku od hrvatskog i engleskog, u japanskom postoje samo dva glagolska vremena - prošlo i neprošlo. Budućnost se izražava kontekstom rečenice, kao recimo u hrv. \textit{sutra idem u Japan} - glagol \textit{idem} je zapravo u prezentu, ali zbog riječi \textit{sutra} to shvaćamo kao plan za sutra - budućnost.
	
	Kao i u hrvatskom, glagole možemo negirati. Dok u hrv. za to koristimo pomoćne riječi (npr. \textit{\underline{ne} vidim, \underline{nisam} jeo}...), u jap. je negacija zapravo nastavak glagola, kao i vrijeme. Osim ova dva, postoji još puno raznih glagolskih oblika koji mijenjaju nastavak glagola, i svi se oni nižu određenim redoslijedom.
	
	\fukudai{Kolokvijalni oblik}
	
	\begin{tabular}{|l|r|r|}
		\hline
		&neprošlost&prošlost\\
		\hline
		potvrdno&だ&だった\\
		\hline
		negirano&じゃ\footnotemark[1]ない&じゃ\footnotemark[1]なかった\\
		\hline
	\end{tabular}

	\footnotetext[1]{じゃ je skraćeno od では, a は je zapravo čestica teme. Svo ovo skraćivanje i spajanje dogodilo se davno i danas nije podložno gramatičkim promjenama.}
	
	\fukudai{Pristojni oblik}
	
	ToDo
	
	\fukudai{Tema rečenice - čestica は}
	
	U hrvatskom jeziku tema je nešto što saznajemo iz zadržaja teksta (npr. koja je tema knjige \textit{Pale sam na svijetu}?). U japanskom, tema je gramatički pojam - dio rečenice koji kaže o čemu ta rečenica (a možda i one poslije) govori. Tema se u rečenici označava česticom koja se piše kao は(ha), ali izgovara kao わ(wa)\footnotemark[2].
	
	\footnotetext[2]{Uvedeno reformom jezika poslije WW2 radi lakšeg čitanja. Kad bi se pisalo わ, u nekim bi se situacijama imenica i čestica mogle dvoznačno shvatiti kao neki nastavci glagola.}
	
	Kao što ćemo vidjeti, tema je često (ali ne uvijek) i subjekt japanske rečenice, ali između teme i subjekta postoje velike razlike. Znajući kako reći koja je tema rečenice, možemo reći i svoje prve potpune rečenice na japanskom!
	
	\begin{reibun}
		\rei{あれは ねこ だ。}{Ono je mačka.}
		\rei{それは わたしの かばん だった。}{To je bila moja torba.}
		\rei{これは りんご じゃない。}{Ovo nije jabuka}
		\rei{あの ひとの なまえは たけし じゃなかった。}{Onaj čovjek se nije zvao Takeši.} (dosl. \textit{Ime onog čovjeka nije bilo Takeši.})
	\end{reibun}

	\fukudai{Subjekt - čestica が}
	
	ToDo
	
	\fukudai{Nadovezivanje na kontekst - čestica も}
	
	ToDo
	
\end{document}