% !TeX document-id = {4c6afbdb-e428-439d-8137-c9df7aed7793}
% !TeX program = xelatex ?me -synctex=0 -interaction=nonstopmode -aux-directory=../tex_aux -output-directory=./release
% !TeX program = xelatex

\documentclass[12pt]{article}

\usepackage{lineno,changepage,lipsum}
\usepackage[colorlinks=true,urlcolor=blue]{hyperref}
\usepackage{fontspec}
\usepackage{xeCJK}
\usepackage{tabularx}
\setCJKfamilyfont{chanto}{AozoraMinchoRegular.ttf}
\setCJKfamilyfont{tegaki}{Mushin.otf}
\usepackage[CJK,overlap]{ruby}
\usepackage{hhline}
\usepackage{multirow,array,amssymb}
\usepackage[croatian]{babel}
\usepackage{soul}
\usepackage[usenames, dvipsnames]{color}
\usepackage{wrapfig,booktabs}
\renewcommand{\rubysep}{0.1ex}
\renewcommand{\rubysize}{0.75}
\usepackage[margin=50pt]{geometry}
\modulolinenumbers[2]

\usepackage{pifont}
\newcommand{\cmark}{\ding{51}}%
\newcommand{\xmark}{\ding{55}}%

\definecolor{faded}{RGB}{100, 100, 100}

\renewcommand{\arraystretch}{1.2}

%\ruby{}{}
%$($\href{URL}{text}$)$

\newcommand{\furigana}[2]{\ruby{#1}{#2}}
\newcommand{\tegaki}[1]{
	\CJKfamily{tegaki}\CJKnospace
	#1
	\CJKfamily{chanto}\CJKnospace
}

\newcommand{\dai}[1]{
	\vspace{20pt}
	\large
	\noindent\textbf{#1}
	\normalsize
	\vspace{20pt}
}

\newcommand{\fukudai}[1]{
	\vspace{10pt}
	\noindent\textbf{#1}
	\vspace{10pt}
}

\newenvironment{bunshou}{
	\vspace{10pt}
	\begin{adjustwidth}{1cm}{3cm}
	\begin{linenumbers}
}{
	\end{linenumbers}
	\end{adjustwidth}
}

\newenvironment{reibun}{
	\vspace{10pt}
	\begin{tabular}{l l}
}{
	\end{tabular}
	\vspace{10pt}
}
\newcommand{\rei}[2]{
	#1&\textit{#2}\\
}
\newcommand{\reinagai}[2]{
	\multicolumn{2}{l}{#1}\\
	\multicolumn{2}{l}{\hspace{10pt}\textit{#2}}\\
}

\newenvironment{mondai}[1]{
	\vspace{10pt}
	#1
	
	\begin{enumerate}
		\itemsep-5pt
	}{
	\end{enumerate}
	\vspace{10pt}
}

\newenvironment{hyou}{
	\begin{itemize}
		\itemsep-5pt
	}{
	\end{itemize}
	\vspace{10pt}
}

\date{\today}

\CJKfamily{chanto}\CJKnospace
\author{Tomislav Mamić}
\begin{document}
	\dai{Spojni glagol}
	
	\fukudai{Teorija}

	Spojni glagol nam omogućuje da izrazimo identitet (npr. \textit{Ono \underline{je} mačka.}), opis (npr. \textit{Nebo \underline{je bilo} plavo.}) ili pripadnost skupini (npr. \textit{Rajčice \underline{su} zapravo voće!}). U japanskom spojni glagol obavlja iste zadaće, ali se, za razliku od hrvatskog glagola \textit{biti}, koristi \textbf{samo kao spojni glagol} dok \textit{biti} ima i drugo značenje - \textit{postojati, biti prisutan}. Zbog toga ćemo morati paziti kad je glagol \textit{biti} spojni glagol, a kad ima svoje pravo značenje.
	
	\fukudai{Prošlost, neprošlost i negacija}
	
	Za razliku od hrvatskog i engleskog, u japanskom postoje samo dva glagolska vremena - prošlo i neprošlo. Budućnost se izražava kontekstom rečenice, kao recimo u hrv. \textit{sutra idem u Japan} - glagol \textit{idem} je zapravo u prezentu, ali zbog riječi \textit{sutra} to shvaćamo kao plan za sutra - budućnost.
	
	Kao i u hrvatskom, glagole možemo negirati. Dok u hrv. za to koristimo pomoćne riječi (npr. \textit{\underline{ne} vidim, \underline{nisam} jeo}...), u jap. je negacija zapravo nastavak glagola, kao i vrijeme. Osim ova dva, postoji još puno raznih glagolskih oblika koji mijenjaju nastavak glagola, i svi se oni nižu određenim redoslijedom.
	
	\fukudai{Kolokvijalni i pristojni oblik}
	
	\begin{tabular}{|l|r|r|c|r|r|}
		\cline{1-3}\cline{5-6}
		&neprošlost&prošlost&&neprošlost&prošlost\\
		\cline{1-3}\cline{5-6}
		potvrdno&だ&だった&&です&でした\\
		\cline{1-3}\cline{5-6}
		negirano&じゃ\footnotemark[1]ない&じゃ\footnotemark[1]なかった&&では\footnotemark[1]ありません&では\footnotemark[1]ありませんでした\\
		\cline{1-3}\cline{5-6}
	\end{tabular}

	\footnotetext[1]{じゃ je skraćeno od では, a は je zapravo čestica teme. Svo ovo skraćivanje i spajanje dogodilo se davno i danas nije podložno gramatičkim promjenama.}
	
	\fukudai{Pristojnost}
	
	U japanskoj kulturi i jeziku, pristojnost je puno bitnija nego kod nas. Svi predikati (glagoli, pridjevi, im.) imaju svoje pristojne oblike. S prijateljima i obitelji razgovarat ćemo kolokvijalno, ali sa strancima i nadređenima uvijek ćemo koristiti pristojne oblike.
	
	Uočit ćemo da su kolokvijalni oblici uvijek kraći od pristojnih - što nam dulje treba da nešto kažemo, to pristojnije.
	
	\fukudai{Tema rečenice - čestica は}
	
	U hrvatskom jeziku tema je nešto što saznajemo iz sadržaja teksta ili razgovora. U japanskom, tema je gramatički pojam - dio rečenice koji kaže o čemu ta rečenica (a možda i one poslije) govori. Tema se u rečenici označava česticom koja se piše kao は(ha), ali izgovara kao わ(wa)\footnotemark[2].
	
	\footnotetext[2]{Uvedeno reformom jezika poslije WW2 radi lakšeg čitanja. Kad bi se pisalo わ, u nekim bi se situacijama imenica i čestica mogle dvoznačno shvatiti kao neki nastavci glagola.}
	
	Kao što ćemo vidjeti, tema je često (ali ne uvijek) i subjekt japanske rečenice, ali između teme i subjekta postoje velike razlike. Znajući kako reći temu, možemo reći i svoje prve potpune rečenice na japanskom!
	
	\begin{reibun}
		\rei{あれは ねこ だ。}{Ono je mačka.}
		\rei{それは わたしの かばん だった。}{To je bila moja torba.}
		\rei{これは りんご じゃない。}{Ovo nije jabuka}
		\rei{あの ひとの なまえは たけし じゃなかった。}{Onaj čovjek se nije zvao Takeši.} (dosl. \textit{Ime onog čovjeka nije bilo Takeši.})
	\end{reibun}

	\begin{mondai}{Pokušajmo sljedeće rečenice prevesti na japanski:}
		\item \textit{Ono je moja kuća.}
		\item \textit{To nije cvijet.}
		\item \textit{Ime ove rijeke je Sava.}
	\end{mondai}

	Tema rečenice nije ograničena samo na jednu rečenicu. Ako u sljedećim rečenicama ne promijenimo temu, sugovornik će pretpostaviti da nastavljamo pričati o istoj. Također, ako uopće ne kažemo temu, vrlo je često pretpostaviti da smo mi (わたし) tema.

	\fukudai{Subjekt - čestica が}
	
	Za razliku od teme, subjekt funkcionira slično kao u hrvatskom - govori točno tko vrši radnju u rečenici. Dok čestica は stavlja naglasak na ono što ćemo reći poslije, čestica が naglašava subjekt kao bitan:
	
	\begin{reibun}
		\rei{わたしは ねこ だ。}{Ja sam mačka.}
		\rei{わたしが ねこ だ。}{\textbf{Ja} sam mačka.}
	\end{reibun}
	
	\fukudai{Nadovezivanje na kontekst - čestica も}
	
	Osnovna upotreba ove čestice je izbjegavanje ponavljanja nečeg što smo već rekli. Iako se može upotrijebiti u različitim ulogama u rečenici, zasad ćemo je koristiti kad zamjenjuje temu ili subjekt:
	
	\begin{reibun}
		\rei{あれは いぬ です。}{Ono je pas.}
		\rei{あれも (いぬ です)。}{I ono isto.}
		\rei{すいかは くだもの だ。}{Lubenice su voće.}
		\rei{りんごも (くだもの だ)。}{I jabuke isto.}
	\end{reibun}

	Kad želimo za više tema reći istu stvar, možemo ih nabrojati česticom も:
	
	\begin{reibun}
		\rei{りんごも すいかも くだもの です。}{I jabuke i lubenice su voće.}
		\rei{わたしも わたしの ともだちも こうこうせい です。}{I ja i moj prijatelj smo srednjoškolci.}
	\end{reibun}

	\newpage
	\fukudai{Vježba}
	
	\begin{mondai}{\ten Prevedi na hrvatski i promijeni pristojnost na japanskom:}
		\item PR: まるは ねこ です。 $\rightarrow$ まるは ねこ だ。\hspace{20pt}\textit{Maru je mačka.}
		\item わたしは はやし だ。
		\item Savaは かわの なまえ です。
		\item わたしの いぬの なまえは じゃまたろう だった。
		\item あの はなの なまえは 「ばら」 でした。
		\item それは あなたの りんご じゃない。
		\item かれの なまえは たけし ではありません。
		\item それは きんぎょ じゃなかった。
		\item わたしの ともだちの とりの なまえは ぽち ではありませんでした。
	\end{mondai}

	\begin{mondai}{\ten Prevedi na japanski pristojno i kolokvijalno:}
		\item \textit{Fuji je planina}.
		\item \textit{Lubenice nisu povrće}.
		\item \textit{Onaj pas je bio moj prijatelj}.
		\item \textit{Onaj konj nije bio moj prijatelj}. (ovo u jap. ne zvuči bezobrazno jer \textit{konj} nije uvreda)
	\end{mondai}
	
\end{document}