% !TeX document-id = {648f75d0-014f-4b4b-b76d-34b561715b14}
% !TeX program = xelatex ?me -synctex=0 -interaction=nonstopmode -aux-directory=../tex_aux -output-directory=./release
% !TeX program = xelatex

\documentclass[12pt]{article}

\usepackage{lineno,changepage,lipsum}
\usepackage[colorlinks=true,urlcolor=blue]{hyperref}
\usepackage{fontspec}
\usepackage{xeCJK}
\usepackage{tabularx}
\setCJKfamilyfont{chanto}{AozoraMinchoRegular.ttf}
\setCJKfamilyfont{tegaki}{Mushin.otf}
\usepackage[CJK,overlap]{ruby}
\usepackage{hhline}
\usepackage{multirow,array,amssymb}
\usepackage[croatian]{babel}
\usepackage{soul}
\usepackage[usenames, dvipsnames]{color}
\usepackage{wrapfig,booktabs}
\renewcommand{\rubysep}{0.1ex}
\renewcommand{\rubysize}{0.75}
\usepackage[margin=50pt]{geometry}
\modulolinenumbers[2]

\usepackage{pifont}
\newcommand{\cmark}{\ding{51}}%
\newcommand{\xmark}{\ding{55}}%

\definecolor{faded}{RGB}{100, 100, 100}

\renewcommand{\arraystretch}{1.2}

%\ruby{}{}
%$($\href{URL}{text}$)$

\newcommand{\furigana}[2]{\ruby{#1}{#2}}
\newcommand{\tegaki}[1]{
	\CJKfamily{tegaki}\CJKnospace
	#1
	\CJKfamily{chanto}\CJKnospace
}

\newcommand{\dai}[1]{
	\vspace{20pt}
	\large
	\noindent\textbf{#1}
	\normalsize
	\vspace{20pt}
}

\newcommand{\fukudai}[1]{
	\vspace{10pt}
	\noindent\textbf{#1}
	\vspace{10pt}
}

\newenvironment{bunshou}{
	\vspace{10pt}
	\begin{adjustwidth}{1cm}{3cm}
	\begin{linenumbers}
}{
	\end{linenumbers}
	\end{adjustwidth}
}

\newenvironment{reibun}{
	\vspace{10pt}
	\begin{tabular}{l l}
}{
	\end{tabular}
	\vspace{10pt}
}
\newcommand{\rei}[2]{
	#1&\textit{#2}\\
}
\newcommand{\reinagai}[2]{
	\multicolumn{2}{l}{#1}\\
	\multicolumn{2}{l}{\hspace{10pt}\textit{#2}}\\
}

\newenvironment{mondai}[1]{
	\vspace{10pt}
	#1
	
	\begin{enumerate}
		\itemsep-5pt
	}{
	\end{enumerate}
	\vspace{10pt}
}

\newenvironment{hyou}{
	\begin{itemize}
		\itemsep-5pt
	}{
	\end{itemize}
	\vspace{10pt}
}

\date{\today}

\CJKfamily{chanto}\CJKnospace
\author{Tomislav Mamić}
\begin{document}
	\dai{Ciljevi i napomene - spojni glagol}
	
	Ovo je prva lekcija u kojoj se od polaznika očekuje da nauče neku tablicu oblika. Bitno je tu tablicu "dobro prodati" da ne izgleda strašno.
	
	\fukudai{Ciljevi}
	
	\begin{hyou}
		\item što je spojni glagol i koja je njegova uloga u jeziku
		\item zašto postoje dva seta nastavaka za istu stvar - pristojnost
		\item zašto nije problem što postoje samo dva vremena
		\item kako su pristojni i kolokvijalni oblik međusobno povezani (radi lakšeg pamćenja)
		\item objasniti tablice i sve pravilnosti i nepravilnosti u njima
		\item uvođenje čestica koje određuju ulogu im. fraza u rečenici - は, が i も
		\item objasniti razliku između teme i subjekta, ali ne ulaziti preduboko u tehničke detalje
	\end{hyou}

	\fukudai{Napomene}
	
	\begin{hyou}
		\item Razlozi pisanja は, a čitanja わ - reforma jezika poslije WW2 da bi se izbjegle dvoznačnosti s う glagolima i imenicama koje koriste isti kanđi (npr. 歌はない \textasciitilde 歌わない).
		\item Objasniti kakva je točno razlika između čestica koje smo prije radili (の, や, と) i ovih novih. Prve se koriste za spajanje u imeničke fraze - stavljaju imenice u odnos s drugim imenicama. Današnje čestice stavljaju imenice u odnos s predikatom - određuju njihovu ulogu u rečenici.
	\end{hyou}
	
	\fukudai{Dodatne pričice}
	
	Originalno u japanskom nije postojao poseban spojni glagol. Porijeklo です je izraz であります u kojem je で čestica, a あります pristojni oblik od ある kakvog i danas koristimo. U pisanom japanskom se i danas kao stilski izbor često za spojni glagol koristi oblik である. Samuraji su umjesto ある govorili ござる (arh. glagol s nepravilnim い oblikom - ござい~) s istim značenjem. Danas se isti glagol vrlo često pojavljuje u pristojnim izrazima, najčešće kao dio ありがとうございます.
\end{document}