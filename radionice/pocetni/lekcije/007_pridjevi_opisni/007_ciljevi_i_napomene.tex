% !TeX document-id = {cce5c86b-1003-484c-8bd2-d6a325b4d365}
% !TeX program = xelatex ?me -synctex=0 -interaction=nonstopmode -aux-directory=../tex_aux -output-directory=./release
% !TeX program = xelatex

\documentclass[12pt]{article}

\usepackage{lineno,changepage,lipsum}
\usepackage[colorlinks=true,urlcolor=blue]{hyperref}
\usepackage{fontspec}
\usepackage{xeCJK}
\usepackage{tabularx}
\setCJKfamilyfont{chanto}{AozoraMinchoRegular.ttf}
\setCJKfamilyfont{tegaki}{Mushin.otf}
\usepackage[CJK,overlap]{ruby}
\usepackage{hhline}
\usepackage{multirow,array,amssymb}
\usepackage[croatian]{babel}
\usepackage{soul}
\usepackage[usenames, dvipsnames]{color}
\usepackage{wrapfig,booktabs}
\renewcommand{\rubysep}{0.1ex}
\renewcommand{\rubysize}{0.75}
\usepackage[margin=50pt]{geometry}
\modulolinenumbers[2]

\usepackage{pifont}
\newcommand{\cmark}{\ding{51}}%
\newcommand{\xmark}{\ding{55}}%

\definecolor{faded}{RGB}{100, 100, 100}

\renewcommand{\arraystretch}{1.2}

%\ruby{}{}
%$($\href{URL}{text}$)$

\newcommand{\furigana}[2]{\ruby{#1}{#2}}
\newcommand{\tegaki}[1]{
	\CJKfamily{tegaki}\CJKnospace
	#1
	\CJKfamily{chanto}\CJKnospace
}

\newcommand{\dai}[1]{
	\vspace{20pt}
	\large
	\noindent\textbf{#1}
	\normalsize
	\vspace{20pt}
}

\newcommand{\fukudai}[1]{
	\vspace{10pt}
	\noindent\textbf{#1}
	\vspace{10pt}
}

\newenvironment{bunshou}{
	\vspace{10pt}
	\begin{adjustwidth}{1cm}{3cm}
	\begin{linenumbers}
}{
	\end{linenumbers}
	\end{adjustwidth}
}

\newenvironment{reibun}{
	\vspace{10pt}
	\begin{tabular}{l l}
}{
	\end{tabular}
	\vspace{10pt}
}
\newcommand{\rei}[2]{
	#1&\textit{#2}\\
}
\newcommand{\reinagai}[2]{
	\multicolumn{2}{l}{#1}\\
	\multicolumn{2}{l}{\hspace{10pt}\textit{#2}}\\
}

\newenvironment{mondai}[1]{
	\vspace{10pt}
	#1
	
	\begin{enumerate}
		\itemsep-5pt
	}{
	\end{enumerate}
	\vspace{10pt}
}

\newenvironment{hyou}{
	\begin{itemize}
		\itemsep-5pt
	}{
	\end{itemize}
	\vspace{10pt}
}

\date{\today}

\CJKfamily{chanto}\CJKnospace
\author{Tomislav Mamić}
\begin{document}
	\dai{Ciljevi i napomene - opisni oblik pridjeva}
	
	Ova se lekcija čvrsto nadovezuje na prethodnu (imenski predikat), i očekuje se znanje oblika iz iste. Svejedno, tablica opisnih oblika je dana u lekciji, s naznačenom razlikom za な pridjeve.
	
	\fukudai{Ciljevi}
	\begin{hyou}
		\item objasniti opisivanje imenica い, な i の pridjevima
		\item vrijeme i negacija opisnog oblika - objasniti što se točno događa
		\item naučiti dijelove tijela
		\item opisivanje ljudskih osobina dijelovima tijela - rečenice oblika\newline <netko>は<neki dio tijela>が<opis>
	\end{hyou}

	\fukudai{Napomene}
	\begin{hyou}
		\item Opisni oblik な i の pridjeva u poz.+ razlikuje se od njihovog predikatnog oblika, ali u svim ostalim situacijama, opisni oblik je jednak kratkom (kolokvijalnom) predikatnom obliku.
		\item Pridjev je kao predikat sam za sebe rečenica, u takvim situacijama vrlo jednostavni opisi su zapravo cijele zavisne \textit{opisne rečenice}, npr:
		
		大きかった木 $\rightarrow$ \textit{drvo koje je bilo veliko}.
		
		U poz.+ obliku, ova se zavisna rečenica u hrvatskom prijevodu gubi jer značenje spada na hrv. pridjev sam za sebe, npr:
		
		大きい木 $\rightarrow$ \textit{veliko drvo},
		
		ali u japanskom je granica između opisa i predikata vrlo tanka.
		\item Nizanje pridjeva ćemo raditi kasnije.
	\end{hyou}

	\fukudai{Dodatne pričice}
	
	Nekad davno su i い pridjevi imali odvojen predikatni (し) i opisni oblik (き), ali s vremenom su se ta dva oblika stopila u današnji い. Tragovi ovih oblika i danas postoje u skamenjenim izrazima i gramatici, npr. (\href{https://jisho.org/search/osorubeshi}{link}), (\href{https://jisho.org/search/arumajiki}{link}). Riječ べき (stari pridjev) i danas je zadržala neke gramatičke funkcije iz prošlosti (oblici べき, べし, べく i べからず), iako upotreba svih osim べき zvuči arhaično.
\end{document}