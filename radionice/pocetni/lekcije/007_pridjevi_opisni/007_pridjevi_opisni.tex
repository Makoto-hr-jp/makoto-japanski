% !TeX document-id = {ffe542aa-da0b-4df4-a930-0f5709189d1b}
% !TeX program = xelatex ?me -synctex=0 -interaction=nonstopmode -aux-directory=../tex_aux -output-directory=./release
% !TeX program = xelatex

\documentclass[12pt]{article}

\usepackage{lineno,changepage,lipsum}
\usepackage[colorlinks=true,urlcolor=blue]{hyperref}
\usepackage{fontspec}
\usepackage{xeCJK}
\usepackage{tabularx}
\setCJKfamilyfont{chanto}{AozoraMinchoRegular.ttf}
\setCJKfamilyfont{tegaki}{Mushin.otf}
\usepackage[CJK,overlap]{ruby}
\usepackage{hhline}
\usepackage{multirow,array,amssymb}
\usepackage[croatian]{babel}
\usepackage{soul}
\usepackage[usenames, dvipsnames]{color}
\usepackage{wrapfig,booktabs}
\renewcommand{\rubysep}{0.1ex}
\renewcommand{\rubysize}{0.75}
\usepackage[margin=50pt]{geometry}
\modulolinenumbers[2]

\usepackage{pifont}
\newcommand{\cmark}{\ding{51}}%
\newcommand{\xmark}{\ding{55}}%

\definecolor{faded}{RGB}{100, 100, 100}

\renewcommand{\arraystretch}{1.2}

%\ruby{}{}
%$($\href{URL}{text}$)$

\newcommand{\furigana}[2]{\ruby{#1}{#2}}
\newcommand{\tegaki}[1]{
	\CJKfamily{tegaki}\CJKnospace
	#1
	\CJKfamily{chanto}\CJKnospace
}

\newcommand{\dai}[1]{
	\vspace{20pt}
	\large
	\noindent\textbf{#1}
	\normalsize
	\vspace{20pt}
}

\newcommand{\fukudai}[1]{
	\vspace{10pt}
	\noindent\textbf{#1}
	\vspace{10pt}
}

\newenvironment{bunshou}{
	\vspace{10pt}
	\begin{adjustwidth}{1cm}{3cm}
	\begin{linenumbers}
}{
	\end{linenumbers}
	\end{adjustwidth}
}

\newenvironment{reibun}{
	\vspace{10pt}
	\begin{tabular}{l l}
}{
	\end{tabular}
	\vspace{10pt}
}
\newcommand{\rei}[2]{
	#1&\textit{#2}\\
}
\newcommand{\reinagai}[2]{
	\multicolumn{2}{l}{#1}\\
	\multicolumn{2}{l}{\hspace{10pt}\textit{#2}}\\
}

\newenvironment{mondai}[1]{
	\vspace{10pt}
	#1
	
	\begin{enumerate}
		\itemsep-5pt
	}{
	\end{enumerate}
	\vspace{10pt}
}

\newenvironment{hyou}{
	\begin{itemize}
		\itemsep-5pt
	}{
	\end{itemize}
	\vspace{10pt}
}

\date{\today}

\CJKfamily{chanto}\CJKnospace
\author{Tomislav Mamić}
\begin{document}
	\dai{Opisni oblik pridjeva}
	
	U prethodnoj smo lekciji naučili što je to imenski predikat i kako se koristi. Naučili smo reći \textit{ona mačka je crna}, a danas ćemo naučiti nešto naizgled jednostavnije - kako reći \textit{crna mačka}.
	
	\fukudai{Teorija - opisni oblik}
	
	Opisni oblik riječi je onaj kojim ta riječ opisuje imenicu. Opis uvijek prethodi imenici na koju se odnosi. I u hrvatskom jeziku, to je istina za jednostavne slučajeve gdje se radi o jednom ili više pridjeva (npr.\textit{crveni auto}, \textit{lijepi plavi Dunav}). Međutim kad opis nije samo pridjev već cijela zavisna rečenica (npr. \textit{mačka koju sam jučer vidio}), ta rečenica slijedi nakon imenske riječi koju opisuje.
	
	U japanskom jeziku, situacija je jednostavnija - kakav god opis bio, uvijek prethodi imenici na koju se odnosi.
	
	U tablici ispod dan je opisni oblik い, な i の pridjeva.
	
	\begin{table}[h]
		\centering
		\begin{tabular}{l l l l}\toprule[2pt]
			+ neprošlost & + prošlost & - neprošlost & - prošlost\\
			\midrule
			\textasciitilde い & \textasciitilde かった & \textasciitilde くない & \textasciitilde くなかった\\
			いい & よかった & よくない & よくなかった\\
			\textasciitilde な* & \textasciitilde だった & \textasciitilde じゃない & \textasciitilde じゃなかった\\
			\textasciitilde の* & \textasciitilde だった & \textasciitilde じゃない & \textasciitilde じゃなかった\\
			\bottomrule[2pt]
		\end{tabular}
	\end{table}

	Vidljivo je da svi oblici osim prvog stupca za な i の odgovaraju kolokvijalnim predikatnim oblicima koje smo naučili u prošloj lekciji. Ovo je dio veće pravilnosti u jeziku koja vrijedi i za glagole, a koja nam omogućuje da na jednostavan način imenicama pridružujemo složene opise. Pridjev いい (\textit{dobro}) u prošlosti je bio よい. Iako se osnovni oblik s vremenom počeo izgovarati kao いい, svi izvedeni oblici dolaze iz よい.
	
	\fukudai{Značenje prošlosti i negacije opisnog oblika}
	
	U hrvatskom jeziku, pridjevi nemaju prošlost i negaciju pa nije jednostavno odrediti što točno takvi pridjevi u japanskom znače. Pogledajmo neke primjere.
	
	\begin{reibun}
		\rei{しろい ねこ}{bijela mačka}
		\rei{たかかった き}{drvo koje je bilo visoko}
		\rei{おおきくない いえ}{kuća koja nije velika}
		\vspace{5pt} % ovo iz nekog razloga stavlja razmak tek red ispod sljedećeg primjera
		\rei{あまくなかった りんご}{jabuka koja nije bila slatka}
		\rei{しずかな こうえん}{tihi park}
		\rei{びょうき だった ともだち}{prijatelj koji je bio bolestan}
		\rei{かんたん じゃない もんだい}{zadatak koji nije jednostavan}
		\rei{きらい じゃなかった とり}{ptica koju nisam mrzio\footnotemark[1]}
	\end{reibun}
	\footnotetext[1]{U jap. ovakav način govora implicira da je govorniku ptica ipak bila pomalo draga.}
	
	Kao što možemo primijetiti u primjerima iznad, prijevod u hrvatskom jeziku je elegantan samo u jednostavnom slučaju gdje pridjev nije ni u prošlosti ni negiran.
	
	\newpage
	\fukudai{Opisivanje ljudi i životinja}
	
	U hrvatskom jeziku uobičajeno je ljude opisivati pridjevima kao i sve ostale imenske riječi, ali u japanskom to nije uvijek tako. Često se umjesto riječi koja označava osobu opisuju dijelovi tijela povezani s osobinom koja se opisuje. Isto vrijedi i za opisivanje životinja.
	
	\begin{reibun}
		\rei{たなかさんは たかい}{Tanaka je visok \xmark}
		\rei{たなかさん\underline{は} せ\underline{が} たかい。}{Tanaka je visok. \cmark \rem{(dosl. \textit{Tanaka je visokog stasa.})}}
		\rei{たかぎさん\underline{は} あし\underline{が} はやい。}{Takagi je brz. \rem{(brzo trči)}}
		\rei{すずきさん\underline{は} あたま\underline{が} いい。}{Suzuki je pametna. \rem{(dosl. \textit{ima dobru glavu})}}
		\rei{たかなしさん\underline{は} め\underline{が} わるい。}{Takanashi loše vidi. \rem{(dosl. \textit{ima loše oči})}}
	\end{reibun}

	Valja primijetiti da su rečenice iznad zapravo potpune - pridjevi u njima su u predikatnom obliku. Nadalje, u njima se pojavljuju i čestica は i が. Iako je u odnosu na hrvatski jezik jako neobično, u ovakvim situacijama temu možemo protumačiti kao \textit{Govoreći o}, npr. \textit{Govoreći o Takagiju, noge su mu brze}\footnotemark[2].
	
	\fukudai{Opisna rečenica}
	
	Ovo je u odnosu na hrvatsku gramatiku poprilično stran koncept pa ćemo ga samo spomenuti, ali moguće je cijelu rečenicu pretvoriti u opis tako da njezin predikat stavimo u opisni oblik:
	
	\vspace{10pt}
	たなかさんは \underline{あしが はやい}。 $\rightarrow$ \underline{あしが はやい} たなかさん
	
	\vspace{10pt}\noindent
	Rečenica koju smo iskoristili kao opis je あしが はやい (dosl. \textit{noge su brze}). S obzirom na to da je s lijeve strane imenici (たなかさん), ta rečenica postaje njezin opis. Nadalje, u opisnim rečenicama se čestica が koja označava subjekt vrlo često mijenja u の. Ovaj の ima znatno drugačiju ulogu od posvojne čestice na koju smo navikli! Pogledajmo nekoliko primjera:
	
	\begin{reibun}
		\rei{あし(が/の) はやい たなかさん}{brzonogi Tanaka}
		\rei{あたま(が/の) いい すずきさん}{pametna Suzuki}
		\rei{め(が/の) わるい たかなしさん}{Takanashi koja loše vidi}
	\end{reibun}

	\fukudai{Vježba}
	
	\vspace{-30pt}
	\begin{mondai}{}
		\item くろい ねこ
		\item あかくない とり
		\item かっこいい たなかさん
		\vspace{5pt}
		\item あの ねこは くろかった。
		\item あれは あかくない とり でした。
		\item たなかさんは かっこよくなかった。
		\vspace{5pt}
		\item あの くろい ねこは あしが はやくない。
		\item あの あたまの いい とりは あかかった です。
		\item あしの はやくない たなかさんは かっこよくなかった。
	\end{mondai}

	
	\footnotetext[2]{Ipak ćemo se truditi prevoditi rečenice u oba smjera tako da budu prirodne. Tako bi spomenuti primjer bilo ljepše prevesti kao \textit{Takagi je brzonog} ili \textit{Takagi ima brze noge}. U spomenutom obliku rečenice, čestica が zapravo označava objekt.}
\end{document}
