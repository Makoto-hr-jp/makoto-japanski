% !TeX document-id = {67e179f1-d360-423d-834a-f52212543b8a}
% !TeX program = xelatex ?me -synctex=0 -interaction=nonstopmode -aux-directory=../tex_aux -output-directory=./release
% !TeX program = xelatex

\documentclass[12pt]{article}

\usepackage{lineno,changepage,lipsum}
\usepackage[colorlinks=true,urlcolor=blue]{hyperref}
\usepackage{fontspec}
\usepackage{xeCJK}
\usepackage{tabularx}
\setCJKfamilyfont{chanto}{AozoraMinchoRegular.ttf}
\setCJKfamilyfont{tegaki}{Mushin.otf}
\usepackage[CJK,overlap]{ruby}
\usepackage{hhline}
\usepackage{multirow,array,amssymb}
\usepackage[croatian]{babel}
\usepackage{soul}
\usepackage[usenames, dvipsnames]{color}
\usepackage{wrapfig,booktabs}
\renewcommand{\rubysep}{0.1ex}
\renewcommand{\rubysize}{0.75}
\usepackage[margin=50pt]{geometry}
\modulolinenumbers[2]

\usepackage{pifont}
\newcommand{\cmark}{\ding{51}}%
\newcommand{\xmark}{\ding{55}}%

\definecolor{faded}{RGB}{100, 100, 100}

\renewcommand{\arraystretch}{1.2}

%\ruby{}{}
%$($\href{URL}{text}$)$

\newcommand{\furigana}[2]{\ruby{#1}{#2}}
\newcommand{\tegaki}[1]{
	\CJKfamily{tegaki}\CJKnospace
	#1
	\CJKfamily{chanto}\CJKnospace
}

\newcommand{\dai}[1]{
	\vspace{20pt}
	\large
	\noindent\textbf{#1}
	\normalsize
	\vspace{20pt}
}

\newcommand{\fukudai}[1]{
	\vspace{10pt}
	\noindent\textbf{#1}
	\vspace{10pt}
}

\newenvironment{bunshou}{
	\vspace{10pt}
	\begin{adjustwidth}{1cm}{3cm}
	\begin{linenumbers}
}{
	\end{linenumbers}
	\end{adjustwidth}
}

\newenvironment{reibun}{
	\vspace{10pt}
	\begin{tabular}{l l}
}{
	\end{tabular}
	\vspace{10pt}
}
\newcommand{\rei}[2]{
	#1&\textit{#2}\\
}
\newcommand{\reinagai}[2]{
	\multicolumn{2}{l}{#1}\\
	\multicolumn{2}{l}{\hspace{10pt}\textit{#2}}\\
}

\newenvironment{mondai}[1]{
	\vspace{10pt}
	#1
	
	\begin{enumerate}
		\itemsep-5pt
	}{
	\end{enumerate}
	\vspace{10pt}
}

\newenvironment{hyou}{
	\begin{itemize}
		\itemsep-5pt
	}{
	\end{itemize}
	\vspace{10pt}
}

\date{\today}

\CJKfamily{chanto}\CJKnospace
\author{Tomislav Mamić}
\begin{document}
	\dai{い oblik}
	
	\fukudai{Teorija}
	
	Japanski naziv ovog oblika je 連用形 (れん.よう.けい dosl. \textit{oblik za uzastopno korištenje}). Ponekad, iako ne često, može se sresti i naziv konjunktiv. Još na prvom satu spomenuli smo kako je japanski jezik "ljepljiv", u smislu da se riječi mogu tvoriti spajanjem i dodavanjem raznih nastavaka. Ključnu ulogu u tome ima upravo ovaj, kao i iz njega izvedeni て oblik.
	
	\fukudai{Tvorba}
	
	U tablici ispod prikazani su い oblici svih glagola:
	
	\begin{table}[h]
		\centering
		\begin{tabular}{l | l l l l | l | l}%\toprule[2pt]
			& \multicolumn{4}{l |}{nepravilni} & 一 & 五 \\
			\midrule
			glagol & いく & くる & ある & する & \textasciitilde る & く, ぐ, す, ぬ, む, ぶ, う, つ, る \\
			い oblik & いき & き & あり & し & \textasciitilde & き, ぎ, し, に, み, び, い, ち, り \\
			%\bottomrule
		\end{tabular}
	\end{table}

	Za 一段 (いち.だん) glagole, い oblik dobijemo micanjem zadnjeg る. Za sve 五段 (ご.だん) glagole vrijedi isto pravilo - zadnji znak hiragane prebacit ćemo iz う reda u pripadni い red. Na nepravilne glagole, kao i obično, moramo pripaziti.
	
	\fukudai{Korištenje}
	
	Vrlo često imenice koje po značenju idu uz glagol nastaju iz njegovog い oblika (npr.~はなす~-~\textit{pričati}~$\rightarrow$~はなし~-~\textit{priča}), ali to je više čest slučaj nego čvrsto gramatičko pravilo.
	
	\vspace{5pt}
	Većinom se ovaj oblik pojavljuje pri spajanju s pomoćnim glagolima i pridjevima, kojih ima jako puno i s raznim značenjima. Pravilo je uvijek da glavni glagol prebacimo u い oblik, a onda na njega dodamo pomoćni glagol ili pridjev, mijenjajući tako originalno značenje glavnog glagola. Tako nastala riječ nasljeđuje gramatičku ulogu pomoćnog glagola / pridjeva i može se dalje spajati i dobivati nastavke. Damo li si oduška, ovako nastale riječi mogu postati prilično dugačke.

	\fukudai{Pristojni oblik glagola \textasciitilde ます}
	
	Dodavanjem pom. glagola ます na い oblik glagola, dobivamo njegov pristojni oblik. Kad razgovaramo s nepoznatima, starijima ili više-manje bilo kime s kim nismo bliski, pazit ćemo da rečenice uvijek završavamo pristojnim predikatima. Za predikate s imenicama i pridjevima, pristojni oblik smo naučili već ranije (です). Da bi nam bio koristan, i za pristojni oblik moramo zapamtiti vrijeme i negaciju:
	
	\begin{table}[h]
		\centering
		\begin{tabular}{l | l l}%\toprule[2pt]
			& nepr. & pr \\
			\midrule
			poz. & ます & ました \\
			neg. & ません & ませんでした \\
			%\bottomrule
		\end{tabular}
	\end{table}

	Zapamtimo da pristojnost rečenice ovisi samo o glavnom predikatu i da se u predikatima zavisnih rečenica pristojni glagoli \textbf{ne pojavljuju} osim u iznimnim situacijama (npr. kad nekog citiramo). Pogledajmo neke primjere:
	
	\begin{reibun}
		\rei{日本に行く。}{}
		\rei{日本に行き\underline{ます}。}{Idem u Japan.}
		\rei{はこの中に りんごが あった。}{}
		\rei{はこの中に りんごが あり\underline{ました}。}{U kutiji je bila jabuka.}
		\rei{どうぶつえんで トラを 見た。}{}
		\rei{どうぶつえんで トラを 見\underline{ました}。}{U zoološkom vrtu sam vidio tigra.}
		\rei{たけしくんに てがみを かかなかった。}{}
		\rei{たけしくんに てがみを かき\underline{ませんでした}。}{Nisam napisao pismo Takešiju.}
	\end{reibun}

	\fukudai{Vježba}
	
	Upristojimo sljedeće rečenice!
	
	\begin{mondai}{Lv. 1}
		\item あの はなは きれい。
		\item そらは あおかった。
		\item あれは ねこ じゃない。
		\item きのうは あたたかくなかった。
	\end{mondai}

	\begin{mondai}{Lv. 2}
		\item あの きれいな はなを 見た。
		\item あおい そらを 見る。
		\item たけしくんは えいがを 見ない。
		\item あの木を 見なかった。
	\end{mondai}

	\begin{mondai}{Lv. 3}
		\item おかねが ない。
		\item すずきさんと 3じかん\footnotemark[1] べんきょうした。
		\item ともだちに てがみを かいた。
		\item その本を よまなかった。
	\end{mondai}

	\begin{mondai}{Lv. 4*}
		\item たけしくんは 「えいがかんに いかない」と いった。
		\item はなこちゃんは 「えいがかんが きらいです」と いわなかった。
		\item あたらしい くるまを かった 田中さんは うれしかった。
		\item はなこちゃんの りんごを たべた たけしくんは かくれた。
	\end{mondai}

	\footnotetext[1]{Dodamo li かん na brojač za vrijeme, pretvaramo količinu iz točnog vremena u raspon.}
\end{document}