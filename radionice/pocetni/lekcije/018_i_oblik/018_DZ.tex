% !TeX document-id = {47d75ff3-439c-408d-b3cb-045668fc0161}
% !TeX program = xelatex ?me -synctex=0 -interaction=nonstopmode -aux-directory=../tex_aux -output-directory=./release
% !TeX program = xelatex

\documentclass[12pt]{article}

\usepackage{lineno,changepage,lipsum}
\usepackage[colorlinks=true,urlcolor=blue]{hyperref}
\usepackage{fontspec}
\usepackage{xeCJK}
\usepackage{tabularx}
\setCJKfamilyfont{chanto}{AozoraMinchoRegular.ttf}
\setCJKfamilyfont{tegaki}{Mushin.otf}
\usepackage[CJK,overlap]{ruby}
\usepackage{hhline}
\usepackage{multirow,array,amssymb}
\usepackage[croatian]{babel}
\usepackage{soul}
\usepackage[usenames, dvipsnames]{color}
\usepackage{wrapfig,booktabs}
\renewcommand{\rubysep}{0.1ex}
\renewcommand{\rubysize}{0.75}
\usepackage[margin=50pt]{geometry}
\modulolinenumbers[2]

\usepackage{pifont}
\newcommand{\cmark}{\ding{51}}%
\newcommand{\xmark}{\ding{55}}%

\definecolor{faded}{RGB}{100, 100, 100}

\renewcommand{\arraystretch}{1.2}

%\ruby{}{}
%$($\href{URL}{text}$)$

\newcommand{\furigana}[2]{\ruby{#1}{#2}}
\newcommand{\tegaki}[1]{
	\CJKfamily{tegaki}\CJKnospace
	#1
	\CJKfamily{chanto}\CJKnospace
}

\newcommand{\dai}[1]{
	\vspace{20pt}
	\large
	\noindent\textbf{#1}
	\normalsize
	\vspace{20pt}
}

\newcommand{\fukudai}[1]{
	\vspace{10pt}
	\noindent\textbf{#1}
	\vspace{10pt}
}

\newenvironment{bunshou}{
	\vspace{10pt}
	\begin{adjustwidth}{1cm}{3cm}
	\begin{linenumbers}
}{
	\end{linenumbers}
	\end{adjustwidth}
}

\newenvironment{reibun}{
	\vspace{10pt}
	\begin{tabular}{l l}
}{
	\end{tabular}
	\vspace{10pt}
}
\newcommand{\rei}[2]{
	#1&\textit{#2}\\
}
\newcommand{\reinagai}[2]{
	\multicolumn{2}{l}{#1}\\
	\multicolumn{2}{l}{\hspace{10pt}\textit{#2}}\\
}

\newenvironment{mondai}[1]{
	\vspace{10pt}
	#1
	
	\begin{enumerate}
		\itemsep-5pt
	}{
	\end{enumerate}
	\vspace{10pt}
}

\newenvironment{hyou}{
	\begin{itemize}
		\itemsep-5pt
	}{
	\end{itemize}
	\vspace{10pt}
}

\date{\today}

\CJKfamily{chanto}\CJKnospace
\author{Tomislav Mamić}
\begin{document}
	\dai{Domaća zadaća - い oblik}
	
	\begin{mondai}{Sljedećim glagolima napišite prošlost i negaciju u pristojnoj i kolokvijalnoj varijanti.}
		\item たべる
		\item 見る
		\item いる
		\vspace{5pt}
		\item ある
		\item いく
		\item くる
		\item する
		\vspace{5pt}
		\item かく、きく
		\item およぐ、さわぐ
		\item はなす、さがす
		\vspace{5pt}
		\item しぬ
		\item よむ、のむ
		\item えらぶ、よぶ
		\vspace{5pt}
		\item かう、いう
		\item まつ、もつ
		\item うる、しる
	\end{mondai}

	\begin{mondai}{Rečenice u nastavku obrnite po pristojnosti - kolokvijalne napišite pristojno, a pristojne kolokvijalno.}
		\item たけしくんは きのう たくさん べんきょう しました。
		\item だれが わたしの ケーキを たべたか?
		\item すずきさんは あたらしい きものを かいました。
		\item 「そのえいが、もう見た」と花子さんは いった。
		\item マルちゃんは つくえの下の はこに います。
		\item 田中さんは「マルちゃんはつくえの下の はこに います」と いった。
		\item *はこの中にいる たけしくんは ぬすんだ はなこちゃんの ケーキを たべました。
	\end{mondai}

	\noindent Rečenice iz prethodnog zadatka prevedite na hrvatski!
\end{document}