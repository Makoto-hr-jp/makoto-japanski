% !TeX document-id = {558e1ef9-1725-4176-abc4-1a135b4325f5}
% !TeX program = xelatex ?me -synctex=0 -interaction=nonstopmode -aux-directory=../tex_aux -output-directory=./release
% !TeX program = xelatex

\documentclass[12pt]{article}

\usepackage{lineno,changepage,lipsum}
\usepackage[colorlinks=true,urlcolor=blue]{hyperref}
\usepackage{fontspec}
\usepackage{xeCJK}
\usepackage{tabularx}
\setCJKfamilyfont{chanto}{AozoraMinchoRegular.ttf}
\setCJKfamilyfont{tegaki}{Mushin.otf}
\usepackage[CJK,overlap]{ruby}
\usepackage{hhline}
\usepackage{multirow,array,amssymb}
\usepackage[croatian]{babel}
\usepackage{soul}
\usepackage[usenames, dvipsnames]{color}
\usepackage{wrapfig,booktabs}
\renewcommand{\rubysep}{0.1ex}
\renewcommand{\rubysize}{0.75}
\usepackage[margin=50pt]{geometry}
\modulolinenumbers[2]

\usepackage{pifont}
\newcommand{\cmark}{\ding{51}}%
\newcommand{\xmark}{\ding{55}}%

\definecolor{faded}{RGB}{100, 100, 100}

\renewcommand{\arraystretch}{1.2}

%\ruby{}{}
%$($\href{URL}{text}$)$

\newcommand{\furigana}[2]{\ruby{#1}{#2}}
\newcommand{\tegaki}[1]{
	\CJKfamily{tegaki}\CJKnospace
	#1
	\CJKfamily{chanto}\CJKnospace
}

\newcommand{\dai}[1]{
	\vspace{20pt}
	\large
	\noindent\textbf{#1}
	\normalsize
	\vspace{20pt}
}

\newcommand{\fukudai}[1]{
	\vspace{10pt}
	\noindent\textbf{#1}
	\vspace{10pt}
}

\newenvironment{bunshou}{
	\vspace{10pt}
	\begin{adjustwidth}{1cm}{3cm}
	\begin{linenumbers}
}{
	\end{linenumbers}
	\end{adjustwidth}
}

\newenvironment{reibun}{
	\vspace{10pt}
	\begin{tabular}{l l}
}{
	\end{tabular}
	\vspace{10pt}
}
\newcommand{\rei}[2]{
	#1&\textit{#2}\\
}
\newcommand{\reinagai}[2]{
	\multicolumn{2}{l}{#1}\\
	\multicolumn{2}{l}{\hspace{10pt}\textit{#2}}\\
}

\newenvironment{mondai}[1]{
	\vspace{10pt}
	#1
	
	\begin{enumerate}
		\itemsep-5pt
	}{
	\end{enumerate}
	\vspace{10pt}
}

\newenvironment{hyou}{
	\begin{itemize}
		\itemsep-5pt
	}{
	\end{itemize}
	\vspace{10pt}
}

\date{\today}

\CJKfamily{chanto}\CJKnospace
\author{Tomislav Mamić}
\begin{document}
	\dai{Ciljevi i napomene - い oblik}
	
	Budući se radi o početnoj grupi, samo ćemo spomenuti naziv 連用形 i rastumačiti njegovo značenje, ali ga nećemo forsirati.
	
	\fukudai{Ciljevi}
	\begin{hyou}
		\item tvorba za sve vrste glagola
		\item napomena o korištenju glagola u い obliku kao imenica
		\item pristojni oblik - prošlost i negacija
	\end{hyou}

	\fukudai{Napomene}
	
	\noindent Ovo je dobra prilika za ponavljanje svih predikata dosad.
	\begin{hyou}
		\item U ovom trenutku, predikati imaju tri neovisne dimenzije - vrijeme, negaciju i pristojnost. To daje 8 različitih kombinacija za svaku kategoriju predikata.
		\item Predikate svrstavamo u tri grube kategorije
		\begin{hyou}
			\item imenice + な / の pridjevi
			\item い pridjevi
			\item glagoli
			\begin{hyou}
				\itemsep0pt
				\item 五段
				\item 一段
				\item 不規則 (行く formalno nije, ali ga stavljamo radi nepr. prošlosti)
				\begin{hyou}
					\itemsep0pt
					\item いく
					\item くる
					\item する
					\item ある
				\end{hyou}
			\end{hyou}
		\end{hyou}
	\end{hyou}

\vspace{-50pt}\noindent Varijanti predikata ukupno ima 3 za imenske (imenica, な pridjev i の pridjev) + 1 い pridjev + 9 五段 glagola + 2 一段 glagola + 4 nepravilna. Tih 19 primjera u 8 mogućih varijanti daje 152 zadatka za potpunu pokrivenost. Ovo nije isplativo ciljati jer ih nitko neće ni smišljati ni riješiti.
\end{document}