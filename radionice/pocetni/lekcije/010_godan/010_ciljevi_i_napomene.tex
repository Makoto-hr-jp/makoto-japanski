% !TeX document-id = {3a3255e8-e152-4d3e-9a90-60b7d35c3883}
% !TeX program = xelatex ?me -synctex=0 -interaction=nonstopmode -aux-directory=../tex_aux -output-directory=./release
% !TeX program = xelatex

\documentclass[12pt]{article}

\usepackage{lineno,changepage,lipsum}
\usepackage[colorlinks=true,urlcolor=blue]{hyperref}
\usepackage{fontspec}
\usepackage{xeCJK}
\usepackage{tabularx}
\setCJKfamilyfont{chanto}{AozoraMinchoRegular.ttf}
\setCJKfamilyfont{tegaki}{Mushin.otf}
\usepackage[CJK,overlap]{ruby}
\usepackage{hhline}
\usepackage{multirow,array,amssymb}
\usepackage[croatian]{babel}
\usepackage{soul}
\usepackage[usenames, dvipsnames]{color}
\usepackage{wrapfig,booktabs}
\renewcommand{\rubysep}{0.1ex}
\renewcommand{\rubysize}{0.75}
\usepackage[margin=50pt]{geometry}
\modulolinenumbers[2]

\usepackage{pifont}
\newcommand{\cmark}{\ding{51}}%
\newcommand{\xmark}{\ding{55}}%

\definecolor{faded}{RGB}{100, 100, 100}

\renewcommand{\arraystretch}{1.2}

%\ruby{}{}
%$($\href{URL}{text}$)$

\newcommand{\furigana}[2]{\ruby{#1}{#2}}
\newcommand{\tegaki}[1]{
	\CJKfamily{tegaki}\CJKnospace
	#1
	\CJKfamily{chanto}\CJKnospace
}

\newcommand{\dai}[1]{
	\vspace{20pt}
	\large
	\noindent\textbf{#1}
	\normalsize
	\vspace{20pt}
}

\newcommand{\fukudai}[1]{
	\vspace{10pt}
	\noindent\textbf{#1}
	\vspace{10pt}
}

\newenvironment{bunshou}{
	\vspace{10pt}
	\begin{adjustwidth}{1cm}{3cm}
	\begin{linenumbers}
}{
	\end{linenumbers}
	\end{adjustwidth}
}

\newenvironment{reibun}{
	\vspace{10pt}
	\begin{tabular}{l l}
}{
	\end{tabular}
	\vspace{10pt}
}
\newcommand{\rei}[2]{
	#1&\textit{#2}\\
}
\newcommand{\reinagai}[2]{
	\multicolumn{2}{l}{#1}\\
	\multicolumn{2}{l}{\hspace{10pt}\textit{#2}}\\
}

\newenvironment{mondai}[1]{
	\vspace{10pt}
	#1
	
	\begin{enumerate}
		\itemsep-5pt
	}{
	\end{enumerate}
	\vspace{10pt}
}

\newenvironment{hyou}{
	\begin{itemize}
		\itemsep-5pt
	}{
	\end{itemize}
	\vspace{10pt}
}

\date{\today}

\CJKfamily{chanto}\CJKnospace
\author{Tomislav Mamić}
\begin{document}
	\dai{Ciljevi i napomene - ごだん glagoli}
	
	\fukudai{Ciljevi}
	
	\vspace{-10pt}
	\begin{hyou}
		\item organizacija ごだん glagola
		\item 4 osnovna oblika s maksimalno učinkovito prikazanim nastavcima
		\item nadovezujući se na prošli tjedan, lokacija s で
		\item sredstvo s で i zašto nema zabune
		\item citiranje s と i podsjetnik za što smo je prije koristili
		\item dobra pokrivenost primjerima u DZ je ključna
	\end{hyou}

	\fukudai{Napomene}
	
	Podjela na いちだん i ごだん nije povijesno jedina. U starom pisanom japanskom, glagoli su se dijelili u 四段, 二段 i 一段 skupine, s nekolicinom nepravilnih koji su, izuzev くる, zapravo bili puno pravilniji od današnjih nepravilnih glagola.
	
	Iz nekih oblika nije moguće jednoznačno rekonstruirati osnovni glagol. Ovo je problem kad ne znamo kontekst ili se radi o nepoznatim riječima, ali imajući to na umu uvijek možemo pretpostaviti što tražimo u rječniku. Dobri primjeri ovog su うった (うつ ili うる?) i よんだ (よむ ili よぶ?). Nasreću, pojava nije toliko česta pa je već malo iskustva dovoljno da ukloni problem. Zapravo su istozvučni glagoli (変える vs. 帰る) i višeznačnost (かける, たつ itd.) daleko veći izvor zabuna.
	
	Najbolje objašnjenje za razliku između わたしと あなた i みえないだと いった je u tome na što se と odnosi. Vjerojatno neće predstavljati problem, ali za svaki slučaj dobro je napomenuti kroz primjere da čestica と kao citat ima smisla samo s glagolima koji prenose neku poruku.
\end{document}