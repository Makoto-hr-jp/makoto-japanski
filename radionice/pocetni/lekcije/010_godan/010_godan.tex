% !TeX document-id = {c0f7fdb5-ccc0-49f2-bd9d-d6b924ac58e3}
% !TeX program = xelatex ?me -synctex=0 -interaction=nonstopmode -aux-directory=../tex_aux -output-directory=./release
% !TeX program = xelatex

\documentclass[12pt]{article}

\usepackage{lineno,changepage,lipsum}
\usepackage[colorlinks=true,urlcolor=blue]{hyperref}
\usepackage{fontspec}
\usepackage{xeCJK}
\usepackage{tabularx}
\setCJKfamilyfont{chanto}{AozoraMinchoRegular.ttf}
\setCJKfamilyfont{tegaki}{Mushin.otf}
\usepackage[CJK,overlap]{ruby}
\usepackage{hhline}
\usepackage{multirow,array,amssymb}
\usepackage[croatian]{babel}
\usepackage{soul}
\usepackage[usenames, dvipsnames]{color}
\usepackage{wrapfig,booktabs}
\renewcommand{\rubysep}{0.1ex}
\renewcommand{\rubysize}{0.75}
\usepackage[margin=50pt]{geometry}
\modulolinenumbers[2]

\usepackage{pifont}
\newcommand{\cmark}{\ding{51}}%
\newcommand{\xmark}{\ding{55}}%

\definecolor{faded}{RGB}{100, 100, 100}

\renewcommand{\arraystretch}{1.2}

%\ruby{}{}
%$($\href{URL}{text}$)$

\newcommand{\furigana}[2]{\ruby{#1}{#2}}
\newcommand{\tegaki}[1]{
	\CJKfamily{tegaki}\CJKnospace
	#1
	\CJKfamily{chanto}\CJKnospace
}

\newcommand{\dai}[1]{
	\vspace{20pt}
	\large
	\noindent\textbf{#1}
	\normalsize
	\vspace{20pt}
}

\newcommand{\fukudai}[1]{
	\vspace{10pt}
	\noindent\textbf{#1}
	\vspace{10pt}
}

\newenvironment{bunshou}{
	\vspace{10pt}
	\begin{adjustwidth}{1cm}{3cm}
	\begin{linenumbers}
}{
	\end{linenumbers}
	\end{adjustwidth}
}

\newenvironment{reibun}{
	\vspace{10pt}
	\begin{tabular}{l l}
}{
	\end{tabular}
	\vspace{10pt}
}
\newcommand{\rei}[2]{
	#1&\textit{#2}\\
}
\newcommand{\reinagai}[2]{
	\multicolumn{2}{l}{#1}\\
	\multicolumn{2}{l}{\hspace{10pt}\textit{#2}}\\
}

\newenvironment{mondai}[1]{
	\vspace{10pt}
	#1
	
	\begin{enumerate}
		\itemsep-5pt
	}{
	\end{enumerate}
	\vspace{10pt}
}

\newenvironment{hyou}{
	\begin{itemize}
		\itemsep-5pt
	}{
	\end{itemize}
	\vspace{10pt}
}

\date{\today}

\CJKfamily{chanto}\CJKnospace
\author{Tomislav Mamić}

\usepackage{tikz}
\newcommand{\en}[1]{
	\begin{tikzpicture}[baseline=(C.base)]
		\node[draw,circle,inner sep=1pt](C){#1};
	\end{tikzpicture}
}

\begin{document}
	\dai{Glagoli III}
	
	\fukudai{ごだん glagoli}
	
	Po količini riječi najbrojnija, po nastavcima najšarenija skupina glagola. Završavaju na devet različitih glasova hiragane podijeljenih prema obliku u prošlosti u tri skupine po tri. Zovu se ごだん (dosl. \textit{pet razina}) zbog pet različitih nastavaka u prošlosti.
	
	\begin{table}[h]
		\centering
		\begin{tabular}{l l l l}\toprule[2pt]
			rječnički oblik & prošlost & negacija & negacija u prošlosti\\
			\midrule
			く & いた & かない & \multirow{9}{90pt}{ない $\rightarrow$ なかった}\\
			ぐ & いだ & がない & \\\vspace{5pt}
			す & した & さない & \\
			ぬ & \multirow{3}{30pt}{んだ} & なない & \\
			む & & まない & \\\vspace{5pt}
			ぶ & & ばない & \\
			う & \multirow{3}{30pt}{った} & \en{わ}\hspace{-2pt}ない & \\
			つ & & たない & \\
			る & & らない & \\
			\bottomrule[2pt]
		\end{tabular}
	\end{table}

	Prva podskupina ima različite repove za prošlost dok je u druge dvije nastavak isti. U početku nam to može stvarati probleme - vidimo li npr. glagol うった, ne možemo znati radi li se o osnovnom obliku うう (nepostojeća riječ), うつ (\textit{udariti}) ili うる (\textit{prodati}). Jedino rješenje ovog problema je iskustvo. Nasreću, glagoli s kojima bi mogla nastati zabuna obično imaju vrlo različita značenja pa iz konteksta znamo koji je ispravan.
	
	Negacija je, iako jedinstvena za svaki od devet nastavaka, opisana vrlo laganim pravilom - zadnji znak hiragane treba promijeniti u varijantu koja na kraju ima samoglasnik \textit{a} i dodati ない. Jedina iznimka (zaokruženo u tablici) su glagoli na う za koje う umjesto u あ prebacujemo u わ.
	
	Naučimo neke korisne ごだん glagole.
	
	\vspace{10pt}
	\begin{tabular}{l l l l l l}
		かく & \textit{napisati} & きく & \textit{čuti}, \textit{pitati} & おく & \textit{ostaviti} (negdje)\\
		およぐ & \textit{plivati} & いそぐ & \textit{požuriti} & ぬぐ & \textit{skinuti} (odjeću sa sebe)\\\vspace{5pt}
		はなす & \textit{pričati}/\textit{pustiti} & だす & \textit{iznijeti}, \textit{izbaciti} & さがす & \textit{potražiti}\\
		しぬ & \multicolumn{5}{l}{\textit{umrijeti} - jedini ぬ glagol u modernom japanskom!}\\
		よむ & \textit{pročitati} & のむ & \textit{popiti} & やすむ & \textit{odmoriti se}\\\vspace{5pt}
		よぶ & \textit{dozvati} & とぶ & \textit{poletjeti}/\textit{skočiti} & あそぶ & \textit{igrati se}\\
		いう & \textit{reći} & かう & \textit{kupiti} & うたう & \textit{pjevati}\\
		まつ & \textit{pričekati} & もつ & \textit{ponijeti}/\textit{imati} (kod sebe) & たつ & \textit{ustati}\\
		はしる & \textit{potrčati} & つくる & \textit{izraditi}, \textit{napraviti} & のる & \textit{voziti se}/\textit{jahati}\\
	\end{tabular}

	\newpage
	\fukudai{で kao mjesto radnje}
	
	Ranije smo naučili da česticu に možemo koristiti kao lokaciju za glagole stanja. Za aktivne glagole, mjesto na kojem se njihova radnja odvija označavamo česticom で. Pogledajmo neke primjere:
	
	\begin{reibun}
		\rei{こうえんで あそぶ。}{Igrati se u parku.}
		\rei{いざかやで さけを のむ。}{Popiti sake u birtiji.}
		\rei{でんしゃで うたう。}{Pjevati u vlaku.}
	\end{reibun}

	Neki glagoli za koje bismo iz hrvatskog intuitivno očekivali česticu で koriste česticu に. Iz perspektive japanskog, to je sasvim logično, ali dok ne steknemo još malo iskustva, vjerojatno će nam biti čudno.
	
	\begin{reibun}
		\rei{うみに およぐ。}{Plivati u moru. \normalfont (točno je i うみで およぐ)}
		\rei{でんしゃに のる。}{Voziti se u vlaku.}
	\end{reibun}

	\fukudai{で kao sredstvo radnje}
	
	U hrvatskom, sredstvo radnje označavamo instrumentalom (npr. \textit{pisati olovkom}). Međutim, u hrvatskom instrumentalom označavamo i živa bića s kojima zajedno obavljamo neku radnju (npr. \textit{razgovarati \underline{s} prijateljem}). Između te dvije upotrebe postoji vrlo bitna razlika - ispred živih bića uvijek se pojavljuje \textit{s}, a ispred stvari nikad. To će nam pomoći da znamo kad instrumental u japanski smijemo prevesti česticom で.
	
	\begin{reibun}
		\rei{えんぴつで かく。}{Pisati olovkom. \cmark}
		\rei{ともだちで そうじする。}{Čistiti prijateljem. \xmark}
	\end{reibun}

	\fukudai{と kao oznaka sudionika}
	
	Kako bismo ispravili problem u primjeru iznad, naučit ćemo još jednu upotrebu čestice と. Ranije smo je koristili za nabrajanje, povezujući međusobno imenice. U ovoj upotrebi, čestica と povezuje svoju imenicu s predikatom:
	
	\begin{reibun}
		\rei{ともだちと そうじする。}{Čistiti s prijateljem. \cmark}
		\rei{ともだちと おとうとと あそぶ。}{Igrati se s prijateljem i mlađim bratom.}
	\end{reibun}

	\fukudai{と kao oznaka doslovnog citata}
	
	Česticu と možemo koristiti i na kraju bilo kakve rečenice ili sintagme da bismo istu označili kao citat. Ovo je vrlo česta upotreba s glagolima čija radnja uključuje neki oblik komunikacije.
	
	\begin{reibun}
		\rei{こうえんで まつと すずきさんが いった。}{G. Suzuki je rekao da će čekati u parku.}
		\rei{やまださんは たなかさんに すきだと いった。}{G. Yamada je rekao G. Tanaki da mu se sviđa.}
	\end{reibun}

	\newpage
	\fukudai{Vježba}
	
	\begin{mondai}{Lv. 1}
		\item かわない。
		\item かいた。
		\item だす。
		\item またない。
		\item しんだ。
	\end{mondai}

	\begin{mondai}{Lv. 2}
		\item あたらしい くつを かわない。
		\item てがみを かいた。
		\item ごみを だす。
		\item ともだちを またない。
		\item とりは しんだ。
	\end{mondai}

	\begin{mondai}{Lv. 3}
		\item あたらしい くつを そのみせで かわない。
		\item おとうさんに ながい てがみを かいた。
		\item やまださんは ごみを ださなかった。
		\item たけしくんは ともだちを またなかった。
		\item すずきさんの きれいな とりは しんだ。
	\end{mondai}

	\begin{mondai}{Lv. 4}
		\item あたらしい くつを そのみせで かわないと むらかみさんは いった。
		\item *おおさかに いる おとうさんに ながい てがみを かいた。
		\item *ごみを ださなかった やまださんは いえを でた。
		\item たけしくんは おとうとと ともだちを まった。
		
		ili
		
		たけしくんは ともだちを おとうとと まった。
		
		U čemu je razlika? :)
		\item すずきさんの きれいな とりは とりかごで しんだ。
	\end{mondai}
\end{document}