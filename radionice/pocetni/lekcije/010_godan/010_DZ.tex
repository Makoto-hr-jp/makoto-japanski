% !TeX document-id = {ee90dfb7-e585-4a4c-a87e-526b36f9ebe2}
% !TeX program = xelatex ?me -synctex=0 -interaction=nonstopmode -aux-directory=../tex_aux -output-directory=./release
% !TeX program = xelatex

\documentclass[12pt]{article}

\usepackage{lineno,changepage,lipsum}
\usepackage[colorlinks=true,urlcolor=blue]{hyperref}
\usepackage{fontspec}
\usepackage{xeCJK}
\usepackage{tabularx}
\setCJKfamilyfont{chanto}{AozoraMinchoRegular.ttf}
\setCJKfamilyfont{tegaki}{Mushin.otf}
\usepackage[CJK,overlap]{ruby}
\usepackage{hhline}
\usepackage{multirow,array,amssymb}
\usepackage[croatian]{babel}
\usepackage{soul}
\usepackage[usenames, dvipsnames]{color}
\usepackage{wrapfig,booktabs}
\renewcommand{\rubysep}{0.1ex}
\renewcommand{\rubysize}{0.75}
\usepackage[margin=50pt]{geometry}
\modulolinenumbers[2]

\usepackage{pifont}
\newcommand{\cmark}{\ding{51}}%
\newcommand{\xmark}{\ding{55}}%

\definecolor{faded}{RGB}{100, 100, 100}

\renewcommand{\arraystretch}{1.2}

%\ruby{}{}
%$($\href{URL}{text}$)$

\newcommand{\furigana}[2]{\ruby{#1}{#2}}
\newcommand{\tegaki}[1]{
	\CJKfamily{tegaki}\CJKnospace
	#1
	\CJKfamily{chanto}\CJKnospace
}

\newcommand{\dai}[1]{
	\vspace{20pt}
	\large
	\noindent\textbf{#1}
	\normalsize
	\vspace{20pt}
}

\newcommand{\fukudai}[1]{
	\vspace{10pt}
	\noindent\textbf{#1}
	\vspace{10pt}
}

\newenvironment{bunshou}{
	\vspace{10pt}
	\begin{adjustwidth}{1cm}{3cm}
	\begin{linenumbers}
}{
	\end{linenumbers}
	\end{adjustwidth}
}

\newenvironment{reibun}{
	\vspace{10pt}
	\begin{tabular}{l l}
}{
	\end{tabular}
	\vspace{10pt}
}
\newcommand{\rei}[2]{
	#1&\textit{#2}\\
}
\newcommand{\reinagai}[2]{
	\multicolumn{2}{l}{#1}\\
	\multicolumn{2}{l}{\hspace{10pt}\textit{#2}}\\
}

\newenvironment{mondai}[1]{
	\vspace{10pt}
	#1
	
	\begin{enumerate}
		\itemsep-5pt
	}{
	\end{enumerate}
	\vspace{10pt}
}

\newenvironment{hyou}{
	\begin{itemize}
		\itemsep-5pt
	}{
	\end{itemize}
	\vspace{10pt}
}

\date{\today}

\CJKfamily{chanto}\CJKnospace
\author{Tomislav Mamić}
\begin{document}
	\dai{Domaća zadaća - ごだん glagoli}
	
	Dopunite tablicu ispod ispravnim oblicima glagola. One glagole kojima ne znate značenje potražite u rječniku i dopišite sa strane.
	
	\vspace{5pt}
	\begin{tabular}{|l|l|l|l|}
		\hline
		rječnički oblik & poz. pr. & neg. nepr. & neg. pr.\\
		\hline
		\multicolumn{4}{|c|}{\textasciitilde ごだん\textasciitilde}\\
		\hline
		&さした&&\\
		&&わからない&\\
		&&&しななかった\\
		かう&&&\\
		&&とばない&\\
		&&&たたなかった\\
		&かいた&&\\
		よむ&&&\\
		&およいだ&&\\
		\hline
		\multicolumn{4}{|c|}{\textasciitilde いちだん\textasciitilde}\\
		\hline
		&みた&&\\
		きる&&&\\
		&&かえない&\\
		&&&かんがえなかった\\
		\hline
		\multicolumn{4}{|c|}{\textasciitilde nepravilni\textasciitilde}\\
		\hline
		&&こない&\\
		&&&なかった\\
		&いった&&\\
		する&&&\\
		\hline
	\end{tabular}

	\newpage
	Prevedite sljedeće rečenice.
	
	\begin{mondai}{Lv. 1}
		\item ともだちと あそんだ。
		\item こうえんに いる。
		\item べんきょうしなかった。
		\item せんせいに きいた。
	\end{mondai}

	\begin{mondai}{Lv. 2}
		\item こうえんで ともだちと あそんだ。
		\item ねこは こうえんに いる。
		\item さつき\footnotemark[1]ちゃんは べんきょうしなかった。
		\item それを せんせいに きいた。
	\end{mondai}
	
	\footnotetext[1]{Žensko ime.}
	
	\begin{mondai}{Lv. 3}
		\item さつきちゃんは こうえんで ともだちと てまりで あそんだ。
		\item あの くろい ねこは ちかくの こうえんに いる。
		\item さつきちゃんも たけし\footnotemark[2]くんも べんきょうしなかった。
		\item たけしくんは てんこうすると せんせいに きいた。
	\end{mondai}

	\footnotetext[2]{Muško ime.}
\end{document}