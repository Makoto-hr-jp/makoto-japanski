% !TeX document-id = {c6e61070-f072-4c33-a3c0-f2b98a26b5bc}
% !TeX program = xelatex ?me -synctex=0 -interaction=nonstopmode -aux-directory=../tex_aux -output-directory=./release
% !TeX program = xelatex

\documentclass[12pt]{article}

\usepackage{lineno,changepage,lipsum}
\usepackage[colorlinks=true,urlcolor=blue]{hyperref}
\usepackage{fontspec}
\usepackage{xeCJK}
\usepackage{tabularx}
\setCJKfamilyfont{chanto}{AozoraMinchoRegular.ttf}
\setCJKfamilyfont{tegaki}{Mushin.otf}
\usepackage[CJK,overlap]{ruby}
\usepackage{hhline}
\usepackage{multirow,array,amssymb}
\usepackage[croatian]{babel}
\usepackage{soul}
\usepackage[usenames, dvipsnames]{color}
\usepackage{wrapfig,booktabs}
\renewcommand{\rubysep}{0.1ex}
\renewcommand{\rubysize}{0.75}
\usepackage[margin=50pt]{geometry}
\modulolinenumbers[2]

\usepackage{pifont}
\newcommand{\cmark}{\ding{51}}%
\newcommand{\xmark}{\ding{55}}%

\definecolor{faded}{RGB}{100, 100, 100}

\renewcommand{\arraystretch}{1.2}

%\ruby{}{}
%$($\href{URL}{text}$)$

\newcommand{\furigana}[2]{\ruby{#1}{#2}}
\newcommand{\tegaki}[1]{
	\CJKfamily{tegaki}\CJKnospace
	#1
	\CJKfamily{chanto}\CJKnospace
}

\newcommand{\dai}[1]{
	\vspace{20pt}
	\large
	\noindent\textbf{#1}
	\normalsize
	\vspace{20pt}
}

\newcommand{\fukudai}[1]{
	\vspace{10pt}
	\noindent\textbf{#1}
	\vspace{10pt}
}

\newenvironment{bunshou}{
	\vspace{10pt}
	\begin{adjustwidth}{1cm}{3cm}
	\begin{linenumbers}
}{
	\end{linenumbers}
	\end{adjustwidth}
}

\newenvironment{reibun}{
	\vspace{10pt}
	\begin{tabular}{l l}
}{
	\end{tabular}
	\vspace{10pt}
}
\newcommand{\rei}[2]{
	#1&\textit{#2}\\
}
\newcommand{\reinagai}[2]{
	\multicolumn{2}{l}{#1}\\
	\multicolumn{2}{l}{\hspace{10pt}\textit{#2}}\\
}

\newenvironment{mondai}[1]{
	\vspace{10pt}
	#1
	
	\begin{enumerate}
		\itemsep-5pt
	}{
	\end{enumerate}
	\vspace{10pt}
}

\newenvironment{hyou}{
	\begin{itemize}
		\itemsep-5pt
	}{
	\end{itemize}
	\vspace{10pt}
}

\date{\today}

\CJKfamily{chanto}\CJKnospace
\author{Tomislav Mamić, Željka Ludošan}
\usepackage{graphicx}
\usepackage{caption}

\begin{document}
	\dai{Osobne zamjenice i oslovljavanje ljudi; čestice と i や}
	
	\dai{Osobne zamjenice}
		
	Izražavanje prvog lica jednine(ja):
	\begin{table}[!h]	
	\begin{tabular}{l l p{400pt}}
		\toprule[2pt]
		Izraz		&Spol	& Ton i/ili situacija u kojoj se koristi\\
		\midrule
		わたし		&M i Ž	&neutralan, formalan\\
		わたくし	&M i Ž	&jedan od najformalnijih oblika\\
		わたくしめ	&M i Ž	&još formalnije od わたくし i izražava poniznost\\
		あたし		&Ž	&može zvučati slatko pa ga češće koriste mlađe žene\\
		ぼく		&M	&donekle neformalni muški, konotaciju djetinjatosti dobije jedino ako je u hiragani\\
		じぶん		&M i Ž	&pojavljuje se češće zadnje vrijeme u modernom japanskom\\
		うち		&M i Ž	&M koriste kada pričaju o svom unutarnjem krugu u malo neformalnijim okolnostima, ali većinom Ž\\
		\bottomrule[2pt]
	\end{tabular}
	\end{table}

	
	Izražavanje drugog lica jednine(ti) za muškarce i žene:
	\begin{table}[!h]
	\begin{tabular}{l l}
		\toprule[2pt]
		あなた&neutralni, kada se piše kanji znakovima tada je formalan\\
		\bottomrule[2pt]
	\end{tabular}
	\end{table}
		
	あなた koristimo za osobe čije ime ne znamo, žene ponekad od milja koriste あなた kad se obraćaju svome mužu.
	
	Korištenje じぶん kao osobne zamjenice 1. lica je u zadnje vrijeme uzelo maha. Prije se koristila u edo periodu uglavnom među vojnicima, a u modernom jeziku se u kansaiju povremeno koristi kao zamjenica za 2. lice. To sve skupa ovu upotrebu čini dosta nespretnom i trebalo bi je izbjegavati.
	
	\vspace{10pt}
	
	U japanskom jeziku, kada govorimo o sebi, najčešće koristimo neku osobnu zamjenicu poput わたし, a kad govorimo o drugim osobama tada češće koristimo njihovo ime koje na kraju ima priljepljen nastavak (npr.-さん,ともきさん) koji određuje relativnu poziciju u društvenoj hijerarhiji i međusobni odnos.
	
	\vspace{10pt}
	Ako znamo ime osobe, tada koristimo njeno ime. 
	
	\vspace{10pt}
	Pristojnije je reći:
	\begin{reibun}
		\rei{\underline{ともきさん}、すいか が すき?}{Tomoki, voliš lubenice?}
	\end{reibun}
	
	nego
	
	\begin{reibun}
		\rei{\underline{あなた}、すいか が すき?\footnotemark[1]}{Ti voliš lubenice?}
	\end{reibun}
	
	\begin{table}
	\begin{tabular}{ l l l}
		\toprule[2pt]
		かれ&on, dečko&neutralni muški\\
		かのじょ&ona, cura&neutralni ženski\\
		\bottomrule[2pt]
	\end{tabular}
	\end{table}

	\footnotetext[1]{Ovo zvuči kao da žena pita muža - \textit{Dragi, voliš li lubenice?}}
	\newpage

	かれ i かのじょ koristimo kada pričamo o trećoj osobi čije ime ne znamo ili nam je ta osoba glavna tema priče pa ime ne moramo ponavljati. Nikad ne koristimo za osobe koje ne poznamo jer izražavaju bliskost. Umjesto toga koristimo あのかた、あのひと...Osim toga, かれ i かのじょ može značiti "cura" i "dečko" u intimnom smislu.
	
	\begin{reibun}
		\rei{\underline{かれ} の にっき}{njegov dnevnik}
		\rei{\underline{わたし} の \underline{かのじょ} の にっき}{dnevnik moje cure}
	\end{reibun}

	Izražavanje drugog lica jednine(ti) na pogrdne ili neprikladne načine: 
	\begin{table}[!h]
	\begin{tabular}{l l l}
		\toprule[2pt]
		Izraz 			&Spol	&Ton i/ili situacija u kojoj se koristi\\ 
		\midrule
		そなた			&M i Ž	&pjesnički „ti“, arhaizam\\
		きみ\footnotemark[2]	&M i Ž	&primarno se koristi kad se stariji obraća puno mlađem od sebe\\
		あんた			&Ž	&arogantniji, češće koriste Ž nego M\\
		おまえ			&M	&pogrdno, ponekad se koristi među muškim prijateljima\\
		てめぇ			&M i Ž 	&pogrdno\\
		きさま			&M i Ž 	&pogrdno, nekoć bio izraz keiga\footnotemark[3]\\
		\bottomrule[2pt]
	\end{tabular}
	\end{table}
	
	
		Ako razgovaramo sa osobom koju smatramo inferiornom ili na istoj razini tada možemo koristiti inferiorne osobne zamjenice, uz napomenu da one mogu zvučati poprilično nepristojno i uvrijediti osobu.
		
	Izražavanje prvog lica jednine(ja) u situacijama koje zahtjevaju pristojnost prema sugovorniku: 
	\begin{table}[!h]	
	\begin{tabular}{l l p{400pt}}
		\toprule[2pt]
		Izraz		&Spol	&Ton i/ili situacija u kojoj se koristi\\
		\midrule
		わたくし	&M i Ž	&formalni\\
		おれ		&M	&najneformalniji muški, u hiragani ima konotaciju kao da osnovnoškolac priča\\
		わし		&M	&gotovo izumrijela riječ, koriste stariji ljudi samo u blizini Hiroshime (ne koriste ju više ni stariji ljudi), sada se koristi kao zezancija i u mangama\\
		わい		&M i Ž 	&koristi se samo u šali\\
		われ		&M i Ž	&najčešće se koristi u pisanom tekstu ili formalnim okolnostima kada se obraća grupi\\
		\bottomrule[2pt]
	\end{tabular}
	\end{table}

	Izražavanje prvog lica množine(mi) za muškarce i žene u formalnim situacijama:
	\begin{table}[!h]
		\begin{tabular}{l p{400pt}}
			\toprule[2pt]
			われわれ&koristi se u formalnim situacijama kada jedna osoba predstavlja grupu, također se može koristiti i われら ali je manje formalno\\
			\bottomrule[2pt]
	\end{tabular}
	\end{table}

	\footnotetext[2]{Osoba koja govori きみ je iznad osobe za koju se to izgovara, ponekad きみ koriste muške osobe kada se obraćaju svojoj jako bliskoj prijateljici}
	\footnotetext[3]{Keigo je napredni pristojni oblik koji se koristi kako bismo iskazali poštovanje i poniznost} 

	\newpage
	
	われ i われわれ koriste tvrtke u poslovnim mail-ovima.\newline
	
	
	Bonus:
	\begin{tabular}{l l l}
		\toprule[2pt]
		せっしゃ&ja&koristili su šoguni\\
		\bottomrule[2pt]
	\end{tabular}
	
	\dai{Nastavci za imena}

	\ten \fukudai{Nastavci vezani uz društevni položaj}

	-さん	koristimo za nepoznate osobe i osobe prema kojima želimo biti pristojni, te općenito za odrasle osobe. Neke imenice već u sebi imaju nastavak さん i obično iskazuju poštovanje prema onom na što se odnose.
	
	\begin{table}[!h]	
	\begin{tabular}{l l}
		\toprule[2pt]
		Izraz&Osoba koja ga koristi\\
		\midrule
		おじょうさん&kćer, mlada dama\\
		みなさん&svi\\
		おおやさん&stanodavac, stanodavka\\
		\bottomrule[2pt]
	\end{tabular}
	\end{table}

	Ovisno o tome govorimo li o ljudima iz naše ili iz tuđe obitelji trebamo koristiti prikladne nazive za članove obitelji. 
	
	\begin{table}[!h]
	\begin{tabular}{l l l l l l l l}
		\toprule[2pt]
		unutra& &van&unutra& &van\\
		ちち&tata&おとうさん&そふ&djed&おじいさん\\
		はは&mama&おかあさん&そぼ&baka&おばあさん\\
		あに、あにき&stariji brat&おにいさん&おじ&stric&おじさん\\
		あね、あねき&starija sestra&おねえさん&おば&teta&おばさん\\
		おとうと&mlađi brat&おとうとさん&おっと&muž&ごしゅじん\\
		いもうと&mlađa sestra&いもうとさん&つま&žena&おくさん\\
		\bottomrule[2pt]
	\end{tabular}
	\end{table}
	
	-さま koristimo za osobe prema kojima želimo iskazati posebno poštovanje. Koristimo li -さま govoreći o sebi, ostavljamo dojam da imamo jako visoko mišljenje o sebi i da smo nepristojni. 

	\begin{table}[!h]
	\begin{tabular}{l l l}
		\toprule[2pt]
		みなさま&(poštovani) svi\\
		おきゃくさま&(poštovani) gost, kupac, klijent\\
		おかあさま&(poštovana) majka\\
		ひめさま&princeza\\
		かみさま&bog\\
		\bottomrule[2pt]
	\end{tabular}
	\end{table}
	\newpage	
		\fukudai{Intimiziranje stvari sa さん i さま}
	
	Japanci ponekad koriste nastavke さん i さま kako bi pokazali poštovanje prema ljudima koji obavljaju neko zanimanje ili opisali nešto što im je drago ili blisko srcu.
	
	\begin{table}[!h]
	\begin{tabular}{l l}
		\toprule[2pt]
		ケーキやさん&simpatična slastičarnica koja nam je draga\\
		てんちょうさん&voditelj trgovine koji se brine za trgovinu i proizvode\\
		おまわりさん&policajac koji patrolira ulicom i održava mir\\
		おひさま&sunašce\\
		\bottomrule[2pt]
	\end{tabular}
	\end{table}


	-ちゃん zvuči djetinjasto, slatko i možemo koristiti među prijateljima. Kada pridodamo ちゃん na kraju nečijeg imena to označava da nam je ta osoba posebno draga. Najčešće se pridodaje curama.\newline
	
	
	-くん se koristi među prijateljima ili za mlađu osobu. Najčešće se koristi na kraju imena muških osoba, ali ako se koristi za cure tada zvuči ozbiljnije i kao iskaz poštovanja. Često se u poslovnom okruženju nadređeni ovako obraćaju zaposlenicama.\newline
	

Odrasle osobe će za svoje prijatelje najčešće koristiti -さん.\newline

	
	-たち dodajemo na osobne zamjenice ili direktno na ime kada želimo govoriti o grupi ljudi i to na način da ga dodamo na ime osobe ili osobnu zamjenicu koja tu grupu predstavlja. Neformalni nastavak od -たち je -ら i koristi se na jednak način, ali zbog toga što zvuči sirovije i koristi se na razne načine, najsigurnije je koristiti -たち.


	\begin{reibun}
		\rei{\underline{ともこたち} と いく。}{Idem s Tomoko i njezinima...}
		\rei{\underline{かのじょたち} が さけんだ。}{Zavrištale su.}
		\rei{\underline{かれら} が みえない。}{Ne vidim ih.}
		\rei{\underline{ぼくら} が そうじ する。}{Mi ćemo počistiti.}
	\end{reibun}

	\ten \fukudai{Nastavci vezani uz zvanja}
	
	Određena zvanja ili pozicije u društvu možemo koristiti kao nastavke na kraju imena, ali i kao imenice same za sebe. U tom slučaju se drugi nastavci na njih ne dodaju.
	
	\begin{table}[!h]	
	\begin{tabular}{l l}
		\toprule[2pt]
		しゃちょう&vlasnik tvrtke\\
		ぶちょう&direktor odjela\\
		しゅにん&glavna odgovorna osoba, šef\\
		せんせい&učitelj, doktor\\
		せんぱい&iskusnija osoba od koje učite\\
		てんちょう&voditelj trgovine\\
		\bottomrule[2pt]
	\end{tabular}
	\end{table}
	
	
	\begin{reibun}
		\rei{\underline{たなかしゅにん}、しょるい を どうぞ!}{Šefe Tanaka, izvolite papire!}
		\rei{\underline{おのせんせい} は げんき です か?}{Je li učitelj Ono u redu?}
		\rei{\underline{けいこちゃん} は \underline{まつだせんぱい} が きらい。}{Keiko mrzi Matsudu.}
	\end{reibun}

\newpage

\dai{Čestice za nabrajanje: と i や}


Česticu と koristimo za nabrajanje stvari ili ljudi na način da stavimo と između imenica ili ljudi koje nabrajamo. Kada nabrajamo moramo nabrojati sve stvari koje su u skupu koji definiramo. 


	\begin{reibun}
		\rei{Kakva pića voli Matsuda?}{}
		\rei{ぎゅうにゅう と おちゃ と みず。}{Mlijeko, zeleni čaj i vodu.}
	\end{reibun}
	
	\noindent Matsuda voli točno ta tri pića.
		
	\begin{reibun}
		\rei{Tko to ide u dućan?}{}
		\rei{まつだ と ももこ。}{Matsuda i Momoko.}
	\end{reibun}
	

Česticu や također koristimo za nabrajanje imenica, ali njome nabrajamo stvari koje samo predstavljaju dio u skupini stvari o kojoj pričamo.

	\begin{reibun}
		\rei{Kakva pića ne voli Matsuda?}{}
		\rei{さけ\footnotemark[4] や ぶどうしゅ や あまみず。}{Pića kao što su sake, vino i kišnica.}
	\end{reibun}

	Matsuda ne voli ta tri pića i druga slična njima.


\footnotetext[4]{さけ je naziv za japansko vino od riže, ali može općenito značiti bilo kakvo alkoholno piće.}
	
\end{document}
