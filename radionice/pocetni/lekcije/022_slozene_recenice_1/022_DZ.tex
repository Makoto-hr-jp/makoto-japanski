% !TeX document-id = {8fdd76bc-b2ff-40b6-8cc2-b7f72c26888d}
% !TeX program = xelatex ?me -synctex=0 -interaction=nonstopmode -aux-directory=../tex_aux -output-directory=./release
% !TeX program = xelatex

\documentclass[12pt]{article}

\usepackage{lineno,changepage,lipsum}
\usepackage[colorlinks=true,urlcolor=blue]{hyperref}
\usepackage{fontspec}
\usepackage{xeCJK}
\usepackage{tabularx}
\setCJKfamilyfont{chanto}{AozoraMinchoRegular.ttf}
\setCJKfamilyfont{tegaki}{Mushin.otf}
\usepackage[CJK,overlap]{ruby}
\usepackage{hhline}
\usepackage{multirow,array,amssymb}
\usepackage[croatian]{babel}
\usepackage{soul}
\usepackage[usenames, dvipsnames]{color}
\usepackage{wrapfig,booktabs}
\renewcommand{\rubysep}{0.1ex}
\renewcommand{\rubysize}{0.75}
\usepackage[margin=50pt]{geometry}
\modulolinenumbers[2]

\usepackage{pifont}
\newcommand{\cmark}{\ding{51}}%
\newcommand{\xmark}{\ding{55}}%

\definecolor{faded}{RGB}{100, 100, 100}

\renewcommand{\arraystretch}{1.2}

%\ruby{}{}
%$($\href{URL}{text}$)$

\newcommand{\furigana}[2]{\ruby{#1}{#2}}
\newcommand{\tegaki}[1]{
	\CJKfamily{tegaki}\CJKnospace
	#1
	\CJKfamily{chanto}\CJKnospace
}

\newcommand{\dai}[1]{
	\vspace{20pt}
	\large
	\noindent\textbf{#1}
	\normalsize
	\vspace{20pt}
}

\newcommand{\fukudai}[1]{
	\vspace{10pt}
	\noindent\textbf{#1}
	\vspace{10pt}
}

\newenvironment{bunshou}{
	\vspace{10pt}
	\begin{adjustwidth}{1cm}{3cm}
	\begin{linenumbers}
}{
	\end{linenumbers}
	\end{adjustwidth}
}

\newenvironment{reibun}{
	\vspace{10pt}
	\begin{tabular}{l l}
}{
	\end{tabular}
	\vspace{10pt}
}
\newcommand{\rei}[2]{
	#1&\textit{#2}\\
}
\newcommand{\reinagai}[2]{
	\multicolumn{2}{l}{#1}\\
	\multicolumn{2}{l}{\hspace{10pt}\textit{#2}}\\
}

\newenvironment{mondai}[1]{
	\vspace{10pt}
	#1
	
	\begin{enumerate}
		\itemsep-5pt
	}{
	\end{enumerate}
	\vspace{10pt}
}

\newenvironment{hyou}{
	\begin{itemize}
		\itemsep-5pt
	}{
	\end{itemize}
	\vspace{10pt}
}

\date{\today}

\CJKfamily{chanto}\CJKnospace
\author{Tomislav Mamić}
\begin{document}
	\dai{Domaća zadaća - Složene rečenice}
	
	\begin{mondai}{Prevedite na hrvatski koristeći se pritom svim mogućim prljavim trikovima, uključujući i grupni rad!}
		\item となりの おばさんは さびしかったから ねこを ひろった。
		
		さびしい - \textit{usamljen}, ひろう - \textit{uzeti, pokupiti} (s ceste)
		
		\item ねこを ひろっても さびしかったので もう いっぴき ひろった。
		\item それが つづいて、いまは ねこを じゅっぴき かっている。
		
		かっている - \textit{imati, brinuti se za} (životinje)
		
		\item むすこが とうきょうに 行ってから おばさんは さびしく なった。
		
		むすこ - \textit{sin}
		
		\item そう ははに きいた わたしは かわいそうだと おもった。
		
		かわいそう(な) - \textit{jadno} (sažaljenje)
		
		\item なのに となりの おばさんは いつも えがおだから、そんなに さびしくないと おもう。
		
		えがお - \textit{osmjeh,} dosl. \textit{nasmijano lice}, そんなに - \textit{tako, toliko} (izraz)
		
		\item わたしは ねこが すきだけど、じゅっぴきは かわない。
		
		\item だが、としを とって さびしく なった おばさんは、かっています。
		
		としを とる - \textit{ostariti} (izraz)
	\end{mondai}
\end{document}