% !TeX document-id = {ca5b62f0-5be9-4410-90f6-61ae462b4ff7}
% !TeX program = xelatex ?me -synctex=0 -interaction=nonstopmode -aux-directory=../tex_aux -output-directory=./release
% !TeX program = xelatex

\documentclass[12pt]{article}

\usepackage{lineno,changepage,lipsum}
\usepackage[colorlinks=true,urlcolor=blue]{hyperref}
\usepackage{fontspec}
\usepackage{xeCJK}
\usepackage{tabularx}
\setCJKfamilyfont{chanto}{AozoraMinchoRegular.ttf}
\setCJKfamilyfont{tegaki}{Mushin.otf}
\usepackage[CJK,overlap]{ruby}
\usepackage{hhline}
\usepackage{multirow,array,amssymb}
\usepackage[croatian]{babel}
\usepackage{soul}
\usepackage[usenames, dvipsnames]{color}
\usepackage{wrapfig,booktabs}
\renewcommand{\rubysep}{0.1ex}
\renewcommand{\rubysize}{0.75}
\usepackage[margin=50pt]{geometry}
\modulolinenumbers[2]

\usepackage{pifont}
\newcommand{\cmark}{\ding{51}}%
\newcommand{\xmark}{\ding{55}}%

\definecolor{faded}{RGB}{100, 100, 100}

\renewcommand{\arraystretch}{1.2}

%\ruby{}{}
%$($\href{URL}{text}$)$

\newcommand{\furigana}[2]{\ruby{#1}{#2}}
\newcommand{\tegaki}[1]{
	\CJKfamily{tegaki}\CJKnospace
	#1
	\CJKfamily{chanto}\CJKnospace
}

\newcommand{\dai}[1]{
	\vspace{20pt}
	\large
	\noindent\textbf{#1}
	\normalsize
	\vspace{20pt}
}

\newcommand{\fukudai}[1]{
	\vspace{10pt}
	\noindent\textbf{#1}
	\vspace{10pt}
}

\newenvironment{bunshou}{
	\vspace{10pt}
	\begin{adjustwidth}{1cm}{3cm}
	\begin{linenumbers}
}{
	\end{linenumbers}
	\end{adjustwidth}
}

\newenvironment{reibun}{
	\vspace{10pt}
	\begin{tabular}{l l}
}{
	\end{tabular}
	\vspace{10pt}
}
\newcommand{\rei}[2]{
	#1&\textit{#2}\\
}
\newcommand{\reinagai}[2]{
	\multicolumn{2}{l}{#1}\\
	\multicolumn{2}{l}{\hspace{10pt}\textit{#2}}\\
}

\newenvironment{mondai}[1]{
	\vspace{10pt}
	#1
	
	\begin{enumerate}
		\itemsep-5pt
	}{
	\end{enumerate}
	\vspace{10pt}
}

\newenvironment{hyou}{
	\begin{itemize}
		\itemsep-5pt
	}{
	\end{itemize}
	\vspace{10pt}
}

\date{\today}

\CJKfamily{chanto}\CJKnospace
\usepackage{wasysym}
\author{Tomislav Mamić}
\begin{document}
	\dai{Složene rečenice}
	
	\fukudai{Teorija}
	
	Rečenice možemo slagati tako da budu zavisne (複文 - ふく.ぶん) ili nezavisne (重文 - じゅう.ぶん). Razlika između jednog i drugog je u hijerarhiji - nezavisno složene rečenice ne gube na značenju ako ih razdvojimo, ali između njih postoji neki odnos, no kod zavisno složenih rečenica, jedna je drugoj podređena. U pravilu, podređena rečenica će odgovarati na pitanje koje možemo formirati preoblikujući nadređenu. Pogledajmo neke primjere:
	
	\begin{reibun}
		\rei{ばんごはんを たべて ねた。}{Pojeo sam večeru i zaspao.}
		\rei{くろくて あしの はやい ねこを 見た。}{Vidio sam brzu crnu mačku.}
	\end{reibun}

	Prva rečenica je nezavisno složena - dvije jednostavne rečenice od kojih se sastoji ne govore ništa jedna o drugoj. U drugoj rečenici, koja u hrvatskom jeziku zapravo i nije složena, ali u japanskom jest, prva podrečenica (くろくて あしがはやい) govori nešto o dijelu druge (ねこ) pa ćemo za nju reći da je podređena drugoj. Za provjeru, možemo pitati どんなねこを見たか.
	
	\fukudai{Veznici s predikatnim oblikom}
	
	Gramatički, veznicima u nastavku je zajedničko to da im prethodi predikat. Za svaki veznik dana su po tri primjera - jedan sa svakom od tri vrste predikata na što je korisno obratiti pozornost.
	
	\vspace{5pt}
	\ten から izriče razlog (\textit{jer}, zavisno slaganje)
	
	\begin{reibun}
		\reinagai{つまらないから えいがを 見ない。}{Ne gledam filmove jer su dosadni.}
		\reinagai{たまねぎを たくさん たべなかったから びょうきに なった。}{Razbolio sam se jer nisam jeo puno luka.}
		\reinagai{しずかだから だれも きづいてくれない。}{Nitko me ne primjećuje jer sam tih.}
	\end{reibun}

	Zapamtimo da se za ovu upotrebu から koristi uglavnom u govoru te da ponekad može zvučati grubo i odavati govornikovu emocionalnu uključenost u ono o čemu se govori\footnotemark[1].
	
	\footnotetext[1]{Što je u kontekstu japanske kulture upravo grubo jer se osjećaji ne pokazuju.}
	
	\vspace{5pt}
	\ten けど・けれど・けれども su veznici suprotnih rečenica (\textit{ali}, nezavisno slaganje)
	
	\begin{reibun}
		\reinagai{うさぎは はやいけど、きつねも はやい。}{Zečevi su brzi, ali i lisice su brze.}
		\reinagai{たけしくんに たべてくださいと いったけど、たべなかった。}{Rekao sam Takešiju da pojede, ali nije (pojeo).}
		\reinagai{はなこさんは しずかだけど、わたしは きづいた。}{Hanako je tiha, ali ja sam je primijetio.}
	\end{reibun}

	Ovi se veznici koriste gotovo isključivo u govornom jeziku, a u pisanom zvuče neformalno i opušteno. Sve tri varijante su jednake po značenju, no けれど i けれども zvuče pristojnije.
	
	\vspace{5pt}
	\ten が je također veznik suprotnih rečenica (\textit{ali}, nezavisno slaganje)
	
	\begin{reibun}
		\reinagai{つまらないが、わたしは きらいじゃない。}{Dosadno je, ali meni nije mrsko.}
		\reinagai{きいてみましたが、せんせいも しりませんでした。}{Pitao sam, ali ni učitelj nije znao.}
		\reinagai{いまは しずかだが、あさは けっこう にぎやかだ。}{Sad je tiho, ali ujutro je prilično živahno.}
	\end{reibun}

	Po značenju, が je isto što i けど, ali formalnije i pretežno dio pisanog jezika.
	
	\fukudai{Veznici s opisnim oblikom}
	
	Za razliku od prethodne skupine, ovim veznicima prethodi opisni oblik. Uočljiva je razlika za imenice i な pridjeve! U konkretnim primjerima, razlog za opisni oblik je to što su nastali kraćenjem もの u の.
	
	\vspace{5pt}
	\ten ので izriče razlog (\textit{jer}, zavisno slaganje)
	
	\begin{reibun}
		\reinagai{つまらないので えいがを 見ません。}{Ne gledam filmove jer su dosadni.}
		\reinagai{たまねぎを たくさん たべなかったので びょうきに なった。}{Razbolio sam se jer nisam jeo puno luka.}
		\reinagai{しずかなので だれも きづいてくれない。}{Nitko me ne primjećuje jer sam tih.}
	\end{reibun}

	Po značenju jednak upotrebi から koju smo ranije naučili, no zvuči blaže, smirenije i pristojnije.
	
	\vspace{5pt}
	\ten のに kao veznik dopusnih rečenica (\textit{iako, unatoč}, zavisno slaganje)
	
	\begin{reibun}
		\reinagai{あたまが いたいのに がっこうへ いった。}{Otišao sam u školu iako me boli glava.}
		\reinagai{たまねぎを たくさん たべたのに びょうきに なった。}{Razbolio sam se iako sam jeo puno luka.}
		\reinagai{ふゆなのに アイスクリームが たべたい。}{Iako je zima, jede mi se sladoled.}
	\end{reibun}

	Ovaj se veznik jednako koristi i u formalnim i u neformalnim situacijama.
	
	\fukudai{Veznici s て oblikom}
	
	Veznicima iz ove skupine prethodi predikat u て obliku.
	
	\vspace{5pt}
	\ten から govori \textit{nakon čega} se događa glavna rečenica (zavisno slaganje)
	
	\begin{reibun}
		\reinagai{えいがを みてから、つまらないと おもった。}{Nakon što sam pogledao film, pomislio sam kako je bezveze.}
		\reinagai{みずにおちてから けっこう ふかいと わかった。}{Nakon što sam upao u vodu, shvatio sam da je prilično duboka.}
	\end{reibun}

	Kao što se da vidjeti iz priloženog, tumačenje から jako varira ovisno o tome što se nalazi ispred. Iz tog je razloga korisno uložiti nešto vremena u razumijevanje vrsta raznih oblika i dijelova rečenice. Uočimo da ova upotreba s imenskim predikatima koji uvijek izriču \textbf{stanje} nema smisla jer je zavisna rečenica \textbf{događaj} koji prethodi glavnoj.
	
	\vspace{5pt}
	\ten も izriče hipotetsku prepreku glavnoj rečenici (zavisno slaganje)
	
	\begin{reibun}
		\reinagai{あたまが いたくても がっこうへ いく。}{Idem u školu makar me boljela glava.}
		\reinagai{たまねぎを たくさん たべても びょうきに なる。}{Razbolit ćeš se čak i ako budeš jeo puno luka.}
		\reinagai{ふゆでも アイスクリームが たべたい ひは ある。}{Ima dana kad mi se jede sladoled iako je zima.}
	\end{reibun}

	Po značenju, ovaj veznik je vrlo blizak のに - oba izražavaju prepreku unatoč kojoj se glavna rečenica događa. Međutim, のに se koristi za stvarne prepreke, a も u situacijama gdje prepreka ne postoji nužno. Posljedično ćemo ovu upotrebu も češće čuti u neprošlim rečenicama, a のに u prošlosti, iako to nije gramatičko pravilo.
	
	\fukudai{Kontekstualni oblici veznika}
	
	U govornom jeziku, vrlo je čest slučaj da se izrečene misli mijenjaju i prekidaju pa se isto događa i izgovorenim rečenicama. Zbog toga veznici često imaju "kontekstualni oblik" kojim započinjemo rečenicu, nadovezujući se na ono što smo prethodno rekli. Ti su oblici nastali stavljanjem spojnog glagola pred veznike:
	
	\begin{reibun}
		\rei{あたまが いたい。}{Boli me glava. \normalfont{(kontekst)}}
		\rei{だから びょういんへ いった。}{Zato sam otišao u bolnicu.}
		\rei{だけど がっこうへ いった。}{Ali, otišao sam u školu.}
		\rei{だが、がっこうへ いった。}{Ali, otišao sam u školu.}
		\rei{なので びょういんへ いった。}{Zato sam otišao u bolnicu.}
		\rei{なのに がっこうへ いった。}{Ali ipak sam otišao u školu.}
		\rei{でも、がっこうへ いった。}{Ali, otišao sam u školu.}
	\end{reibun}

	To je jako puno \textit{ali} koje treba naučiti razlikovati \smiley. Sjetimo se da ne postoji preslikavanje \textit{jedan za jedan} između hrvatskih i japanskih veznika i da to što u hrvatskom često različite oblike japanskog prevedemo na sličan ili isti način ne znači da u japanskom među njima nema razlike.
	
	Pokušajmo se usredotočiti na shvaćanje što rečenice znače \textbf{na japanskom}!
	
\end{document}