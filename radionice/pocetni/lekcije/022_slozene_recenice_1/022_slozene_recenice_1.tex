% !TeX document-id = {ca5b62f0-5be9-4410-90f6-61ae462b4ff7}
% !TeX program = xelatex ?me -synctex=0 -interaction=nonstopmode -aux-directory=../tex_aux -output-directory=./release
% !TeX program = xelatex

\documentclass[12pt]{article}

\usepackage{lineno,changepage,lipsum}
\usepackage[colorlinks=true,urlcolor=blue]{hyperref}
\usepackage{fontspec}
\usepackage{xeCJK}
\usepackage{tabularx}
\setCJKfamilyfont{chanto}{AozoraMinchoRegular.ttf}
\setCJKfamilyfont{tegaki}{Mushin.otf}
\usepackage[CJK,overlap]{ruby}
\usepackage{hhline}
\usepackage{multirow,array,amssymb}
\usepackage[croatian]{babel}
\usepackage{soul}
\usepackage[usenames, dvipsnames]{color}
\usepackage{wrapfig,booktabs}
\renewcommand{\rubysep}{0.1ex}
\renewcommand{\rubysize}{0.75}
\usepackage[margin=50pt]{geometry}
\modulolinenumbers[2]

\usepackage{pifont}
\newcommand{\cmark}{\ding{51}}%
\newcommand{\xmark}{\ding{55}}%

\definecolor{faded}{RGB}{100, 100, 100}

\renewcommand{\arraystretch}{1.2}

%\ruby{}{}
%$($\href{URL}{text}$)$

\newcommand{\furigana}[2]{\ruby{#1}{#2}}
\newcommand{\tegaki}[1]{
	\CJKfamily{tegaki}\CJKnospace
	#1
	\CJKfamily{chanto}\CJKnospace
}

\newcommand{\dai}[1]{
	\vspace{20pt}
	\large
	\noindent\textbf{#1}
	\normalsize
	\vspace{20pt}
}

\newcommand{\fukudai}[1]{
	\vspace{10pt}
	\noindent\textbf{#1}
	\vspace{10pt}
}

\newenvironment{bunshou}{
	\vspace{10pt}
	\begin{adjustwidth}{1cm}{3cm}
	\begin{linenumbers}
}{
	\end{linenumbers}
	\end{adjustwidth}
}

\newenvironment{reibun}{
	\vspace{10pt}
	\begin{tabular}{l l}
}{
	\end{tabular}
	\vspace{10pt}
}
\newcommand{\rei}[2]{
	#1&\textit{#2}\\
}
\newcommand{\reinagai}[2]{
	\multicolumn{2}{l}{#1}\\
	\multicolumn{2}{l}{\hspace{10pt}\textit{#2}}\\
}

\newenvironment{mondai}[1]{
	\vspace{10pt}
	#1
	
	\begin{enumerate}
		\itemsep-5pt
	}{
	\end{enumerate}
	\vspace{10pt}
}

\newenvironment{hyou}{
	\begin{itemize}
		\itemsep-5pt
	}{
	\end{itemize}
	\vspace{10pt}
}

\date{\today}

\CJKfamily{chanto}\CJKnospace
\author{Tomislav Mamić}
\begin{document}
	\dai{Složene rečenice}
	
	\fukudai{Teorija}
	
	Rečenice možemo spajati ili gnijezditi. Razlika između jednog i drugog je u hijerarhiji - spojene rečenice su neovisne jedna o drugoj, ali između njih postoji neki odnos, no kad jednu rečenicu ugnijezdimo u drugu, ona joj postaje podređena. U pravilu, podređena rečenica će odgovarati na pitanje koje možemo formirati iz nadređene rečenice. Pogledajmo neke primjere:
	
	\begin{reibun}
		\rei{ばんごはんを たべて ねた。}{Pojeo sam večeru i zaspao.}
		\rei{くろくて あしの はやい ねこを 見た。}{Vidio sam brzu crnu mačku.}
	\end{reibun}

	Prva rečenica je nezavisno složena - dvije jednostavne rečenice od kojih se sastoji ne govore ništa jedna o drugoj. Reći ćemo da su te rečenice spojene. U drugoj rečenici (koja u hrvatskom jeziku zapravo i nije složena, ali u japanskom jest!), prva rečenica (くろくて あしがはやい) govori nešto o dijelu druge rečenice (ねこ) pa ćemo za nju reći da je ugniježđena u drugu. Za provjeru, možemo pitati どんなねこを見たか.
	
\end{document}