% !TeX document-id = {af30b7e4-e9ec-4b52-9be4-7a05c0561a1b}
% !TeX program = xelatex ?me -synctex=0 -interaction=nonstopmode -aux-directory=../../tex_aux -output-directory=./release
% !TeX program = xelatex

\documentclass[12pt]{article}

\usepackage{lineno,changepage,lipsum}
\usepackage[colorlinks=true,urlcolor=blue]{hyperref}
\usepackage{fontspec}
\usepackage{xeCJK}
\usepackage{tabularx}
\setCJKfamilyfont{chanto}{AozoraMinchoRegular.ttf}
\setCJKfamilyfont{tegaki}{Mushin.otf}
\usepackage[CJK,overlap]{ruby}
\usepackage{hhline}
\usepackage{multirow,array,amssymb}
\usepackage[croatian]{babel}
\usepackage{soul}
\usepackage[usenames, dvipsnames]{color}
\usepackage{wrapfig,booktabs}
\renewcommand{\rubysep}{0.1ex}
\renewcommand{\rubysize}{0.75}
\usepackage[margin=50pt]{geometry}
\modulolinenumbers[2]

\usepackage{pifont}
\newcommand{\cmark}{\ding{51}}%
\newcommand{\xmark}{\ding{55}}%

\definecolor{faded}{RGB}{100, 100, 100}

\renewcommand{\arraystretch}{1.2}

%\ruby{}{}
%$($\href{URL}{text}$)$

\newcommand{\furigana}[2]{\ruby{#1}{#2}}
\newcommand{\tegaki}[1]{
	\CJKfamily{tegaki}\CJKnospace
	#1
	\CJKfamily{chanto}\CJKnospace
}

\newcommand{\dai}[1]{
	\vspace{20pt}
	\large
	\noindent\textbf{#1}
	\normalsize
	\vspace{20pt}
}

\newcommand{\fukudai}[1]{
	\vspace{10pt}
	\noindent\textbf{#1}
	\vspace{10pt}
}

\newenvironment{bunshou}{
	\vspace{10pt}
	\begin{adjustwidth}{1cm}{3cm}
	\begin{linenumbers}
}{
	\end{linenumbers}
	\end{adjustwidth}
}

\newenvironment{reibun}{
	\vspace{10pt}
	\begin{tabular}{l l}
}{
	\end{tabular}
	\vspace{10pt}
}
\newcommand{\rei}[2]{
	#1&\textit{#2}\\
}
\newcommand{\reinagai}[2]{
	\multicolumn{2}{l}{#1}\\
	\multicolumn{2}{l}{\hspace{10pt}\textit{#2}}\\
}

\newenvironment{mondai}[1]{
	\vspace{10pt}
	#1
	
	\begin{enumerate}
		\itemsep-5pt
	}{
	\end{enumerate}
	\vspace{10pt}
}

\newenvironment{hyou}{
	\begin{itemize}
		\itemsep-5pt
	}{
	\end{itemize}
	\vspace{10pt}
}

\date{\today}

\CJKfamily{chanto}\CJKnospace
\author{autor}
\begin{document}
	\dai{Ciljevi i napomene za uvodno predavanje}
	
	S obzirom da se radi o uvodnom predavanju, nema nekih konkretnih stvari koje treba prenijeti polaznicima. Osnovna ideja je pričati o japanskom jeziku općenito i dati pregled onog što će se događati na radionicama.
	
	\fukudai{Napomene}
	\begin{hyou}
		\item naglasiti da japanski nije težak nego drugačiji, i da zato ne zahtjeva puno pameti nego puno rada
		\item na radionicama se u početku fokusiramo na gramatiku - cilj je razumjeti što je ispravno, a što ne i kako jezik funkcionira
		\item objasniti da svake godine najmanje pola ljudi ne doživi kraj radionica i da je razlog tome dobrim dijelom neredovit rad
		\item domaće zadaće su jako bitne, naglasiti da ih se predaje
		\item poticati komunikaciju i zatirati sram, reći svima da odmah kažu što ne razumiju
		\item potaknut ih da nauce i koriste japansko pismo od pocetka, hiraganu u pocetku, a kasnije kanji te da se ne uce na romaji
	\end{hyou}

	\fukudai{Zgodne pričice koje nisu na listiću}
	
	Samoglasnici su uglavnom slični hrvatskim, ali \textit{u} je malo drugačiji. Za razliku od hrvatskog koje je otvoreno, japansko \textit{u} je "spljošteno" - stražnji dio jezika je bliže nepcu.
	
	\vspace{10pt}
	Na neke glasove utječe to koji se glasovi nalaze oko njih, primjer \textit{h} u \textit{himna} vs. \textit{hangar}. Glas ん zapravo ima nekoliko varijanti ovisno o tome što je okolo, ali ne pokušavati objasniti koje su varijante i kada, samo dati primjer recimo みまん koji često postane \textit{mimam}. Glumce koji posuđuju glasove (pogotovo djevojke) uče da svaki ん na kraju riječi izgovaraju kao \textit{m} jer je više 可愛い.
	
	\vspace{10pt}
	Kroz povijest se dio glasova izgubio, pogotovo u narječjima koja je pregazila modernizacija jezika. Danas još uvijek postoje narječja koja imaju pokoji nestandardni glas, ali ih malo ljudi govori.
	
	\vspace{10pt}
	Hiraganu su u srednjem vijeku smislile i koristile dvorske žene. Često se kaže da dok su muškarci pisali na lošem kineskom, žene su pisale na dobrom japanskom. Priča o bolnom uvozu ideograma u jezik koji nije napravljen za ideograme.
	
	\vspace{10pt}
	Naglasiti kako je redoslijed poteza u japanskom neizmjerno bitan. Ako ima ljevaka, naglasiti da su neki potezi u hiragani jako neprirodni pa je tu okej pomalo varati (iako što manje to bolje), ali da u kanđiju nema varanja.
	
	\fukudai{Domaća zadaća}
	
	Zadati polaznicima da se informiraju o hiragani i glasovnom sustavu - neka pročitaju sami po internetu što mogu i nek na YT pokušaju poslušati izgovor glasova. Ako se osjećaju sposobno, slobodno neka i sami počnu učiti. Sljedeća lekcija će službeno pokriti pisanje i izgovor hiragane.
\end{document}
