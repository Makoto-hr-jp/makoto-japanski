% !TeX document-id = {aa1964cd-e4d1-4c8c-8134-6010424df1b4}
% !TeX program = xelatex ?me -synctex=0 -interaction=nonstopmode -aux-directory=../../tex_aux -output-directory=./release
% !TeX program = xelatex

\documentclass[12pt]{article}

\usepackage{lineno,changepage,lipsum}
\usepackage[colorlinks=true,urlcolor=blue]{hyperref}
\usepackage{fontspec}[ Path =../../../ ]
\usepackage{xeCJK}
\usepackage{tabularx}
\usepackage{graphicx}
\setCJKfamilyfont{chanto}{AOZORAMINCHOREGULAR_0.TTF}%
\setCJKfamilyfont{tegaki}{Mushin.otf}%
\usepackage[CJK,overlap]{ruby}
\usepackage{hhline}
\usepackage{multirow,array,amssymb}
\usepackage[croatian]{babel}
\usepackage{soul}
\usepackage[usenames, dvipsnames]{color}
\usepackage{wrapfig,booktabs}
\usepackage{calc}
\renewcommand{\rubysep}{0.1ex}
\renewcommand{\rubysize}{0.75}
\usepackage[margin=50pt]{geometry}
\usepackage{hyperref}
\modulolinenumbers[2]

\date{\today}

\usepackage{fancyhdr}
\pagestyle{fancy}
\fancyhf{}
\fancyhead[LE,RO]{\thepage}
\makeatletter
\fancyhead[RE,LO]{rev. \@date 誠}
\makeatother

\usepackage{pifont}
\newcommand{\cmark}{\ding{51}}%
\newcommand{\xmark}{\ding{55}}%

\newcommand{\dosl}{{\normalfont dosl. }}%
\newcommand{\rem}[1]{{\normalfont #1 }}%

\definecolor{faded}{RGB}{100, 100, 100}

\renewcommand{\arraystretch}{1.2}

%\ruby{}{}
%$($\href{URL}{text}$)$

\newcommand{\furigana}[2]{\ruby{#1}{#2}}
\newcommand{\tegaki}[1]{
	\CJKfamily{tegaki}\CJKnospace
	#1
	\CJKfamily{chanto}\CJKnospace
}

\newcommand{\dai}[1]{
	\vspace{20pt}
	\large
	\noindent\textbf{#1}
	\normalsize
	\vspace{20pt}
}

\newcommand{\fukudai}[1]{
	\vspace{10pt}
	\noindent\textbf{#1}
	\vspace{10pt}
}

\newenvironment{bunshou}{
	\vspace{10pt}
	\begin{adjustwidth}{1cm}{3cm}
	\begin{linenumbers}
}{
	\end{linenumbers}
	\end{adjustwidth}
}

\newenvironment{reibun}[1][]{
	\vspace{10pt}
	#1
	
	\begin{tabular}{l l}
}{
	\end{tabular}
	\vspace{10pt}
}
\newcommand{\rei}[2]{
	#1&\textit{#2}\\
}
\newcommand{\reinagai}[2]{
	\multicolumn{2}{l}{#1}\\
	\multicolumn{2}{l}{\hspace{10pt}\textit{#2}}\\
}

\newenvironment{mondai}[1]{
	\vspace{10pt}
	\noindent #1
	
	\begin{enumerate}
		\itemsep-5pt
	}{
	\end{enumerate}
}

\newenvironment{hyou}{
	\begin{itemize}
		\itemsep-5pt
	}{
	\end{itemize}
	\vspace{10pt}
}

\newcommand{\juuyou}[2][20pt]{
	\vspace{5pt}
		\noindent\hspace{#1}\parbox[c]{\textwidth-#1-#1}{\centering\textit{#2}}
	\vspace{5pt}
}

\newcommand{\ten}{
	\vspace{5pt}
	\noindent\hspace{-10pt}$\bullet$
}

\CJKfamily{chanto}\CJKnospace

\frenchspacing
\author{Tomislav Mamić}
\begin{document}
	\dai{Uvod u radionice japanskog jezika}
	
	\fukudai{Cilj radionica}
	
	Učenje japanskog je posao za cijeli život, a ove početne radionice trebale bi vam služiti kao pomoć u prvim koracima na tom putu. Osim uvoda u pismo, glasovni sustav i gramatiku, na radionicama ćemo se dotaknuti i kulturno-povijesnih tema, kao i dobrih strategija za samostalno učenje koje su primjenjive ne samo na japanski jezik nego i šire.
	
	\fukudai{Organizacija}
	
	Radionice su ugrubo podijeljene na teme koje sadrže nekoliko predavanja. Svaki tjedan započinje predavanjem koje služi kao uvod u samostalni rad. Na vama je odgovornost da preko tjedna uložite nešto svog vremena u vježbu i istraživanje novih stvari o kojima pričamo na predavanju.
	
	\fukudai{Domaće zadaće}
	
	U samostalnom radu vodit će vas domaće zadaće koje su sastavni dio predavanja i kao takve su obavezne. Osim što vam daju uvid u to koliko ste dobro savladali novo gradivo, zadaće služe i predavačima kako bismo mogli ponavljati stvari koje su bile teške ili nejasne i brže prijeći preko stvari koje su svima bile jasne i lagane.
	
	\dai{Obilježja japanskog jezika}
	
	\fukudai{Zvuk}
	\begin{itemize}
		\itemsep-5pt
		\item 5 samoglasnika sličnih hrvatskim
		\item \textasciitilde15 suglasnika od kojih većina ima hrvatski ekvivalent
		\item suglasnici se nikad ne pojavljuju sami - uvijek u paru suglasnik+samoglasnik
		\item moraički jezik (hrv. je slogovni) - svaka mora jednako traje
		\item sa svim varijacijama, u japanskom postoji jedva nešto više od 100 različitih mora - manje jedinstvenih kombinacija za riječi
	\end{itemize}

	
\end{document}