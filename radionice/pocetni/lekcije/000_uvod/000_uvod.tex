% !TeX document-id = {aa1964cd-e4d1-4c8c-8134-6010424df1b4}
% !TeX program = xelatex ?me -synctex=0 -interaction=nonstopmode -aux-directory=../../tex_aux -output-directory=./release
% !TeX program = xelatex

\documentclass[12pt]{article}

\usepackage{lineno,changepage,lipsum}
\usepackage[colorlinks=true,urlcolor=blue]{hyperref}
\usepackage{fontspec}
\usepackage{xeCJK}
\usepackage{tabularx}
\setCJKfamilyfont{chanto}{AozoraMinchoRegular.ttf}
\setCJKfamilyfont{tegaki}{Mushin.otf}
\usepackage[CJK,overlap]{ruby}
\usepackage{hhline}
\usepackage{multirow,array,amssymb}
\usepackage[croatian]{babel}
\usepackage{soul}
\usepackage[usenames, dvipsnames]{color}
\usepackage{wrapfig,booktabs}
\renewcommand{\rubysep}{0.1ex}
\renewcommand{\rubysize}{0.75}
\usepackage[margin=50pt]{geometry}
\modulolinenumbers[2]

\usepackage{pifont}
\newcommand{\cmark}{\ding{51}}%
\newcommand{\xmark}{\ding{55}}%

\definecolor{faded}{RGB}{100, 100, 100}

\renewcommand{\arraystretch}{1.2}

%\ruby{}{}
%$($\href{URL}{text}$)$

\newcommand{\furigana}[2]{\ruby{#1}{#2}}
\newcommand{\tegaki}[1]{
	\CJKfamily{tegaki}\CJKnospace
	#1
	\CJKfamily{chanto}\CJKnospace
}

\newcommand{\dai}[1]{
	\vspace{20pt}
	\large
	\noindent\textbf{#1}
	\normalsize
	\vspace{20pt}
}

\newcommand{\fukudai}[1]{
	\vspace{10pt}
	\noindent\textbf{#1}
	\vspace{10pt}
}

\newenvironment{bunshou}{
	\vspace{10pt}
	\begin{adjustwidth}{1cm}{3cm}
	\begin{linenumbers}
}{
	\end{linenumbers}
	\end{adjustwidth}
}

\newenvironment{reibun}{
	\vspace{10pt}
	\begin{tabular}{l l}
}{
	\end{tabular}
	\vspace{10pt}
}
\newcommand{\rei}[2]{
	#1&\textit{#2}\\
}
\newcommand{\reinagai}[2]{
	\multicolumn{2}{l}{#1}\\
	\multicolumn{2}{l}{\hspace{10pt}\textit{#2}}\\
}

\newenvironment{mondai}[1]{
	\vspace{10pt}
	#1
	
	\begin{enumerate}
		\itemsep-5pt
	}{
	\end{enumerate}
	\vspace{10pt}
}

\newenvironment{hyou}{
	\begin{itemize}
		\itemsep-5pt
	}{
	\end{itemize}
	\vspace{10pt}
}

\date{\today}

\CJKfamily{chanto}\CJKnospace
\author{Tomislav Mamić}
\begin{document}
	\dai{Uvod u radionice japanskog jezika}
	
	\fukudai{Cilj radionica}
	
	Učenje japanskog je posao za cijeli život, a ove početne radionice trebale bi vam služiti kao pomoć pri prvim koracima na tom putu. Osim uvoda u pismo, glasovni sustav i gramatiku, na radionicama ćemo se dotaknuti i kulturno-povijesnih tema, kao i dobrih strategija za samostalno učenje koje su primjenjive ne samo na japanski jezik nego i šire.
	
	\fukudai{Organizacija}
	
	Radionice su ugrubo podijeljene na teme koje sadrže nekoliko predavanja. Svaki tjedan započinje predavanjem koje služi kao uvod u samostalni rad. Na vama je odgovornost da preko tjedna uložite nešto svog vremena u vježbu i istraživanje novih stvari o kojima pričamo na predavanju.
	
	\fukudai{Domaće zadaće}
	
	U samostalnom radu vodit će vas domaće zadaće koje su sastavni dio predavanja i kao takve su obavezne. Osim što vam daju uvid u to koliko ste dobro savladali novo gradivo, zadaće služe i predavačima kako bismo mogli ponavljati stvari koje su bile teške ili nejasne i brže prijeći preko stvari koje su svima bile jasne i lagane.
	
	\dai{Obilježja japanskog jezika}
	
	\fukudai{Zvuk}
	\begin{hyou}
		\item 5 samoglasnika sličnih hrvatskim
		\item \textasciitilde15 suglasnika od kojih većina ima hrvatski ekvivalent
		\item suglasnici se nikad ne pojavljuju sami - uvijek u paru suglasnik+samoglasnik
		\item moraički jezik (hrv. je slogovni) - svaka mora jednako traje
		\item sa svim varijacijama, u japanskom postoji jedva nešto više od 100 različitih mora - manje jedinstvenih kombinacija za riječi
		\item more mogu imati uzlazni ili silazni naglasak
		\item naglasak jako varira od narječja do narječja
	\end{hyou}

	\fukudai{Pismo}
	\begin{hyou}
		\item paralelno se koriste 3 pisma:
		\begin{itemize}
			\itemsep-5pt
			\item hiragana - ひらがな
			\item katakana - カタカナ
			\item kanji - 漢字
		\end{itemize}
		\item hiragana i katakana su glasovna pisma - simboli predstavljaju izgovor
		\item kanji znakovi su ideogrami - ne predstavljaju izgovor nego značenje
		\item ne postoje razmaci - granice među riječima vidimo iz prijelaza među pismima
	\end{hyou}

	\fukudai{Gramatika}
	\begin{hyou}
		\item dosta jednostavnija nego hrvatska, ali \textbf{potpuno} drugačija
		\item aglutinativni (\textit{ljepljiv}) jezik - na riječi se dodaju predmetci i nastavci
		\item ne postoje rod, broj ni padež
		\item odnos imenskih riječi u rečenici određuju čestice
		\item postoje samo dva glagolska vremena - prošlost i neprošlost
		\item glagoli uvijek na kraju rečenice (SOP, hrvatski je standardno SPO)
		\item jedine promjenjive riječi su glagoli i i-pridjevi
		\item granica između glagola i pridjeva je jako tanka
		\item imenice imaju razne posebne uloge koje u hrvatskom nemaju
	\end{hyou}

	\fukudai{Kratka povijest}
	\begin{hyou}
		\item slično kao i kod slavenskih jezika, japanski također jako dugo nije imao pismo
		\item pretpostavlja se da se jezik počeo razvijati prije otprilike 2200 godina, te da su ga donijeli doseljenici
		\item u 8. st. započinje kulturna razmjena s Kinom - uvoz pisma
		\item u početku se koriste samo kanji ideogrami koji nisu dobro prilagođeni za japanski jezik
		\item velik utjecaj kineske književnosti na razvoj japanskog jezika
		\item kasnije dvorske žene razvijaju svoje jednostavnije fonetsko pismo (hiragana)
		\item kroz srednji i rani novi vijek događaju se razna pojednostavljenja fonetike
		\item u 19. i 20. st. jezik počinje sličiti današnjem japanskom
		\item poslije 2. svj. rata poslijednja veća reforma jezika - pojednostavljen sustav pisanja i određen standardni skup kanji znakova
		\item poslije 2. svj. rata velik priljev stranih riječi
	\end{hyou}

	\fukudai{Pristojnost i poštovanje}
	\begin{hyou}
		\item vrlo važni kulturni koncepti - imaju velik utjecaj na jezik
		\item relativni društveni položaj govornika i sugovornika određuje način na koji govore
		\item mlađi, neiskusniji i oni niže u hijerarhiji se svojim nadređenima uvijek pristojno obraćaju
		\item prijatelji i članovi obitelji međusobno razgovaraju kolokvijalno
		\item u službenim situacijama koristi se način govora koji sugovorniku iskazuje poštovanje
	\end{hyou}
\end{document}
