% !TeX document-id = {2292039b-dec6-4836-a40b-2951248bf75a}
% !TeX program = xelatex ?me -synctex=0 -interaction=nonstopmode -aux-directory=../../tex_aux -output-directory=./release
% !TeX program = xelatex

\documentclass[12pt]{article}

\usepackage{lineno,changepage,lipsum}
\usepackage[colorlinks=true,urlcolor=blue]{hyperref}
\usepackage{fontspec}
\usepackage{xeCJK}
\usepackage{tabularx}
\setCJKfamilyfont{chanto}{AozoraMinchoRegular.ttf}
\setCJKfamilyfont{tegaki}{Mushin.otf}
\usepackage[CJK,overlap]{ruby}
\usepackage{hhline}
\usepackage{multirow,array,amssymb}
\usepackage[croatian]{babel}
\usepackage{soul}
\usepackage[usenames, dvipsnames]{color}
\usepackage{wrapfig,booktabs}
\renewcommand{\rubysep}{0.1ex}
\renewcommand{\rubysize}{0.75}
\usepackage[margin=50pt]{geometry}
\modulolinenumbers[2]

\usepackage{pifont}
\newcommand{\cmark}{\ding{51}}%
\newcommand{\xmark}{\ding{55}}%

\definecolor{faded}{RGB}{100, 100, 100}

\renewcommand{\arraystretch}{1.2}

%\ruby{}{}
%$($\href{URL}{text}$)$

\newcommand{\furigana}[2]{\ruby{#1}{#2}}
\newcommand{\tegaki}[1]{
	\CJKfamily{tegaki}\CJKnospace
	#1
	\CJKfamily{chanto}\CJKnospace
}

\newcommand{\dai}[1]{
	\vspace{20pt}
	\large
	\noindent\textbf{#1}
	\normalsize
	\vspace{20pt}
}

\newcommand{\fukudai}[1]{
	\vspace{10pt}
	\noindent\textbf{#1}
	\vspace{10pt}
}

\newenvironment{bunshou}{
	\vspace{10pt}
	\begin{adjustwidth}{1cm}{3cm}
	\begin{linenumbers}
}{
	\end{linenumbers}
	\end{adjustwidth}
}

\newenvironment{reibun}{
	\vspace{10pt}
	\begin{tabular}{l l}
}{
	\end{tabular}
	\vspace{10pt}
}
\newcommand{\rei}[2]{
	#1&\textit{#2}\\
}
\newcommand{\reinagai}[2]{
	\multicolumn{2}{l}{#1}\\
	\multicolumn{2}{l}{\hspace{10pt}\textit{#2}}\\
}

\newenvironment{mondai}[1]{
	\vspace{10pt}
	#1
	
	\begin{enumerate}
		\itemsep-5pt
	}{
	\end{enumerate}
	\vspace{10pt}
}

\newenvironment{hyou}{
	\begin{itemize}
		\itemsep-5pt
	}{
	\end{itemize}
	\vspace{10pt}
}

\date{\today}

\CJKfamily{chanto}\CJKnospace
\author{autor}
\begin{document}
	\dai{Domaća zadaća - hiragana}
	
	Korisni online materijali:
	\begin{hyou}
		\item tablica s animiranim redoslijedom poteza (\href{https://yosida.com/en/hiragana.html}{link})
		\item pouzdan eng. jap. rječnik (\href{https://jisho.org/}{link})
		\item jap. rječnik izgovora - potreban upis na hiragani (\href{http://www.gavo.t.u-tokyo.ac.jp/ojad/search}{link})
	\end{hyou}
	
	\begin{mondai}{Pročitajte sljedeće riječi i pronađite u rječniku njihovo značenje. Sve riječi su imenice iz prirode, ostale istozvučnice zanemarite za sada.}
		\item かめ
		\item いし
		\item くも
		\item かぜ
		\item おか
	\end{mondai}

	\begin{mondai}{Pročitajte sljedeće riječi obraćajući pažnju na glasove やゃ, ゆゅ i よょ.}
		\item しゅくだい - \textit{domaća zadaća}
		\item しょうらい - \textit{budućnost}
		\item りゅう - \textit{zmaj} (posebno kineski)
		\item りゆう - \textit{razlog}
	\end{mondai}

	\begin{mondai}{Pročitajte sljedeće riječi (regije i gradovi) pazeći na duljinu samoglasnika i glotalnu stanku っ.}
		\item おおさか
		\item きょうと
		\item とうかまち
		\item きゅうしゅう
		\item ほっかいどう
		\item さっぽろう
	\end{mondai}

	\begin{mondai}{Zapišite izgovor sljedećih riječi na hiragani.}
		\item \textit{mrak} - ya.mi
		\item \textit{svjetlo} - hi.ka.ri
		\item \textit{Mjesec} - tsu.ki
		\item \textit{Sunce} - ta.i.yo.u
		\item \textit{Zemlja} - chi.kyu.u
		\item \textit{nebo} - so.ra
		\item \textit{mliječni put} - a.ma.no.ga.wa
	\end{mondai}
\end{document}