% !TeX document-id = {87c116f6-681a-4b3d-9fc9-28ee9ca2f882}
% !TeX program = xelatex ?me -synctex=0 -interaction=nonstopmode -aux-directory=../../tex_aux -output-directory=./release
% !TeX program = xelatex

\documentclass[12pt]{article}

\usepackage{lineno,changepage,lipsum}
\usepackage[colorlinks=true,urlcolor=blue]{hyperref}
\usepackage{fontspec}
\usepackage{xeCJK}
\usepackage{tabularx}
\setCJKfamilyfont{chanto}{AozoraMinchoRegular.ttf}
\setCJKfamilyfont{tegaki}{Mushin.otf}
\usepackage[CJK,overlap]{ruby}
\usepackage{hhline}
\usepackage{multirow,array,amssymb}
\usepackage[croatian]{babel}
\usepackage{soul}
\usepackage[usenames, dvipsnames]{color}
\usepackage{wrapfig,booktabs}
\renewcommand{\rubysep}{0.1ex}
\renewcommand{\rubysize}{0.75}
\usepackage[margin=50pt]{geometry}
\modulolinenumbers[2]

\usepackage{pifont}
\newcommand{\cmark}{\ding{51}}%
\newcommand{\xmark}{\ding{55}}%

\definecolor{faded}{RGB}{100, 100, 100}

\renewcommand{\arraystretch}{1.2}

%\ruby{}{}
%$($\href{URL}{text}$)$

\newcommand{\furigana}[2]{\ruby{#1}{#2}}
\newcommand{\tegaki}[1]{
	\CJKfamily{tegaki}\CJKnospace
	#1
	\CJKfamily{chanto}\CJKnospace
}

\newcommand{\dai}[1]{
	\vspace{20pt}
	\large
	\noindent\textbf{#1}
	\normalsize
	\vspace{20pt}
}

\newcommand{\fukudai}[1]{
	\vspace{10pt}
	\noindent\textbf{#1}
	\vspace{10pt}
}

\newenvironment{bunshou}{
	\vspace{10pt}
	\begin{adjustwidth}{1cm}{3cm}
	\begin{linenumbers}
}{
	\end{linenumbers}
	\end{adjustwidth}
}

\newenvironment{reibun}{
	\vspace{10pt}
	\begin{tabular}{l l}
}{
	\end{tabular}
	\vspace{10pt}
}
\newcommand{\rei}[2]{
	#1&\textit{#2}\\
}
\newcommand{\reinagai}[2]{
	\multicolumn{2}{l}{#1}\\
	\multicolumn{2}{l}{\hspace{10pt}\textit{#2}}\\
}

\newenvironment{mondai}[1]{
	\vspace{10pt}
	#1
	
	\begin{enumerate}
		\itemsep-5pt
	}{
	\end{enumerate}
	\vspace{10pt}
}

\newenvironment{hyou}{
	\begin{itemize}
		\itemsep-5pt
	}{
	\end{itemize}
	\vspace{10pt}
}

\date{\today}

\CJKfamily{chanto}\CJKnospace
\author{autor}
\begin{document}
	\dai{Glasovni sustav japanskog - hiragana}
	
	\fukudai{Organizacija glasova}
	
	U japanskom jeziku glasovi dolaze u \textit{m\={o}rama}. M\={o}ra je izgovorna jedinica slična slogu, no za razliku od slogova, m\={o}rama se duljina nikad ne mijenja. Kad ih izgovaramo, m\={o}re su kao note. Ovisno o situaciji ćemo im mijenjati visinu tona (npr. kraj upitne rečenice vs. izjavne), ali nikad trajanje.
	
	Općenito, m\={o}ra se sastoji od jezgre (samoglasnik), početka (eng. \textit{coda}) i kraja (eng. \textit{onset}), ali u japanskom ne postoje m\={o}re sa suglasničkim krajem. Npr. u riječi \textit{stol}, jezgra je \textit{o}, početak je suglasnički skup \textit{st}, a kraj je \textit{l}.
	
	Također, zabranjeni su suglasnički skupovi - dva suglasnika ne smiju stajati jedan do drugog. Tako je ranije spomenutu riječ \textit{stol} u japanskom nemoguće točno zapisati iz dva razloga (koja?). Zbog ovih ograničenja, broj mogućih m\={o}ra je iznenađujuće malen u odnosu na broj mogućih slogova u hrvatskom.
	
	Dopuštene su m\={o}re koje nemaju ni početka ni kraja - dakle sami samoglasnici. Osim samoglasnika, sam može stajati i nosni glas ん. Ovaj glas u hrvatskom jeziku ne tretiramo posebno, ali u japanskom zapravo predstavlja više različitih glasova (n, ŋ, m). Ove varijacije su nam prirodne. Pokušajmo izgovoriti riječi \textit{kondo}, \textit{konkai} i \textit{senpai} obraćajući pažnju na glas \textit{n}. Zbog toga nosni glas ん u japanskom ne smatramo suglasnikom, ali ni samoglasnikom.
	
	\fukudai{Zapis glasova}
	
	Japanski jezik ne zapisuje pojedinačne glasove nego m\={o}re. U tablici ispod nalazi se 46 osnovnih znakova pisma hiragana. M\={o}re u istom stupcu počinju istim suglasnikom, one u istom redu završavaju istim samoglasnikom. Jedina iznimka ovome je ranije spomenuti ん.
	
	\vspace{10pt}
	\begin{tabular}{|l|r|r|r|r|r|r|r|r|r|r|r|}
		\hline
		&\textasciitilde&k&s&t&n&m&r&h&j&v&\textasciitilde\\
		\hline
		a&あ&か&さ&た&な&ま&ら&は&や&わ&ん\\
		i&い&き&し&ち&に&み&り&ひ&&&\\
		u&う&く&す&つ&ぬ&む&る&ふ&ゆ&&\\
		e&え&け&せ&て&ね&め&れ&へ&&&\\
		o&お&こ&そ&と&の&も&ろ&ほ&よ&を&\\
		\hline
	\end{tabular}
	
\end{document}