% !TeX document-id = {ca3061d2-1783-4bc5-8ae3-d3eb3a9d6fcc}
% !TeX program = xelatex ?me -synctex=0 -interaction=nonstopmode -aux-directory=../../tex_aux -output-directory=./release
% !TeX program = xelatex

\documentclass[12pt]{article}

\usepackage{lineno,changepage,lipsum}
\usepackage[colorlinks=true,urlcolor=blue]{hyperref}
\usepackage{fontspec}[ Path =../../../ ]
\usepackage{xeCJK}
\usepackage{tabularx}
\usepackage{graphicx}
\setCJKfamilyfont{chanto}{AOZORAMINCHOREGULAR_0.TTF}%
\setCJKfamilyfont{tegaki}{Mushin.otf}%
\usepackage[CJK,overlap]{ruby}
\usepackage{hhline}
\usepackage{multirow,array,amssymb}
\usepackage[croatian]{babel}
\usepackage{soul}
\usepackage[usenames, dvipsnames]{color}
\usepackage{wrapfig,booktabs}
\usepackage{calc}
\renewcommand{\rubysep}{0.1ex}
\renewcommand{\rubysize}{0.75}
\usepackage[margin=50pt]{geometry}
\usepackage{hyperref}
\modulolinenumbers[2]

\date{\today}

\usepackage{fancyhdr}
\pagestyle{fancy}
\fancyhf{}
\fancyhead[LE,RO]{\thepage}
\makeatletter
\fancyhead[RE,LO]{rev. \@date 誠}
\makeatother

\usepackage{pifont}
\newcommand{\cmark}{\ding{51}}%
\newcommand{\xmark}{\ding{55}}%

\newcommand{\dosl}{{\normalfont dosl. }}%
\newcommand{\rem}[1]{{\normalfont #1 }}%

\definecolor{faded}{RGB}{100, 100, 100}

\renewcommand{\arraystretch}{1.2}

%\ruby{}{}
%$($\href{URL}{text}$)$

\newcommand{\furigana}[2]{\ruby{#1}{#2}}
\newcommand{\tegaki}[1]{
	\CJKfamily{tegaki}\CJKnospace
	#1
	\CJKfamily{chanto}\CJKnospace
}

\newcommand{\dai}[1]{
	\vspace{20pt}
	\large
	\noindent\textbf{#1}
	\normalsize
	\vspace{20pt}
}

\newcommand{\fukudai}[1]{
	\vspace{10pt}
	\noindent\textbf{#1}
	\vspace{10pt}
}

\newenvironment{bunshou}{
	\vspace{10pt}
	\begin{adjustwidth}{1cm}{3cm}
	\begin{linenumbers}
}{
	\end{linenumbers}
	\end{adjustwidth}
}

\newenvironment{reibun}[1][]{
	\vspace{10pt}
	#1
	
	\begin{tabular}{l l}
}{
	\end{tabular}
	\vspace{10pt}
}
\newcommand{\rei}[2]{
	#1&\textit{#2}\\
}
\newcommand{\reinagai}[2]{
	\multicolumn{2}{l}{#1}\\
	\multicolumn{2}{l}{\hspace{10pt}\textit{#2}}\\
}

\newenvironment{mondai}[1]{
	\vspace{10pt}
	\noindent #1
	
	\begin{enumerate}
		\itemsep-5pt
	}{
	\end{enumerate}
}

\newenvironment{hyou}{
	\begin{itemize}
		\itemsep-5pt
	}{
	\end{itemize}
	\vspace{10pt}
}

\newcommand{\juuyou}[2][20pt]{
	\vspace{5pt}
		\noindent\hspace{#1}\parbox[c]{\textwidth-#1-#1}{\centering\textit{#2}}
	\vspace{5pt}
}

\newcommand{\ten}{
	\vspace{5pt}
	\noindent\hspace{-10pt}$\bullet$
}

\CJKfamily{chanto}\CJKnospace

\frenchspacing
\author{autor}
\begin{document}
	\dai{Ciljevi i napomene - hiragana}
	
	Krećem od pretpostavke da je najbitniji dio hiragane izgovor glasova. Zapis pojedinih znakova polaznici mogu sami naučiti iz resursa, ali objašnjenja izgovora u odnosu na hrvatski vrlo vjerojatno nigdje drugdje neće naći pa je fokus na tome.
	
	\fukudai{Ciljevi}
	
	\begin{hyou}
		\item objasniti glasovni sustav
		\item organizacija hiragane i zašto je baš dobra tako kako je
		\item pisanje i izgovor
		\begin{itemize}
			\itemsep-5pt
			\item pokazati pisanje i naglasiti važnost redoslijeda i smjera poteza
			\item objasniti izgovor i nagovoriti ljude da pokušaju izgovarati glasove i riječi
			\item objasniti i istaknuti "nepravilnosti" u glasovima
		\end{itemize}
		\item objasniti malo ゃ, ゅ i ょ
		\item glotalna stanka っ
		\item dugi samoglasnici
	\end{hyou}

	\fukudai{Napomene}
	
	\begin{hyou}
		\item objasniti pojam m\={o}re je najbolje preko nota iste duljine
		\item usporediti s hrvatskim naglasno-slogovnim sustavom - M\={o}ra ima jednu varijablu (visina tona), a hrvatski slog dvije (visina tona i duljina jezgre). Zato u japanskom postoji samo silazni i uzlazni naglasak, dok u hrvatskom postoje 4 varijante za sve kombinacije (kratkouzlazni, kratkosilazni, dugouzlazni, dugosilazni).
		\item kod izgovora posebnu pažnju posvetiti sljedećem:
		\begin{itemize}
			\itemsep-5pt
			\item \textit{u} nije duboko ni zaobljeno
			\item \textit{\'{s}}, \textit{ć} i \textit{đ} su mekani
		\end{itemize}
	\end{hyou}
\end{document}