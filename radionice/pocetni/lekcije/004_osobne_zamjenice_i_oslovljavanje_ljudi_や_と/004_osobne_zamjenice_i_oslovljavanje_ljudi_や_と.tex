% !TeX document-id = {c6e61070-f072-4c33-a3c0-f2b98a26b5bc}
% !TeX program = xelatex ?me -synctex=0 -interaction=nonstopmode -aux-directory=../tex_aux -output-directory=./release
% !TeX program = xelatex

\documentclass[12pt]{article}

\usepackage{lineno,changepage,lipsum}
\usepackage[colorlinks=true,urlcolor=blue]{hyperref}
\usepackage{fontspec}
\usepackage{xeCJK}
\usepackage{tabularx}
\setCJKfamilyfont{chanto}{AozoraMinchoRegular.ttf}
\setCJKfamilyfont{tegaki}{Mushin.otf}
\usepackage[CJK,overlap]{ruby}
\usepackage{hhline}
\usepackage{multirow,array,amssymb}
\usepackage[croatian]{babel}
\usepackage{soul}
\usepackage[usenames, dvipsnames]{color}
\usepackage{wrapfig,booktabs}
\renewcommand{\rubysep}{0.1ex}
\renewcommand{\rubysize}{0.75}
\usepackage[margin=50pt]{geometry}
\modulolinenumbers[2]

\usepackage{pifont}
\newcommand{\cmark}{\ding{51}}%
\newcommand{\xmark}{\ding{55}}%

\definecolor{faded}{RGB}{100, 100, 100}

\renewcommand{\arraystretch}{1.2}

%\ruby{}{}
%$($\href{URL}{text}$)$

\newcommand{\furigana}[2]{\ruby{#1}{#2}}
\newcommand{\tegaki}[1]{
	\CJKfamily{tegaki}\CJKnospace
	#1
	\CJKfamily{chanto}\CJKnospace
}

\newcommand{\dai}[1]{
	\vspace{20pt}
	\large
	\noindent\textbf{#1}
	\normalsize
	\vspace{20pt}
}

\newcommand{\fukudai}[1]{
	\vspace{10pt}
	\noindent\textbf{#1}
	\vspace{10pt}
}

\newenvironment{bunshou}{
	\vspace{10pt}
	\begin{adjustwidth}{1cm}{3cm}
	\begin{linenumbers}
}{
	\end{linenumbers}
	\end{adjustwidth}
}

\newenvironment{reibun}{
	\vspace{10pt}
	\begin{tabular}{l l}
}{
	\end{tabular}
	\vspace{10pt}
}
\newcommand{\rei}[2]{
	#1&\textit{#2}\\
}
\newcommand{\reinagai}[2]{
	\multicolumn{2}{l}{#1}\\
	\multicolumn{2}{l}{\hspace{10pt}\textit{#2}}\\
}

\newenvironment{mondai}[1]{
	\vspace{10pt}
	#1
	
	\begin{enumerate}
		\itemsep-5pt
	}{
	\end{enumerate}
	\vspace{10pt}
}

\newenvironment{hyou}{
	\begin{itemize}
		\itemsep-5pt
	}{
	\end{itemize}
	\vspace{10pt}
}

\date{\today}

\CJKfamily{chanto}\CJKnospace
\author{Tomislav Mamić, Željka Ludošan}
\begin{document}
	\dai{Osobne zamjenice i oslovljavanje ljudi; čestice と i や}
	
	\dai{Osobne zamjenice}
		
	M - koriste muške osobe, Ž - koriste ženske osobe
	
	
	\begin{tabular}{|l|l|p{400pt}|}
		\hline
		わたし&ja&neutralan, formalan, M i Ž\\\hline
		わたくし&ja&jedan od najformalnijih oblika, M i Ž\\\hline
		わたくしめ&ja&još formalnije od わたくし i izražava poniznost, M i Ž\\\hline
		あたし&ja&može zvučati slatko pa ga češće koriste mlađe žene, Ž\\\hline
		ぼく&ja&donekle neformalni muški, konotaciju djetinjatosti dobije jedino ako je u hiragani, M\\\hline
		じぶん&ja&pojavljuje se češće zadnje vrijeme u modernom japanskom, M i Ž\\\hline
		うち&ja&M koriste kada pričaju o svom unutarnjem krugu u malo neformalnijim okolnostima, inače koriste i M i Ž, ali većinom Ž\\\hline
		あたい&ja&djetinjasto わたし、u anime-ima i ostalim medijima se koristi kad netko pokušava siliti da bude sladak al im nikako nejde, Ž\\\hline
	\end{tabular}
	
	\vspace{10pt}
	
		\begin{tabular}{|l|l|l|}
		\hline
		あなた&ti&neutralni, kada se piše kanji znakovima tada je formalan, M i Ž\\\hline
		\end{tabular}
		
		
	\vspace{10pt}
	
	あなた koristimo za osobe čije ime ne znamo, žene ponekad od milja koriste あなた kad se obraćaju svome mužu.
	
	Korištenje じぶん kao osobne zamjenice 1. lica je u zadnje vrijeme uzelo maha. Prije se koristila u edo periodu uglavnom među vojnicima, a u modernom jeziku se u kansaiju povremeno koristi kao zamjenica za 2. lice. To sve skupa ovu upotrebu čini dosta nespretnom i trebalo bi je izbjegavati.
	
	\vspace{10pt}
	
	U japanskom jeziku, kada govorimo o sebi, najčešće koristimo neku osobnu zamjenicu poput わたし, a kad govorimo o drugim osobama tada češće koristimo njihovo ime koje na kraju ima priljepljen nastavak (npr.-さん,ともきさん) koji određuje našu i njihovu poziciju u društvenoj hijerarhiji.
	
	\vspace{10pt}
	Ako znamo ime osobe, tada koristimo njeno ime. 
	
	\vspace{10pt}
	Pristojnije je reći:
	\begin{reibun}
		\rei{\underline{ともきさん}、すいか が すき?}{Tomoki, voliš lubenice?}
	\end{reibun}
	
	nego
	
	\begin{reibun}
		\rei{\underline{あなた}、すいか が すき?\footnotemark[1]}{Ti voliš lubenice?}
	\end{reibun}

	\footnotetext[1]{Ovo zvuči kao da žena pita muža - \textit{Dragi, voliš li lubenice?}}
	
	\begin{tabular}{|l|l|l|}
		\hline
		かれ&on, dečko&neutralni muški\\\hline
		かのじょ&ona, cura&neutralni ženski\\\hline
	\end{tabular}
	
	\vspace{10pt}
	
	かれ i かのじょ koristimo kada pričamo o trećoj osobi čije ime neznamo ili nam je ta osoba glavna tema priče pa ime ne moramo ponavljati. Nikad ne koriste za osobe koje neznamo jer izražavaju bliskost.Umjesto toga koristi se koristi se あのかた、あのひと...Osim toga,かれ i かのじょ može značiti "cura" i "dečko" u intimnom smislu.
	
	\begin{reibun}
		\rei{\underline{かれ} の にっき}{njegov dnevnik}
		\rei{\underline{わたし} の \underline{かのじょ} の にっき}{dnevnik moje cure}
	\end{reibun}
	
	\begin{tabular}{|l|l|l|}
		\hline
		そなた&ti&pjesnički „ti“, arhaizam\\\hline
		きみ\footnotemark[2]&ti&koristi se za M i Ž pogotovo kad se stariji obraća puno mlađem od sebe\\\hline
		あんた&ti&arogantniji, češće koriste Ž nego M\\\hline
		おまえ&ti&pogrdno, ponekad se koristi među muškim prijateljima, M\\\hline
		てめぇ&ti&pogrdno, M i Ž\\\hline
		きさま&ti&pogrdno, nekoć bio izraz keiga, M i Ž\\\hline
	\end{tabular}	
	
	\vspace{10pt}
	
	\footnotetext[2]{Osoba koja govori きみ je iznad osobe za koju se to izgovara, ponekad きみ koriste muške osobe kada se obraćaju svojoj jako bliskoj prijateljici}
	
		Ako razgovaramo sa osobom koja nam je inferiorna ili na istoj razini tada možemo koristiti inferiorne osobne zamjenice, uz napomenu da one mogu zvučati poprilično nepristojno i uvrijediti osobu.
		
		\vspace{15pt}
	
	\vspace{10pt}
	
	
	
	\vspace{10pt}
	
	\begin{tabular}{|l|l|p{400pt}|}
		\hline
		わたくし&ja&formalni\\\hline
		おれ&ja&najneformalniji muški, u hiragani ima konotaciju kao da osnovnoškolac priča, M\\\hline
		わし&ja&gotovo izumrijela riječ, koriste stariji ljudi samo u blizini Hiroshime (ne koriste ju više ni stariji ljudi), sada se koristi kao zezancija i u mangama, M\\\hline
		わい&ja&koristi se samo u šali\\\hline
		われ&ja&najčešće se koristi u pisanom tekstu ili formalnim okolnostima kada se obraća grupi, M i Ž\\\hline
		われわれ&mi&koristi se u formalnim situacijama kada jedna osoba predstavlja grupu, također se može koristiti i われら ali je manje formalno, M i Ž\\\hline
	\end{tabular}

	\vspace{10pt}	
	
	われ i われわれ koriste tvrtke u poslovnim mail-ovima.
	
	\vspace{10pt}
	
	Bonus:
	\begin{tabular}{|l|l|l|}
		\hline
		せっしゃ&ja&koristili su shogunati\\\hline
	\end{tabular}
	
	\newpage
	
	\dai{Nastavci za imena}

	\ten \fukudai{Nastavci vezani uz društevni položaj}



	-さん	koristimo za nepoznate osobe i osobe prema kojima želimo biti pristojni, te općenito za odrasle osobe. Neke imenice već u sebi imaju nastavak さん i obično iskazuju poštovanje prema onom na što se odnose.
	
	
	\vspace{10pt}	
	
	\begin{tabular}{|l|l|}
		\hline
		おじょうさん&kćer, mlada dama\\\hline
		みなさん&svi\\\hline
		おおやさん&stanodavac, stanodavka\\\hline

	\end{tabular}
	
	\vspace{10pt}
	
	Kada pričamo o obitelji, koristimo drugačije nazive ovisno o tome da li govorimo o svojoj ili tuđoj obitelji. Kada govorimo o svojoj obitelji, govorimo o ljudima koji pripadaju našem unutarnjem krugu, a kada govorimo o tuđoj obitelji, govorimo o ljudima koji pripadaju vanjskom krugu i sukladno s time postoje drugačiji nazivi za članove obitelji.
	
	\vspace{10pt}	
	
	\begin{tabular}{|l|l|l||l|l|l|l|l|}
		\hline
		unutra& &van&unutra& &van\\\hline
		ちち&tata&おとうさん&そふ&djed&おじいさん\\\hline
		はは&mama&おかあさん&そぼ&baka&おばあさん\\\hline
		あに、あにき&stariji brat&おにいさん&おじ&stric&おじさん\\\hline
		あね、あねき&starija sestra&おねえさん&おば&teta&おばさん\\\hline
		おとうと&mlađi brat&おとうとさん&おっと&muž&ごしゅじん\\\hline
		いもうと&mlađa sestra&いもうとさん&つま&žena&おくさん\\\hline
	\end{tabular}
	
	-さま koristimo za osobe prema kojima želimo iskazati posebno poštovanje i veoma je nepristojno koristiti -さま kada govorimo o sebi jer to daje dojam da imamo jako visoko mišljenje o sebi.
	
	\vspace{10pt}
	
	\begin{tabular}{|l|l|l|}
		\hline
		みなさま&(poštovani) svi\\\hline
		おきゃくさま&(poštovani) gost, kupac, klijent\\\hline
		おかあさま&(poštovana) majka\\\hline
		ひめさま&princeza\\\hline
		かみさま&bog\\\hline
	\end{tabular}
	
	\vspace{10pt}
	
		\fukudai{Intimiziranje stvari sa さん i さま}
	
	Japanci ponekad koriste nastavke さん i さま kako bi pokazali poštovanje prema zaposlenicima ili opisali nešto što im je drago ili blisko srcu.
	
	\vspace{10pt}	
	

\begin{tabular}{|l|p{400pt}|}
		\hline
		ケーキやさん&simpatična slastičarnica koja nam je draga\\\hline
		てんちょうさん&voditelj trgovine koji se brine za trgovinu i proizvode\\\hline
		おまわりさん&policajac koji patrolira ulicom i održava mir\\\hline
		おひさま&sunašce\\\hline
	\end{tabular}

	\vspace{10pt}	
	\newpage
	
	-ちゃん zvuči djetinjasto, slatko i možemo koristiti među prijateljima. Kada pridodamo ちゃん na kraju nečijeg imena to označava da nam je ta osoba posebno draga. Najčešće se pridodaje curama.
	
	\vspace{10pt}	
	
	-くん se koristi među prijateljima ili za mlađu osobu. Najčešće se koristi na kraju imena muških osoba, ali ako se koristi za cure tada zvuči ozbiljnije i kao iskaz poštovanja. Često se u poslovnom okruženju nadređeni ovako obraćaju zaposlenicama.
	
	\vspace{10pt}

Odrasle osobe će za svoje prijatelje najčešće koristiti -さん.

	\vspace{10pt}
	
	-たち dodajemo na osobne zamjenice ili direktno na ime kada želimo govoriti o grupi ljudi i to na način da ga dodamo na ime osobe ili osobnu zamjenicu koja tu grupu predstavlja. Neformalni nastavak od -たち je -ら i koristi se na jednak način, ali zbog toga što zvuči sirovije i koristi se na razne načine, najsigurnije je koristiti -たち.


	\begin{reibun}
		\rei{\underline{ともこたち} と いく。}{Idem s Tomoko i njezinima...}
		\rei{\underline{かのじょたち} が さけんだ。}{Zavrištale su.}
		\rei{\underline{かれら} が みえない。}{Ne vidim ih.}
		\rei{\underline{ぼくら} が そうじ する。}{Mi ćemo počistiti.}
	\end{reibun}

	\ten \fukudai{Nastavci vezani uz zvanja}
	
	Određena zvanja ili pozicije u društvu možemo koristiti kao nastavke na kraju imena, ali i kao imenice same za sebe. U tom slučaju se drugi nastavci na njih ne dodaju.
	
	\vspace{10pt}
	
	\begin{tabular}{|l|l|}
		\hline
		しゃちょう&vlasnik tvrtke\\\hline
		ぶちょう&direktor odjela\\\hline
		しゅにん&glavna odgovorna osoba, šef\\\hline
		せんせい&učitelj, doktor\\\hline
		せんぱい&iskusnija osoba od koje učite\\\hline
		てんちょう&voditelj trgovine\\\hline
	\end{tabular}
	
	
	
	\begin{reibun}
		\rei{\underline{たなかしゅにん}、しょるい を どうぞ!}{Šefe Tanaka, izvolite papire!}
		\rei{\underline{おのせんせい} は げんき です か?}{Je li učitelj Ono u redu?}
		\rei{\underline{けいこちゃん} は \underline{まつだせんぱい} が きらい。}{Keiko mrzi Matsudu.}
	\end{reibun}

\newpage

\dai{Čestice za nabrajanje: と i や}


Česticu と koristimo za nabrajanje stvari ili ljudi na način da stavimo と između imenica ili ljudi koje nabrajamo.


	\begin{reibun}
		\rei{Kakva pića voli Matsuda?}{}
		\rei{ぎゅうにゅう と おちゃ と みず。}{Mlijeko, zeleni čaj i vodu.}
		\rei{Matsuda voli točno ta tri pića.}{}
		\rei{Tko to ide u dućan?}{}
		\rei{まつだ と ももこ。}{Matsuda i Momoko.}
	\end{reibun}
	
	\vspace{10pt}

Česticu や također koristimo za nabrajanje imenica, ali njome nabrajamo stvari koje samo predstavljaju dio u skupini stvari o kojoj pričamo.

	\begin{reibun}
		\rei{Kakva pića ne voli Matsuda?}{}
		\rei{さけ\footnotemark[3] や ぶどうしゅ や あまみず。}{Pića kao što su sake, vino i kišnica.}
		\rei{Matsuda ne voli ta tri pića i druga slična njima.}{}
	\end{reibun}

\footnotetext[3]{さけ je naziv za japansko vino od riže, ali može općenito značiti bilo kakvo alkoholno piće.}
	
\end{document}