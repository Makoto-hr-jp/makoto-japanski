% !TeX program = xelatex ?me -synctex=0 -interaction=nonstopmode -aux-directory=../tex_aux -output-directory=./release
% !TeX program = xelatex

\documentclass[12pt]{article}

\usepackage{lineno,changepage,lipsum}
\usepackage[colorlinks=true,urlcolor=blue]{hyperref}
\usepackage{fontspec}
\usepackage{xeCJK}
\usepackage{tabularx}
\setCJKfamilyfont{chanto}{AozoraMinchoRegular.ttf}
\setCJKfamilyfont{tegaki}{Mushin.otf}
\usepackage[CJK,overlap]{ruby}
\usepackage{hhline}
\usepackage{multirow,array,amssymb}
\usepackage[croatian]{babel}
\usepackage{soul}
\usepackage[usenames, dvipsnames]{color}
\usepackage{wrapfig,booktabs}
\renewcommand{\rubysep}{0.1ex}
\renewcommand{\rubysize}{0.75}
\usepackage[margin=50pt]{geometry}
\modulolinenumbers[2]

\usepackage{pifont}
\newcommand{\cmark}{\ding{51}}%
\newcommand{\xmark}{\ding{55}}%

\definecolor{faded}{RGB}{100, 100, 100}

\renewcommand{\arraystretch}{1.2}

%\ruby{}{}
%$($\href{URL}{text}$)$

\newcommand{\furigana}[2]{\ruby{#1}{#2}}
\newcommand{\tegaki}[1]{
	\CJKfamily{tegaki}\CJKnospace
	#1
	\CJKfamily{chanto}\CJKnospace
}

\newcommand{\dai}[1]{
	\vspace{20pt}
	\large
	\noindent\textbf{#1}
	\normalsize
	\vspace{20pt}
}

\newcommand{\fukudai}[1]{
	\vspace{10pt}
	\noindent\textbf{#1}
	\vspace{10pt}
}

\newenvironment{bunshou}{
	\vspace{10pt}
	\begin{adjustwidth}{1cm}{3cm}
	\begin{linenumbers}
}{
	\end{linenumbers}
	\end{adjustwidth}
}

\newenvironment{reibun}{
	\vspace{10pt}
	\begin{tabular}{l l}
}{
	\end{tabular}
	\vspace{10pt}
}
\newcommand{\rei}[2]{
	#1&\textit{#2}\\
}
\newcommand{\reinagai}[2]{
	\multicolumn{2}{l}{#1}\\
	\multicolumn{2}{l}{\hspace{10pt}\textit{#2}}\\
}

\newenvironment{mondai}[1]{
	\vspace{10pt}
	#1
	
	\begin{enumerate}
		\itemsep-5pt
	}{
	\end{enumerate}
	\vspace{10pt}
}

\newenvironment{hyou}{
	\begin{itemize}
		\itemsep-5pt
	}{
	\end{itemize}
	\vspace{10pt}
}

\date{\today}

\CJKfamily{chanto}\CJKnospace
\author{Tomislav Mamić, Željka Ludošan}
\begin{document}
	\dai{Osobne zamjenice i oslovljavanje ljudi; čestice や i と}
	
	\fukudai{Osobne zamjenice}
	
	\begin{tabular}{|l|l|l|}
		\hline
		わたし&ja&neutralni\\\hline
		あなた&ti&neutralni\\\hline
	\end{tabular}
	
	\vspace{10pt}
	
	あなた koristimo za osobe čije ime neznamo, žene iz milja ponekad あなた koriste kada se obraćaju svome mužu.
	
	\vspace{10pt}	
	
	U japanskom jeziku, kada govorimo o sebi najčešće  koristimo neku osobnu zamjenicu poput わたし, a kad govorimo o drugim osobama tada češće koristimo njihovo ime koje na kraju ima priljepljen nastavak (npr.-さん,ともきさん) koji određuje naš odnos prema toj osobi. 
	
	\vspace{10pt}
	Ako znamo ime osobe, tada koristimo njeno ime. 
	
	\vspace{10pt}
	Pristojnije je reći:
	\begin{reibun}
		\rei{ともきさん、すいか が すき?}{Tomoki, voliš lubenice?}
	\end{reibun}
	
	za razliku od
	
	
	\begin{reibun}
		\rei{あなた、すいか が すき?}{Ti voliš lubenice?}
	\end{reibun}
	
	\begin{tabular}{|l|l|l|}
		\hline
		かれ&on, dečko&neutralni muški\\\hline
		かのじょ&ona, cura&neutralni ženski\\\hline
		かれら&oni&neutralni, ali češće koristimo za skupinu muških osoba\\\hline
	\end{tabular}
	
	\vspace{10pt}
	
	かれ i かのじょ koristimo kada pričamo o trećoj osobi čije ime neznamo ili nam je ta osoba glavna tema priče pa ime ne moramo ponavljati. Osim toga, može značiti "cura" i "dečko" u intimnom smislu.
	
	\begin{reibun}
		\rei{かれ の にっき}{njegov dnevnik}
		\rei{わたし の かのじょ の にっき}{dnevnik moje cure}
	\end{reibun}
	
	
	\begin{tabular}{|l|l|l|}
		\hline
		きみ&ti&inferiorni ženski\\\hline
		あんた&ti&inferiorni muški\\\hline
		おまえ&ti&inferiorni\\\hline
	\end{tabular}
	
	\vspace{10pt}
	
		Ako razgovaramo sa osobom koja nam je inferiorna ili na istoj razini tada možemo koristiti inferiorne osobne zamjenice, uz napomenu da one mogu zvučati poprilično nepristojno i uvrijediti osobu.
		
	\vspace{10pt}
	
	\begin{tabular}{|l|l|l|}
		\hline
		てめえ&ti&veoma uvredljivo\\\hline
		きさま&ti&veoma uvredljivo\\\hline
	\end{tabular}
	
	\vspace{10pt}
	
		てめえ i きさま se koriste često u fikciji, ali u stvarnom svijetu koriste se samo ako se želite za ozbiljno potući jer su te osobne zamjenice toliko uvredljive da su praktički uvrede same po sebi.
		
		\vspace{100pt}
		
	\begin{tabular}{|l|l|l|}
		\hline
		あたし&ja&ženstveniji ženski\\\hline
		うち&ja&blagi\\\hline
		ぼく&ja&blagi muški\\\hline
	\end{tabular}
	
	\vspace{10pt}
	
	あたし može zvuči slatko, pa ga češće koriste mlađe žene. ぼく koriste dječaci, ali i muškarci kad žele zvučati pristojnije.
	
	
	\vspace{10pt}
	
	\begin{tabular}{|l|l|l|}
		\hline
		わたくし&ja&formalni\\\hline
		おれ&ja&arogantni muški\\\hline
		わし&ja&koriste stari muškarci\\\hline
		われ&ja&formalni ponosni\\\hline
		われわれ&mi&formalni ponosni\\\hline
	\end{tabular}

	\vspace{10pt}	
	
	われ i われわれ koriste tvrtke u poslovnim mail-ovima.
	
	\fukudai{Nastavci za imena}
	
	\ten{Nastavci vezani uz društevni položaj}

	\vspace{10pt}

	-さん	koristimo za nepoznate osobe i osobe prema kojima želimo biti pristojni, te općenito za odrasle osobe. Neke imenice već u sebi imaju nastavak さん i obično su iskaz poštovanja prema tuđem članu obitelji.
	
	\vspace{10pt}	
	
	\begin{tabular}{|l|l||l||l|}
		\hline
		おとうさん&otac&おじさん&stric,stari čovjek\\\hline
		おかあさん&majka&おばさん&teta,stara žena\\\hline
		おにいさん&stariji brat&おくさん&žena (od muža)\\\hline
		おねえさん&starija sestra&みなさん&svi\\\hline
		おじいさん&stric&おおやさん&stanodavac, stanodavka\\\hline
		おばあさん&teta&じょうさん&kćer, mlada dama\\\hline
	\end{tabular}
	
	\vspace{10pt}
	
	
	-さま koristimo za osobe prema kojima želimo iskazati posebno poštovanje.
	
	\vspace{10pt}
	
	\begin{tabular}{|l|l|l|}
		\hline
		みなさま&(poštovani) svi\\\hline
		おきゃくさま&(poštovani) gost, kupac, klijent\\\hline
		おかあさま&(poštovana) majka\\\hline
		ひめさま&princeza\\\hline
		かみさま&Bog\\\hline
	\end{tabular}
	
	\vspace{10pt}	
	
	\begin{reibun}
		\rei{おきゃくさま の おくさん は ダーク チョコレート が すき}{Kupčeva žena voli tamnu čokoladu.}
	\end{reibun}
	
	\vspace{200pt}	
	
	
	-ちゃん zvuči djetinjasto, slatko i možemo koristiti među prijateljima. Kada pridodamo ちゃん na kraju nečijeg imena to označava da nam je ta osoba posebno draga. Najčešće se pridodaje curama.
	
	\vspace{10pt}	
	
	-くん se koristi među prijateljima ili za mlađu osobu. Najčešće se koristi na kraju imena muških osoba, ali ako se koristi za cure tada zvuči ozbiljnije i kao iskaz poštovanja.
	
	\vspace{10pt}

Odrasle osobe će za svoje prijatelje najčešće koristiti -さん.

	\vspace{10pt}
	
	-たち dodajemo na osobne zamjenice ili direktno na ime kada želimo govoriti o grupi ljudi i to na način da na ime osobe ili osobnu zamjenicu koja predstavlja tu grupu dodamo nastavak -たち.


	\begin{reibun}
		\rei{ともこたち と いく。}{Idem s Tomoko i njezinima...}
		\rei{かのじょたち が さけんだ。}{Zavrištale su.}
	\end{reibun}

	\vspace{10pt}

	\fukudai{Intimiziranje stvari sa さん i さま}
	
	Japanci ponekad koriste nastavke さん i さま kako bi dočarali prisnost ili opisali nešto što im je drago ili blisko srcu.
	
	\vspace{10pt}	
	

\begin{tabular}{|l|p{400pt}l|}
		\hline
		ケーキやさん&simpatična slastičarnica koja nam je draga\\\hline
		てんちょうさん&dragi voditelj trgovine 
		
		(npr. Netko u čiju trgovinu često dolazite pa imate osjećaj kao da se poznajete)\\\hline
		おまわりさん&dragi policajac koji patrolira ulicom\\\hline
		おひさま&sunašce\\\hline
	\end{tabular}

	\vspace{10pt}	
	
	\ten{Nastavci vezani uz zvanja}
	
	Određena zvanja ili pozicije u društvu možemo koristiti kao nastavke na kraju imena:
	
	\vspace{10pt}
	
	\begin{tabular}{|l|l|}
		\hline
		しゃちょう&vlasnik tvrtke\\\hline
		ぶちょう&direktor odjela\\\hline
		しゅにん&glavna odgovorna osoba, šef\\\hline
		せんせい&učitelj\\\hline
		せんぱい&iskusnija osoba od koje učite\\\hline
		てんちょう&voditelj trgovine\\\hline
	\end{tabular}
	
	
	
	\begin{reibun}
		\rei{たなかしゅにん、しょるい を どうぞ!}{Šefe Tanaka, izvolite papire!}
		\rei{おのせんせい は げんき です か?}{Je li učitelj Ono u redu?}
		\rei{けいこちゃん は まつだせんぱい が きらい。}{Keiko mrzi Matsudu.}
	\end{reibun}

\fukudai{Čestice や i と}
wip
	
\end{document}