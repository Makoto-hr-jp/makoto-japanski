% !TeX document-id = {048a8486-122a-4c5e-ab1d-890bfad423dd}
% !TeX program = xelatex ?me -synctex=0 -interaction=nonstopmode -aux-directory=../tex_aux -output-directory=./release
% !TeX program = xelatex

\documentclass[12pt]{article}

\usepackage{lineno,changepage,lipsum}
\usepackage[colorlinks=true,urlcolor=blue]{hyperref}
\usepackage{fontspec}[ Path =../../../ ]
\usepackage{xeCJK}
\usepackage{tabularx}
\usepackage{graphicx}
\setCJKfamilyfont{chanto}{AOZORAMINCHOREGULAR_0.TTF}%
\setCJKfamilyfont{tegaki}{Mushin.otf}%
\usepackage[CJK,overlap]{ruby}
\usepackage{hhline}
\usepackage{multirow,array,amssymb}
\usepackage[croatian]{babel}
\usepackage{soul}
\usepackage[usenames, dvipsnames]{color}
\usepackage{wrapfig,booktabs}
\usepackage{calc}
\renewcommand{\rubysep}{0.1ex}
\renewcommand{\rubysize}{0.75}
\usepackage[margin=50pt]{geometry}
\usepackage{hyperref}
\modulolinenumbers[2]

\date{\today}

\usepackage{fancyhdr}
\pagestyle{fancy}
\fancyhf{}
\fancyhead[LE,RO]{\thepage}
\makeatletter
\fancyhead[RE,LO]{rev. \@date 誠}
\makeatother

\usepackage{pifont}
\newcommand{\cmark}{\ding{51}}%
\newcommand{\xmark}{\ding{55}}%

\newcommand{\dosl}{{\normalfont dosl. }}%
\newcommand{\rem}[1]{{\normalfont #1 }}%

\definecolor{faded}{RGB}{100, 100, 100}

\renewcommand{\arraystretch}{1.2}

%\ruby{}{}
%$($\href{URL}{text}$)$

\newcommand{\furigana}[2]{\ruby{#1}{#2}}
\newcommand{\tegaki}[1]{
	\CJKfamily{tegaki}\CJKnospace
	#1
	\CJKfamily{chanto}\CJKnospace
}

\newcommand{\dai}[1]{
	\vspace{20pt}
	\large
	\noindent\textbf{#1}
	\normalsize
	\vspace{20pt}
}

\newcommand{\fukudai}[1]{
	\vspace{10pt}
	\noindent\textbf{#1}
	\vspace{10pt}
}

\newenvironment{bunshou}{
	\vspace{10pt}
	\begin{adjustwidth}{1cm}{3cm}
	\begin{linenumbers}
}{
	\end{linenumbers}
	\end{adjustwidth}
}

\newenvironment{reibun}[1][]{
	\vspace{10pt}
	#1
	
	\begin{tabular}{l l}
}{
	\end{tabular}
	\vspace{10pt}
}
\newcommand{\rei}[2]{
	#1&\textit{#2}\\
}
\newcommand{\reinagai}[2]{
	\multicolumn{2}{l}{#1}\\
	\multicolumn{2}{l}{\hspace{10pt}\textit{#2}}\\
}

\newenvironment{mondai}[1]{
	\vspace{10pt}
	\noindent #1
	
	\begin{enumerate}
		\itemsep-5pt
	}{
	\end{enumerate}
}

\newenvironment{hyou}{
	\begin{itemize}
		\itemsep-5pt
	}{
	\end{itemize}
	\vspace{10pt}
}

\newcommand{\juuyou}[2][20pt]{
	\vspace{5pt}
		\noindent\hspace{#1}\parbox[c]{\textwidth-#1-#1}{\centering\textit{#2}}
	\vspace{5pt}
}

\newcommand{\ten}{
	\vspace{5pt}
	\noindent\hspace{-10pt}$\bullet$
}

\CJKfamily{chanto}\CJKnospace

\frenchspacing
\author{Tomislav Mamić, Željka Ludošan}

\begin{document}
	\dai{Imenice i pokazne zamjenice}
	
	\vspace{10pt}
	\begin{tabular}{|l|l|l||l||l||l||l||l||l|}
		\hline
		植物&しょくぶつ&biljke&動物&どうぶつ&životinje&果物&くだもの&voće\\\hline
		木&き&drvo&鳥&とり&ptica&りんご&&jabuka\\\hline
		川&かわ&rijeka&庭鳥&にわとり&kokoš&バナナ&&banana\\\hline
		空&そら&nebo&犬&いぬ&pas&なし&&kruška\\\hline
		葉&は&list&猫&ねこ&mačka&梅&うめ&šljiva\\\hline
		土&つち&zemlja&虫&むし&kukac&いちご&&jagoda\\\hline
		丘&おか&brijeg&蚊&か&komarac&オレンジ&&naranča\\\hline
		庭&にわ&dvorište&馬&うま&konj\\\hline
	\end{tabular}
	\vspace{10pt}

	
	\fukudai{Pokazne zamjenice - これ、それ、あれ\footnotemark[1]}

	Pokazne zamjenice これ、それ、あれ ponašaju se kao imenice i koriste se za ukazivanje na neku stvar, osobu, događaj ili čak apstraktan pojam. U japanskom jeziku pokazne zamjenice veoma ovise o tome koliko je stvar na koju se ukazuje udaljena od govornika.
	
	\vspace{10pt}
	\begin{tabular}{|l|l|l|}
		\hline
		これ&ovo&ukazuje na nešto što je relativno blizu nama\\\hline
		それ&ono&ukazuje na nešto što je blizu sugovorniku, a udaljeno od nas\\\hline
		あれ&ono tamo&ukazuje na nešto što je udaljeno i od nas i od sugovornika\\\hline
	\end{tabular}
	\vspace{10pt}
	
	Zamislimo da u ruci držimo jabuku. Zatim nam dođe netko i pita nas što je to.
	Tada možemo reći \textit{これはりんごです。 Ovo je jabuka}. Ako tu istu jabuku dodamo toj drugoj osobi tada možemo reći:	\textit{それはりんごです。 Ono je jabuka.} Ako smo ukrali tu jabuku i ne želimo da nas ulove možemo baciti jabuku u susjedovo dvorište i reći \textit{あれはりんごです。Ono tamo je jabuka.}
	
		\footnotetext[1]{Iako se これ、それ、あれ i この、その、あの mogu pisati sa kanji znakovima, ali u pravilu se pišu u hiragani.}
		\footnotetext[2]{木の葉 se zapravo češće čita kao このは}
	
		\fukudai{Posvojna zamjenica - の}
	
	Čestica の može se koristiti u razne svrhe. Kao posvojnu zamjenicu koristimo ju za opisivanje imenica i ukazivanje na pripadnost.

	
	\juuyou[200pt]{<riječ kojom opisujemo>の<riječ koja se opisuje>}
	
	\begin{reibun}
		\rei{\furigana{花}{はな}の\furigana{色}{いろ}}{Boja cvijeta.}
	\end{reibun}
	
	Imenica “boja” opisuje se sa imenicom “cvijet”, pa tako 花の 色 postaje “boja od cvijeta”.
	
	\begin{reibun}
		\rei{\furigana{川}{かわ}の\furigana{名前}{なまえ}}{Ime rijeke.}
		\rei{\furigana{木}{き}の\furigana{葉}{は}\footnotemark[1]}{Lišće od drva.}
	\end{reibun}
		
	\vspace{200pt}
		
	Osim imenica mogu se koristiti i pokazne zamjenice, ali tada one mijenjaju oblik u この、その、あの.\footnotemark[1]

	\vspace{10pt}
	\begin{tabular}{|l|l|l|}
		\hline
		これ&→&この\\\hline
		それ&→&その\\\hline
		あれ&→&あの\\\hline
	\end{tabular}
	\vspace{10pt}
	
	\begin{reibun}
		\rei{このオレンジ}{Ova naranča.}
		\rei{あの\furigana{鳥}{とり}}{Ona tamo ptica.}
		\rei{となりの\furigana{家}{いえ}}{Kuća pokraj.}
		\rei{あそこの\furigana{建物}{たてもの} }{Tamošnja zgrada.}
		\rei{この\furigana{椅子}{いす}の\furigana{高}{たか}さは1メートルです。}{Visina ove stolice je 1 metar.}
	\end{reibun}
	
	Imenica \textit{visina} opisuje se sa imenicom \textit{stolica}, pa tako \textit{椅子の高さ} postaje \textit{visina od stolice}.

	\begin{reibun}
		\rei{この椅子は\furigana{学校}{がっこう}の椅子です。}{Ovo je školska stolica.}
	\end{reibun}
	

	Osim imenica mogu se koristiti i osobne zamjenice.
	
	\begin{reibun}
		\rei{\furigana{私}{わたし}の\furigana{猫}{ねこ}}{Moja mačka.}
		\rei{\furigana{田中}{たなか}さんの\furigana{本}{ほん}}{Knjiga od Tanake.}
		\rei{この本は田中さんの本です。}{Ova knjiga je Tanakina knjiga.}
	\end{reibun}
	
	Također, mogu se koristiti i vremenske oznake.
	
	\begin{reibun}
		\rei{\furigana{今日}{きょう}の\furigana{天気}{てんき} }{Vrijeme danas.}
		\rei{\furigana{来年}{らいねん}の\furigana{誕生日}{たんじょうび} }{Rođendan sljedeće godine.}
	\end{reibun}
	
	Posvojna zamjenica の može se koristiti za spajanje više međusobno povezanih opisa za redom.
	
	\begin{reibun}
		\rei{\furigana{友達}{ともだち}の\furigana{部屋}{へや}の\furigana{窓}{まど}。}{Prozor od prijateljeve sobe.
}
	\end{reibun}


	Ovdje se nalaze dva opisa.  \textit{友達の部屋 Soba od prijatelja} i \textit{部屋の窓 Prozor od sobe}.
	
	
	\vspace{200pt}
	\normalsize \textbf{Primjeri za vježbu}

	\furigana{昨日}{きのう}のニューズ
	\vspace{20pt}
	
	
	バナナの\furigana{味}{あじ}
	\vspace{20pt}
	
	
	その\furigana{人}{ひと}の\furigana{言い方}{いいかた}
	\vspace{20pt}
	
	
	\furigana{今週}{こんしゅう}の\furigana{日曜日}{にちようび}
	\vspace{20pt}
	
	
	\furigana{今}{いま}の\furigana{所}{ところ}
	\vspace{20pt}
	
	
	\furigana{皆}{みんな}の\furigana{空}{そら}
	
	


\end{document}