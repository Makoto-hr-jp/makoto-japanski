% !TeX document-id = {8145854b-39f5-4ea7-9dce-e23885b92db3}
% !TeX program = xelatex ?me -synctex=0 -interaction=nonstopmode -aux-directory=../tex_aux -output-directory=./release
% !TeX program = xelatex

\documentclass[12pt]{article}

\usepackage{lineno,changepage,lipsum}
\usepackage[colorlinks=true,urlcolor=blue]{hyperref}
\usepackage{fontspec}[ Path =../../../ ]
\usepackage{xeCJK}
\usepackage{tabularx}
\usepackage{graphicx}
\setCJKfamilyfont{chanto}{AOZORAMINCHOREGULAR_0.TTF}%
\setCJKfamilyfont{tegaki}{Mushin.otf}%
\usepackage[CJK,overlap]{ruby}
\usepackage{hhline}
\usepackage{multirow,array,amssymb}
\usepackage[croatian]{babel}
\usepackage{soul}
\usepackage[usenames, dvipsnames]{color}
\usepackage{wrapfig,booktabs}
\usepackage{calc}
\renewcommand{\rubysep}{0.1ex}
\renewcommand{\rubysize}{0.75}
\usepackage[margin=50pt]{geometry}
\usepackage{hyperref}
\modulolinenumbers[2]

\date{\today}

\usepackage{fancyhdr}
\pagestyle{fancy}
\fancyhf{}
\fancyhead[LE,RO]{\thepage}
\makeatletter
\fancyhead[RE,LO]{rev. \@date 誠}
\makeatother

\usepackage{pifont}
\newcommand{\cmark}{\ding{51}}%
\newcommand{\xmark}{\ding{55}}%

\newcommand{\dosl}{{\normalfont dosl. }}%
\newcommand{\rem}[1]{{\normalfont #1 }}%

\definecolor{faded}{RGB}{100, 100, 100}

\renewcommand{\arraystretch}{1.2}

%\ruby{}{}
%$($\href{URL}{text}$)$

\newcommand{\furigana}[2]{\ruby{#1}{#2}}
\newcommand{\tegaki}[1]{
	\CJKfamily{tegaki}\CJKnospace
	#1
	\CJKfamily{chanto}\CJKnospace
}

\newcommand{\dai}[1]{
	\vspace{20pt}
	\large
	\noindent\textbf{#1}
	\normalsize
	\vspace{20pt}
}

\newcommand{\fukudai}[1]{
	\vspace{10pt}
	\noindent\textbf{#1}
	\vspace{10pt}
}

\newenvironment{bunshou}{
	\vspace{10pt}
	\begin{adjustwidth}{1cm}{3cm}
	\begin{linenumbers}
}{
	\end{linenumbers}
	\end{adjustwidth}
}

\newenvironment{reibun}[1][]{
	\vspace{10pt}
	#1
	
	\begin{tabular}{l l}
}{
	\end{tabular}
	\vspace{10pt}
}
\newcommand{\rei}[2]{
	#1&\textit{#2}\\
}
\newcommand{\reinagai}[2]{
	\multicolumn{2}{l}{#1}\\
	\multicolumn{2}{l}{\hspace{10pt}\textit{#2}}\\
}

\newenvironment{mondai}[1]{
	\vspace{10pt}
	\noindent #1
	
	\begin{enumerate}
		\itemsep-5pt
	}{
	\end{enumerate}
}

\newenvironment{hyou}{
	\begin{itemize}
		\itemsep-5pt
	}{
	\end{itemize}
	\vspace{10pt}
}

\newcommand{\juuyou}[2][20pt]{
	\vspace{5pt}
		\noindent\hspace{#1}\parbox[c]{\textwidth-#1-#1}{\centering\textit{#2}}
	\vspace{5pt}
}

\newcommand{\ten}{
	\vspace{5pt}
	\noindent\hspace{-10pt}$\bullet$
}

\CJKfamily{chanto}\CJKnospace

\frenchspacing
\author{Tomislav Mamić}
\begin{document}
	\dai{Pitanja i završne čestice}
	
	U prethodnim smo lekcijama naučili rečenici dodati razne informacije. Često smo to čineći naglašavali na koje pitanje odgovara koji dio rečenice - sada je vrijeme da naučimo ta pitanja i postaviti.
	
	\fukudai{Čestice na kraju rečenice}
	
	Dosad smo naučili razne čestice koje dolaze uz imenice i određuju njihovu ulogu u rečenici. Osim takvih, postoje i čestice koje dolaze na kraju rečenice, iza predikata. One na razne načine nadopunjuju značenje ili mu dodaju osjećaje govornika. U ovoj ćemo lekciji naučiti tri najvažnije i najčešće: よ, ね i か.
	
	\fukudai{Naglašavanje česticom よ}
	
	Kad se pojavi na kraju rečenice, ova čestica sugovorniku na nju svraća pozornost. Koristimo je kad želimo istaknuti da rečenica sadrži neku novu informaciju:
	
	\begin{reibun}
		\rei{たけしくんは けさ、日本にかえった。}{Takeši se jutros vratio u Japan.}
		\rei{たけしくんは けさ、日本にかえった\underline{よ}。}{Hej, Takeši se jutros vratio u Japan!}
	\end{reibun}

	Koliko je naglasak dramatičan uvelike ovisi i o načinu na koji je rečenica izgovorena - na ovo treba dobro obratiti pozornost pri slušanju. Ponekad se ova čestica može pojaviti i odmah iza imenice (kao čestice na koje smo dosad navikli). U tom slučaju radi kao vokativ u hrvatskom jeziku:
	
	\begin{reibun}
		\rei{たけし\underline{よ}、日本にかえれ\footnotemark[1]。}{Hej Takeši, vrati se u Japan.}
	\end{reibun}

	\footnotetext[1]{Ovaj oblik glagola koji završava na \textit{\textasciitilde e} još neko vrijeme nećemo učiti, ali radi se o imperativu koji često može zvučati grubo i nepristojno.}
	
	\fukudai{Traženje potvrde česticom ね}
	
	Vrlo slično hrvatskom, dodamo li na kraju rečenice ね, očekujemo od sugovornika da nam potvrdi ono što smo upravo rekli. Kao i ranije, intonacija je jako bitna, ali i taj je aspekt vrlo sličan hrvatskom. Završi li rečenica uzlaznom intonacijom, zaista od sugovornika očekujemo potvrdu, no ako je intonacija silazna, radi se o retoričkom pitanju.
	
	\begin{reibun}
		\rei{あの\furigana{花}{はな}は きれいです。}{Onaj cvijet je lijep.}
		\rei{あの花は きれいです\underline{ね}。}{Onaj cvijet je lijep, zar ne?}
	\end{reibun}

	\fukudai{Upitna rečenica česticom か}
	
	Osnovna funkcija ove čestice slična je u hrvatskom znaku \textit{?} (upitnik, nije greška u ispisu). Da bi bile gramatički potpune, sve upitne rečenice u japanskom trebaju na kraju imati česticu か, ali u kolokvijalnom govoru ona se često izostavlja kad je iz ostatka rečenice jasno da se radi o pitanju. U jednostavnim situacijama, ova čestica pretvara rečenicu u da/ne pitanje:
	
	\begin{reibun}
		\rei{花子さんは たけしくんを みた。}{Hanako je vidjela Takešija.}
		\rei{花子さんは たけしくんを みた\underline{か}。}{Je li Hanako vidjela Takešija?}
	\end{reibun}
	\begin{reibun}
		\rei{たけしくんは サルマを たべた。}{Takeši je pojeo sarmu.}
		\rei{たけしくんは サルマを たべた\underline{か}。}{Je li Takeši pojeo sarmu?}
	\end{reibun}

	Ako rečenica završava neprošlim kolokvijalnim oblikom spojnog glagola (zdravo seljački だ), onda je običaj iz upitne rečenice taj だ izbaciti:
	
	\begin{reibun}
		\rei{花子さんは サルマが きらい\underline{だ}。}{Hanako mrzi sarmu.}
		\rei{花子さんは さるまが きらい\underline{か}。}{Mrzi li Hanako sarmu?}
	\end{reibun}

	\fukudai{Složenije upitne rečenice}
	
	Naučili smo česticom か postavljati da/ne pitanja, no što ako želimo neku precizniju informaciju? U jezicima na koje smo navikli, postavljanje pitanja uključuje i preslagivanje redoslijeda riječi u rečenici, no u japanskom to nije slučaj. Pitanje ćemo napraviti tako da na mjesto gdje bi se u rečenici pojavila informacija koja nas zanima postavimo odgovaraju upitnu riječ:
	
	\begin{reibun}
		\rei{すずきさんは \underline{ともだち}と はなした。}{Suzuki je razgovarala s prijateljem.}
		\rei{すずきさんは \underline{だれ}と はなした\underline{か}。}{S kim je Suzuki razgovarala?}
	\end{reibun}

	U nastavku se nalaze korisne upitne riječi za informacije koje zasad znamo izreći.
	
	\vspace{10pt}
	\begin{table}[h]
		\centering
		\begin{tabular}{l l l}\toprule[2pt]
			riječ & zamjenjuje & primjer odgovora\\
			\midrule
			だれ & \textit{tko} & すずきさん、あの人\\
			なん/なに & \textit{što} & りんご、はこ\\
			どれ & \textit{koje} & あれ、これ\\
			どの<nešto> & \textit{koje}<nešto> & このねこ、そのいえ\\
			どこ & \textit{gdje} & そこ、いえのうしろ\\
			どんな<nešto> & \textit{kakvo}<nešto> & きれいな、あかい\\
			どう & \textit{kako} & きれいに、あかく\\
			なん + <brojač> & \textit{koliko} & 三人、五匹\\
			\bottomrule
		\end{tabular}
	\end{table}

	\vspace{5pt}
	Zamjena tražene informacije veoma je doslovna - jedino na što trebamo obratiti pozornost je da upitne riječi nikad ne smiju biti označene česticom は - nju ćemo pretvoriti u が.

	\fukudai{Vježba}
	
	\begin{mondai}{Prevedite sljedeća pitanja na hrvatski i odgovorite na njih na japanskom.}
		\item だれが たけしくんの ケーキを たべたか。
		\item 花子さんは なにを たべたか。
		\item どれが あなたの ものですか。
		\item どんなアイスクリームを かったか。
		\item その かばん、どこで かったか。
		\item パーティに ともだちが なん人 きたか。
		\item コーヒー、どうですか。
	\end{mondai}
\end{document}