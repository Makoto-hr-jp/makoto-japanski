% !TeX document-id = {560a1cad-8395-444a-8f36-5986f0ba0d2b}
% !TeX program = xelatex ?me -synctex=0 -interaction=nonstopmode -aux-directory=../tex_aux -output-directory=./release
% !TeX program = xelatex

\documentclass[12pt]{article}

\usepackage{lineno,changepage,lipsum}
\usepackage[colorlinks=true,urlcolor=blue]{hyperref}
\usepackage{fontspec}
\usepackage{xeCJK}
\usepackage{tabularx}
\setCJKfamilyfont{chanto}{AozoraMinchoRegular.ttf}
\setCJKfamilyfont{tegaki}{Mushin.otf}
\usepackage[CJK,overlap]{ruby}
\usepackage{hhline}
\usepackage{multirow,array,amssymb}
\usepackage[croatian]{babel}
\usepackage{soul}
\usepackage[usenames, dvipsnames]{color}
\usepackage{wrapfig,booktabs}
\renewcommand{\rubysep}{0.1ex}
\renewcommand{\rubysize}{0.75}
\usepackage[margin=50pt]{geometry}
\modulolinenumbers[2]

\usepackage{pifont}
\newcommand{\cmark}{\ding{51}}%
\newcommand{\xmark}{\ding{55}}%

\definecolor{faded}{RGB}{100, 100, 100}

\renewcommand{\arraystretch}{1.2}

%\ruby{}{}
%$($\href{URL}{text}$)$

\newcommand{\furigana}[2]{\ruby{#1}{#2}}
\newcommand{\tegaki}[1]{
	\CJKfamily{tegaki}\CJKnospace
	#1
	\CJKfamily{chanto}\CJKnospace
}

\newcommand{\dai}[1]{
	\vspace{20pt}
	\large
	\noindent\textbf{#1}
	\normalsize
	\vspace{20pt}
}

\newcommand{\fukudai}[1]{
	\vspace{10pt}
	\noindent\textbf{#1}
	\vspace{10pt}
}

\newenvironment{bunshou}{
	\vspace{10pt}
	\begin{adjustwidth}{1cm}{3cm}
	\begin{linenumbers}
}{
	\end{linenumbers}
	\end{adjustwidth}
}

\newenvironment{reibun}{
	\vspace{10pt}
	\begin{tabular}{l l}
}{
	\end{tabular}
	\vspace{10pt}
}
\newcommand{\rei}[2]{
	#1&\textit{#2}\\
}
\newcommand{\reinagai}[2]{
	\multicolumn{2}{l}{#1}\\
	\multicolumn{2}{l}{\hspace{10pt}\textit{#2}}\\
}

\newenvironment{mondai}[1]{
	\vspace{10pt}
	#1
	
	\begin{enumerate}
		\itemsep-5pt
	}{
	\end{enumerate}
	\vspace{10pt}
}

\newenvironment{hyou}{
	\begin{itemize}
		\itemsep-5pt
	}{
	\end{itemize}
	\vspace{10pt}
}

\date{\today}

\CJKfamily{chanto}\CJKnospace
\usepackage{xcolor}
\author{Tomislav Mamić}
\begin{document}
	\dai{Količina I}
	
	Prije susreta s civilizacijom Kine, japanski jezik imao je svoj sustav brojanja koji je i danas prisutan u jeziku, ali u tragovima. U modernom jeziku uglavnom se koriste brojevi "uvezeni" iz Kine zajedno s kanji znakovima. Od starog sustava dovoljno je znati samo brojeve do deset, dok ćemo na novom sustavu naučiti brojati dokle god to praktično ima smisla.
	
	\fukudai{Osnovni brojevi}
	
	Tablica ispod prikazuje stare brojeve s nastavkom za brojanje neživih stvari (つ). Ovi se brojevi koriste i danas. Valja primijetiti dvije stvari - pišu se kineskim znakom na kojeg je dodan brojač, a broj deset nema brojač već se on podrazumijeva kad se koristi staro čitanje.

	Izuzev ovih deset brojeva, ostaci starog sustava mogu se naći u imenima i skamenjenim izrazima (npr. や.お.よろず - dosl. \textit{osam deset desettisuća} = 800 000, u starijim tekstovima izraz za \textit{jako puno}), kao i u nekim brojačima što ćemo vidjeti kasnije.
	
	Paralelno su prikazani i novi brojevi koji uz sebe nemaju brojač, a kojima se broje i veće količine. U zagradama se nalaze čitanja koja se koriste rjeđe i u posebnim situacijama.
	
	\vspace{5pt}
	\begin{table}[h]
		\centering
		\begin{tabular}{r l l l l}\toprule[2pt]
			rimski & novo čitanje & kanji & staro čitanje & kanji\\
			\midrule
			1			& いち & 一 & ひと.つ & 一つ \\
			2			& に & 二 & ふた.つ & 二つ \\
			3			& さん & 三 & みっ.つ & 三つ \\
			4			& よん(し) & 四 & よっ.つ & 四つ \\
			5			& ご & 五 & いつ.つ & 五つ \\
			6			& ろく & 六 & むっ.つ & 六つ \\
			7			& なな(しち) & 七 & なな.つ & 七つ \\
			8			& はち & 八 & やっ.つ & 八つ \\
			9			& きゅう(く) & 九 & ここの.つ & 九つ \\
			10			& じゅう & 十 & とお & 十 \\
			100			& ひゃく & 百 &  &  \\
			1000		& せん & 千 &  &  \\
			10 000		& まん & 万 &  &  \\
			100 000 000	& おく & 億 &  &  \\
			\bottomrule
		\end{tabular}
	\end{table}

	\vspace{5pt}
	Pogledamo li pomno veće brojeve, uočit ćemo da je svaki sljedeći imenovani broj 10 000 puta (4~nule) veći od prethodnog. U hrvatskom jeziku, imenovani brojevi uvećavaju se za po 1000 puta (3~nule). Ovo je u početku poprilično neugodna stvar na koju će se trebati naviknuti jer veći brojevi osim prevođenja zahtjevaju i preračunavanje u glavi.

	\newpage
	\fukudai{Kombiniranje u veće brojeve}
	
	Osnovni princip po kojem se brojevi čitaju identičan je onom u hrvatskom jeziku. Kao i kod nas, pri spajanju se ponekad događaju glasovne promjene, a broj se izgovara od najveće znamenke prema nižima bez ikakvih iznenađenja. Pogledajmo nekoliko primjera:
	
	\begin{reibun}
		\rei{四十二}{četrdeset dva}
		\rei{三百六十四}{tristo šezdeset četiri}
		\rei{五千六百三十三}{pet tisuća šesto trideset tri}
	\end{reibun}

	Gledajući znakove vjerojatno bismo mogli odrediti o kojem se broju radi, no čitanje je dodatno zakomplicirano glasovnim promjenama. U tablici ispod navedene su kombinacije koje se mijenjaju. Za brojeve iznad tisuću (まん, おく) glasovne promjene se ne događaju.
	
	\vspace{5pt}
	\begin{table}[h]
		\centering
		\begin{tabular}{r l r r r}\toprule[2pt]
			rimski & osnovno čitanje & 十 (じゅう) & 百 (ひゃく) & 千 (せん)\\
			\midrule
			1			& いち & $\varnothing$ & $\varnothing$ & \colorbox{blue!10}{いっ.せん} \\
			2			& に & に.じゅう & に.ひゃく & に.せん \\
			3			& さん & さん.じゅう & \colorbox{blue!10}{さん.びゃく} & \colorbox{blue!10}{さん.ぜん} \\
			4			& よん & よん.じゅう & よん.ひゃく & よん.せん \\
			5			& ご & ご.じゅう & ご.ひゃく & ご.せん \\
			6			& ろく & ろく.じゅう & \colorbox{blue!10}{ろっ.ぴゃく} & ろく.せん \\
			7			& なな & なな.じゅう & なな.ひゃく & なな.せん \\
			8			& はち & はち.じゅう & \colorbox{blue!10}{はっ.ぴゃく} & \colorbox{blue!10}{はっ.せん} \\
			9			& きゅう & きゅう.じゅう & きゅう.ひゃく & きゅう.せん \\
			\bottomrule
		\end{tabular}
	\end{table}
	
\end{document}