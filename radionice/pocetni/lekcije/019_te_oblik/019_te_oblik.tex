% !TeX document-id = {889ad5e7-f888-488d-90dc-2167e337f651}
% !TeX program = xelatex ?me -synctex=0 -interaction=nonstopmode -aux-directory=../tex_aux -output-directory=./release
% !TeX program = xelatex

\documentclass[12pt]{article}

\usepackage{lineno,changepage,lipsum}
\usepackage[colorlinks=true,urlcolor=blue]{hyperref}
\usepackage{fontspec}[ Path =../../../ ]
\usepackage{xeCJK}
\usepackage{tabularx}
\usepackage{graphicx}
\setCJKfamilyfont{chanto}{AOZORAMINCHOREGULAR_0.TTF}%
\setCJKfamilyfont{tegaki}{Mushin.otf}%
\usepackage[CJK,overlap]{ruby}
\usepackage{hhline}
\usepackage{multirow,array,amssymb}
\usepackage[croatian]{babel}
\usepackage{soul}
\usepackage[usenames, dvipsnames]{color}
\usepackage{wrapfig,booktabs}
\usepackage{calc}
\renewcommand{\rubysep}{0.1ex}
\renewcommand{\rubysize}{0.75}
\usepackage[margin=50pt]{geometry}
\usepackage{hyperref}
\modulolinenumbers[2]

\date{\today}

\usepackage{fancyhdr}
\pagestyle{fancy}
\fancyhf{}
\fancyhead[LE,RO]{\thepage}
\makeatletter
\fancyhead[RE,LO]{rev. \@date 誠}
\makeatother

\usepackage{pifont}
\newcommand{\cmark}{\ding{51}}%
\newcommand{\xmark}{\ding{55}}%

\newcommand{\dosl}{{\normalfont dosl. }}%
\newcommand{\rem}[1]{{\normalfont #1 }}%

\definecolor{faded}{RGB}{100, 100, 100}

\renewcommand{\arraystretch}{1.2}

%\ruby{}{}
%$($\href{URL}{text}$)$

\newcommand{\furigana}[2]{\ruby{#1}{#2}}
\newcommand{\tegaki}[1]{
	\CJKfamily{tegaki}\CJKnospace
	#1
	\CJKfamily{chanto}\CJKnospace
}

\newcommand{\dai}[1]{
	\vspace{20pt}
	\large
	\noindent\textbf{#1}
	\normalsize
	\vspace{20pt}
}

\newcommand{\fukudai}[1]{
	\vspace{10pt}
	\noindent\textbf{#1}
	\vspace{10pt}
}

\newenvironment{bunshou}{
	\vspace{10pt}
	\begin{adjustwidth}{1cm}{3cm}
	\begin{linenumbers}
}{
	\end{linenumbers}
	\end{adjustwidth}
}

\newenvironment{reibun}[1][]{
	\vspace{10pt}
	#1
	
	\begin{tabular}{l l}
}{
	\end{tabular}
	\vspace{10pt}
}
\newcommand{\rei}[2]{
	#1&\textit{#2}\\
}
\newcommand{\reinagai}[2]{
	\multicolumn{2}{l}{#1}\\
	\multicolumn{2}{l}{\hspace{10pt}\textit{#2}}\\
}

\newenvironment{mondai}[1]{
	\vspace{10pt}
	\noindent #1
	
	\begin{enumerate}
		\itemsep-5pt
	}{
	\end{enumerate}
}

\newenvironment{hyou}{
	\begin{itemize}
		\itemsep-5pt
	}{
	\end{itemize}
	\vspace{10pt}
}

\newcommand{\juuyou}[2][20pt]{
	\vspace{5pt}
		\noindent\hspace{#1}\parbox[c]{\textwidth-#1-#1}{\centering\textit{#2}}
	\vspace{5pt}
}

\newcommand{\ten}{
	\vspace{5pt}
	\noindent\hspace{-10pt}$\bullet$
}

\CJKfamily{chanto}\CJKnospace

\frenchspacing
\author{Tomislav Mamić}
\begin{document}
	\dai{て oblik}
	
	Još jedan spojni oblik glagola sličan い obliku. Gledajući tvorbu, nastaje iz い oblika baš kao i prošlost. Za dobivanje prošlosti, dodavali smo nastavak \textasciitilde~た, za ovaj ćemo oblik dodati nastavak \textasciitilde~て. Ovo će uzrokovati identične promjene na ごだん i nepravilnim glagolima, tako da tvorbu て oblika dobivamo gratis ako smo dobro naučili prošlost - samo ne završava samoglasnikom \textit{a} nego \textit{e}!
	
	\fukudai{Imperativ}
	
	Kad glagol u て obliku završava rečenicu, tumači se kao jednostavni imperativ. Ovakav govor je kolokvijalan i nije pristojan prema nepoznatim ljudima, a češće ga koriste žene. Dodavanjem pristojnog glagola ください, ovaj se imperativ ublažava u zamolbu (koja je većinom retorička i zapravo se još uvijek radi o blagoj naredbi) i u takvom obliku može se pristojno koristiti prema nepoznatim ljudima i u formalnim situacijama.
	
	\begin{reibun}
		\rei{りんごを たべて。}{Pojedi jabuku.}
		\rei{りんごを たべて ください。}{Molim te, pojedi jabuku.}
	\end{reibun}

	\fukudai{Sastavni veznik}
	
	Glagolima u て obliku možemo više rečenica povezati u uzročno-posljedični i/ili vremenski slijed. Ova uzastopnost je vrlo bitna za razumijevanje. U hrvatskom jeziku, tako spojene rečenice odgovaraju nezavisno složenim sastavnim rečenicama (\textit{i}, \textit{pa}).
	
	\begin{reibun}
		\rei{くつを はいた。いえを でました。}{Obukao sam cipele. Izišao sam iz kuće.}
		\rei{くつを はいて、いえを でました。}{Obukao sam cipele i izišao iz kuće.}
		\rei{おなかが すいた。りんごを たべた。}{Ogladnio sam. Pojeo sam jabuku.}
		\rei{おなかが すいて、りんごを たべた。}{Ogladnio sam pa sam pojeo jabuku.}
	\end{reibun}

	\fukudai{Trenutno stanje s pomoćnim glagolom いる}
	
	Kako već znamo, いる je glagol stanja koji koristimo kao \textit{biti, postojati} za živa bića. Kao pomoćni glagol uz て oblik, mijenja značenje glavnog glagola tako da opisuje trenutno stanje stvari:
	
	\begin{reibun}
		\rei{そらを みる。}{Gledati nebo. / Gledat ću nebo.}
		\rei{そらを みている。}{Gledam nebo. \textnormal{(upravo sad)}}
	\end{reibun}

	Budući je vrijeme u japanskom klizno u odnosu na glavni glagol, moguće je reći i
	
	\begin{reibun}
		\rei{そらを みていた。}{Gledao sam nebo. \textnormal{(upravo onda u prošlosti)}}
	\end{reibun}

	što se razlikuje od dosad nam poznatog
	
	\begin{reibun}
		\rei{そらを みた。}{Vidio sam nebo.}
	\end{reibun}

	Zbog osnovne ideje iza ovog oblika (opis trenutnog stanja), negiramo li glagol いる, možemo značenje protumačiti na dva načina:
	
	\begin{reibun}
		\rei{(まだ)そらを みていない。}{Upravo \textbf{ne} gledam nebo. \textnormal{ili} \textnormal{(još uvijek)} Nisam pogledao nebo.}
	\end{reibun}

	U praksi je vrlo čest slučaj da se dio u zagradama implicira. Tako je uobičajena razmjena sljedećeg formata:
	
	\begin{reibun}
		\rei{ひるごはんは (もう) たべたか?}{Jesi li \textnormal{(već)} ručao?}
		\rei{いえ、 (まだ) たべていない。}{Ne, nisam \textnormal{(još)}.}
	\end{reibun}

	Dopustimo li tome da "otkliže" u prošlost, dobit ćemo situaciju koja je iz perspektive hrvatskog jezika komplicirana, ali u japanskom ostaje jednostavna ako zapamtimo da て+いる izriče trenutno stanje (u kojem god trenutku ono bilo).
	
	\begin{reibun}
		\rei{そのときは まだ、 ひるごはんを たべていなかった。}{U tom trenutku još nisam bio ručao.}
	\end{reibun}

	\fukudai{Vježba}
	
	\begin{mondai}{Recite sljedeće rečenice na japanskom. U zagradama se nalaze glagoli koje smo rjeđe spominjali, a koji bi vam mogli u tome pomoći.}
		\item Očisti svoju sobu. (そうじする)
		\item Molim te iznesi smeće. (だす)
		\item Jutros sam ustao, pojeo doručak i otišao u dućan. (おきる)
		\item Molim te očisti sobu i iznesi smeće.
		\item Spavao sam.
		\item Još nisam iznio smeće.
		\item Igram se s mačkom. (あそぶ)
		\item Pričam s prijateljima.
		\item Još to nisam ispričao prijateljima.
		\item Suzuki tada još nije bio oženjen. (けっこんする)
	\end{mondai}
\end{document}