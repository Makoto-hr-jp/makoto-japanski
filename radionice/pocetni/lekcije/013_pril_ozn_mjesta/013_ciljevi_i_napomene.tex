% !TeX document-id = {25246ded-d1c2-4903-9fda-e0fc6790645f}
% !TeX program = xelatex ?me -synctex=0 -interaction=nonstopmode -aux-directory=../tex_aux -output-directory=./release
% !TeX program = xelatex

\documentclass[12pt]{article}

\usepackage{lineno,changepage,lipsum}
\usepackage[colorlinks=true,urlcolor=blue]{hyperref}
\usepackage{fontspec}
\usepackage{xeCJK}
\usepackage{tabularx}
\setCJKfamilyfont{chanto}{AozoraMinchoRegular.ttf}
\setCJKfamilyfont{tegaki}{Mushin.otf}
\usepackage[CJK,overlap]{ruby}
\usepackage{hhline}
\usepackage{multirow,array,amssymb}
\usepackage[croatian]{babel}
\usepackage{soul}
\usepackage[usenames, dvipsnames]{color}
\usepackage{wrapfig,booktabs}
\renewcommand{\rubysep}{0.1ex}
\renewcommand{\rubysize}{0.75}
\usepackage[margin=50pt]{geometry}
\modulolinenumbers[2]

\usepackage{pifont}
\newcommand{\cmark}{\ding{51}}%
\newcommand{\xmark}{\ding{55}}%

\definecolor{faded}{RGB}{100, 100, 100}

\renewcommand{\arraystretch}{1.2}

%\ruby{}{}
%$($\href{URL}{text}$)$

\newcommand{\furigana}[2]{\ruby{#1}{#2}}
\newcommand{\tegaki}[1]{
	\CJKfamily{tegaki}\CJKnospace
	#1
	\CJKfamily{chanto}\CJKnospace
}

\newcommand{\dai}[1]{
	\vspace{20pt}
	\large
	\noindent\textbf{#1}
	\normalsize
	\vspace{20pt}
}

\newcommand{\fukudai}[1]{
	\vspace{10pt}
	\noindent\textbf{#1}
	\vspace{10pt}
}

\newenvironment{bunshou}{
	\vspace{10pt}
	\begin{adjustwidth}{1cm}{3cm}
	\begin{linenumbers}
}{
	\end{linenumbers}
	\end{adjustwidth}
}

\newenvironment{reibun}{
	\vspace{10pt}
	\begin{tabular}{l l}
}{
	\end{tabular}
	\vspace{10pt}
}
\newcommand{\rei}[2]{
	#1&\textit{#2}\\
}
\newcommand{\reinagai}[2]{
	\multicolumn{2}{l}{#1}\\
	\multicolumn{2}{l}{\hspace{10pt}\textit{#2}}\\
}

\newenvironment{mondai}[1]{
	\vspace{10pt}
	#1
	
	\begin{enumerate}
		\itemsep-5pt
	}{
	\end{enumerate}
	\vspace{10pt}
}

\newenvironment{hyou}{
	\begin{itemize}
		\itemsep-5pt
	}{
	\end{itemize}
	\vspace{10pt}
}

\date{\today}

\CJKfamily{chanto}\CJKnospace
\author{Tomislav Mamić}
\begin{document}
	\dai{Ciljevi i napomene - priložne oznake mjesta}
	
	\fukudai{Ciljevi}
	
	\begin{hyou}
		\item Objasniti da se odnosi među imenicama u japanskom određuju drugim imenicama, a ne prijedlozima kao u hrvatskom.
		\item Naučiti i isprobati razne odnosne imenice.
		\item Objasniti čestice から, まで i へ u prostornom kontekstu.
		\item Kroz primjere se igrati i lokacijom i glagolima radi ponavljanja i utvrđivanja.
	\end{hyou}

	\fukudai{Napomene}
	
	\begin{hyou}
		\item Iako je samo izricanje lokacije jednostavno, načini na koje se povezuje s ostatkom rečenice znaju biti zakučasti. Neki glagoli za indirektni objekt umjesto に mogu koristiti から kako bi naglasili da je nešto \textit{od nekog}, moguće je pričati o lokaciji (lokacija je tema, npr. ここからは遠すぎるけど、あそこからならちゃんと見える。) i kombinirati je s drugim složenijim česticama (npr. だけ, しか, なら). Kako se ne bismo zapetljali u objašnjavanje tih stvari, samo ćemo natuknuti da se nešto takvo može, ali ćemo te primjere izbjegavati.
		\item U objašnjavanju posebno se fokusirati na korištenje glagola s に i で radi ponavljanja.
	\end{hyou}

	\fukudai{Pričice}
	
	U hrvatskoj lingvistici pojam \textit{odnosne imenice} uopće ne postoji. Razlog tome je jednostavan - te kategorije riječi u hrvatskom nema, a naša lingvistika je praktički nepostojeća. Izraz \textit{odnosne imenice} je prijevod engleskog \textit{relational nouns}. S obzirom da su \textit{imenice} i da određuju \textit{odnose} među riječima, učinilo mi se prikladno tako ga prevesti.
	
	Pri tome valja obratiti pažnju na pojam iz hrvatske gramatike - \textit{odnosne zamjenice} - koje nažalost imaju puno slabije opravdanje zvati se \textit{odnosne}, ali svejedno mogu navesti onog tko za njih zna na pomisao da između odnosnih zamjenica i odnosnih imenica postoji veza.
\end{document}