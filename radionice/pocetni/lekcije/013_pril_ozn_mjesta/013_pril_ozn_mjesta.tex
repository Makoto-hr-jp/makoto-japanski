% !TeX document-id = {35e39c97-dff0-45bf-83e6-4c483f09c6af}
% !TeX program = xelatex ?me -synctex=0 -interaction=nonstopmode -aux-directory=../tex_aux -output-directory=./release
% !TeX program = xelatex

\documentclass[12pt]{article}

\usepackage{lineno,changepage,lipsum}
\usepackage[colorlinks=true,urlcolor=blue]{hyperref}
\usepackage{fontspec}
\usepackage{xeCJK}
\usepackage{tabularx}
\setCJKfamilyfont{chanto}{AozoraMinchoRegular.ttf}
\setCJKfamilyfont{tegaki}{Mushin.otf}
\usepackage[CJK,overlap]{ruby}
\usepackage{hhline}
\usepackage{multirow,array,amssymb}
\usepackage[croatian]{babel}
\usepackage{soul}
\usepackage[usenames, dvipsnames]{color}
\usepackage{wrapfig,booktabs}
\renewcommand{\rubysep}{0.1ex}
\renewcommand{\rubysize}{0.75}
\usepackage[margin=50pt]{geometry}
\modulolinenumbers[2]

\usepackage{pifont}
\newcommand{\cmark}{\ding{51}}%
\newcommand{\xmark}{\ding{55}}%

\definecolor{faded}{RGB}{100, 100, 100}

\renewcommand{\arraystretch}{1.2}

%\ruby{}{}
%$($\href{URL}{text}$)$

\newcommand{\furigana}[2]{\ruby{#1}{#2}}
\newcommand{\tegaki}[1]{
	\CJKfamily{tegaki}\CJKnospace
	#1
	\CJKfamily{chanto}\CJKnospace
}

\newcommand{\dai}[1]{
	\vspace{20pt}
	\large
	\noindent\textbf{#1}
	\normalsize
	\vspace{20pt}
}

\newcommand{\fukudai}[1]{
	\vspace{10pt}
	\noindent\textbf{#1}
	\vspace{10pt}
}

\newenvironment{bunshou}{
	\vspace{10pt}
	\begin{adjustwidth}{1cm}{3cm}
	\begin{linenumbers}
}{
	\end{linenumbers}
	\end{adjustwidth}
}

\newenvironment{reibun}{
	\vspace{10pt}
	\begin{tabular}{l l}
}{
	\end{tabular}
	\vspace{10pt}
}
\newcommand{\rei}[2]{
	#1&\textit{#2}\\
}
\newcommand{\reinagai}[2]{
	\multicolumn{2}{l}{#1}\\
	\multicolumn{2}{l}{\hspace{10pt}\textit{#2}}\\
}

\newenvironment{mondai}[1]{
	\vspace{10pt}
	#1
	
	\begin{enumerate}
		\itemsep-5pt
	}{
	\end{enumerate}
	\vspace{10pt}
}

\newenvironment{hyou}{
	\begin{itemize}
		\itemsep-5pt
	}{
	\end{itemize}
	\vspace{10pt}
}

\date{\today}

\CJKfamily{chanto}\CJKnospace
\author{Tomislav Mamić}
\begin{document}
	\dai{Priložne oznake mjesta}
	
	U hrvatskom jeziku lokaciju izričemo koristeći posebnu vrstu riječi - prijedloge - u kombinaciji s lokativom. U japanskom prijedlozi ne postoje, ali čestice に i で obavljaju funkciju lokativa kako smo naučili ranije. Ono što nismo naučili je kako odrediti prostorne odnose među imenicama.
	
	\fukudai{Imenice umjesto prijedloga}
	
	U jezicima u kojima nema prijedloga, njihovu funkciju obavlja posebna kategorija imenica (\textit{odnosne imenice} od eng. \textit{relational noun}). Budući je japanski takav jezik, položaj u prostoru izricat ćemo na pomalo drugačiji način u odnosu na ono na što smo navikli. Pogledajmo za početak neke osnovne odnosne imenice:
	
	\vspace{10pt}
	\begin{tabular}{l l l l l l}
		ここ&\textit{ovdje}&そこ&\textit{tamo}&あそこ&\textit{ondje}\\
		うえ&\textit{gore}&した&\textit{dolje}&&\\
		みぎ&\textit{desno}&ひだり&\textit{lijevo}&&\\
		まえ&\textit{ispred}&うしろ&\textit{iza}&&\\
		なか&\textit{unutar}&そと&\textit{izvan}&&\\
	\end{tabular}

	\vspace{10pt}
	S obzirom da se radi o imenicama, način na koji se u japanski prevodi recimo \textit{u kutiji} je u početku prilično neintuitivan - \textit{u 'unutar' od kutije}. Pogledajmo neke primjere:
	
	\begin{reibun}
		\rei{木の\furigana{上}{うえ}に}{nad drvetom \dosl iznad od drveta}
		\rei{いえの ひだりに}{desno od kuće \rem{- ovo je vrlo doslovno i u hrvatskom!}}
		\rei{たかぎさんの まえで}{pred Takagijem}
		\rei{はこの\furigana{中}{なか}に}{u kutiji}
	\end{reibun}

	\fukudai{Čestica から kao početna točka}
	
	Iako ova čestica ima razne upotrebe, po ideji su sve slične - から označava početnu točku ili početak neke radnje. Prijedlog kojim ćemo prevesti ovu česticu, slično kao i kod に i で, ovisi o glagolu - iskoristit ćemo ono što najtočnije prenosi značenje iz japanskog. Pogledajmo neke primjere:
	
	\begin{reibun}
		\rei{はこの中から}{iz kutije}
		\rei{やまの上から}{\dosl od iznad planine \rem{- ovisi o glagolu, može biti npr.}s planine}
		\rei{いえを\furigana{出}{で}た。}{Izišao sam iz kuće.}
		\rei{いえから出た。}{Izišao sam iz kuće.}
	\end{reibun}

	\fukudai{Čestica まで kao krajnja točka}
	
	Analogno čestici から, まで označava završnu točku ili kraj neke radnje:
	
	\begin{reibun}
		\rei{あの木まで}{do onog drveta}
		\rei{たけしくんの いえの うしろまで はしった。}{Otrčao sam iza Takešijeve kuće.}
		\rei{いえから がっこうまで あるく。}{Hodat ću od kuće do škole.}
	\end{reibun}

	\fukudai{Vježba}
	
	Prevedite sljedeće rečenice na hrvatski:

\end{document}