% !TeX program = xelatex ?me -synctex=0 -interaction=nonstopmode -aux-directory=../tex_aux -output-directory=./release
% !TeX program = xelatex

\documentclass[12pt]{article}

\usepackage{lineno,changepage,lipsum}
\usepackage[colorlinks=true,urlcolor=blue]{hyperref}
\usepackage{fontspec}
\usepackage{xeCJK}
\usepackage{tabularx}
\setCJKfamilyfont{chanto}{AozoraMinchoRegular.ttf}
\setCJKfamilyfont{tegaki}{Mushin.otf}
\usepackage[CJK,overlap]{ruby}
\usepackage{hhline}
\usepackage{multirow,array,amssymb}
\usepackage[croatian]{babel}
\usepackage{soul}
\usepackage[usenames, dvipsnames]{color}
\usepackage{wrapfig,booktabs}
\renewcommand{\rubysep}{0.1ex}
\renewcommand{\rubysize}{0.75}
\usepackage[margin=50pt]{geometry}
\modulolinenumbers[2]

\usepackage{pifont}
\newcommand{\cmark}{\ding{51}}%
\newcommand{\xmark}{\ding{55}}%

\definecolor{faded}{RGB}{100, 100, 100}

\renewcommand{\arraystretch}{1.2}

%\ruby{}{}
%$($\href{URL}{text}$)$

\newcommand{\furigana}[2]{\ruby{#1}{#2}}
\newcommand{\tegaki}[1]{
	\CJKfamily{tegaki}\CJKnospace
	#1
	\CJKfamily{chanto}\CJKnospace
}

\newcommand{\dai}[1]{
	\vspace{20pt}
	\large
	\noindent\textbf{#1}
	\normalsize
	\vspace{20pt}
}

\newcommand{\fukudai}[1]{
	\vspace{10pt}
	\noindent\textbf{#1}
	\vspace{10pt}
}

\newenvironment{bunshou}{
	\vspace{10pt}
	\begin{adjustwidth}{1cm}{3cm}
	\begin{linenumbers}
}{
	\end{linenumbers}
	\end{adjustwidth}
}

\newenvironment{reibun}{
	\vspace{10pt}
	\begin{tabular}{l l}
}{
	\end{tabular}
	\vspace{10pt}
}
\newcommand{\rei}[2]{
	#1&\textit{#2}\\
}
\newcommand{\reinagai}[2]{
	\multicolumn{2}{l}{#1}\\
	\multicolumn{2}{l}{\hspace{10pt}\textit{#2}}\\
}

\newenvironment{mondai}[1]{
	\vspace{10pt}
	#1
	
	\begin{enumerate}
		\itemsep-5pt
	}{
	\end{enumerate}
	\vspace{10pt}
}

\newenvironment{hyou}{
	\begin{itemize}
		\itemsep-5pt
	}{
	\end{itemize}
	\vspace{10pt}
}

\date{\today}

\CJKfamily{chanto}\CJKnospace
\author{Katja Kržišnik}
\begin{document}
	\kaisetsu
	\dai{Domaća zadaća - ごだん glagoli}
		
	Dopunite tablicu ispod ispravnim oblicima glagola. One glagole kojima ne znate značenje potražite u rječniku i dopišite sa strane.
	
	\vspace{5pt}
	\begin{tabular}{|l|l|l|l|}
		\hline
		rječnički oblik & poz. pr. & neg. nepr. & neg. pr.\\
		\hline
		\multicolumn{4}{|c|}{\textasciitilde ごだん\textasciitilde}\\
		\hline
		さす&さした&ささない&ささなかった\\
		わかる&わかった&わからない&わからなかった\\
		しぬ&しんだ&しなない&しななかった\\
		かう&かった&かわない&かわなかった\\
		とぶ&とんだ&とばない&とばなかった\\
		たつ&たった&たたない&たたなかった\\
		かく&かいた&かかない&かかなかった\\
		よむ&よんだ&よまない&よまなかった\\
		およぐ&およいだ&およがない&およがなかった\\
		\hline
		\multicolumn{4}{|c|}{\textasciitilde いちだん\textasciitilde}\\
		\hline
		みる&みた&みない&みなかった\\
		きる&きった&きらない&きらなかった\\
		かえる&かえた&かえない&かえなかった\\
		かんがえる&かんがえた&かんがえない&かんがえなかった\\
		\hline
		\multicolumn{4}{|c|}{\textasciitilde nepravilni\textasciitilde}\\
		\hline
		くる&きた&こない&こなかった\\
		いる&いた&ない&なかった\\
		いく&いった&いかない&いかなかった\\
		する&した&しない&しなかった\\
		\hline
	\end{tabular}
	
	\newpage
	\fukudai{Prevedite sljedeće rečenice.}
	
	\begin{kaitou}[Lv. 1]
		\kai{ともだちと あそんだ。}{Igrala sam se s prijateljem.}
		\kai{こうえんに いる。}{U parku sam.}
		\kai{べんきょうしなかった。}{Nisam učila.}
		\kai{せんせいに きいた。}{Čula sam od profesora.}
	\end{kaitou}
	
	\begin{kaitou}[Lv. 2]
		\kai{こうえんで ともだちと あそんだ。}{U parku sam se igrala s prijateljem.}
		\kai{ねこは こうえんに いる。}{Mačka je u parku.}
		\kai{さつきちゃんは べんきょうしなかった。}{Satsuki nije učila.}
		\kai{それを せんせいに きいた。}{To sam čula od profesora.}
	\end{kaitou}
		
	\begin{kaitou}[Lv. 3]
		\kainagai{さつきちゃんは こうえんで ともだちと てまりで あそんだ。}{Satsuki se u parku s prijateljem igrala sa temari loptom.}
		\kainagai{あの くろい ねこは ちかくの こうえんに いる。}{Ona crna mačka je u obližnjem parku.}
		\kainagai{さつきちゃんも たけしくんも べんきょうしなかった。}{Ni Satsuki ni Takeši nisu učili.}
		\kainagai{たけしくんは てんこうすると せんせいに きいた。}{Takeši je čuo od profesora da će promijeniti školu.}
	\end{kaitou}
	
\end{document}