% !TeX program = xelatex ?me -synctex=0 -interaction=nonstopmode -aux-directory=../tex_aux -output-directory=./release
% !TeX program = xelatex

\documentclass[12pt]{article}

\usepackage{lineno,changepage,lipsum}
\usepackage[colorlinks=true,urlcolor=blue]{hyperref}
\usepackage{fontspec}
\usepackage{xeCJK}
\usepackage{tabularx}
\setCJKfamilyfont{chanto}{AozoraMinchoRegular.ttf}
\setCJKfamilyfont{tegaki}{Mushin.otf}
\usepackage[CJK,overlap]{ruby}
\usepackage{hhline}
\usepackage{multirow,array,amssymb}
\usepackage[croatian]{babel}
\usepackage{soul}
\usepackage[usenames, dvipsnames]{color}
\usepackage{wrapfig,booktabs}
\renewcommand{\rubysep}{0.1ex}
\renewcommand{\rubysize}{0.75}
\usepackage[margin=50pt]{geometry}
\modulolinenumbers[2]

\usepackage{pifont}
\newcommand{\cmark}{\ding{51}}%
\newcommand{\xmark}{\ding{55}}%

\definecolor{faded}{RGB}{100, 100, 100}

\renewcommand{\arraystretch}{1.2}

%\ruby{}{}
%$($\href{URL}{text}$)$

\newcommand{\furigana}[2]{\ruby{#1}{#2}}
\newcommand{\tegaki}[1]{
	\CJKfamily{tegaki}\CJKnospace
	#1
	\CJKfamily{chanto}\CJKnospace
}

\newcommand{\dai}[1]{
	\vspace{20pt}
	\large
	\noindent\textbf{#1}
	\normalsize
	\vspace{20pt}
}

\newcommand{\fukudai}[1]{
	\vspace{10pt}
	\noindent\textbf{#1}
	\vspace{10pt}
}

\newenvironment{bunshou}{
	\vspace{10pt}
	\begin{adjustwidth}{1cm}{3cm}
	\begin{linenumbers}
}{
	\end{linenumbers}
	\end{adjustwidth}
}

\newenvironment{reibun}{
	\vspace{10pt}
	\begin{tabular}{l l}
}{
	\end{tabular}
	\vspace{10pt}
}
\newcommand{\rei}[2]{
	#1&\textit{#2}\\
}
\newcommand{\reinagai}[2]{
	\multicolumn{2}{l}{#1}\\
	\multicolumn{2}{l}{\hspace{10pt}\textit{#2}}\\
}

\newenvironment{mondai}[1]{
	\vspace{10pt}
	#1
	
	\begin{enumerate}
		\itemsep-5pt
	}{
	\end{enumerate}
	\vspace{10pt}
}

\newenvironment{hyou}{
	\begin{itemize}
		\itemsep-5pt
	}{
	\end{itemize}
	\vspace{10pt}
}

\date{\today}

\CJKfamily{chanto}\CJKnospace
\author{Katja Kržišnik}
\begin{document}
	\kaisetsu
	
	\dai{Glagoli III}
	
	\fukudai{Vježba}
	
	\begin{kaitou}[Lv. 1]
		\kai{かわない。}{Neću kupiti.}
		\kai{かいた。}{Napisala sam.}
		\kai{だす。}{Izvadit ću. (Nije jedino značenje, ovaj glagol ima ih puno)}
		\kai{またない。}{Neću čekati.}
		\kai{しんだ。}{Umrla je.}
	\end{kaitou}
	
	\begin{kaitou}[Lv. 2]
		\kai{あたらしい くつを かわない。}{Neću kupiti nove cipele.}
		\kai{てがみを かいた。}{Napisao sam pismo.}
		\kai{ごみを だす。}{Bacit ću smeće.}
		\kai{ともだちを またない。}{Neću čekati prijatelja.}
		\kai{とりは しんだ。}{Ptica je umrla.}
	\end{kaitou}
	
	\begin{kaitou}[Lv. 3]
		\kainagai{あたらしい くつを そのみせで かわない。}{Neću kupiti nove cipele u tom dućanu.}
		\kainagai{おとうさんに ながい てがみを かいた。}{Napisala sam dugo pismo ocu.}
		\kainagai{やまださんは ごみを ださなかった。}{Gospodin Jamada nije bacio smeće.}
		\kainagai{たけしくんは ともだちを またなかった。}{Takeši nije čekao prijatelja.}
		\kainagai{すずきさんの きれいな とりは しんだ。}{Lijepa ptica gospodina Suzukija je umrla.}
	\end{kaitou}

\newpage
	
	\begin{kaitou}[Lv. 4]
		\kaimecchanagai{あたらしい くつを そのみせで かわないと むらかみさんは いった。}{Gospodin Murakami je rekao da u tom dućanu neće kupiti nove cipele.}{}
		\kaimecchanagai{*おおさかに いる おとうさんに ながい てがみを かいた。}{Napisala sam dugo pismo ocu koji je u Osaki.}{}
		\kaimecchanagai{*ごみを ださなかった やまださんは いえを でた。}{Gospodin Jamada koji nije bacio smeće je izašao iz kuće.}{}
		\kaimecchanagai{すずきさんの きれいな とりは とりかごで しんだ。}{Lijepa ptica gospodina Suzukija umrla je u kavezu.}{}
		\kaimecchanagai{たけしくんは おとうとと ともだちを まった。}{ili たけしくんは ともだちを おとうとと まった。}{Takeši je s mlađim bratom čekao prijatelja.}
	\end{kaitou}

                 \textbf{5 - objašnjenje:} Razlika je u tome što se prva rečenica može shvatiti na dva načina, a druga ne. Prva rečenica još se može shavtiti i kao: Takeši je čekao prijatelja i mlađeg brata.
	
\end{document}