% !TeX program = xelatex ?me -synctex=0 -interaction=nonstopmode -aux-directory=../tex_aux -output-directory=./release
% !TeX program = xelatex

\documentclass[12pt]{article}

\usepackage{lineno,changepage,lipsum}
\usepackage[colorlinks=true,urlcolor=blue]{hyperref}
\usepackage{fontspec}
\usepackage{xeCJK}
\usepackage{tabularx}
\setCJKfamilyfont{chanto}{AozoraMinchoRegular.ttf}
\setCJKfamilyfont{tegaki}{Mushin.otf}
\usepackage[CJK,overlap]{ruby}
\usepackage{hhline}
\usepackage{multirow,array,amssymb}
\usepackage[croatian]{babel}
\usepackage{soul}
\usepackage[usenames, dvipsnames]{color}
\usepackage{wrapfig,booktabs}
\renewcommand{\rubysep}{0.1ex}
\renewcommand{\rubysize}{0.75}
\usepackage[margin=50pt]{geometry}
\modulolinenumbers[2]

\usepackage{pifont}
\newcommand{\cmark}{\ding{51}}%
\newcommand{\xmark}{\ding{55}}%

\definecolor{faded}{RGB}{100, 100, 100}

\renewcommand{\arraystretch}{1.2}

%\ruby{}{}
%$($\href{URL}{text}$)$

\newcommand{\furigana}[2]{\ruby{#1}{#2}}
\newcommand{\tegaki}[1]{
	\CJKfamily{tegaki}\CJKnospace
	#1
	\CJKfamily{chanto}\CJKnospace
}

\newcommand{\dai}[1]{
	\vspace{20pt}
	\large
	\noindent\textbf{#1}
	\normalsize
	\vspace{20pt}
}

\newcommand{\fukudai}[1]{
	\vspace{10pt}
	\noindent\textbf{#1}
	\vspace{10pt}
}

\newenvironment{bunshou}{
	\vspace{10pt}
	\begin{adjustwidth}{1cm}{3cm}
	\begin{linenumbers}
}{
	\end{linenumbers}
	\end{adjustwidth}
}

\newenvironment{reibun}{
	\vspace{10pt}
	\begin{tabular}{l l}
}{
	\end{tabular}
	\vspace{10pt}
}
\newcommand{\rei}[2]{
	#1&\textit{#2}\\
}
\newcommand{\reinagai}[2]{
	\multicolumn{2}{l}{#1}\\
	\multicolumn{2}{l}{\hspace{10pt}\textit{#2}}\\
}

\newenvironment{mondai}[1]{
	\vspace{10pt}
	#1
	
	\begin{enumerate}
		\itemsep-5pt
	}{
	\end{enumerate}
	\vspace{10pt}
}

\newenvironment{hyou}{
	\begin{itemize}
		\itemsep-5pt
	}{
	\end{itemize}
	\vspace{10pt}
}

\date{\today}

\CJKfamily{chanto}\CJKnospace
\author{Katja Kržišnik}
\begin{document}
	\dai{Domaća zadaća - prilozi}
	\kaisetsu
	
	\begin{kaitou}[Probajte nekoliko puta napisati sljedeće znakove. Koja su njihova značenja?]
		\kai{小}{malen}
		\kai{大}{velik}
		\kai{白}{bijel}
		\kai{子}{dijete}
		\kai{女}{žena}
		\kai{男}{muškarac}
		\kai{父}{otac}
		\kai{母}{majka}
	\end{kaitou}

    \begin{kaitou}[Sljedeće rečenice pročitajte na glas nekoliko puta i prevedite na hrvatski.]
    	\kainagai{さつきちゃんも たけしくんも こんなに 大きく なった。}{I Satsuki i Takeši su narasli ovako veliki.}
    	\kainagai{すずきさんは こえを 小さく した。}{Gospodin Suzuki je snizio glas.}
    	\kainagai{山田さんの かおは 白く なった。}{Gospodin Jamada je problijedio.}
    	\kainagai{「月がきれい」と、男の子が はっきり いう。}{“Mjesec je lijep”, reče mladić jasno.}
    	\kainagai{まつださんの お父さんは「女の人は すうがくが ぜんぜんわからない」という。}{Otac gospodina Matsude kaže: “Žene uopće ne razumiju matematiku”.}
    	\kainagai{母の りょうりは すごく おいしい。}{Majka kuha jako ukusno.}
    \end{kaitou} 

\footnotetext[1]{月がきれい je poetičan način za reći volim te.}
   
\end{document}