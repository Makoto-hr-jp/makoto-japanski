% !TeX program = xelatex ?me -synctex=0 -interaction=nonstopmode -aux-directory=../tex_aux -output-directory=./release
% !TeX program = xelatex

\documentclass[12pt]{article}

\usepackage{lineno,changepage,lipsum}
\usepackage[colorlinks=true,urlcolor=blue]{hyperref}
\usepackage{fontspec}
\usepackage{xeCJK}
\usepackage{tabularx}
\setCJKfamilyfont{chanto}{AozoraMinchoRegular.ttf}
\setCJKfamilyfont{tegaki}{Mushin.otf}
\usepackage[CJK,overlap]{ruby}
\usepackage{hhline}
\usepackage{multirow,array,amssymb}
\usepackage[croatian]{babel}
\usepackage{soul}
\usepackage[usenames, dvipsnames]{color}
\usepackage{wrapfig,booktabs}
\renewcommand{\rubysep}{0.1ex}
\renewcommand{\rubysize}{0.75}
\usepackage[margin=50pt]{geometry}
\modulolinenumbers[2]

\usepackage{pifont}
\newcommand{\cmark}{\ding{51}}%
\newcommand{\xmark}{\ding{55}}%

\definecolor{faded}{RGB}{100, 100, 100}

\renewcommand{\arraystretch}{1.2}

%\ruby{}{}
%$($\href{URL}{text}$)$

\newcommand{\furigana}[2]{\ruby{#1}{#2}}
\newcommand{\tegaki}[1]{
	\CJKfamily{tegaki}\CJKnospace
	#1
	\CJKfamily{chanto}\CJKnospace
}

\newcommand{\dai}[1]{
	\vspace{20pt}
	\large
	\noindent\textbf{#1}
	\normalsize
	\vspace{20pt}
}

\newcommand{\fukudai}[1]{
	\vspace{10pt}
	\noindent\textbf{#1}
	\vspace{10pt}
}

\newenvironment{bunshou}{
	\vspace{10pt}
	\begin{adjustwidth}{1cm}{3cm}
	\begin{linenumbers}
}{
	\end{linenumbers}
	\end{adjustwidth}
}

\newenvironment{reibun}{
	\vspace{10pt}
	\begin{tabular}{l l}
}{
	\end{tabular}
	\vspace{10pt}
}
\newcommand{\rei}[2]{
	#1&\textit{#2}\\
}
\newcommand{\reinagai}[2]{
	\multicolumn{2}{l}{#1}\\
	\multicolumn{2}{l}{\hspace{10pt}\textit{#2}}\\
}

\newenvironment{mondai}[1]{
	\vspace{10pt}
	#1
	
	\begin{enumerate}
		\itemsep-5pt
	}{
	\end{enumerate}
	\vspace{10pt}
}

\newenvironment{hyou}{
	\begin{itemize}
		\itemsep-5pt
	}{
	\end{itemize}
	\vspace{10pt}
}

\date{\today}

\CJKfamily{chanto}\CJKnospace
\author{Katja Kržišsnik}
\begin{document}
	\kaisetsu
	\dai{Domaća zadaća - いちだん glagoli}
	
	\begin{kaitou}[Pronađite značenja sljedećih glagola:]
		\kai{かんがえる}{misliti}
		\kai{いれる}{staviti u/napraviti(kavu,čaj,...)}
		\kai{きがえる}{presvući se}
		\kai{ねる}{spavati/leći}
		\kai{かりる}{posuditi od}
	\end{kaitou}
	
	\begin{kaitou}[Prevedite sljedeće rečenice na japanski:]
		\kai{Vidjelo se more.}{うみが みえた。}
		\kai{Vidjeli su more.}{かれらは うみを みた。}
		\kai{Mačka je zaspala.}{ねこは ねた。}
		\kai{Takeši se presvukao.}{たけしは ふくを きがえた。}
		\kai{Gospodin Suzuki neće nositi kimono.}{すずきさんは きものを きない。}
		\kai{Nisam o tome razmišljao.}{かんがえてなかった。}
		\kai{Mala Akiko je posudila knjigu.}{あきこちゃんは ほんを かりた。}
	\end{kaitou}
	
	\begin{kaitou}[Prevedite sljedeće rečenice na hrvatski:]
		\kainagai{おおきい くまは あなを でた。}{Veliki medvjed je izašao iz rupe.}
		\kainagai{ちいさい ねずみは きの はこを あけた。}{Mali miš je otvorio drvenu kutiju.}
		\kainagai{ともだちの いぬを みた。}{Vidjela sam prijateljevog psa.}
		\kainagai{わたしの ねこが ともだちの いぬの えさを たべた。}{Moja mačka je pojela hranu prijateljevog psa.}
	\end{kaitou}
	
\end{document}