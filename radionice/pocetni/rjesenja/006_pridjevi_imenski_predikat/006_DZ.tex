% !TeX program = xelatex ?me -synctex=0 -interaction=nonstopmode -aux-directory=../tex_aux -output-directory=./release
% !TeX program = xelatex

\documentclass[12pt]{article}

\usepackage{lineno,changepage,lipsum}
\usepackage[colorlinks=true,urlcolor=blue]{hyperref}
\usepackage{fontspec}
\usepackage{xeCJK}
\usepackage{tabularx}
\setCJKfamilyfont{chanto}{AozoraMinchoRegular.ttf}
\setCJKfamilyfont{tegaki}{Mushin.otf}
\usepackage[CJK,overlap]{ruby}
\usepackage{hhline}
\usepackage{multirow,array,amssymb}
\usepackage[croatian]{babel}
\usepackage{soul}
\usepackage[usenames, dvipsnames]{color}
\usepackage{wrapfig,booktabs}
\renewcommand{\rubysep}{0.1ex}
\renewcommand{\rubysize}{0.75}
\usepackage[margin=50pt]{geometry}
\modulolinenumbers[2]

\usepackage{pifont}
\newcommand{\cmark}{\ding{51}}%
\newcommand{\xmark}{\ding{55}}%

\definecolor{faded}{RGB}{100, 100, 100}

\renewcommand{\arraystretch}{1.2}

%\ruby{}{}
%$($\href{URL}{text}$)$

\newcommand{\furigana}[2]{\ruby{#1}{#2}}
\newcommand{\tegaki}[1]{
	\CJKfamily{tegaki}\CJKnospace
	#1
	\CJKfamily{chanto}\CJKnospace
}

\newcommand{\dai}[1]{
	\vspace{20pt}
	\large
	\noindent\textbf{#1}
	\normalsize
	\vspace{20pt}
}

\newcommand{\fukudai}[1]{
	\vspace{10pt}
	\noindent\textbf{#1}
	\vspace{10pt}
}

\newenvironment{bunshou}{
	\vspace{10pt}
	\begin{adjustwidth}{1cm}{3cm}
	\begin{linenumbers}
}{
	\end{linenumbers}
	\end{adjustwidth}
}

\newenvironment{reibun}{
	\vspace{10pt}
	\begin{tabular}{l l}
}{
	\end{tabular}
	\vspace{10pt}
}
\newcommand{\rei}[2]{
	#1&\textit{#2}\\
}
\newcommand{\reinagai}[2]{
	\multicolumn{2}{l}{#1}\\
	\multicolumn{2}{l}{\hspace{10pt}\textit{#2}}\\
}

\newenvironment{mondai}[1]{
	\vspace{10pt}
	#1
	
	\begin{enumerate}
		\itemsep-5pt
	}{
	\end{enumerate}
	\vspace{10pt}
}

\newenvironment{hyou}{
	\begin{itemize}
		\itemsep-5pt
	}{
	\end{itemize}
	\vspace{10pt}
}

\date{\today}

\CJKfamily{chanto}\CJKnospace
\author{Katja Kržišnik}
\begin{document}
	\kaisetsu
	\dai{Domaća zadaća - predikatni oblik pridjeva}
	
	\begin{mondai}{1. Prevedite sljedeće rečenice na hrvatski, a u japanskom im obrnite pristojnost (pristojne prepišite kolokvijalno i obratno).}
		\item あの ねこは くろい。 \newline あの ねこは くろいです。Ona mačka je crna. 
		\item たなかさんの にわは ひろかったです。 \newline たなかさんの にわは ひろかった。 Tanakin vrt je bio prostran. 
		\item ことしの あきは さむくない。 \newline ことしの あきは さむくありません。 Jesen ove godine nije hladna.
		\item きょねんの なつは あつくなかった。 \newline きょねんの なつは あつくありませんでした。 Ljeto prošle godine nije bilo vruće. 
		\item たけしくんの いもうとさんも おとうとさんも びょうき でした。 \newline たけしくんの いもうとさんも おとうとさんも びょうきだった。 Takešijevi i mlađi brat i mlađa sestra su bolesni.
		\item くまや いのししは きけん だ。 \newline くまや いのししは きけん です。 Medvjedi, divlje svinje i slične životinje su opasni.
		\item よるの ろっぽんぎは しずか ではありません。 \newline よるの ろっぽんぎは しずかじゃない。 Ropongi noću nije tih. 
		\item ひきがやくんの ともだちは へん じゃなかった。 \newline ひきがやくんの ともだちは へん ではありませんでした。 Hikigajin prijatelj nije bio čudan.
	\end{mondai}
	
	\begin{kaitou}[Lv. 1]
		\kai{テスト}{test}
		\kai{テスト}{test}
		\kainagai{めっちゃ長いテストのはずだったけど、何を書けばいいかわかんなくなってきた。}{test}
		\kai{テスト}{test}
		\kainagai{めっちゃ長いテストのはずだったけど、何を書けばいいかわかんなくなってきた。}{test}
		\kainagai{めっちゃ長いテストのはずだったけど、何を書けばいいかわかんなくなってきた。}{test}
		\kai{テスト}{test}
	\end{kaitou}

	\begin{kaitou}[Lv. 2]
		\kai{テスト}{test3}
		\kai{テスト}{test4}
	\end{kaitou}
	
	\begin{mondai}{2. Prevedite sljedeće rečenice na japanski u pristojnom i kolokvijalnom obliku.}
		\item Bio sam bolestan. \newline びょうきでした。 びょうきだった。
		\item Medvjedi su veliki. \newline くまはおおきいです。 くまはおおきい。
		\item WC je daleko. \newline トイレはとおいです。 トイレはとおい。
		\item Mačka Takešijeve prijateljice nije bila bijela. \newline たけしくんのともだちのねこはしろくではありませんでした。 たけしくんのともだちのねこはしろくなかった。
	\end{mondai}
\end{document}