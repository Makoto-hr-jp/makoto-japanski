% !TeX program = xelatex ?me -synctex=0 -interaction=nonstopmode -aux-directory=../tex_aux -output-directory=./release
% !TeX program = xelatex

\documentclass[12pt]{article}

\usepackage{lineno,changepage,lipsum}
\usepackage[colorlinks=true,urlcolor=blue]{hyperref}
\usepackage{fontspec}[ Path =../../../ ]
\usepackage{xeCJK}
\usepackage{tabularx}
\usepackage{graphicx}
\setCJKfamilyfont{chanto}{AOZORAMINCHOREGULAR_0.TTF}%
\setCJKfamilyfont{tegaki}{Mushin.otf}%
\usepackage[CJK,overlap]{ruby}
\usepackage{hhline}
\usepackage{multirow,array,amssymb}
\usepackage[croatian]{babel}
\usepackage{soul}
\usepackage[usenames, dvipsnames]{color}
\usepackage{wrapfig,booktabs}
\usepackage{calc}
\renewcommand{\rubysep}{0.1ex}
\renewcommand{\rubysize}{0.75}
\usepackage[margin=50pt]{geometry}
\usepackage{hyperref}
\modulolinenumbers[2]

\date{\today}

\usepackage{fancyhdr}
\pagestyle{fancy}
\fancyhf{}
\fancyhead[LE,RO]{\thepage}
\makeatletter
\fancyhead[RE,LO]{rev. \@date 誠}
\makeatother

\usepackage{pifont}
\newcommand{\cmark}{\ding{51}}%
\newcommand{\xmark}{\ding{55}}%

\newcommand{\dosl}{{\normalfont dosl. }}%
\newcommand{\rem}[1]{{\normalfont #1 }}%

\definecolor{faded}{RGB}{100, 100, 100}

\renewcommand{\arraystretch}{1.2}

%\ruby{}{}
%$($\href{URL}{text}$)$

\newcommand{\furigana}[2]{\ruby{#1}{#2}}
\newcommand{\tegaki}[1]{
	\CJKfamily{tegaki}\CJKnospace
	#1
	\CJKfamily{chanto}\CJKnospace
}

\newcommand{\dai}[1]{
	\vspace{20pt}
	\large
	\noindent\textbf{#1}
	\normalsize
	\vspace{20pt}
}

\newcommand{\fukudai}[1]{
	\vspace{10pt}
	\noindent\textbf{#1}
	\vspace{10pt}
}

\newenvironment{bunshou}{
	\vspace{10pt}
	\begin{adjustwidth}{1cm}{3cm}
	\begin{linenumbers}
}{
	\end{linenumbers}
	\end{adjustwidth}
}

\newenvironment{reibun}[1][]{
	\vspace{10pt}
	#1
	
	\begin{tabular}{l l}
}{
	\end{tabular}
	\vspace{10pt}
}
\newcommand{\rei}[2]{
	#1&\textit{#2}\\
}
\newcommand{\reinagai}[2]{
	\multicolumn{2}{l}{#1}\\
	\multicolumn{2}{l}{\hspace{10pt}\textit{#2}}\\
}

\newenvironment{mondai}[1]{
	\vspace{10pt}
	\noindent #1
	
	\begin{enumerate}
		\itemsep-5pt
	}{
	\end{enumerate}
}

\newenvironment{hyou}{
	\begin{itemize}
		\itemsep-5pt
	}{
	\end{itemize}
	\vspace{10pt}
}

\newcommand{\juuyou}[2][20pt]{
	\vspace{5pt}
		\noindent\hspace{#1}\parbox[c]{\textwidth-#1-#1}{\centering\textit{#2}}
	\vspace{5pt}
}

\newcommand{\ten}{
	\vspace{5pt}
	\noindent\hspace{-10pt}$\bullet$
}

\CJKfamily{chanto}\CJKnospace

\frenchspacing
\author{Katja Kržišnik}
\begin{document}
	\kaisetsu
	
	\dai{Upotrebe て oblika I - Vježba}
	
	     \begin{kaitou}[Lv. 1]
	     	\kai{のんでみた。}{Pokušala sam popiti.}
	     	\kai{そうじ しておいた。}{Očistio sam. (da ne moram poslije)}
	     	\kai{よんであげた。}{Pročitao sam. (nešto za nekoga)}
	     	\kai{おしえてくれた。}{Rekao mi je. (dobronamjerno)}
	     	\kai{かってもらった。}{Tražio sam ga da pobijedi. (za mene)}
	     \end{kaitou}
     
         \begin{kaitou}[Lv. 2]
         	\kai{このおちゃを のんでみましたか。}{Jeste li probali ovaj čaj?}
         	\kai{いえを そうじ しておくと きめた。}{Odlučio sam počistiti kuću. }
         	\kai{先生は 子供たちに えほんを よんであげた。}{Učitelj je djeci pročitao slikovnicu.}
         	\kai{ともだちが わたしに ぶつりがくを おしえてくれた。}{Prijatelj mi je pokazao fiziku.}
         	\kai{かのじょに ぎゅうにゅうを かってもらった。}{Zamolio sam ju da mi kupi mlijeko.}
         \end{kaitou}
     
         \begin{kaitou}[Lv. 3]
         	\kainagai{このおちゃを のんでみたいですか。}{Želite li probati ovaj čaj?}
         	\kainagai{いえを そうじ しておくと きめて せんざいを かいに いった。}{Nakon što sam odlučio počistiti kuću otišao sam kupiti deterdžent.}
         	\kainagai{先生は まいにち 子供たちに えほんを よんであげていました。}{Učitelj je svaki dan djeci čitao slikovnice.}
         	\kainagai{ぶつりがくを おしえてくれて ありがとうと ともだちに いいました。}{Zahvalila sam se prijatelju što mi je pokazao fiziku.}
         	\kainagai{*かのじょに かいわすれた ぎゅうにゅうを かいにいってもらった。}{Zamolio sam ju da ode kupiti mlijeko koje sam ja zaboravio kupiti.}
         \end{kaitou}
\end{document}