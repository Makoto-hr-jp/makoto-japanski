% !TeX program = xelatex ?me -synctex=0 -interaction=nonstopmode -aux-directory=../tex_aux -output-directory=./release
% !TeX program = xelatex

\documentclass[12pt]{article}

\usepackage{lineno,changepage,lipsum}
\usepackage[colorlinks=true,urlcolor=blue]{hyperref}
\usepackage{fontspec}
\usepackage{xeCJK}
\usepackage{tabularx}
\setCJKfamilyfont{chanto}{AozoraMinchoRegular.ttf}
\setCJKfamilyfont{tegaki}{Mushin.otf}
\usepackage[CJK,overlap]{ruby}
\usepackage{hhline}
\usepackage{multirow,array,amssymb}
\usepackage[croatian]{babel}
\usepackage{soul}
\usepackage[usenames, dvipsnames]{color}
\usepackage{wrapfig,booktabs}
\renewcommand{\rubysep}{0.1ex}
\renewcommand{\rubysize}{0.75}
\usepackage[margin=50pt]{geometry}
\modulolinenumbers[2]

\usepackage{pifont}
\newcommand{\cmark}{\ding{51}}%
\newcommand{\xmark}{\ding{55}}%

\definecolor{faded}{RGB}{100, 100, 100}

\renewcommand{\arraystretch}{1.2}

%\ruby{}{}
%$($\href{URL}{text}$)$

\newcommand{\furigana}[2]{\ruby{#1}{#2}}
\newcommand{\tegaki}[1]{
	\CJKfamily{tegaki}\CJKnospace
	#1
	\CJKfamily{chanto}\CJKnospace
}

\newcommand{\dai}[1]{
	\vspace{20pt}
	\large
	\noindent\textbf{#1}
	\normalsize
	\vspace{20pt}
}

\newcommand{\fukudai}[1]{
	\vspace{10pt}
	\noindent\textbf{#1}
	\vspace{10pt}
}

\newenvironment{bunshou}{
	\vspace{10pt}
	\begin{adjustwidth}{1cm}{3cm}
	\begin{linenumbers}
}{
	\end{linenumbers}
	\end{adjustwidth}
}

\newenvironment{reibun}{
	\vspace{10pt}
	\begin{tabular}{l l}
}{
	\end{tabular}
	\vspace{10pt}
}
\newcommand{\rei}[2]{
	#1&\textit{#2}\\
}
\newcommand{\reinagai}[2]{
	\multicolumn{2}{l}{#1}\\
	\multicolumn{2}{l}{\hspace{10pt}\textit{#2}}\\
}

\newenvironment{mondai}[1]{
	\vspace{10pt}
	#1
	
	\begin{enumerate}
		\itemsep-5pt
	}{
	\end{enumerate}
	\vspace{10pt}
}

\newenvironment{hyou}{
	\begin{itemize}
		\itemsep-5pt
	}{
	\end{itemize}
	\vspace{10pt}
}

\date{\today}

\CJKfamily{chanto}\CJKnospace
\author{Katja Kržišnik}
\begin{document}
	\kaisetsu
	
	\dai{Domaća zadaća - pitanja i završne čestice}
	
	    \begin{kaitou}
	    	\kai{サルマが すき ですか。}{はい、とても すき です。}
	    	\kai{すしが すき ですか。}{はい、すき です。}
	    	\kai{\furigana{学生}{がく.せい} ですか。}{いいえ、\furigana{学生}{がく.せい} ではないです。}
	    	\kai{\furigana{大学生}{だい.がく.せい} ですか。}{はい、\furigana{大学生}{だい.がく.せい} です。}
	    	\kai{さいきん、どうぶつえんに いったか。}{いいえ、いってない けど いきたい。}
	    	\kai{(あなたは)だれですか。}{わたしは たけし です。}
	    	\kai{まいにち だれと ひるごはんを たべるか。}{まいにち ねこと ひるごはんを たべる。}
	    	\kai{けさ、何を したか。}{けさは たくさん べんきょう した。}
	    	\kai{さいきん、何を かったか。}{おととい シャツを 二枚 かった。}
	    	\kai{サルマ、すし、ラーメン、どれを たべるか。}{どれ でも たべる。}
	    	\kai{どの くだものが いちばん すきか。}{バナナが いちばん すき。}
	    	\kai{どこで うまれたか。}{ザグレブ で。}
	    	\kai{こんどのにちようび、どこへ いくか。}{日本に いく。}
	    	\kai{あなたの へやは どんな へや ですか。}{しょくぶつが たくさん いる へや です。}
	    	\kai{どんな人が すきですか。}{おもしろいひとが すき です。}
	    	\kai{さかなを おはしで どう たべるか。}{ふつうに たべる。}
	    	\kai{(あなたは)ともだちが 何人 いるか。}{ともだちが 三人 いる。}
	    	\kai{猫が 何匹 いるか。}{猫が 二匹 いる。}
	    \end{kaitou}
	    	
\end{document}