% !TeX program = xelatex ?me -synctex=0 -interaction=nonstopmode -aux-directory=../tex_aux -output-directory=./release
% !TeX program = xelatex

\documentclass[12pt]{article}

\usepackage{lineno,changepage,lipsum}
\usepackage[colorlinks=true,urlcolor=blue]{hyperref}
\usepackage{fontspec}[ Path =../../../ ]
\usepackage{xeCJK}
\usepackage{tabularx}
\usepackage{graphicx}
\setCJKfamilyfont{chanto}{AOZORAMINCHOREGULAR_0.TTF}%
\setCJKfamilyfont{tegaki}{Mushin.otf}%
\usepackage[CJK,overlap]{ruby}
\usepackage{hhline}
\usepackage{multirow,array,amssymb}
\usepackage[croatian]{babel}
\usepackage{soul}
\usepackage[usenames, dvipsnames]{color}
\usepackage{wrapfig,booktabs}
\usepackage{calc}
\renewcommand{\rubysep}{0.1ex}
\renewcommand{\rubysize}{0.75}
\usepackage[margin=50pt]{geometry}
\usepackage{hyperref}
\modulolinenumbers[2]

\date{\today}

\usepackage{fancyhdr}
\pagestyle{fancy}
\fancyhf{}
\fancyhead[LE,RO]{\thepage}
\makeatletter
\fancyhead[RE,LO]{rev. \@date 誠}
\makeatother

\usepackage{pifont}
\newcommand{\cmark}{\ding{51}}%
\newcommand{\xmark}{\ding{55}}%

\newcommand{\dosl}{{\normalfont dosl. }}%
\newcommand{\rem}[1]{{\normalfont #1 }}%

\definecolor{faded}{RGB}{100, 100, 100}

\renewcommand{\arraystretch}{1.2}

%\ruby{}{}
%$($\href{URL}{text}$)$

\newcommand{\furigana}[2]{\ruby{#1}{#2}}
\newcommand{\tegaki}[1]{
	\CJKfamily{tegaki}\CJKnospace
	#1
	\CJKfamily{chanto}\CJKnospace
}

\newcommand{\dai}[1]{
	\vspace{20pt}
	\large
	\noindent\textbf{#1}
	\normalsize
	\vspace{20pt}
}

\newcommand{\fukudai}[1]{
	\vspace{10pt}
	\noindent\textbf{#1}
	\vspace{10pt}
}

\newenvironment{bunshou}{
	\vspace{10pt}
	\begin{adjustwidth}{1cm}{3cm}
	\begin{linenumbers}
}{
	\end{linenumbers}
	\end{adjustwidth}
}

\newenvironment{reibun}[1][]{
	\vspace{10pt}
	#1
	
	\begin{tabular}{l l}
}{
	\end{tabular}
	\vspace{10pt}
}
\newcommand{\rei}[2]{
	#1&\textit{#2}\\
}
\newcommand{\reinagai}[2]{
	\multicolumn{2}{l}{#1}\\
	\multicolumn{2}{l}{\hspace{10pt}\textit{#2}}\\
}

\newenvironment{mondai}[1]{
	\vspace{10pt}
	\noindent #1
	
	\begin{enumerate}
		\itemsep-5pt
	}{
	\end{enumerate}
}

\newenvironment{hyou}{
	\begin{itemize}
		\itemsep-5pt
	}{
	\end{itemize}
	\vspace{10pt}
}

\newcommand{\juuyou}[2][20pt]{
	\vspace{5pt}
		\noindent\hspace{#1}\parbox[c]{\textwidth-#1-#1}{\centering\textit{#2}}
	\vspace{5pt}
}

\newcommand{\ten}{
	\vspace{5pt}
	\noindent\hspace{-10pt}$\bullet$
}

\CJKfamily{chanto}\CJKnospace

\frenchspacing
\author{Katja Kržišnik}
\begin{document}
	\kaisetsu
	
	\dai{Domaća zadaća - い oblik}
	
	\noindent Sljedećim glagolima napišite prošlost i negaciju u pristojnoj i kolokvijalnoj varijanti.
	\vspace{15pt}
	
	     \begin{tabular}{l l l l l}\toprule[2pt]
	     	glagol &  prošlost & negacija & pristojna prošlost & pristojna negacija \\
	     	\midrule
	     	たべる & たべた & たべない & たべました & たべません \\
	     	見る & 見た & 見ない & 見ました & 見ません \\
	     	\vspace{5pt}
	     	いる & いた & いない & いました & いません \\
	     	ある & あった & ない & ありました & ありません \\
	     	いく & いった & いかない & いきました & いきません \\
	     	くる & きた & こない & きました & きません \\
	     	\vspace{5pt}
	     	する & した & しない & しました & しません \\
	     	かく & かいた & かかない & かきました & かきません \\
	     	きく & きいいた & きかない & ききました & ききません \\
	     	およぐ & およいだ & およがない & およぎました & およぎません \\
	     	さわぐ & さわいだ & さわがない & さわぎました & さわぎません \\
	     	はなす & はなした & はなさない & はなしました & はなしません \\
	     	\vspace{5pt}
	     	さがす & さがした & さがさない & さがしました & さがしません \\
	     	しぬ & しんだ & しなない & しにました & しにません \\
	     	よむ & よんだ & よまない & よみました & よみません \\
	     	のむ & のんだ & のまない & のみました & のみません \\
	     	えらぶ& えらんだ & えらばない & えらびました & えらびません \\
	     	\vspace{5pt}
	     	よぶ & よんだ & よばない & よびました & よびません \\
	     	かう & かった & かわない & かいました & かいません \\
	     	いう & いった & いわない & いいました & いいません \\
	     	まつ & まった & またない & まちました & まちません \\
	     	もつ & もった & もたない & もちました & もちません \\
	     	うる & うった & うらない & うりました & うりません \\
	     	しる & しった & しらない & しりました & しりません \\
	     	\bottomrule
	     \end{tabular}
     
     \newpage
     
         \begin{kaitou}[Rečenice u nastavku obrnite po pristojnosti - kolokvijalne napišite pristojno, a pristojne kolokvijalno. Rečenice prevedite na hrvatski!]
         	\kaimecchanagai{たけしくんは きのう たくさん べんきょう しました。}{たけしくんは きのう たくさん べんきょう した。}{Takeši je jučer puno učio.}
         	\kaimecchanagai{だれが わたしの ケーキを たべたか?}{だれが わたしの ケーキを たべましたか?}{Tko je pojeo moju tortu?}
         	\kaimecchanagai{すずきさんは あたらしい きものを かいました。}{すずきさんは あたらしい きものを かった。}{Gospođa Suzuki je kupila novi kimono.}
         	\kaimecchanagai{「そのえいが、もう見た」と 花子さんは いった。}{「そのえいが、もう見ました」と 花子さんは いいました。}{Hanako je rekla: „Taj film sam već vidjela.”.}
         	\kaimecchanagai{マルちゃんは つくえの下の はこに います。}{マルちゃんは つくえの下の はこに いる。}{Maru je u kutiji ispod stola.}
         	\kaimecchanagai{田中さんは 「マルちゃんは つくえの下の はこに います」と いった。}{田中さんは 「マルちゃんは つくえの下の はこに います」と いいました。}{Gospodin Tanaka je rekao: „Maru je u kutiji ispod stola.”.}
         	\kaimecchanagai{*はこの中にいる たけしくんは ぬすんだ はなこちゃんの ケーキを たべました。}{*はこの中にいる たけしくんは ぬすんだ はなこちゃんの ケーキを たべた。}{Takeši koji je u kutiji je pojeo Hanakinu tortu koju je ukrao.}
         \end{kaitou}
\end{document}