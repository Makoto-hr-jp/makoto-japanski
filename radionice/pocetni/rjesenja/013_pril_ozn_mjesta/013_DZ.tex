% !TeX program = xelatex ?me -synctex=0 -interaction=nonstopmode -aux-directory=../tex_aux -output-directory=./release
% !TeX program = xelatex

\documentclass[12pt]{article}

\usepackage{lineno,changepage,lipsum}
\usepackage[colorlinks=true,urlcolor=blue]{hyperref}
\usepackage{fontspec}
\usepackage{xeCJK}
\usepackage{tabularx}
\setCJKfamilyfont{chanto}{AozoraMinchoRegular.ttf}
\setCJKfamilyfont{tegaki}{Mushin.otf}
\usepackage[CJK,overlap]{ruby}
\usepackage{hhline}
\usepackage{multirow,array,amssymb}
\usepackage[croatian]{babel}
\usepackage{soul}
\usepackage[usenames, dvipsnames]{color}
\usepackage{wrapfig,booktabs}
\renewcommand{\rubysep}{0.1ex}
\renewcommand{\rubysize}{0.75}
\usepackage[margin=50pt]{geometry}
\modulolinenumbers[2]

\usepackage{pifont}
\newcommand{\cmark}{\ding{51}}%
\newcommand{\xmark}{\ding{55}}%

\definecolor{faded}{RGB}{100, 100, 100}

\renewcommand{\arraystretch}{1.2}

%\ruby{}{}
%$($\href{URL}{text}$)$

\newcommand{\furigana}[2]{\ruby{#1}{#2}}
\newcommand{\tegaki}[1]{
	\CJKfamily{tegaki}\CJKnospace
	#1
	\CJKfamily{chanto}\CJKnospace
}

\newcommand{\dai}[1]{
	\vspace{20pt}
	\large
	\noindent\textbf{#1}
	\normalsize
	\vspace{20pt}
}

\newcommand{\fukudai}[1]{
	\vspace{10pt}
	\noindent\textbf{#1}
	\vspace{10pt}
}

\newenvironment{bunshou}{
	\vspace{10pt}
	\begin{adjustwidth}{1cm}{3cm}
	\begin{linenumbers}
}{
	\end{linenumbers}
	\end{adjustwidth}
}

\newenvironment{reibun}{
	\vspace{10pt}
	\begin{tabular}{l l}
}{
	\end{tabular}
	\vspace{10pt}
}
\newcommand{\rei}[2]{
	#1&\textit{#2}\\
}
\newcommand{\reinagai}[2]{
	\multicolumn{2}{l}{#1}\\
	\multicolumn{2}{l}{\hspace{10pt}\textit{#2}}\\
}

\newenvironment{mondai}[1]{
	\vspace{10pt}
	#1
	
	\begin{enumerate}
		\itemsep-5pt
	}{
	\end{enumerate}
	\vspace{10pt}
}

\newenvironment{hyou}{
	\begin{itemize}
		\itemsep-5pt
	}{
	\end{itemize}
	\vspace{10pt}
}

\date{\today}

\CJKfamily{chanto}\CJKnospace
\author{Katja Kržišnik}
\begin{document}
	
	\kaisetsu
	
	\dai{Domaća zadaća - priložne oznake mjesta}
	
	\begin{kaitou}[Pronađite značenja sljedećih znakova i probajte ih nekoliko puta napisati:]
		\kai{右}{desno}
		\kai{左}{lijevo}
		\kai{中}{sredina}
		\kai{上}{gore}
		\kai{下}{dolje}
		\kai{行}{ići}
		\kai{来}{doći}
		\kai{出}{izaći}
    \end{kaitou}

	\begin{kaitou}[Napišite nekoliko rečenica o crnoj mački tako da se u njima barem jednom pojavi svaki od kanji znakova iz prethodnog zadatka. Dakle maksimalan broj rečenica koje smijete napisati je 8 (jedan znak u svakoj), a minimalan 1 (svi znakovi u jednoj).]
		\kainagai{かべの 上に \furigana{黒}{くろ}い猫が いる。}{Na zidu je crna mačka.}
		\kainagai{\furigana{黒}{くろ}い猫は白いはこの中にはいらない。}{Crna mačka neće ući u bijelu kutiju.}
		\kainagai{\furigana{黒}{くろ}い猫は いすの 下から ソファまで 来た。}{Crna mačka je od ispod stolice došla do sofe.}
		\kainagai{まえ 白かった \furigana{黒}{くろ}い猫は 家を 出た。}{Crna mačka koja je prije bila bijela je izašla iz kuće.}
		\kainagai{\furigana{黒}{くろ}い猫のクロは にわの 右の 木の 左に ある 犬の となりに 行く。}{Crni mačak Kuro ide pored psa koji je lijevo od desnog vrtnog stabla.}
	\end{kaitou}

\newpage

     \begin{kaitou}[Opišite ukratko svoju sobu. Imena stvari koja ne znate pokušajte pronaći u rječniku. Bonus bodovi za svaki iskorišteni kanji znak :)]
         \kainagai{\furigana{私}{わたし}の\furigana{部屋}{へや}は\furigana{広}{ひろ}くない。}{Moja soba nije prostrana.}
         \kainagai{\furigana{壁}{かべ}は\furigana{緑色}{みどりいろ}。}{Zidovi su zeleni.}
         \kainagai{\furigana{部屋}{へや}の中に\furigana{布団}{ふとん}とテーブルと\furigana{色々}{いろいろ}なクッションがある。}{U sobi su futon, stol i razno razni jastuci.}
         \kainagai{\furigana{椅子}{いす}はない。}{Nema stolica.}
         \kainagai{\furigana{棚}{たね}の上に\furigana{石}{いし}や本、\furigana{植物}{しょくぶつ}や\furigana{小物}{こもの}がある。}{Na policama je kamenje i knjige, biljke, drangulije i slične stvari.}
     \end{kaitou}
    
\end{document}