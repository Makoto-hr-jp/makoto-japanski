% !TeX program = xelatex ?me -synctex=0 -interaction=nonstopmode -aux-directory=../tex_aux -output-directory=./release
% !TeX program = xelatex

\documentclass[12pt]{article}

\usepackage{lineno,changepage,lipsum}
\usepackage[colorlinks=true,urlcolor=blue]{hyperref}
\usepackage{fontspec}
\usepackage{xeCJK}
\usepackage{tabularx}
\setCJKfamilyfont{chanto}{AozoraMinchoRegular.ttf}
\setCJKfamilyfont{tegaki}{Mushin.otf}
\usepackage[CJK,overlap]{ruby}
\usepackage{hhline}
\usepackage{multirow,array,amssymb}
\usepackage[croatian]{babel}
\usepackage{soul}
\usepackage[usenames, dvipsnames]{color}
\usepackage{wrapfig,booktabs}
\renewcommand{\rubysep}{0.1ex}
\renewcommand{\rubysize}{0.75}
\usepackage[margin=50pt]{geometry}
\modulolinenumbers[2]

\usepackage{pifont}
\newcommand{\cmark}{\ding{51}}%
\newcommand{\xmark}{\ding{55}}%

\definecolor{faded}{RGB}{100, 100, 100}

\renewcommand{\arraystretch}{1.2}

%\ruby{}{}
%$($\href{URL}{text}$)$

\newcommand{\furigana}[2]{\ruby{#1}{#2}}
\newcommand{\tegaki}[1]{
	\CJKfamily{tegaki}\CJKnospace
	#1
	\CJKfamily{chanto}\CJKnospace
}

\newcommand{\dai}[1]{
	\vspace{20pt}
	\large
	\noindent\textbf{#1}
	\normalsize
	\vspace{20pt}
}

\newcommand{\fukudai}[1]{
	\vspace{10pt}
	\noindent\textbf{#1}
	\vspace{10pt}
}

\newenvironment{bunshou}{
	\vspace{10pt}
	\begin{adjustwidth}{1cm}{3cm}
	\begin{linenumbers}
}{
	\end{linenumbers}
	\end{adjustwidth}
}

\newenvironment{reibun}{
	\vspace{10pt}
	\begin{tabular}{l l}
}{
	\end{tabular}
	\vspace{10pt}
}
\newcommand{\rei}[2]{
	#1&\textit{#2}\\
}
\newcommand{\reinagai}[2]{
	\multicolumn{2}{l}{#1}\\
	\multicolumn{2}{l}{\hspace{10pt}\textit{#2}}\\
}

\newenvironment{mondai}[1]{
	\vspace{10pt}
	#1
	
	\begin{enumerate}
		\itemsep-5pt
	}{
	\end{enumerate}
	\vspace{10pt}
}

\newenvironment{hyou}{
	\begin{itemize}
		\itemsep-5pt
	}{
	\end{itemize}
	\vspace{10pt}
}

\date{\today}

\CJKfamily{chanto}\CJKnospace
\author{Katja Kržišnik}
\begin{document}
	\kaisetsu
	
	\dai{Domaća zadaća - Složene rečenice}
	
	     \begin{kaitou}[Prevedite na hrvatski koristeći se pritom svim mogućim prljavim trikovima, uključujući i grupni rad!]
	     	\kainagai{となりの おばさんは さびしかったから ねこを ひろった。}{Teta susjeda je pokupila mačku s ulice jer je bila usamljena.}
	     	\kainagai{ねこを ひろっても さびしかったので もう いっぴき ひろった。}{Pošto je bila usamljena čak i nakon što je pokupila mačku pokupila je još jednu.}
	     	\kainagai{それが つづいて、いまは ねこを じゅっぴき かっている。}{To se nastavilo i sad ima deset mačaka.}
	     	\kainagai{むすこが とうきょうに 行ってから おばさんは さびしく なった。}{Teta je postala usamljena otkad joj je sin otišao u Tokyo.}
	     	\kainagai{そう ははに きいた わたしは かわいそうだと おもった。}{Ja koji sam to čuo od mame sam se sažalio.}
	     	\kainagai{なのに となりの おばさんは いつも えがおだから、そんなに さびしくないと おもう。}{Ipak, pošto je teta susjeda uvijek nasmiješena, mislim da nije tako usamljena.}
	     	\kainagai{わたしは ねこが すきだけど、じゅっぴきは かわない。}{Iako volim mačke, ne bih imao deset.}
	     	\kainagai{だが、としを とって さびしく なった おばさんは、かっています。}{Ali teta koja je ostarila ima.}
	     \end{kaitou}
\end{document}