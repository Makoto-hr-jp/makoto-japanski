% !TeX program = xelatex ?me -synctex=0 -interaction=nonstopmode -aux-directory=../tex_aux -output-directory=./release
% !TeX program = xelatex

\documentclass[12pt]{article}

\usepackage{lineno,changepage,lipsum}
\usepackage[colorlinks=true,urlcolor=blue]{hyperref}
\usepackage{fontspec}
\usepackage{xeCJK}
\usepackage{tabularx}
\setCJKfamilyfont{chanto}{AozoraMinchoRegular.ttf}
\setCJKfamilyfont{tegaki}{Mushin.otf}
\usepackage[CJK,overlap]{ruby}
\usepackage{hhline}
\usepackage{multirow,array,amssymb}
\usepackage[croatian]{babel}
\usepackage{soul}
\usepackage[usenames, dvipsnames]{color}
\usepackage{wrapfig,booktabs}
\renewcommand{\rubysep}{0.1ex}
\renewcommand{\rubysize}{0.75}
\usepackage[margin=50pt]{geometry}
\modulolinenumbers[2]

\usepackage{pifont}
\newcommand{\cmark}{\ding{51}}%
\newcommand{\xmark}{\ding{55}}%

\definecolor{faded}{RGB}{100, 100, 100}

\renewcommand{\arraystretch}{1.2}

%\ruby{}{}
%$($\href{URL}{text}$)$

\newcommand{\furigana}[2]{\ruby{#1}{#2}}
\newcommand{\tegaki}[1]{
	\CJKfamily{tegaki}\CJKnospace
	#1
	\CJKfamily{chanto}\CJKnospace
}

\newcommand{\dai}[1]{
	\vspace{20pt}
	\large
	\noindent\textbf{#1}
	\normalsize
	\vspace{20pt}
}

\newcommand{\fukudai}[1]{
	\vspace{10pt}
	\noindent\textbf{#1}
	\vspace{10pt}
}

\newenvironment{bunshou}{
	\vspace{10pt}
	\begin{adjustwidth}{1cm}{3cm}
	\begin{linenumbers}
}{
	\end{linenumbers}
	\end{adjustwidth}
}

\newenvironment{reibun}{
	\vspace{10pt}
	\begin{tabular}{l l}
}{
	\end{tabular}
	\vspace{10pt}
}
\newcommand{\rei}[2]{
	#1&\textit{#2}\\
}
\newcommand{\reinagai}[2]{
	\multicolumn{2}{l}{#1}\\
	\multicolumn{2}{l}{\hspace{10pt}\textit{#2}}\\
}

\newenvironment{mondai}[1]{
	\vspace{10pt}
	#1
	
	\begin{enumerate}
		\itemsep-5pt
	}{
	\end{enumerate}
	\vspace{10pt}
}

\newenvironment{hyou}{
	\begin{itemize}
		\itemsep-5pt
	}{
	\end{itemize}
	\vspace{10pt}
}

\date{\today}

\CJKfamily{chanto}\CJKnospace
\author{Katja Kržišnik}
\begin{document}
	\kaisetsu
	
	\dai{Upotrebe い oblika I - Vježba}
	
	      
	      \begin{kaitou}
	      	\kainagai{そのとき、たけしくんは かえりたいと おもいはじめた。}{U tom trenu, Takeši je počeo misliti da se želi vratiti.}
	      	\kainagai{日本語の どうしは わかりにくい ですか?}{Jesu li japanski glagoli teški za razumjeti?}
	      	\kainagai{まえの本を よみおわって、こんどは よみやすい本を かりに いきます。}{Završila sam s čitanjem prošle knjige i ovaj put ću ići posuditi lakšu knjigu za čitanje.}
	      	\kainagai{先生が たけしくんに 「あとで しょくいんしつに きなさい」と いいました。}{Profesor je Takešiju rekao: „Poslije dođi u zbornicu.”.}
	      	\kainagai{まいしゅう ここに すしを たべに くるよ。}{Doći ću ovdje svaki tjedan jesti suši.}
	      	\kainagai{すずきさんは きょねん ひっこして、あそびに こなく なった。}{Suzuki se prošle godine preselio i otad se više ne dolazi družiti.}
	      	\kainagai{*日本から もどった ともだちの はなしを きいて わたしも いきたく なりました。}{Čuvši priče prijatelja koji se vratio iz Japana i ja sam poželio otići.}
	      \end{kaitou}
	
\end{document}