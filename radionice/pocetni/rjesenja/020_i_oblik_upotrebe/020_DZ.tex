% !TeX program = xelatex ?me -synctex=0 -interaction=nonstopmode -aux-directory=../tex_aux -output-directory=./release
% !TeX program = xelatex

\documentclass[12pt]{article}

\usepackage{lineno,changepage,lipsum}
\usepackage[colorlinks=true,urlcolor=blue]{hyperref}
\usepackage{fontspec}[ Path =../../../ ]
\usepackage{xeCJK}
\usepackage{tabularx}
\usepackage{graphicx}
\setCJKfamilyfont{chanto}{AOZORAMINCHOREGULAR_0.TTF}%
\setCJKfamilyfont{tegaki}{Mushin.otf}%
\usepackage[CJK,overlap]{ruby}
\usepackage{hhline}
\usepackage{multirow,array,amssymb}
\usepackage[croatian]{babel}
\usepackage{soul}
\usepackage[usenames, dvipsnames]{color}
\usepackage{wrapfig,booktabs}
\usepackage{calc}
\renewcommand{\rubysep}{0.1ex}
\renewcommand{\rubysize}{0.75}
\usepackage[margin=50pt]{geometry}
\usepackage{hyperref}
\modulolinenumbers[2]

\date{\today}

\usepackage{fancyhdr}
\pagestyle{fancy}
\fancyhf{}
\fancyhead[LE,RO]{\thepage}
\makeatletter
\fancyhead[RE,LO]{rev. \@date 誠}
\makeatother

\usepackage{pifont}
\newcommand{\cmark}{\ding{51}}%
\newcommand{\xmark}{\ding{55}}%

\newcommand{\dosl}{{\normalfont dosl. }}%
\newcommand{\rem}[1]{{\normalfont #1 }}%

\definecolor{faded}{RGB}{100, 100, 100}

\renewcommand{\arraystretch}{1.2}

%\ruby{}{}
%$($\href{URL}{text}$)$

\newcommand{\furigana}[2]{\ruby{#1}{#2}}
\newcommand{\tegaki}[1]{
	\CJKfamily{tegaki}\CJKnospace
	#1
	\CJKfamily{chanto}\CJKnospace
}

\newcommand{\dai}[1]{
	\vspace{20pt}
	\large
	\noindent\textbf{#1}
	\normalsize
	\vspace{20pt}
}

\newcommand{\fukudai}[1]{
	\vspace{10pt}
	\noindent\textbf{#1}
	\vspace{10pt}
}

\newenvironment{bunshou}{
	\vspace{10pt}
	\begin{adjustwidth}{1cm}{3cm}
	\begin{linenumbers}
}{
	\end{linenumbers}
	\end{adjustwidth}
}

\newenvironment{reibun}[1][]{
	\vspace{10pt}
	#1
	
	\begin{tabular}{l l}
}{
	\end{tabular}
	\vspace{10pt}
}
\newcommand{\rei}[2]{
	#1&\textit{#2}\\
}
\newcommand{\reinagai}[2]{
	\multicolumn{2}{l}{#1}\\
	\multicolumn{2}{l}{\hspace{10pt}\textit{#2}}\\
}

\newenvironment{mondai}[1]{
	\vspace{10pt}
	\noindent #1
	
	\begin{enumerate}
		\itemsep-5pt
	}{
	\end{enumerate}
}

\newenvironment{hyou}{
	\begin{itemize}
		\itemsep-5pt
	}{
	\end{itemize}
	\vspace{10pt}
}

\newcommand{\juuyou}[2][20pt]{
	\vspace{5pt}
		\noindent\hspace{#1}\parbox[c]{\textwidth-#1-#1}{\centering\textit{#2}}
	\vspace{5pt}
}

\newcommand{\ten}{
	\vspace{5pt}
	\noindent\hspace{-10pt}$\bullet$
}

\CJKfamily{chanto}\CJKnospace

\frenchspacing
\author{Katja Kržišnik}
\begin{document}
	\kaisetsu
	
	\dai{Domaća zadaća - upotrebe い oblika I}
	
	\noindent Prevedite sljedeće rečenice. Pokušajte ih rastaviti na dijelove i napisati / skicirati njihov međusobni odnos. Zadaci su varijacije na oblik rečenica iz vježbe za pripadni listić! (5. zad. mi se učinio lagan pa sam ga preskočio)
	
    	\begin{kaitou}[Zad. 1]
    		\kainagai{すしが たべたい。}{Jede mi se suši.}
    		\kainagai{花子さんは すしが たべたいと いった。}{Hanako je rekla da joj se jede suši.}
    		\kainagai{そのとき、花子さんは すしが たべたいと いった。}{Hanako je tada rekla da joj se jede suši.}
    		\kainagai{そのとき、花子さんは すしが たべたいと おもった。}{Hanako je tada pomislila da joj se jede suši.}
    		\kainagai{そのとき、花子さんは すしが たべたいと おもいはじめた。}{Hanako je tada počela misliti da joj se jede suši.}
    	\end{kaitou}
    
        \begin{kaitou}[Zad. 2]
        	\kainagai{たけしくんは すうがくが わからない。}{Takeši ne razumije matematiku.}
        	\kainagai{すうがくは わかりにくい。}{Matematika je teška za razumjeti.}
        	\kainagai{たけしくんは すうがくが わかりにくいと 先生に いいました。}{Takeši je rekao profesoru da je matematika teška za razumjeti.}
        \end{kaitou}
    
        \begin{kaitou}[Zad. 3]
        	\kainagai{まえの 本を よんだ。}{Pročitao sam prošlu knjigu.}
        	\kainagai{まえの 本を よみおわった。}{Završio sam s čitanjem prošle knjige.}
        	\kainagai{こんどは よみやすい本を かります。}{Sljedeći put ću posuditi knjigu koja je lakša za čitati.}
        	\kainagai{こんどは よみやすい本を かりに いきます。}{Ovaj put idem posuditi knjigu koja je lakša za čitati.}
        	\kainagai{まえの 本を よみおわって、こんどは よみやすい本を かりに いきます。}{Završila sam s čitanjem prošle knjige i ovaj put ću ići posuditi lakšu knjigu za čitati.}
        \end{kaitou}
    
    \newpage
      
        \begin{kaitou}[Zad. 4]
        	\kainagai{あとで しょくいんしつに きなさい。}{Poslije dođi u zbornicu.}
        	\kainagai{先生は 「あとで しょくいんしつ に きなさい」と たけしくんに いいました。}{Profesor je Takešiju rekao: „Poslije dođi u zbornicu.”.}
        	\kainagai{先生は きょうしつを でました。}{Profesor je izašao iz učionice.}
        	\kainagai{\parbox{.90\textwidth}{けさ、先生は「あとで しょくいんしつに きなさい」と たけしくんに いって きょうしつを でた。}}{Jutros je profesor rekao Takešiju: „Poslije dođi u zbornicu.” i izašao iz učionice.}
        \end{kaitou}
    
        \begin{kaitou}[Zad. 6]
        	\kainagai{すずきさんが あそびに くる。}{Suzuki će se doći družiti.}
        	\kainagai{すずきさんが まいにち あそびに きています。}{Suzuki se svaki dan dolazi družiti.}
        	\kainagai{きょねん、すずきさんは ひっこした。}{Prošle godine Suzuki se preselila.}
        	\kainagai{きょねんまで すずきさんが まいにち あそびに きていました。}{Do prošle godine Suzuki se svaki dan dolazila družiti.}
        	\kainagai{ひっこして、 もう あそびに こない。}{Preselila se pa se više neće dolaziti družiti.}
        	\kainagai{すずきさんは あそびに こなく なった。}{Suzuki se više ne dolazi družiti.}
        \end{kaitou}
    
        \begin{kaitou}[Zad. 7]
        	\kainagai{ともだちの はなしを きいた。}{Saslušao sam prijatelja.}
        	\kainagai{ わたしも いきたい。}{I ja želim ići.}
        	\kainagai{ともだちの はなしを きいて、 わたしも いきたいと おもいました。}{Saslušao sam prijatelja pa sam i ja pomislio da želim ići.}
        	\kainagai{ともだちが 日本から もどった。}{Prijatelj se vratio iz Japana.}
        	\kainagai{日本から もどった ともだちが いる。}{Imam prijatelja koji se vratio iz japana.}
        	\kainagai{日本から もどった ともだちの はなしを きいて、 わたしも いきたいと おもいはじめた。}{Saslušao sam prijatelja koji se vratio iz Japana pa sam i ja počeo željeti otići.}
        \end{kaitou}
\end{document}