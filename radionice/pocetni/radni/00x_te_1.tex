% !TeX program = xelatex

\documentclass[12pt]{article}

\usepackage{lineno,changepage,lipsum}
\usepackage[colorlinks=true,urlcolor=blue]{hyperref}
\usepackage{fontspec}
\usepackage{xeCJK}
\usepackage{tabularx}
\setCJKfamilyfont{chanto}{AozoraMinchoRegular.ttf}
\setCJKfamilyfont{tegaki}{Mushin.otf}
\usepackage[CJK,overlap]{ruby}
\usepackage{hhline}
\usepackage{multirow,array,amssymb}
\usepackage[croatian]{babel}
\usepackage{soul}
\usepackage[usenames, dvipsnames]{color}
\usepackage{wrapfig,booktabs}
\renewcommand{\rubysep}{0.1ex}
\renewcommand{\rubysize}{0.75}
\usepackage[margin=50pt]{geometry}
\modulolinenumbers[2]

\usepackage{pifont}
\newcommand{\cmark}{\ding{51}}%
\newcommand{\xmark}{\ding{55}}%

\definecolor{faded}{RGB}{100, 100, 100}

\renewcommand{\arraystretch}{1.2}

%\ruby{}{}
%$($\href{URL}{text}$)$

\newcommand{\furigana}[2]{\ruby{#1}{#2}}
\newcommand{\tegaki}[1]{
	\CJKfamily{tegaki}\CJKnospace
	#1
	\CJKfamily{chanto}\CJKnospace
}

\newcommand{\dai}[1]{
	\vspace{20pt}
	\large
	\noindent\textbf{#1}
	\normalsize
	\vspace{20pt}
}

\newcommand{\fukudai}[1]{
	\vspace{10pt}
	\noindent\textbf{#1}
	\vspace{10pt}
}

\newenvironment{bunshou}{
	\vspace{10pt}
	\begin{adjustwidth}{1cm}{3cm}
	\begin{linenumbers}
}{
	\end{linenumbers}
	\end{adjustwidth}
}

\newenvironment{reibun}{
	\vspace{10pt}
	\begin{tabular}{l l}
}{
	\end{tabular}
	\vspace{10pt}
}
\newcommand{\rei}[2]{
	#1&\textit{#2}\\
}
\newcommand{\reinagai}[2]{
	\multicolumn{2}{l}{#1}\\
	\multicolumn{2}{l}{\hspace{10pt}\textit{#2}}\\
}

\newenvironment{mondai}[1]{
	\vspace{10pt}
	#1
	
	\begin{enumerate}
		\itemsep-5pt
	}{
	\end{enumerate}
	\vspace{10pt}
}

\newenvironment{hyou}{
	\begin{itemize}
		\itemsep-5pt
	}{
	\end{itemize}
	\vspace{10pt}
}

\date{\today}

\CJKfamily{chanto}\CJKnospace

\author{Tomislav Mamić}

\begin{document}
	
	\dai{て oblik}
	
	U japanskom jeziku postoje dva spojna oblika (\textit{konjunktiv}) glagola - い i て oblik. Bitni su za svakodnevnu upotrebu glagola i pridjeva, a pojavljuju se u većini složenijih izraza i rečenica. Kao što im naziv kaže, služe za spajanje riječi i izraza u složenije cjeline.
	
	\fukudai{Tvorba}
	
	S obzirom da već znamo prošli oblik svih glagola, て konjunktiv dobivamo besplatno! Jedina razlika u odnosu na prošlost je da umjesto た i だ govorimo て i で. Ovo vrijedi za sve glagole bez obzira na skupinu.
	
	Za pridjeve je priča neznatno složenija, ali još uvijek jako jednostavna. な i の pridjevi mijenjaju nastavak u で (\textit{konj.} spojnog glagola), a い pridjevi prelaze u prilog (い$\rightarrow$く) na koji se dodaje nastavak て.
	
	\vspace{10pt}
	\begin{tabular}{|l|l|l|}
		\hline
		&osnovni oblik&pravilo\\\hline
		glagoli&prošlost た/だ&た$\rightarrow$て, だ$\rightarrow$で\\\hline
		な/の pridjevi&predikatni oblik だ&だ$\rightarrow$で\\\hline
		い pridjevi&predikatni oblik い&い$\rightarrow$く+て\\\hline
	\end{tabular}

	\fukudai{Niz događaja}
	
	Više rečenica možemo nanizati u vremenski slijed događaja povezujući ih て oblikom. Jedino obvezujuće pravilo je to da rečenice poredamo onako kako su se događale. Na primjer:
	
	\begin{reibun}
		\rei{けさ 6じに おきた。}{Jutros sam ustao u 6.}
		\rei{トストをたべた。}{Pojeo sam tost.}
		\rei{7じに いえを でた。}{U 7 sam izišao iz kuće.}
		\reinagai{けさ 6きに おき\underline{て} トストを たべ\underline{て} 7じに いえを でた。}{Jutros sam ustao u 6, pojeo tost i u 7 izišao iz kuće.}
	\end{reibun}

	Zbog prirode て oblika, u tako nanizane glagole nije moguće upakirati vrijeme. Zbog toga je ponekad potrebno naglasiti \textit{kad} se neka od rečenica događa:
	
	\begin{reibun}
		\rei{ゆうべは おそく かえった。}{Sinoć sam se kasno vratio.}
		\rei{きょう いちにちじゅう ねむかった。}{Danas sam cijeli dan bio pospan.}
		\rei{あしたは はやく ねる。}{Sutra ću rano na spavanje.}
		\reinagai{ゆうべは おそく かえって きょう いちにちじゅう ねむくて あしたは はやく ねる。}{Sinoć sam se kasno vratio pa sam danas cijeli dan bio pospan, a sutra ću rano na spavanje.}
	\end{reibun}

	Ovakvo nizanje u hrvatskom se da prevesti kao niz sastavnih (\textit{i, pa, te, ni}) rečenica, ali ne odgovara im u potpunosti značenjem.
	
	\newpage
	
	Iako u て oblik glagola nije ugrađeno vrijeme, negativni glagoli imaju svoj て oblik. Dapače, imaju dva koji se različito koriste:
	
	\begin{reibun}
		\rei{ない$\rightarrow$なくて}{kao običan い pridjev, koristi se u složenijim izrazima}
		\rei{ない$\rightarrow$ないで}{posebno za glagole, označava da se radnja glagola nije dogodila}
	\end{reibun}

	Zasad ćemo se zadržati na drugom od ta dva oblika jer nam je neposredno koristan, no s vremenom ćemo naučiti koristiti oba.
	
	\begin{reibun}
		\rei{てを あらわないで ごはんを たべた。}{Jeo sam ručak a da nisam oprao ruke.}
		\rei{べんきょう しないで テストを \underline{うけて} おちた。}{\underline{Pisao} sam test bez da sam učio i pao sam.}
	\end{reibun}

	\fukudai{Nizanje opisnih oblika}
	
	Dosad smo već koristili pridjeve da kažemo stvari poput \textit{novi auto} ili \textit{crveni auto}, ali nismo naučili kako reći \textit{novi crveni auto} ili \textit{Auto je nov i crven.} Za nizanje pridjeva koristimo njihov て oblik, pazeći pritom da zadnji pridjev bude u opisnom obliku. Ovaj mehanizam je zapravo isti kao i nizanje rečenica jer pridjevi u predikatnom obliku zapravo jesu potpune rečenice.
	
	\begin{reibun}
		\rei{あかい。}{Crveno je.}
		\rei{あたらしい。}{Novo je.}
		\rei{あかくて あたらしい。}{Crveno je i novo.}
		\rei{あかくて、あたらしい くるま。}{Crveni novi auto.}
	\end{reibun}

	Negativni pridjevi imaju svoj て oblik bez ikakvih skrivenih iznenađenja:
	
	\begin{reibun}
		\rei{あかくて あたらしくない くるま}{crveni ne-baš-novi auto}
		\rei{しずかで ちいさい こども}{tiho malo dijete}
		\rei{せが たかくて ハンサムな おとこ}{visok zgodan tip}
	\end{reibun}

	S obzirom da glagoli i pridjevi u opisnom obliku igraju po istim pravilima, moguće je do određene granice miješati pridjeve i opisne rečenice, ali tu već zalazimo u teritorij gdje nije lako povući liniju između gramatički ispravnog i prirodnog/smislenog.
	
	\begin{reibun}
		\rei{しずかで けんかしない ひと}{tiha osoba koja se ne svadi \cmark}
		\rei{けんかしなくて しずかな ひと}{ovo ne zvuči dobro \xmark}
	\end{reibun}

	\fukudai{Nesvršeni oblik glagola}
	
	Dosad smo već nekoliko puta spominjali kako su svi glagoli koji označavaju radnju u japanskom po vidu svršeni. Tako npr. たべる u osnovi znači \textit{pojesti}, a ne \textit{jesti}. Nesvršeni oblik možemo dobiti dodavanjem pomoćnog glagola いる na て oblik:
	
	\begin{reibun}
		\rei{アイスクリームを たべる。}{Pojest ću sladoled.}
		\rei{アイスクリームを たべて\underline{いる}。}{Jedem sladoled.}
	\end{reibun}

	\newpage
	Vrlo je bitno primijetiti nekoliko stvari u ovoj situaciji:
	
	\begin{hyou}
		\item na て oblik možemo nalijepiti pomoćni glagol
		\item pomoćni glagoli mijenjaju značenje cijelog izraza, ali njihovo osnovno značenje ne mora biti korisno u pogađaju krajnjeg značenja
		\item u kombinaciji s pom. glagolom て oblik je nepromjenjiv, ali ukupan izraz se zdesna ponaša kao glagol
		\item pomoćni glagol nosi svu gramatiku cijelog izraza
	\end{hyou}
	
	Na ovaj se način rješava mutna granica između sadašnjosti i budućnosti u japanskom jeziku. Nadalje, ako tako složeni glagol negiramo, možemo dobiti naizgled neočekivano značenje:
	
	\begin{reibun}
		\rei{たけしさんは けっこん しない。}{Takeši se neće ženiti.}
		\rei{たけしさんは けっこん していない。}{Takeši (još) nije oženjen.}
	\end{reibun}

	Prebacimo li ove primjere u prošlost, dobivamo još neočekivanih značenja:
	
	\begin{reibun}
		\rei{アイスクリームを たべていた。}{Jeo sam sladoled. (onda, u tom trenutku)}
		\rei{たけしさんは けっこん していなかった。}{Takeši (onda) nije bio oženjen.}
	\end{reibun}
	
	Razlog ovim značenjima je to što ovaj oblik glagola zapravo idejno označava stanje. Ikonski primjeri krive upotrebe ovog oblika dolaze od dva slična glagola - しる \textit{znati} i わかる \textit{razumjeti}. 
	
	\begin{reibun}
		\rei{しっている。}{Znam.}
		\rei{わかっている。}{Znam/razumijem. (i pusti me na miru)}
		\rei{しらない。}{Ne znam. (i nije me briga)}
		\rei{わからない。}{Ne znam/ne razumijem.}
	\end{reibun}

	\fukudai{Ženski imperativ}
	
	Glagoli u て obliku sami za sebe zapravo impliciraju naredbu. Razlog tome je u sljedećem odlomku, no o ovoj upotrebi treba zapamtiti nekoliko stvari:
	
	\begin{hyou}
		\item može varirati od uobičajenog kolokvijalnog govora preko blage ljutnje do zajedljivosti
		\item dečki koji ne žele zvučati kao transvestiti bi ga trebali izbjegavati - zvuči jako ženski
		\item za neg. glagole koristi se で oblik
	\end{hyou}
	
	\vspace{-10pt}
	\begin{reibun}
		\rei{みて。}{Gle!}
		\rei{わたしの りんごを かえして。}{Vrati moju jabuku.}
		\rei{わたしの りんごを たべないで。}{Nemoj jesti moje jabuke.}
	\end{reibun}

	\newpage
	\fukudai{Zamolba}
	
	Ako na て oblik glagola dodamo pom. glagol ください\footnotemark[1], izražavamo molbu. S obzirom da je značenje molbe sadržano u pom. glagolu, ako ga maknemo dobijemo ženski imperativ. Vrlo suptilno.
	
	\begin{reibun}
		\rei{みて ください。}{Molim te pogledaj.}
		\rei{わたしの りんごを かえして ください。}{Molim te vrati mi jabuku.}
		\rei{わたしの りんごを たべないで ください。}{Molim te nemoj jesti moje jabuke.}
	\end{reibun}

	\fukudai{Pokušaj}
	
	Kad na て oblik glagola dodamo みる kao pom. glagol, dobivamo značenje pokušaja neke radnje u smislu \textit{probat ću pa vidjeti kako ispadne}. Iako na prvi pogled možda ne zvuči kao nešto što bismo često koristili, zapravo je u japanskom jako prirodna i česta pojava.
	
	\begin{reibun}
		\rei{ほうれんそうを たべて みるか。}{Hoćeš probati špinat? (pojesti) \cmark}
		\rei{クッキーを つくって みた。}{Probao sam napraviti kolačiće. \cmark}
		\rei{えおうごくから にげて みた。}{Probao sam pobjeći iz zatvora. \xmark}
	\end{reibun}

	Zadnji primjer je zamka jer u hrvatskom ne postoji dublji kontekst vezan uz glagol "probati", ali u japanskom ovaj izraz zvuči vrlo ležerno, \textit{probao sam pa što bude, onako usput} i zvuči jako neprirodno kad ga koristimo za ozbiljne stvari do kojih nam je stalo, kao što je recimo bijeg iz zatvora.
	
	\fukudai{Vježba}
	
	\vspace{-20pt}
	\begin{mondai}{}
		\item たなかくんは にほんに かえって けっこん した。
		\item たけしくんは けっこん しないで にほんに かえった。
		\item すうがくを べんきょう して みて すぐに あきらめた。
		\item べんきょう して いい だいがくに はいって ください。
		\item にほんごが よく わからなくて とうきょうで みちに まよった。
		\item けんか しないで。
		\item ここに サインを ください。\footnotemark[2]
		\item あおくて きれいな とりを みて しゃしんを とった。
		\item きょうすけくんは せが たかくて あたまが よくて かっこいい。
		\item さくらちゃんは あたまが よくて かみが きれいで かわいい。
	\end{mondai}
	
	\footnotetext[1]{glagol くださる u い obliku, naučit ćemo nekad kasnije}
	\footnotetext[2]{uočite način na koji je glagol ください upotrijebljen sam}
	
\end{document}