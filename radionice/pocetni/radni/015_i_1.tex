% !TeX program = xelatex
% !TeX program = xelatex

\documentclass[12pt]{article}

\usepackage{lineno,changepage,lipsum}
\usepackage[colorlinks=true,urlcolor=blue]{hyperref}
\usepackage{fontspec}
\usepackage{xeCJK}
\usepackage{tabularx}
\setCJKfamilyfont{chanto}{AozoraMinchoRegular.ttf}
\setCJKfamilyfont{tegaki}{Mushin.otf}
\usepackage[CJK,overlap]{ruby}
\usepackage{hhline}
\usepackage{multirow,array,amssymb}
\usepackage[croatian]{babel}
\usepackage{soul}
\usepackage[usenames, dvipsnames]{color}
\usepackage{wrapfig,booktabs}
\renewcommand{\rubysep}{0.1ex}
\renewcommand{\rubysize}{0.75}
\usepackage[margin=50pt]{geometry}
\modulolinenumbers[2]

\usepackage{pifont}
\newcommand{\cmark}{\ding{51}}%
\newcommand{\xmark}{\ding{55}}%

\definecolor{faded}{RGB}{100, 100, 100}

\renewcommand{\arraystretch}{1.2}

%\ruby{}{}
%$($\href{URL}{text}$)$

\newcommand{\furigana}[2]{\ruby{#1}{#2}}
\newcommand{\tegaki}[1]{
	\CJKfamily{tegaki}\CJKnospace
	#1
	\CJKfamily{chanto}\CJKnospace
}

\newcommand{\dai}[1]{
	\vspace{20pt}
	\large
	\noindent\textbf{#1}
	\normalsize
	\vspace{20pt}
}

\newcommand{\fukudai}[1]{
	\vspace{10pt}
	\noindent\textbf{#1}
	\vspace{10pt}
}

\newenvironment{bunshou}{
	\vspace{10pt}
	\begin{adjustwidth}{1cm}{3cm}
	\begin{linenumbers}
}{
	\end{linenumbers}
	\end{adjustwidth}
}

\newenvironment{reibun}{
	\vspace{10pt}
	\begin{tabular}{l l}
}{
	\end{tabular}
	\vspace{10pt}
}
\newcommand{\rei}[2]{
	#1&\textit{#2}\\
}
\newcommand{\reinagai}[2]{
	\multicolumn{2}{l}{#1}\\
	\multicolumn{2}{l}{\hspace{10pt}\textit{#2}}\\
}

\newenvironment{mondai}[1]{
	\vspace{10pt}
	#1
	
	\begin{enumerate}
		\itemsep-5pt
	}{
	\end{enumerate}
	\vspace{10pt}
}

\newenvironment{hyou}{
	\begin{itemize}
		\itemsep-5pt
	}{
	\end{itemize}
	\vspace{10pt}
}

\date{\today}

\CJKfamily{chanto}\CJKnospace
\author{Tomislav Mamić}

\begin{document}
	\dai{い oblik I}
	
	\fukudai{Teorija}
	
	Ovo je jedan od osnovnih oblika glagola u japanskom. Iz い oblika se tvore prošlost i て oblik, a i sam い oblik ima jako puno primjena. Osim što se koristi za spajanje s pomoćnim glagolima (slično て obliku), služi i kao osnova za jako puno drugih složenijih glagolskih oblika.
	
	\fukudai{Tvorba}
	
	Iako se u japanskom naziva \furigana{連用形}{れんようけい}\ (dosl. \textit{oblik za uzastopnu upotrebu}), na radionicama ga zovemo い oblik zbog asocijacije na tvorbu koja je iz rječničkog oblika savršeno pravilna. Za sve いちだん glagole, dovoljno je \textbf{s kraja ukloniti る}. Za ごだん glagole, jedina potrebna informacija za tvorbu je:
	
	\juuyou{Iz rječničkog oblika, zadnju m\={o}ru hiragane prebaciti tako da suglasnik ostane isti, a samoglasnik prijeđe iz stupca う u stupac い.}
	
	Pri tome naravno moramo poštivati oblike suglasnika u hiragani, pa će recimo かく prijeći u かき (suglasnik \textit{k} ostaje isti), a たつ u たち (suglasnik \textit{c (ts)} postaje \textit{ć (ch)}). Tablica lijevo sadrži primjere pravilnih glagola, u desnoj su い oblici nepravilnih.
	
	\vspace{10pt}
	\begin{tabular}{|l|l|l|}
		\hline
		\textbf{vrsta}&\textbf{glagol}&\textbf{い oblik}\\
		\hline
		\multirow{2}{50pt}{いちだん}&みる&み\\\cline{2-3}
		&たべる&たべ\\
		\hline
		\multirow{4}{50pt}{ごだん}&のむ&のみ\\\cline{2-3}
		&しる&しり\\\cline{2-3}
		&かう&かい\\\cline{2-3}
		&はなす&はなし\\
		\hline
	\end{tabular}\hspace{10pt}
	\begin{tabular}{|l|l|}
		\hline
		\textbf{glagol}&\textbf{い oblik}\\
		\hline
		いく&いき\\
		\hline
		くる&き\\
		\hline
		する&し\\
		\hline
		ある&あり\\
		\hline
	\end{tabular}
	
	\fukudai{Pristojnost u japanskom jeziku - registri govora}
	
	Kad pričamo s nepoznatim ljudima, starijima ili onima koji su nam na bilo koji način nadređeni u danoj situaciji, u japanskom ćemo uvijek koristiti pristojni govor. Postoji više različitih vrsta pristojnog govora ovisno o relativnom društvenom položaju govornika i sugovornika. Za razmišljanje o njima, zgodno nam je promotriti pojam registra koji se definira otprilike ovako:
	
	\juuyou{Registar je podskup prihvatljivih riječi i gramatičkih oblika za korištenje u određenoj prigodi.}
	
	Registri barem donekle postoje u svakom jeziku. U hrvatskom recimo riječ \textit{gospodin}, kao i upotreba množine iz pristojnosti pripada pristojnom, formalnom registru dok riječ \textit{seronja} baš i ne. U sklopu početnih radionica bavit ćemo se isključivo osnovnim pristojnim registrom (\furigana{丁寧語}{ていねいご}) jer je najrašireniji i najkorisniji. Među učenicima japanskog poznat je i kao です/ます zbog pristojnih oblika predikata.
	
	\newpage
	\fukudai{Pristojni oblik glagola \textasciitilde ます}
	
	Kao dio pristojnog registra u japanskom, mijenjaju se \textbf{svi predikatni oblici}\footnotemark[1]. Već smo naučili kolokvijalni i pristojni spojni glagol (だ/です), kao i pristojne imenske predikate s pridjevima. Da bismo upristojili glagole, na njihov い oblik dodajemo nastavak (vodi se kao pom. glagol) ます. Taj nastavak na sebe preuzima svu gramatiku, ali s obzirom da je upotreba uglavnom ograničena na predikatne oblike, pristojni oblici nemaju punu složenost kolokvijalnih glagola. Pogledajmo oblike glagola ます:
	
	\vspace{10pt}
	\begin{tabular}{|l|l|l|}
		\hline
		&\textbf{pozitivni}&\textbf{negativni}\\
		\hline
		\textbf{neprošlost}&ます&ません\\
		\hline
		\textbf{prošlost}&ました&ませんでした\\
		\hline
	\end{tabular}

	\vspace{10pt}
	Uz navedene osnovne oblike, ます se u jako formalnim situacijama pojavljuje i u spojnim oblicima (い i て) kao まし i まして.
	
	\footnotetext[1]{Predikatni oblici su oni koji mogu završiti nezavisnu rečenicu. U japanskom su predikatni i opisni oblik jako slični pa na to treba obratiti pažnju.}
	
	\begin{reibun}[Pogledajmo nekoliko primjera pristojnih predikatnih oblika:]
		\rei{たべる$\rightarrow$たべます}{pojesti}
		\rei{みない$\rightarrow$みません}{ne pogledati}
		\rei{はしった$\rightarrow$はしりました}{potrčao je}
		\rei{なかった$\rightarrow$ありませんでした}{nije bilo}
	\end{reibun}
	
	\begin{reibun}[Primjenjivo je i na složenije predikatne oblike:]
		\rei{はなしていなかった$\rightarrow$はなしていませんでした}{nije pričao}
		\rei{けっこん していない$\rightarrow$けっこん していません}{nije oženjen}
	\end{reibun}

	\fukudai{Vježba}
	
	\begin{mondai}{Upristojimo sljedeće rečenice:}
		\item すうがくを べんきょう して みて すぐに あきらめた。
		\item さくらちゃんは あたまが よくて かみが きれいで かわいい。
		\item きのう、きょねん なくした ゆびわを さがしている ひとに あった。
		\item きょうすけさんは ことしの いちがつに にほんに もどって しがつに きょうこさんと けっこんする。
		\item 「まる」は ねこの なまえ だった。
	\end{mondai}
\end{document}