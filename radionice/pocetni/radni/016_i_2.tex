% TeX program = xelatex
% !TeX program = xelatex

\documentclass[12pt]{article}

\usepackage{lineno,changepage,lipsum}
\usepackage[colorlinks=true,urlcolor=blue]{hyperref}
\usepackage{fontspec}
\usepackage{xeCJK}
\usepackage{tabularx}
\setCJKfamilyfont{chanto}{AozoraMinchoRegular.ttf}
\setCJKfamilyfont{tegaki}{Mushin.otf}
\usepackage[CJK,overlap]{ruby}
\usepackage{hhline}
\usepackage{multirow,array,amssymb}
\usepackage[croatian]{babel}
\usepackage{soul}
\usepackage[usenames, dvipsnames]{color}
\usepackage{wrapfig,booktabs}
\renewcommand{\rubysep}{0.1ex}
\renewcommand{\rubysize}{0.75}
\usepackage[margin=50pt]{geometry}
\modulolinenumbers[2]

\usepackage{pifont}
\newcommand{\cmark}{\ding{51}}%
\newcommand{\xmark}{\ding{55}}%

\definecolor{faded}{RGB}{100, 100, 100}

\renewcommand{\arraystretch}{1.2}

%\ruby{}{}
%$($\href{URL}{text}$)$

\newcommand{\furigana}[2]{\ruby{#1}{#2}}
\newcommand{\tegaki}[1]{
	\CJKfamily{tegaki}\CJKnospace
	#1
	\CJKfamily{chanto}\CJKnospace
}

\newcommand{\dai}[1]{
	\vspace{20pt}
	\large
	\noindent\textbf{#1}
	\normalsize
	\vspace{20pt}
}

\newcommand{\fukudai}[1]{
	\vspace{10pt}
	\noindent\textbf{#1}
	\vspace{10pt}
}

\newenvironment{bunshou}{
	\vspace{10pt}
	\begin{adjustwidth}{1cm}{3cm}
	\begin{linenumbers}
}{
	\end{linenumbers}
	\end{adjustwidth}
}

\newenvironment{reibun}{
	\vspace{10pt}
	\begin{tabular}{l l}
}{
	\end{tabular}
	\vspace{10pt}
}
\newcommand{\rei}[2]{
	#1&\textit{#2}\\
}
\newcommand{\reinagai}[2]{
	\multicolumn{2}{l}{#1}\\
	\multicolumn{2}{l}{\hspace{10pt}\textit{#2}}\\
}

\newenvironment{mondai}[1]{
	\vspace{10pt}
	#1
	
	\begin{enumerate}
		\itemsep-5pt
	}{
	\end{enumerate}
	\vspace{10pt}
}

\newenvironment{hyou}{
	\begin{itemize}
		\itemsep-5pt
	}{
	\end{itemize}
	\vspace{10pt}
}

\date{\today}

\CJKfamily{chanto}\CJKnospace
\author{Tomislav Mamić}

\begin{document}
	\dai{い oblik II}
	
	\fukudai{Izražavanje želje s \textasciitilde たい}
	
	Da bismo u japanskom rekli da želimo nešto učiniti, glagol koji želimo učiniti prebacujemo u い oblik i dodajemo mu nastavak たい. Tako dobiveni glagolski oblik zapravo \textbf{postaje い pridjev} što ima bitne posljedice za njegovu upotrebu.
	
	\begin{reibun}
		\rei{たべる}{jesti}
		\rei{たべたい}{(ono što) želim jesti}
	\end{reibun}

	Kao što možemo vidjeti iz prijevoda iznad, perspektiva se znatno promijenila u novom obliku. Kao pridjev, たい oblik \textbf{opisuje imenice} kao nešto na čemu bismo htjeli izvršiti radnju glagola:
	
	\begin{reibun}
		\rei{たべたい りんご}{jabuka koju želim pojesti}
		\rei{みたい えいが}{film kojeg želim pogledati}
		\rei{わすれたい よる}{noć koju želim zaboraviti}
	\end{reibun}

	Kao i svi drugi pridjevi, たい oblik može se pojaviti i u predikatu. U ovom slučaju događa se nešto što možda izgleda kao detalj, ali je gramatički vrlo čudno. Naime, svi imenski predikati su po prirodi \textbf{neprelazni} što znači da \textbf{ne mogu imati direktni objekt}. Zato se događa "promjena perspektive" u rečenici - sam fenomen koji rečenica opisuje ostaje isti, ali je potrebno presložiti uloge. Slično je kao upotreba pridjeva すき kojeg smo ranije radili. Pogledajmo primjere:
	
	\begin{reibun}
		\rei{わたしは りんご\underline{を} たべる $\rightarrow$ わたしは りんご\underline{が} たべたい}{}
		\rei{ともだちに ほん\underline{を} あげた $\rightarrow$ ともだちに ほん\underline{が} あげたかった}{}
	\end{reibun}

	Međutim, u svakodnevnoj upotrebi stvari su malo složenije. Iako je gramatički netočno, u zadnje je vrijeme u porastu upotreba direktnog objekta (を) s pridjevima ovog tipa. U drugom primjeru se pojavljuje neizravni objekt (に - \textit{kome?}). U ovakvoj situaciji još je veća sklonost govornika da ne mijenjaju を$\rightarrow$が (vidi tablicu\footnotemark[1]).
	
	\footnotetext[1]{Ovakve rezultate pretraživanja možete dobiti korištenjem google-a i sličnih servisa za slučaj kad niste sigurni što je prirodnije reći.}
	
	\vspace{10pt}
	\begin{tabular}{|l|l|}
		\hline
		\textbf{izraz}&\textbf{broj rezultata}\\
		\hline
		"が食べたい"&17.5M\\
		"を食べたい"&14.7M\\
		\hline
		"に*があげたい"&1.8M\\
		"に*をあげたい"&7.3M\\
		\hline
	\end{tabular}

	\vspace{10pt}
	Bez obzira kojim se stilom odlučite izražavati, bitno je zapamtiti da razlike u značenju nema. Nadalje, preporuka kuhara je da uvijek počnete korištenjem najčešćih izraza pa ih onda mijenjate po želji/potrebi kad dobijete bolji osjećaj o tome kako jezik funkcionira.
	
	\newpage
	\fukudai{Ići i doći nešto raditi}
	
	Da bismo rekli da \textit{idemo <nešto>}, u japanskom koristimo izraz <nešto u い obliku>にいく. S desne strane, taj je izraz za sve gramatičke potrebe zapravo glagol いく:
	
	\begin{reibun}
		\rei{あそびに いくよ。}{Idem se igrati!}
		\rei{まいにち 11じに ひるごはんを たべに いく。}{Svaki dan u 11 idem jesti ručak.}
		\reinagai{さいきん、ともだちと あそびに いかない。}{U zadnje vrijeme se ne idem zabavljati s prijateljima.}
		\rei{けっきょく、えいがを みに いかなかった。}{Na kraju nismo išli gledati film.}
	\end{reibun}

	Po potpuno istom principu možemo koristiti i glagol くる za \textit{doći <nešto>}:
	
	\begin{reibun}
		\rei{また あそびに きてください。}{Dođi se opet igrati. (impl. sa mnom)}
		\rei{りんごを とりに きました。}{Došao sam brati jabuke.}
	\end{reibun}

	\fukudai{Vježba}
	
	\begin{mondai}{Lv. 1}
		\item いきません。
		\item みたい。
		\item あそぶ。
	\end{mondai}

	\begin{mondai}{Lv. 2}
		\item あそこへ いきません。
		\item えいがが みたい。
		\item すずきちゃんと あそぶ。
	\end{mondai}

	\begin{mondai}{Lv. 3}
		\item おばの いえへ いきません。
		\item あの かなしい えいがが みたい。
		\item すずきちゃんと こうえんで あそぶ。
	\end{mondai}

	\begin{mondai}{Lv. 4}
		\item おばの いえへ いきたくありません。
		\item あの かなしい えいがを きみと みたい です。
		\item また すずきちゃんと きょねんの なつ いっしょに あそんだ こうえんに あそびに いきたい。
	\end{mondai}
\end{document}