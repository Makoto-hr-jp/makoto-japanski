% !TeX document-id = {5a8b4c70-9f53-4387-96a0-5094a066db12}
% !TeX program = xelatex ?me -synctex=0 -interaction=nonstopmode -aux-directory=../tex_aux -output-directory=./release
% !TeX program = xelatex

\documentclass[12pt]{article}

\usepackage{lineno,changepage,lipsum}
\usepackage[colorlinks=true,urlcolor=blue]{hyperref}
\usepackage{fontspec}
\usepackage{xeCJK}
\usepackage{tabularx}
\setCJKfamilyfont{chanto}{AozoraMinchoRegular.ttf}
\setCJKfamilyfont{tegaki}{Mushin.otf}
\usepackage[CJK,overlap]{ruby}
\usepackage{hhline}
\usepackage{multirow,array,amssymb}
\usepackage[croatian]{babel}
\usepackage{soul}
\usepackage[usenames, dvipsnames]{color}
\usepackage{wrapfig,booktabs}
\renewcommand{\rubysep}{0.1ex}
\renewcommand{\rubysize}{0.75}
\usepackage[margin=50pt]{geometry}
\modulolinenumbers[2]

\usepackage{pifont}
\newcommand{\cmark}{\ding{51}}%
\newcommand{\xmark}{\ding{55}}%

\definecolor{faded}{RGB}{100, 100, 100}

\renewcommand{\arraystretch}{1.2}

%\ruby{}{}
%$($\href{URL}{text}$)$

\newcommand{\furigana}[2]{\ruby{#1}{#2}}
\newcommand{\tegaki}[1]{
	\CJKfamily{tegaki}\CJKnospace
	#1
	\CJKfamily{chanto}\CJKnospace
}

\newcommand{\dai}[1]{
	\vspace{20pt}
	\large
	\noindent\textbf{#1}
	\normalsize
	\vspace{20pt}
}

\newcommand{\fukudai}[1]{
	\vspace{10pt}
	\noindent\textbf{#1}
	\vspace{10pt}
}

\newenvironment{bunshou}{
	\vspace{10pt}
	\begin{adjustwidth}{1cm}{3cm}
	\begin{linenumbers}
}{
	\end{linenumbers}
	\end{adjustwidth}
}

\newenvironment{reibun}{
	\vspace{10pt}
	\begin{tabular}{l l}
}{
	\end{tabular}
	\vspace{10pt}
}
\newcommand{\rei}[2]{
	#1&\textit{#2}\\
}
\newcommand{\reinagai}[2]{
	\multicolumn{2}{l}{#1}\\
	\multicolumn{2}{l}{\hspace{10pt}\textit{#2}}\\
}

\newenvironment{mondai}[1]{
	\vspace{10pt}
	#1
	
	\begin{enumerate}
		\itemsep-5pt
	}{
	\end{enumerate}
	\vspace{10pt}
}

\newenvironment{hyou}{
	\begin{itemize}
		\itemsep-5pt
	}{
	\end{itemize}
	\vspace{10pt}
}

\date{\today}

\CJKfamily{chanto}\CJKnospace
\author{Tomislav Mamić}
\begin{document}
	\dai{Kauzativ i pasiv}
	
	\fukudai{Tvorba}
	
	\begin{wraptable}{r}{200pt}
		\centering
		\begin{tabular}{l r r}\toprule[2pt]
			rječnički oblik & kauzativ & pasiv\\
			(辞書形) & (使役) & (受動態)\\
			\midrule
			\multicolumn{3}{c}{nepravilni}\\
			\midrule
			いく & いかせる & いかれる\\
			& いかす & \\
			くる & こさせる & こられる\\
			ある & nema\footnotemark[1] & nema\footnotemark[1]\\
			する & させる & される\\
			\midrule
			\multicolumn{3}{c}{一段}\\
			\midrule
			\textasciitilde る & \textasciitilde させる & \textasciitilde される\footnotemark[2]\\
			\midrule
			\multicolumn{3}{c}{五段}\\
			\midrule
			\textasciitilde く & \textasciitilde かせる & \textasciitilde かれる\\
			\textasciitilde ぐ & \textasciitilde がせる & \textasciitilde がれる\\\vspace{5pt}
			\textasciitilde す & \textasciitilde させる & \textasciitilde される\\
			\textasciitilde ぬ & \textasciitilde なせる & \textasciitilde なれる\\
			\textasciitilde む & \textasciitilde ませる & \textasciitilde まれる\\\vspace{5pt}
			\textasciitilde ぶ & \textasciitilde ばせる & \textasciitilde ばれる\\
			\textasciitilde う & \textasciitilde わせる & \textasciitilde われる\\
			\textasciitilde つ & \textasciitilde たせる & \textasciitilde たれる\\
			\textasciitilde る & \textasciitilde らせる & \textasciitilde られる\\
			\bottomrule[2pt]
		\end{tabular}
	\end{wraptable}

	Kako se da uočiti iz tablice desno, osnova tvorbe kauzativa i pasiva je あ stupac, kao za negaciju. Kauzativ dobijemo dodavanjem せる, a pasiv pomoću れる. Iz tablice se također da uočiti da glagol いく ima 2 oblika kauzativa. Iako u tablici nisu navedeni jer bi zauzimalo puno mjesta, a i pravilno je, zapravo svi 五段 glagoli osim す također imaju dva kauzativa. Moguće je naime umjesto せる staviti す. Bitno je da, ako odlučite koristiti す, znate da zvuči pomalo kolokvijalno i da nije dobra praksa iz kauzativa na す dalje raditi složenije oblike. Dakle u bilo kojoj situaciji gdje treba na kauzativ dodati još promjena, bolja opcija je せる.
	
\end{document}