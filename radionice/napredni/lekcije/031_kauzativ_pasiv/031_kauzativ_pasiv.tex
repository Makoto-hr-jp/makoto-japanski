% !TeX document-id = {5a8b4c70-9f53-4387-96a0-5094a066db12}
% !TeX program = xelatex ?me -synctex=0 -interaction=nonstopmode -aux-directory=../tex_aux -output-directory=./release
% !TeX program = xelatex

\documentclass[12pt]{article}

\usepackage{lineno,changepage,lipsum}
\usepackage[colorlinks=true,urlcolor=blue]{hyperref}
\usepackage{fontspec}
\usepackage{xeCJK}
\usepackage{tabularx}
\setCJKfamilyfont{chanto}{AozoraMinchoRegular.ttf}
\setCJKfamilyfont{tegaki}{Mushin.otf}
\usepackage[CJK,overlap]{ruby}
\usepackage{hhline}
\usepackage{multirow,array,amssymb}
\usepackage[croatian]{babel}
\usepackage{soul}
\usepackage[usenames, dvipsnames]{color}
\usepackage{wrapfig,booktabs}
\renewcommand{\rubysep}{0.1ex}
\renewcommand{\rubysize}{0.75}
\usepackage[margin=50pt]{geometry}
\modulolinenumbers[2]

\usepackage{pifont}
\newcommand{\cmark}{\ding{51}}%
\newcommand{\xmark}{\ding{55}}%

\definecolor{faded}{RGB}{100, 100, 100}

\renewcommand{\arraystretch}{1.2}

%\ruby{}{}
%$($\href{URL}{text}$)$

\newcommand{\furigana}[2]{\ruby{#1}{#2}}
\newcommand{\tegaki}[1]{
	\CJKfamily{tegaki}\CJKnospace
	#1
	\CJKfamily{chanto}\CJKnospace
}

\newcommand{\dai}[1]{
	\vspace{20pt}
	\large
	\noindent\textbf{#1}
	\normalsize
	\vspace{20pt}
}

\newcommand{\fukudai}[1]{
	\vspace{10pt}
	\noindent\textbf{#1}
	\vspace{10pt}
}

\newenvironment{bunshou}{
	\vspace{10pt}
	\begin{adjustwidth}{1cm}{3cm}
	\begin{linenumbers}
}{
	\end{linenumbers}
	\end{adjustwidth}
}

\newenvironment{reibun}{
	\vspace{10pt}
	\begin{tabular}{l l}
}{
	\end{tabular}
	\vspace{10pt}
}
\newcommand{\rei}[2]{
	#1&\textit{#2}\\
}
\newcommand{\reinagai}[2]{
	\multicolumn{2}{l}{#1}\\
	\multicolumn{2}{l}{\hspace{10pt}\textit{#2}}\\
}

\newenvironment{mondai}[1]{
	\vspace{10pt}
	#1
	
	\begin{enumerate}
		\itemsep-5pt
	}{
	\end{enumerate}
	\vspace{10pt}
}

\newenvironment{hyou}{
	\begin{itemize}
		\itemsep-5pt
	}{
	\end{itemize}
	\vspace{10pt}
}

\date{\today}

\CJKfamily{chanto}\CJKnospace
\author{Tomislav Mamić}
\begin{document}
	\dai{Kauzativ i pasiv}
	
	\fukudai{Tvorba}
	
	\begin{wraptable}[21]{r}{200pt}
		\centering
		\begin{tabular}{l r r}\toprule[2pt]
			rječnički oblik & kauzativ & pasiv\\
			(辞書形) & (使役) & (受動態)\\
			\midrule
			\multicolumn{3}{c}{nepravilni}\\
			\midrule
			いく & いかせる & いかれる\\
			& いかす & \\
			くる & こさせる & こられる\\
			ある & nema\footnotemark[1] & nema\footnotemark[1]\\
			する & させる & される\\
			\midrule
			\multicolumn{3}{c}{一段}\\
			\midrule
			\textasciitilde る & \textasciitilde させる & \textasciitilde られる\footnotemark[2]\\
			\midrule
			\multicolumn{3}{c}{五段}\\
			\midrule
			\textasciitilde く & \textasciitilde かせる & \textasciitilde かれる\\
			\textasciitilde ぐ & \textasciitilde がせる & \textasciitilde がれる\\\vspace{5pt}
			\textasciitilde す & \textasciitilde させる & \textasciitilde される\\
			\textasciitilde ぬ & \textasciitilde なせる & \textasciitilde なれる\\
			\textasciitilde む & \textasciitilde ませる & \textasciitilde まれる\\\vspace{5pt}
			\textasciitilde ぶ & \textasciitilde ばせる & \textasciitilde ばれる\\
			\textasciitilde う & \textasciitilde わせる & \textasciitilde われる\\
			\textasciitilde つ & \textasciitilde たせる & \textasciitilde たれる\\
			\textasciitilde る & \textasciitilde らせる & \textasciitilde られる\\
			\bottomrule[2pt]
		\end{tabular}
	\end{wraptable}

	\footnotetext[1]{Kauzativ i pasiv glagola ある ne nalazi baš puno praktičnih primjena.}
	\footnotetext[2]{Ne pomiješati s potencijalom - \textit{potencijalno opasno} HA ha.}

	Kako se da uočiti iz tablice desno, osnova tvorbe kauzativa i pasiva je あ stupac (未然形), kao za negaciju. Kauzativ dobijemo dodavanjem せる, a pasiv pomoću れる. Iz tablice se također da uočiti da glagol いく ima 2 oblika kauzativa. Iako u tablici nisu navedeni, zapravo svi 五段 glagoli osim \textasciitilde~す također imaju dva kauzativa. Moguće je naime umjesto せる staviti す. Bitno je da, ako odlučite koristiti す, znate da zvuči pomalo kolokvijalno i da nije dobra praksa iz kauzativa na す dalje raditi složenije oblike. Dakle u bilo kojoj situaciji gdje treba na kauzativ dodati još promjena, bolja opcija je せる.
	
	\fukudai{Značenje kauzativa}
	
	Ovisno o kontekstu, može se shvatiti kao prisila
	
	\begin{reibun}
		\reinagai{こどもに にんじんを たべさせた。}{Natjerao sam dijete da jede mrkve.}
	\end{reibun}

	ili kao dopuštenje
	
	\begin{reibun}
		\reinagai{いぬを そとに いかせた。}{Pustio sam psa van.}
	\end{reibun}

	Uočimo kako se u kauzativu događa pomak u ulogama - subjekt (は/が) postaje meta radnje (に), a novi subjekt je onaj tko dopušta ili prisiljava na radnju.

	\fukudai{Značenje pasiva}
	
	Iako oblik zovemo \textit{pasiv}, postoji dosta situacija gdje uopće nema tu gramatičku funkciju. Kao \mbox{敬語} oblik, izražava poštovanje prema glagolima čiji je vršitelj radnje sugovornik, a njime možemo izraziti i nezadovoljstvo onim što se dogodilo iz perspektive onog kome se to dogodilo. U ovom slučaju onaj tko radnju pretrpi ostaje subjekt, stvarni vršitelj radnje označen je česticom に, a direktni objekt se ne mijenja.
	
	\begin{reibun}
		\rei{鈴木さん\underline{は}リンゴ\underline{を}食べられて泣いている。}{Suzuki je netko pojeo jabuku pa plače.}
		\rei{それ\underline{を}先生\underline{に}よく言われています。}{Prof. mi to često govori.}
	\end{reibun}

	\newpage
	Da bi pasiv zaista gramatički bio pasiv, mora doći do zamjene perspektive subjekta i objekta tako da direktni objekt postaje subjekt (を $\rightarrow$ は/が), a subjekt dobiva česticu に kao vršitelj radnje (は/が $\rightarrow$ に).
	
	\begin{reibun}
		\rei{リンゴ\underline{を}食べた。 $\rightarrow$ リンゴ\underline{が}食べられた。}{Pojeo sam jabuku. $\rightarrow$ Jabuka je pojedena.}
		\reinagai{先生\underline{が}松田さんのノート\underline{を}見つけた。 $\rightarrow$ 松田さんのノート\underline{は}先生\underline{に}見つけられた\footnotemark[3]。}{Profesor je pronašao Matsudinu bilježnicu. $\rightarrow$ Matsudina bilježnica je pronađena.}
	\end{reibun}
	
	Uočimo kako je u drugom primjeru u japanskom vrlo prirodno dodati informaciju o vršitelju radnje dok je u hrvatskom svaki pokušaj istog (npr. \textit{Matsudina bilježnica je pronađena od (strane) profesora.}) zločin protiv jezika.

	\footnotetext[3]{Ovdje je prirodno reći i 見つかった.}
	
	\fukudai{Pasiv kauzativa}
	
	Znači samo kombinaciju kauzativa i pasiva - \textit{dopušteno mi je} / \textit{natjeran sam} nešto učiniti. Oblik je kompozicija kauzativa i pasiva - pasiv(kauzativ(glagol)). S obzirom da završava pasivom, može imati i onu konotaciju da govornik nije sretan situacijom
	
	\begin{reibun}
		\reinagai{子供の\furigana{頃}{ころ}、ニンジンを食べさせられていた。}{Kad sam bio dijete, tjerali su me da jedem mrkve. \rem{(i nisam bio sretan zbog toga)}}
		\rei{日本では、\furigana{酒}{さけ}を\underline{飲ませられる}ことが多い。}{U japanu često budeš prisiljen piti alkohol.}
	\end{reibun}

	Moguće je koristiti i skraćeni kauzativ, ali ne smijemo zaboraviti da ga 一段 i 五段 す glagoli nemaju. Znači smijemo reći
	
	\begin{reibun}
		\rei{日本では、\furigana{酒}{さけ}を\underline{飲まされる}ことが多い。}{U japanu često budeš prisiljen piti alkohol.}
	\end{reibun}

	\newpage
	\fukudai{Vježba}
	
	\begin{mondai}{Lv. 1}
		\item 先生に言われた。
		\item 生徒達に\furigana{掃除}{そうじ}させました。
		\item たけしくんは2時間も\furigana{待}{ま}たされた。
	\end{mondai}

	\begin{mondai}{Lv. 2}
		\item \furigana{今朝}{けさ}、学校に\furigana{着}{つ}いたとたん、先生に\furigana{油}{あぶら}を\furigana{搾}{しぼ}られた。
		\item \furigana{悪}{わる}い生徒達に\furigana{庭}{にわ}を掃除させました。\vspace{7pt}
		\item たけしくんは彼女に2時間も待たされた。
	\end{mondai}

	\begin{mondai}{Lv. 3}
		\item 今朝学校に着いたとたん、先生に油を搾られたたけしくんは何が\furigana{起}{お}きているか分からなかった。
		\item 悪い生徒達に庭の掃除をさせた先生は\furigana{同僚}{どうりょう}と\hspace{10pt}\furigana{一緒}{いっしょ}にそれを見ながら\furigana{笑}{わら}っていました。
		\item 彼女に2時間も待たされて\furigana{怒}{おこ}っていたたけしくんは\furigana{結局}{けっきょく}\hspace{15pt}\furigana{帰}{かえ}ってしまった。
	\end{mondai}
	
\end{document}
