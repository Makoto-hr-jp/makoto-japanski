% !TeX document-id = {c5b1505a-f30f-4093-80bc-284ee6feff4a}
% !TeX program = xelatex ?me -synctex=0 -interaction=nonstopmode -aux-directory=../tex_aux -output-directory=./release
% !TeX program = xelatex

\documentclass[12pt]{article}

\usepackage{lineno,changepage,lipsum}
\usepackage[colorlinks=true,urlcolor=blue]{hyperref}
\usepackage{fontspec}
\usepackage{xeCJK}
\usepackage{tabularx}
\setCJKfamilyfont{chanto}{AozoraMinchoRegular.ttf}
\setCJKfamilyfont{tegaki}{Mushin.otf}
\usepackage[CJK,overlap]{ruby}
\usepackage{hhline}
\usepackage{multirow,array,amssymb}
\usepackage[croatian]{babel}
\usepackage{soul}
\usepackage[usenames, dvipsnames]{color}
\usepackage{wrapfig,booktabs}
\renewcommand{\rubysep}{0.1ex}
\renewcommand{\rubysize}{0.75}
\usepackage[margin=50pt]{geometry}
\modulolinenumbers[2]

\usepackage{pifont}
\newcommand{\cmark}{\ding{51}}%
\newcommand{\xmark}{\ding{55}}%

\definecolor{faded}{RGB}{100, 100, 100}

\renewcommand{\arraystretch}{1.2}

%\ruby{}{}
%$($\href{URL}{text}$)$

\newcommand{\furigana}[2]{\ruby{#1}{#2}}
\newcommand{\tegaki}[1]{
	\CJKfamily{tegaki}\CJKnospace
	#1
	\CJKfamily{chanto}\CJKnospace
}

\newcommand{\dai}[1]{
	\vspace{20pt}
	\large
	\noindent\textbf{#1}
	\normalsize
	\vspace{20pt}
}

\newcommand{\fukudai}[1]{
	\vspace{10pt}
	\noindent\textbf{#1}
	\vspace{10pt}
}

\newenvironment{bunshou}{
	\vspace{10pt}
	\begin{adjustwidth}{1cm}{3cm}
	\begin{linenumbers}
}{
	\end{linenumbers}
	\end{adjustwidth}
}

\newenvironment{reibun}{
	\vspace{10pt}
	\begin{tabular}{l l}
}{
	\end{tabular}
	\vspace{10pt}
}
\newcommand{\rei}[2]{
	#1&\textit{#2}\\
}
\newcommand{\reinagai}[2]{
	\multicolumn{2}{l}{#1}\\
	\multicolumn{2}{l}{\hspace{10pt}\textit{#2}}\\
}

\newenvironment{mondai}[1]{
	\vspace{10pt}
	#1
	
	\begin{enumerate}
		\itemsep-5pt
	}{
	\end{enumerate}
	\vspace{10pt}
}

\newenvironment{hyou}{
	\begin{itemize}
		\itemsep-5pt
	}{
	\end{itemize}
	\vspace{10pt}
}

\date{\today}

\CJKfamily{chanto}\CJKnospace
\author{Ivan Petranović}
\begin{document}
	\dai{Ciljevi i napome - keigo}
	
	\fukudai{Ciljevi}
	
	\begin{hyou}
		\item objasniti pojam "registar" u jezikoslovlju
		\item objasniti zašto postoji keigo(寧語)
		\item objasniti vrste keigo-a (丁寧語, 尊敬語, 謙譲語)
		\item objasniti gdje se koja vrsta koristi
		\item naučiti prepoznati vrste keigo-a
		\item naučiti česte oblike za svaku vrstu keigo-a
		\item keigo glagoli
	\end{hyou}

	\fukudai{Napomene}
	
	\begin{hyou}
		\item bitno je naučiti prepoznati i razumjeti keigo puno više nego ga znati koristiti
		\item cilj je u eventualnoj situaciji posjeta Japanu znati preživjeti susret s tetom u dućanu ili konobarom koji koristi keigo.
	\end{hyou}

	\fukudai{Dodatne pričice}
	
	Registar je skup riječi koji se koristi pri određenom društvenom kontekstu. Npr. u hrvatskom jeziku izrazi "Bok!", "Dobar dan!" i "Havaljen Isus i Marija." pripadaju različitim registrima. Korištenje riječi iz neprigodnog registra zvuči neobično, a ponekad i uvredljivo. Različiti jezici imaju različit broj i opseg registara. Za razliku od hrvatskog, u japanskom jeziku registri bitno utječu i na gramatiku, a ne samo na odabir riječi. S time smo se već susreli u vidu pristojnih i kolokvijalnih oblika, no tomu nije kraj.
	
	Keigo je naziv za registre koji se koriste u formalnoj komunikaciji. Njih se u normalnom razgovoru između poznanika ili stranaca ne koristi već uglavnnom u komunikaciji poslovnog ili profesionalnog karaktera, primjerice prilikom obraćanja kupcima u dućanu, gostima u restoranu ili suradnicima u tvrtki.
	
	丁寧語 je običan pristojan japanski, koristi ga se u običnim razgovorima sa strancima te na televiziji ili u novinama. Tehnički je i to keigo, ali se ponekad ne smatra "pravim" keigo-om.
	
	尊敬語 je "jezik s poštovanjem", njime se opisuju akcije druge osobe prema kojoj se iskazuje posebno poštovanje (kupac, klijent, šef?). Također se njime moli i navodi istu osobu da nešto radi.
	
	謙譲語 je "ponizni jezik", njime se osobi prema kojoj se iskazuje poštovanje opisuju akcije samog govornika ili drugih članova grupe kojoj pripada.
\end{document}