% !TeX document-id = {5a8b4c70-9f53-4387-96a0-5094a066db12}
% !TeX program = xelatex ?me -synctex=0 -interaction=nonstopmode -aux-directory=../tex_aux -output-directory=./release
% !TeX program = xelatex

\documentclass[12pt]{article}

\usepackage{lineno,changepage,lipsum}
\usepackage[colorlinks=true,urlcolor=blue]{hyperref}
\usepackage{fontspec}
\usepackage{xeCJK}
\usepackage{tabularx}
\setCJKfamilyfont{chanto}{AozoraMinchoRegular.ttf}
\setCJKfamilyfont{tegaki}{Mushin.otf}
\usepackage[CJK,overlap]{ruby}
\usepackage{hhline}
\usepackage{multirow,array,amssymb}
\usepackage[croatian]{babel}
\usepackage{soul}
\usepackage[usenames, dvipsnames]{color}
\usepackage{wrapfig,booktabs}
\renewcommand{\rubysep}{0.1ex}
\renewcommand{\rubysize}{0.75}
\usepackage[margin=50pt]{geometry}
\modulolinenumbers[2]

\usepackage{pifont}
\newcommand{\cmark}{\ding{51}}%
\newcommand{\xmark}{\ding{55}}%

\definecolor{faded}{RGB}{100, 100, 100}

\renewcommand{\arraystretch}{1.2}

%\ruby{}{}
%$($\href{URL}{text}$)$

\newcommand{\furigana}[2]{\ruby{#1}{#2}}
\newcommand{\tegaki}[1]{
	\CJKfamily{tegaki}\CJKnospace
	#1
	\CJKfamily{chanto}\CJKnospace
}

\newcommand{\dai}[1]{
	\vspace{20pt}
	\large
	\noindent\textbf{#1}
	\normalsize
	\vspace{20pt}
}

\newcommand{\fukudai}[1]{
	\vspace{10pt}
	\noindent\textbf{#1}
	\vspace{10pt}
}

\newenvironment{bunshou}{
	\vspace{10pt}
	\begin{adjustwidth}{1cm}{3cm}
	\begin{linenumbers}
}{
	\end{linenumbers}
	\end{adjustwidth}
}

\newenvironment{reibun}{
	\vspace{10pt}
	\begin{tabular}{l l}
}{
	\end{tabular}
	\vspace{10pt}
}
\newcommand{\rei}[2]{
	#1&\textit{#2}\\
}
\newcommand{\reinagai}[2]{
	\multicolumn{2}{l}{#1}\\
	\multicolumn{2}{l}{\hspace{10pt}\textit{#2}}\\
}

\newenvironment{mondai}[1]{
	\vspace{10pt}
	#1
	
	\begin{enumerate}
		\itemsep-5pt
	}{
	\end{enumerate}
	\vspace{10pt}
}

\newenvironment{hyou}{
	\begin{itemize}
		\itemsep-5pt
	}{
	\end{itemize}
	\vspace{10pt}
}

\date{\today}

\CJKfamily{chanto}\CJKnospace
\author{Ivan Petranović}
\begin{document}
	\dai{Keigo}
	
	\fukudai{Teorija - registri}
	
	U različitim društvenim kontekstima koristimo drugačije riječi. Primjerice, kada želimo nekoga pozdraviti, ovisno o našem odnosu s osobom koju pozdravljamo možemo upotrijebiti bilo koju od sljedećih fraza.
	\begin{hyou}
	\item "Bok!"
	\item "Dobar dan!"
	\item "Hvaljen Isus i Marija!"
	\item "Oj!"
	\item ...
	\end{hyou}
	Sve te fraze imaju isto značenje i istu funkciju: pozdravljanje osobe. Razlike u odabiru riječi i izraza dolaze iz međusobnog odnosa sugovornika, kao i prigode u kojoj se nalaze. Različiti skupovi riječi koji se koriste ovisno o društvenom kontekstu u kojem se rabe se zovu registri. U registre mogu ući i gramatičke značajke. U hrvatskom je jeziku najočitija značajka to što se osobama koje poštujemo obraćamo u drugom licu množine: Vi, dok se osobama s kojima smo prisni obraćamo u drugom licu jednine.
	Za razliku od hrvatskog društva, u japanskom je društvu relativan odnos među ljudima u društvenoj hijerarhiji važniji dio svakodnevnog života, pa je kada pričamo japanski puno bitnije paziti kojim registrom pričamo.
	Dva glavna registra japanskog jezika se zovu ため\furigana{口}{ぐち} i \furigana{敬語}{けいご}. ため口 se koristi u razgovoru s osobama istog ili nižeg društvenog statusa od govornika te povremeno s osobama s kojima je govornik jako blizak (npr. s roditeljima). U većini ostalih situacija se koristi 敬語.

	\fukudai{Vrste 敬語-a}
	
	U modernom se japanskom govoru može susresti tri vrste 敬語-a. Prva od njih je \furigana{丁寧語}{ていねいご}. S njim se već regularno susrećemo u japanskom jer je to uobičajeni pristojni japanski govor. Koristi se u neformalnim razgovorima sa strancima, kao i na televiziji i u novinama.
		
	Druge dvije vrste 敬語-a su \furigana{尊敬語}{そんけいご} i \furigana{謙譲語}{けんじょうご}. Ta dva oblika zajedno čine najformalniji registar japanskog koji se redovito koristi. Susreće ih se u poslovnoj ili profesionalnoj komunikaciji, primjerice pri obraćanju kupcima u dućanu. S obzirom da ćemo se, ukoliko posjetimo Japan, naći u ulozi kupca, dobro je znati prepoznati i razumijeti 敬語 izraze.
		
	\fukudai{丁寧語}
	
	丁寧語 smo već upoznali i redovito ga koristimo. Riječ je o uobičajenim pristojnim oblicima glagola koji se tvore dodavanjem nastavka ます na い-oblik glagola te korištenjem spojnog glagola です.
	
	\fukudai{謙譲語 i 尊敬語}
	
	謙譲語 (doslovno "ponizni jezik") se koristi za opis radnji koje čini govornik ili neki govornikov suradnik. (govornikov 内)
	尊敬語 (doslovno "jezik iz poštovanja") se koristi za opis radnji koje čini kupac, putnik ili osoba u nekoj sličnoj situaciji. (govornikov 外)
	
	謙譲語 oblici glagola se tvore tako da se い-obliku glagola stavi prefiks お te mu se doda する ili いたす.
	
	Primjer
	
	尊敬語 oblici glagola se tvore tako da se い-obliku glagola stavi prefiks お te mu se doda になる.
	
	Primjer
	
	\fukudai{敬語 glagoli}

	Dok se većina glagola može prebaciti u viši registar koristeći gore navedena pravila, neke izrazito česti glagoli imaju zasebne 敬語 parnjake.
	
	Insert tablica here
	
	Prilikom korištenja tih glagola je potrebno paziti da se njih ne provuče kroz gornje algoritme jer bi to stvorilo dupli 敬語. Ukoliko se osobi potkrade dupli 敬語 sugovornik može dobiti dojam kao da mu se govornik ruga.
	
	\fukudai{Vježba}
\end{document}