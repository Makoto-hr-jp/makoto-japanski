% !TeX document-id = {5a8b4c70-9f53-4387-96a0-5094a066db12}
% !TeX program = xelatex ?me -synctex=0 -interaction=nonstopmode -aux-directory=../tex_aux -output-directory=./release
% !TeX program = xelatex

\documentclass[12pt]{article}

\usepackage{lineno,changepage,lipsum}
\usepackage[colorlinks=true,urlcolor=blue]{hyperref}
\usepackage{fontspec}
\usepackage{xeCJK}
\usepackage{tabularx}
\setCJKfamilyfont{chanto}{AozoraMinchoRegular.ttf}
\setCJKfamilyfont{tegaki}{Mushin.otf}
\usepackage[CJK,overlap]{ruby}
\usepackage{hhline}
\usepackage{multirow,array,amssymb}
\usepackage[croatian]{babel}
\usepackage{soul}
\usepackage[usenames, dvipsnames]{color}
\usepackage{wrapfig,booktabs}
\renewcommand{\rubysep}{0.1ex}
\renewcommand{\rubysize}{0.75}
\usepackage[margin=50pt]{geometry}
\modulolinenumbers[2]

\usepackage{pifont}
\newcommand{\cmark}{\ding{51}}%
\newcommand{\xmark}{\ding{55}}%

\definecolor{faded}{RGB}{100, 100, 100}

\renewcommand{\arraystretch}{1.2}

%\ruby{}{}
%$($\href{URL}{text}$)$

\newcommand{\furigana}[2]{\ruby{#1}{#2}}
\newcommand{\tegaki}[1]{
	\CJKfamily{tegaki}\CJKnospace
	#1
	\CJKfamily{chanto}\CJKnospace
}

\newcommand{\dai}[1]{
	\vspace{20pt}
	\large
	\noindent\textbf{#1}
	\normalsize
	\vspace{20pt}
}

\newcommand{\fukudai}[1]{
	\vspace{10pt}
	\noindent\textbf{#1}
	\vspace{10pt}
}

\newenvironment{bunshou}{
	\vspace{10pt}
	\begin{adjustwidth}{1cm}{3cm}
	\begin{linenumbers}
}{
	\end{linenumbers}
	\end{adjustwidth}
}

\newenvironment{reibun}{
	\vspace{10pt}
	\begin{tabular}{l l}
}{
	\end{tabular}
	\vspace{10pt}
}
\newcommand{\rei}[2]{
	#1&\textit{#2}\\
}
\newcommand{\reinagai}[2]{
	\multicolumn{2}{l}{#1}\\
	\multicolumn{2}{l}{\hspace{10pt}\textit{#2}}\\
}

\newenvironment{mondai}[1]{
	\vspace{10pt}
	#1
	
	\begin{enumerate}
		\itemsep-5pt
	}{
	\end{enumerate}
	\vspace{10pt}
}

\newenvironment{hyou}{
	\begin{itemize}
		\itemsep-5pt
	}{
	\end{itemize}
	\vspace{10pt}
}

\date{\today}

\CJKfamily{chanto}\CJKnospace
\author{Ivan Petranović}
\begin{document}
	\dai{Keigo}
	
	\fukudai{Teorija - registri}
		U različitim društvenim kontekstima koristimo drugačije riječi. Primjerice, kada želimo nekoga pozdraviti, ovisno o našem odnosu s osobom koju pozdravljamo možemo upotrijebiti bilo koju od sljedećih fraza.
		\begin{hyou}
		\item "Bok!"
		\item "Dobar dan!"
		\item "Hvaljen Isus i Marija!"
		\item "Oj!"
		\item ...
		\end{hyou}
	Sve te fraze imaju isto semantičko značenje i istu funkciju: pozdravljanje osobe. Jedina je razlika društveni status osobe kojoj je upućena. Različiti skupovi riječi koji se koriste ovisno o društvenom kontekstu u kojem se rabe se zovu registri. U registre mogu ući i gramatičke značajke. U hrvatskom je jeziku najočitija značajka to što se osobama koje poštujemo obraćamo u drugom licu množine: Vi, dok se osobama s kojima smo prisno obraćamo u drugom licu jednine.
	Za razliku od hrvatskog društva, u japanskom je društvu relativan odnos među ljudima u društvenoj hijerarhiji važniji dio svakodnevnog života, pa je kada pričamo japanski puno bitnije paziti kojim registrom pričamo.
	Dva glavna registra japanskog jezika se zovu ため\furigana{口}{ぐち} i \furigana{敬語}{けいご}. ため口 se koristi u razgovoru s osobama istog ili nižeg društvenog statusa od govornika te povremeno s osobama s kojima je govornik jako blizak (npr. s roditeljima). U većini ostalih situacija se koristi.

	\fukudai{Vrste 敬語-a}
		U modernom se japanskom govoru može susresti tri vrste 敬語-a. Prva od njih je \furigana{丁寧語}{ていねいご}. S njim se već regularno susrećemo u japanskom jer je to uobičajeni pristojni japanski govor. Koristi se u neformalnim razgovorima sa strancima, kao i na televiziji i u novinama.
		
		Druge dvije vrste 敬語-a su \furigana{尊敬語}{そんけいご} i \furigana{謙譲語}{けんじょうご}. Ta dva oblika zajedno čine najformalniji registar japanskog koji se redovito koristi. Susreće ih se u poslovnoj ili profesionalnoj komunikaciji, primjerice pri obraćanju kupcima u dućana. S obzirom da ćemo se, ukoliko posjetimo Japan, naći u ulozi kupca, dobro je znati prepoznati i razumijeti 敬語 izraze.
	
	\fukudai{謙譲語 i 尊敬語}
		\furigana{謙譲語}{けんじょうご} (doslovno "ponizni jezik") se koristi za opis radnji koje čini govornik ili neki govornikov suradnik. (govornikov 内)
		\furigana{尊敬語}{そんけいご}
\end{document}