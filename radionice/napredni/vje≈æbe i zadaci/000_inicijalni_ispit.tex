% !TeX program = xelatex

\documentclass[12pt]{article}

\usepackage[colorlinks=true,urlcolor=blue]{hyperref}
\usepackage{xeCJK}
\usepackage{tabularx}
\setCJKmainfont{AozoraMinchoRegular.ttf}
\usepackage[CJK,overlap]{ruby}
\usepackage{hhline}
\usepackage{multirow,array,amssymb}
\usepackage[croatian]{babel}
\usepackage{soul}
\renewcommand{\rubysep}{0.1ex}
\renewcommand{\rubysize}{0.55}
\author{Tomislav Mamić}
\title{Inicijalna provjera}

%\ruby{}{}
%$($\href{URL}{text}$)$

\begin{document}

	\subsection{Imenski predikat}
	
	\paragraph{1.} Povežite rečenice i sintagme s njihovim točnim prijevodima.
	
	\begin{tabularx}{\textwidth}{X X}
		Ono je mačka.&あれは ねこでした。\\
		Ono nije bila mačka.&これは ねこです。\\
		Ono je bila mačka.&これは ねこじゃない。\\
		Ovo nije mačka.&あれは ねこじゃなかった。\\
		Ovo je mačka.&あれは ねこだ。\\
	\end{tabularx}

	\subsection{Pridjevi}

	\paragraph{2.} Povežite rečenice i sintagme s njihovim točnim prijevodima.
	
	\begin{tabularx}{\textwidth}{X X}
		Ovaj auto je crven.&これは あかい くるまじゃない。\\
		Ovo nije crveni auto.&これは あかくなかった くるまだ。\\
		Ovo je bio crveni auto.&これは あかい くるまだった。\\
		Ovo je auto koji nije bio crven.&あかくない くるま\\
		auto koji nije crven&この くるまは あかいです。\\
		auto koji je bio crven&あかくなかった くるま\\
	\end{tabularx}

	\paragraph{3.} Posložite pridjeve u zadane sintagme na hrvatskom.
	
	\begin{tabularx}{\textwidth}{X X}
		Pridjevi:&Sintagme:\\
		あおい&lijepa plava ptica\\
		あかい&plava lijepa ptica\\
		きれい&crveno voće koje mrzim\\
		きらい&nezrelo voće koje nije fino\\
		しずか&lijepa, crvena, fina jabuka\\
		おいしい&tiho plavo more\\
	\end{tabularx}

	\subsection{Glagoli}
	
	\paragraph{4.} Povežite rečenice i točne prijevode.
	
	\begin{tabularx}{\textwidth}{X X}
		りんごを たべます。&(Tu) je mačka.\\
		とりを みた。&Vidjela se ptica.\\
		とりが みえました。&Pojest ću jabuku.\\
		ねこが いる。&Vidio sam pticu.\\
		ねこが いません。&Nema mačke.\\
	\end{tabularx}
	
	\newpage
	\paragraph{5.} Dopunite tablicu glagolskih oblika.
	
	\begin{tabularx}{\textwidth}{X X X X}
		Glagol:&Prošlost:&Negacija:&い oblik:\\
		かく&&&\\
		およぐ&&&\\
		ころす&&&\\
		しぬ&&&\\
		よむ&&&\\
		えらぶ&&&\\
		うたう&&&\\
		たつ&&&\\
		はしる&&&\\
	\end{tabularx}
	
	\paragraph{6.} Precrtajte netočne glagolske oblike u tablici.
	
	\begin{tabularx}{\textwidth}{X X X X}
		Glagol:&て oblik:&Negacija:&い oblik:\\
		する&して&すない&し\\
		くる&きて&きない&き\\
		ある&あて&あらない&あり\\
		いく&いって&いかない&いき\\
	\end{tabularx}

	\subsection{Lokacija}
	
	\paragraph{7.} Iz zadanih dijelova sastavite 5 suvislih rečenica. Svaki dio morate iskoristiti barem jednom.
	
	\begin{tabularx}{\textwidth}{X X X}
		ねこが&木の&なかに\\
		こうえんで&はこの&うえに\\
		あそんだ。&したで&いる。\\
		こどもたちは&ねた。&いえの\\
	\end{tabularx}
	
	\subsection{Vrijeme}
	
	\paragraph{8.} Precrtajte netočne dijelove u rečenicama.
	
	\begin{tabularx}{\textwidth}{X}
		きのうに ともだちに あった。\\
		けさ6時に おきた。\\
		しごとはあさ8時まで ごご4時からです。\\
		おととし23さいに なった。\\
	\end{tabularx}

	\newpage
	\subsection{て oblik}
	
	\paragraph{9.} Dovršite rečenice pravilnim oblikom za zadani prijevod.
	
	\begin{tabularx}{\textwidth}{X X X}
		つくえの上に本が(1)。&おく& \textit{Na stolu je (ostavljena) knjiga.}\\
		子供たちは外で(2)。&あそぶ& \textit{Djeca se vani igraju.}\\
		きのう友達に(4)(5)(6).&あう かえる ねる&\textit{Jučer sam se našao s prijateljem, vratio se kući i spavao.}\\
		しずかに(7)ください。&する&\textit{Molim te budi tiho.}\\
	\end{tabularx}

	\subsection{い oblik}
	
	\paragraph{10.} Upristojite zadane rečenice.
	
	\begin{tabularx}{\textwidth}{X}
		おれはナルトだってばよ。\\
		なっとうを たべたことは ない。\\
		しけんは あしただと せんせいが いった。\\
		しね。\\
	\end{tabularx}

	\paragraph{11.} Povežite glagole i prijevode.
	
	\begin{tabularx}{\textwidth}{X X}
		はしります&\textit{lako za pojesti}\\
		はしりたい&\textit{potrčao (sam)}\\
		はしりだした&\textit{nastaviti jesti}\\
		たべやすい&\textit{teško za pojesti}\\
		たべにくい&\textit{trčati}\\
		たべつづける&\textit{želim trčati}\\
	\end{tabularx}

	\subsection{Napredni glagolski oblici}
	
	\paragraph{12.} Prevedite na hrvatski. U slučaju dvoznačnih rečenica prihvaćam oba odgovora.
	
	\begin{tabularx}{\textwidth}{X}
		どこまではしれますか?\\
		せんせいが わたしに 「まちなさい」と いった。\\
		わたしは せんせいに 「まちなさい」と いわれた。\\
		わたしは せんせいに 「まちなさい」と いわせた。\\
		せんせいは わたしに 「まちなさい」と いわせられた。\\
		こどものころ にんじんを たべさせられた。\\
	\end{tabularx}

	\subsection{Rečenice složene opisom}
	
	\paragraph{13.} U stupcima su navedeni dijelovi koje valja presložiti i česticama povezati u suvisle rečenice. Jedan stupac, jedna rečenica. Ako neke dijelove ne znate iskoristiti, izostavite ih, ali svi dijelovi se daju povezati.
	
	\begin{tabularx}{\textwidth}{X X X X}
		です&ある&えいが&おなじ\\
		なまえ&あそんでいた&たけしくん&すずきさん\\
		みた&こうえん&みなかった&3ねん\\
		ねこ&うしろ&しし&かいしゃ\\
		マル&えき&うごけなかった&まえ\\
		きのう&まえ&こわすぎて&みたい\\
		&よく&&おもいます\\
		&&&はなして\\
		&&&やめた\\
	\end{tabularx}

	\paragraph{14.} Prevedite rečenice iz prethodnog zadatka na hrvatski.
	
	\subsection{Rečenice složene veznicima}
	
	\paragraph{15.} Umetnite veznik koji nedostaje da bi rečenica odgovarala prijevodu s desne strane.
	
	\begin{tabularx}{\textwidth}{X X}
		べんきょうしなかった(1)だいじょうぶだった(3)つぎはどうなる(4)わからない。&Iako nisam učio, ispalo je dobro, ali ne znam kako će biti drugi put.\\
		たけしはわたしのおもちゃをぬすんだ(5)おかあさんにいいつけた(6)あとで(いたいめをみた)*。&Tužio sam Takešija mami jer mi je ukrao igračku, ali sam kasnije nadrapao*.
	\end{tabularx}

\flushright おつかれさまー
\end{document}