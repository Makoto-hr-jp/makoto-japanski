% !TeX document-id = {0c1c4a5c-e31a-46dd-900d-a126304b4a2f}
% !TeX program = xelatex ?me -synctex=0 -interaction=nonstopmode -aux-directory=../tex_aux -output-directory=./release
% !TeX program = xelatex

\documentclass[12pt]{article}

\usepackage{lineno,changepage,lipsum}
\usepackage[colorlinks=true,urlcolor=blue]{hyperref}
\usepackage{fontspec}[ Path =../../../ ]
\usepackage{xeCJK}
\usepackage{tabularx}
\usepackage{graphicx}
\setCJKfamilyfont{chanto}{AOZORAMINCHOREGULAR_0.TTF}%
\setCJKfamilyfont{tegaki}{Mushin.otf}%
\usepackage[CJK,overlap]{ruby}
\usepackage{hhline}
\usepackage{multirow,array,amssymb}
\usepackage[croatian]{babel}
\usepackage{soul}
\usepackage[usenames, dvipsnames]{color}
\usepackage{wrapfig,booktabs}
\usepackage{calc}
\renewcommand{\rubysep}{0.1ex}
\renewcommand{\rubysize}{0.75}
\usepackage[margin=50pt]{geometry}
\usepackage{hyperref}
\modulolinenumbers[2]

\date{\today}

\usepackage{fancyhdr}
\pagestyle{fancy}
\fancyhf{}
\fancyhead[LE,RO]{\thepage}
\makeatletter
\fancyhead[RE,LO]{rev. \@date 誠}
\makeatother

\usepackage{pifont}
\newcommand{\cmark}{\ding{51}}%
\newcommand{\xmark}{\ding{55}}%

\newcommand{\dosl}{{\normalfont dosl. }}%
\newcommand{\rem}[1]{{\normalfont #1 }}%

\definecolor{faded}{RGB}{100, 100, 100}

\renewcommand{\arraystretch}{1.2}

%\ruby{}{}
%$($\href{URL}{text}$)$

\newcommand{\furigana}[2]{\ruby{#1}{#2}}
\newcommand{\tegaki}[1]{
	\CJKfamily{tegaki}\CJKnospace
	#1
	\CJKfamily{chanto}\CJKnospace
}

\newcommand{\dai}[1]{
	\vspace{20pt}
	\large
	\noindent\textbf{#1}
	\normalsize
	\vspace{20pt}
}

\newcommand{\fukudai}[1]{
	\vspace{10pt}
	\noindent\textbf{#1}
	\vspace{10pt}
}

\newenvironment{bunshou}{
	\vspace{10pt}
	\begin{adjustwidth}{1cm}{3cm}
	\begin{linenumbers}
}{
	\end{linenumbers}
	\end{adjustwidth}
}

\newenvironment{reibun}[1][]{
	\vspace{10pt}
	#1
	
	\begin{tabular}{l l}
}{
	\end{tabular}
	\vspace{10pt}
}
\newcommand{\rei}[2]{
	#1&\textit{#2}\\
}
\newcommand{\reinagai}[2]{
	\multicolumn{2}{l}{#1}\\
	\multicolumn{2}{l}{\hspace{10pt}\textit{#2}}\\
}

\newenvironment{mondai}[1]{
	\vspace{10pt}
	\noindent #1
	
	\begin{enumerate}
		\itemsep-5pt
	}{
	\end{enumerate}
}

\newenvironment{hyou}{
	\begin{itemize}
		\itemsep-5pt
	}{
	\end{itemize}
	\vspace{10pt}
}

\newcommand{\juuyou}[2][20pt]{
	\vspace{5pt}
		\noindent\hspace{#1}\parbox[c]{\textwidth-#1-#1}{\centering\textit{#2}}
	\vspace{5pt}
}

\newcommand{\ten}{
	\vspace{5pt}
	\noindent\hspace{-10pt}$\bullet$
}

\CJKfamily{chanto}\CJKnospace

\frenchspacing
\author{Marko Miličić}

\begin{document}
	\dai{\furigana{形式名詞3}{けいしきめいし}}
	
	\fukudai{はず}

	\begin{reibun}
		\rei{もうそろそろ着く\underline{はず}だ}{Trebalo bi uskoro stići.}
		\rei{彼が知らない\underline{はず}がありません}{Nema šanse da on ne zna.}
		\rei{会議は午後の\underline{はず}です}{Sastanak bi trebalo biti popodne.}
		\rei{時間に間に合いますか。その\underline{はず}です}{Hoćeš li stići na vrijeme? Trebao bi.}
	\end{reibun}
			
		\begin{tabular}{|l|l|}
		\hline
		断定&zaključak\\\hline
		確信&uvjerenje\\\hline
		\end{tabular}


	\fukudai{つもり}

	\begin{reibun}
		\rei{もう一度やる\underline{つもり}です}{Imam namjeru to učiniti još jednom.}
		\rei{二度と行く\underline{つもり}はありません}{Nemam namjere više ikad ići.}
		\rei{この絵は何をかいたのですか。}{Što si nacrtao na ovoj slici?}
		\rei{猫の\underline{つもり}ですが...}{Imao sam namjere da bude mačka, ali...}
		\rei{上手くだました\underline{つもり}だったが...}{Mislio sam da sam ga prevario,ali...}
		\rei{上手くだます\underline{つもり}だったが...}{Imao sam namjere prevariti ga, ali...}
		\rei{あなたも買いますか。その\underline{つもり}です}{Hoćeš li i ti kupiti? To mi je namjera.}
	\end{reibun}
				
		\begin{tabular}{|l|l|}
		\hline
		勧誘&pozivanje\\\hline
		意志表現&izražavanje volje\\\hline
		\end{tabular}


	\newpage
	\fukudai{の}
					
		
	\begin{reibun}
	\rei{その大きいケーキをください。}{Molim Vas tu veliku tortu.}
	\rei{その大きい\underline{の}をください。}{Molim Vas tu veliku.(「の」はケーキの代わり)}
	\rei{私のケーキは大きいです。}{Moja torta je velika. }
	\rei{私\underline{の}は大きいです。}{Moja je velika.(「の」はケーキの省略)}
	\rei{ケーキを作る\underline{の}が好きです。}{Volim raditi torte.(名詞化)}
	\rei{ケーキが大きい\underline{の}\footnotemark[1]はいいことです。}{Dobra je stvar što je torta velika.}
	\rei{彼女がここへ来た\underline{の}は4日です。8日ではありません。}{Ona je ovamo došla četvrtog. Ne osmog.}
	\rei{彼女は4日にここへ来た}{Ona je četvrtog došla ovamo.}
	\end{reibun}
	\footnotetext[1]{複文を作るために}
	
	\begin{tabular}{|l|l|}
		\hline
		名詞化&nominalizacija\\\hline
		強調&naglašavanje\\\hline
		省略&skraćivanje\\\hline
		複文&kompleksne rečenice\\\hline
	\end{tabular}

	\fukudai{ため}
	
	\begin{reibun}
	\rei{人の\underline{ため}に作る}{Napraviti za dobrobit ljudi.}
	\rei{会社の\underline{ため}に働く}{Raditi za dobrobit tvrtke.}
	\rei{あなたの\underline{ため}になる}{Bit će za tvoju dobrobit.}
	\rei{苦しい練習は自分の\underline{ため}です。人の\underline{ため}ではありません。}{Vježbam teško za svoju dobrobit. Ne za tuđu.}
	\rei{雨の\underline{ため}に中止になりました。}{Zaustavljeno je zbog kiše.}
	\end{reibun}
	
	\begin{tabular}{|l|l|}
		\hline
		原因&uzrok\\\hline
		理由&razlog\\\hline
		利益&dobit\\\hline
	\end{tabular}
	
	\begin{reibun}
	\rei{崖崩れでJRが不通となった。\underline{このため/そのため}、首都圏の足が大きく混乱した。}{}
	\end{reibun}

	\textit{JR je prestao voziti zbog odrona stijene. Zbog toga, stvorio se kaos (u smislu transporta) na području glavnog grada.}
	
	\begin{reibun}
	\rei{崖崩れでJRが不通となった\underline{ため}、首都圏の足が大きく混乱した。}{}
	\end{reibun}

	\textit{Zbog toga što je JR prestao voziti zbog odrona stijene, stvorio se kaos (u smislu transporta) na području glavnog grada.}
	
	\begin{reibun}

	\rei{何の\underline{ため}に勉強しますか。進学の\underline{ため}に勉強する。}{Zašto učiš? Učim da mogu upisati faks.}
	\rei{誰の\underline{ため}に働くのですか。}{Za čiju korist radiš?}
	\end{reibun}
	
	
	\newpage
	\fukudai{せいで、おかげで}\footnotemark[2]
	
	\begin{reibun}
	\rei{失敗した\underline{せいで}、みんなに迷惑をかけました。}{Doveo sam ljude u neugodnu situaciju zbog svog neuspjeha.}
	\rei{全てはあなたの\underline{せい}\footnotemark[3]だよ!}{Sve si ti kriv!}
	\end{reibun}
	
	\begin{reibun}
	\rei{みんなが手伝ってくれた\underline{おかげで}早く終わりました。}{}
	\end{reibun}

	\textit{Brzo smo završili zahvaljujući tome što su svi pomogli.}	
	
	\vspace{10pt}
	\begin{tabular}{|l|l|}
		\hline
		比較&uspoređivanje\\\hline
		理由&razlog\\\hline
	\end{tabular}
	
	\footnotetext[2]{原因の「ため」と同じだが感情を表すー用法はもっと広い}
	\footnotetext[3]{この例文において「ため」は使えない}

	\fukudai{まま}	
	
	\begin{reibun}
	\rei{まだパジャマの\underline{まま}です。}{Još sam u piđami.}
	\rei{きゅうりを皮の\underline{まま}食べる。}{Jedem krastavce dok još imaju kožu.}
	\rei{この\underline{まま}ずっとここにいます。}{Bit ću stalno ovdje ovako.}
	\rei{その\underline{まま}、働かないで!}{Ne radi tako!}
	\rei{その\underline{まま}で働かないで!}{Ne radi tako!}
	\rei{靴の\underline{まま}、家に入る}{ući u kuću noseći cipele}
	\rei{その\underline{まま}に\footnotemark[4]してください。}{Učini to tako molim te.}
	\rei{熱い\underline{まま}、お皿に盛ります}{servirati tanjur dok je vrući}
	\rei{長い\underline{まま}で\footnotemark[5]は食べにくいので短く切ります。}{Dok je dugačko je teško za jesti pa ću odrezati na kraće.}
	\end{reibun}
	
	\footnotetext[4]{「なる・する」は大抵「に」を使う}
	\footnotetext[5]{「ままで」では様子、様態を表す}
	
	
	\fukudai{とおり(道、やり方、行き方、それにしたがって、それと同じように)}

	\begin{reibun}
	\rei{あの占いの\underline{とおり}です。}{Tako je kako je u onom proročanstvu.}
	\rei{地図の\underline{とおり}に行きます。}{Idem kako bi karta kaže.}
	\rei{私の言ってる\underline{とおり}にして下さい。}{Molim te, učini kako ti kažem.}
	\rei{その\underline{とおり}です。}{Tako je.}
	\rei{結果はご覧の\underline{とおり}になりました。}{Ishod je ispao kao što vidiš.}
	\end{reibun}


	\newpage
	\fukudai{ほう}
	
	\begin{reibun}
	\rei{こっちの\underline{方}が大きい。}{Ovaj je veći.}
	\rei{あなたは行かない\underline{方}がいいかも。}{Možda bolje da ti ne ideš.}
	\rei{それは私の\underline{方}でなんとかします。}{Ja ću nešto učiniti u vezi toga.}
	\rei{この\underline{方}が早いです。}{Ovako je brže.}
	\end{reibun}	


	xどのほう se ne kaže
	
	\vspace{10pt}
	\begin{tabular}{|l|l|}
		\hline
		片側&jedna strana nečega\\\hline
		比較&uspoređivanje\\\hline
	\end{tabular}

	\fukudai{たび}
	
	
	Aが起きる時、いつもBが起きることを表す
	
	\begin{reibun}
	\rei{食事の\underline{たび}に、手を洗います。}{Svaki put kad idem jesti, operem ruke.}
	\rei{うちに来る\underline{たび}に、何か持っていってしまう。}{Svaki put kad dođem kući, nešto odnesem.}
	\end{reibun}

	\fukudai{かわり}
	

	\begin{reibun}
	\rei{このコインはお金の\underline{かわり}です。}{Ovaj novčić je zamjena za novac.}
	\rei{彼女の\underline{かわり}の人はいませんか?!}{Nema li nekog tko ju može zamjeniti?!}
	\rei{これの\underline{かわり}にはなにを使いましょうか。}{Što da koristimo umjesto ovoga?}
	\end{reibun}	
	
	\begin{reibun}
	\rei{数学を教えてもらう\underline{かわり}に英語の勉強を手伝いました。}{}
	\end{reibun}

\textit{U zamjenu za instrukcije iz matematike, pomogao sam u učenju engleskog.}

\begin{reibun}
	\rei{手伝ってもいいけど、その\underline{かわり}に何をしてくれる?}{}
	\end{reibun}

\textit{Mogu ti pomoći, ali što ćeš mi napraviti zauzvrat?}

\end{document}