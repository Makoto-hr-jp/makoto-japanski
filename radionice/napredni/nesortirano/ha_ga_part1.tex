% !TeX document-id = {0c1c4a5c-e31a-46dd-900d-a126304b4a2f}
% !TeX program = xelatex ?me -synctex=0 -interaction=nonstopmode -aux-directory=../tex_aux -output-directory=./release
% !TeX program = xelatex

\documentclass[12pt]{article}

\usepackage{lineno,changepage,lipsum}
\usepackage[colorlinks=true,urlcolor=blue]{hyperref}
\usepackage{fontspec}
\usepackage{xeCJK}
\usepackage{tabularx}
\setCJKfamilyfont{chanto}{AozoraMinchoRegular.ttf}
\setCJKfamilyfont{tegaki}{Mushin.otf}
\usepackage[CJK,overlap]{ruby}
\usepackage{hhline}
\usepackage{multirow,array,amssymb}
\usepackage[croatian]{babel}
\usepackage{soul}
\usepackage[usenames, dvipsnames]{color}
\usepackage{wrapfig,booktabs}
\renewcommand{\rubysep}{0.1ex}
\renewcommand{\rubysize}{0.75}
\usepackage[margin=50pt]{geometry}
\modulolinenumbers[2]

\usepackage{pifont}
\newcommand{\cmark}{\ding{51}}%
\newcommand{\xmark}{\ding{55}}%

\definecolor{faded}{RGB}{100, 100, 100}

\renewcommand{\arraystretch}{1.2}

%\ruby{}{}
%$($\href{URL}{text}$)$

\newcommand{\furigana}[2]{\ruby{#1}{#2}}
\newcommand{\tegaki}[1]{
	\CJKfamily{tegaki}\CJKnospace
	#1
	\CJKfamily{chanto}\CJKnospace
}

\newcommand{\dai}[1]{
	\vspace{20pt}
	\large
	\noindent\textbf{#1}
	\normalsize
	\vspace{20pt}
}

\newcommand{\fukudai}[1]{
	\vspace{10pt}
	\noindent\textbf{#1}
	\vspace{10pt}
}

\newenvironment{bunshou}{
	\vspace{10pt}
	\begin{adjustwidth}{1cm}{3cm}
	\begin{linenumbers}
}{
	\end{linenumbers}
	\end{adjustwidth}
}

\newenvironment{reibun}{
	\vspace{10pt}
	\begin{tabular}{l l}
}{
	\end{tabular}
	\vspace{10pt}
}
\newcommand{\rei}[2]{
	#1&\textit{#2}\\
}
\newcommand{\reinagai}[2]{
	\multicolumn{2}{l}{#1}\\
	\multicolumn{2}{l}{\hspace{10pt}\textit{#2}}\\
}

\newenvironment{mondai}[1]{
	\vspace{10pt}
	#1
	
	\begin{enumerate}
		\itemsep-5pt
	}{
	\end{enumerate}
	\vspace{10pt}
}

\newenvironment{hyou}{
	\begin{itemize}
		\itemsep-5pt
	}{
	\end{itemize}
	\vspace{10pt}
}

\date{\today}

\CJKfamily{chanto}\CJKnospace
\author{Marko Miličić, Željka Ludošan}

\begin{document}
	\dai{は&が1}
	
	\fukudai{\furigana{名詞修飾}{めいししゅうしょく}}{- Modifikacija/ukrašavanje imenica složenim rečenicama}

	\begin{reibun}
		\rei{A:これはなんですか}{Što je ovo?}
		\rei{B:時計です}{Sat.}
	\end{reibun}
	
	\begin{reibun}
		\rei{A:これはなんですか}{Što je ovo?}
		\rei{B:これ\underline{は}時計です。私\underline{は}東京で買いました。}{Ovo je sat. Kupio sam ga u Tokiju.}
		
		--->	
		\rei{B:これ\underline{は}私\underline{が}東京で買った時計です。}{Ovo je sat kojeg sam kupio u Tokiju.}
	\end{reibun}
			
		\begin{tabular}{|l|}
		\hline
		Unutar modifikacije ne može biti は\\\hline
		\end{tabular}

	\begin{mondai}{Zadaci(točni odgovori nalaze se na sljedećoj stranici):}
		\item 私はよく聞きます + これは音楽です。
		\item 私は昨日作りました + それは料理です。
		\item 毎日、彼は読みます + あれは本です。
	\end{mondai}

	\fukudai{\furigana{対比}{たいひ}}{- Komparacija/usporedba}

	\begin{reibun}
		\rei{A:Bさんはひらがなが書けますか}{Gospon B, znate li pisati hiraganu?}
		\rei{B:はい、書けます}{Da, znam.}
		\rei{A:じゃ、漢字は?}{A kanji?}
		\rei{B:書けません}{Ne znam.}
		
		--->
		\rei{B:ひらがな\underline{は}書けますが、漢字\underline{は}書けません}{Hiraganu znam pisati, ali kanji ne znam.}
	\end{reibun}
				
	\begin{mondai}{Zadaci(točni odgovori nalaze se na sljedećoj stranici):}
		\item お酒を辞めた + タバコを辞めない
		\item 昨日、風邪 + 今日、元気
		\item 
		花子:田中くんって猫__好き?
		
		田中:ん~ 犬__好き。
	\end{mondai}


	\newpage
	\fukudai{\furigana{最低限}{さいていげん}}{- Minimum}
					
		
	\begin{reibun}
	\reinagai{A:その写真って何?人がめっちゃいるんだが...}{Kakva je to slika? Fakat je dosta ljudi na njoj...}
	\reinagai{B:あっ、これはAさんが来なかったパーティーの写真ですよ。}{A, ovo je slika partija na kojeg A nije došao.}
	\reinagai{A:えっ!まじっすか・・・こんなに人が来たのか・・・20人じゃないね。30人でもないなぁ。うーん、40人\underline{は}いるよね。}{Huh?! Fakat...? Toliko ih je došlo...? Nije 20. Nije ni 30. Uhum, barem 40 ljudi ima, zar ne?}
	\reinagai{B:ん~、42人だったような・・・}{Uhum, čini mi se da je bilo 42 ljudi.}
	\end{reibun}
	
	\begin{mondai}{Zadaci(točni odgovori nalaze se na sljedećoj stranici):}
		\item 
		A:漢字__日本__生活する上で2千字__覚えてほしい。
		
		B:えっ!2千字__?!		
		\item 結婚式 + 1000人が来た
	\end{mondai}
	
	\fukudai{\furigana{目前}{もくぜん}・\furigana{話題}{わだい}(\furigana{取り立て}{とりたて})}{- Tema(skupljanje,zahtijevanje)}
	
		\begin{reibun}
		\reinagai{A:カギはどこにおいたのかな。みつからないなぁ。}{Gdje sam li ostavio ključ? Ne mogu ga naći...}
		\reinagai{A:カギ\underline{は}ここにあります。}{Ključ je ovdje.}

	\end{reibun}
	
	\begin{mondai}{Zadaci(točni odgovori nalaze se na sljedećoj stranici):}
		\item 彼__どこですか?彼__教室にいます。
		\item そこにいす__ある <=> 田中さんの部屋にいす__ある?
	\end{mondai}
	
	
	\begin{mondai}{Odgovori (名詞修飾)}
		\item これは私がよく聞く音楽です。
		\item それは私が昨日作った料理です。
		\item あれは彼が毎日読んでいる本です。
	\end{mondai}
	
	\begin{mondai}{Odgovori (対比)}
		\item お酒は辞めたがタバコは辞めない。
		\item 昨日は風邪だったが今日は元気。
		\item 
		花子:田中くんって猫が好き?
		
		田中:ん~ 犬は好き。
	\end{mondai}


	\newpage
	\fukudai{\furigana{強調}{きょうちょう}}{- Naglasak}
	
	\begin{reibun}
	\rei{A:田中さん\underline{が}来たら、行こう}{Ako dođe tanaka, idemo.}
	\rei{B:えっ、田中さん\underline{は}もう行っちゃたよ。}{Huh, Tanaka je već otišao.}
	\end{reibun}
	

	\begin{mondai}{Zadaci(točni odgovori nalaze se na sljedećoj stranici):}
		\item 
		A:田中__誰ですか?
		
		B:私__田中です。
		\item たけし__クラスで一番不細工だ。
	\end{mondai}
	
	\fukudai{\furigana{未知}{みち}・\furigana{既知}{きち}}{- Poznato/Nepoznato}
	
	\begin{reibun}
	\reinagai{昔々ある所におじいさんとおばあさん\underline{が}いました。おじいさん\underline{は}山へ、おばあさん\underline{は}川へ行きました。}{Davno, davno, na nekom mjestu, bili su djed i baka. Djed je otišao prema planini, a baka prema rijeci.}
	\end{reibun}
	
	\begin{mondai}{Zadaci(točni odgovori nalaze se na sljedećoj stranici):}
		\item 
		A:すみません、ここに財布__あったと思うんですが・・・
		
		B:どんな財布ですか?
		
		A:その財布__黒くて、長いです。		
		\item 	
		A:こっちに面白い所__あるんですよ。
		
		B:何をするんですか?
		
		A:そこ__遊んだり、食べたり、色々できるんです。
	\end{mondai}
	
	\begin{mondai}{Odgovori (最低限)}
		\item 
		A:漢字は日本で生活する上で2千字は覚えてほしい。
		
		B:えっ!2千字も?!		
		\item 結婚式に1000人は来た。
	\end{mondai}
	
	\begin{mondai}{Odgovori (目前・話題)}
		\item 彼はどこですか?彼は教室にいます。
		\item そこにいすがある <=> 田中さんの部屋にいすはある?
	\end{mondai}
	
	\newpage
	
	\begin{mondai}{Odgovori (強調)}
		\item 
		A:田中は誰ですか?
		
		B:私が田中です。
		\item たけしがクラスで一番不細工だ。
	\end{mondai}
	
	\begin{mondai}{Odgovori (未知・既知)}
		\item 
		A:すみません、ここに財布があったと思うんですが・・・
		
		B:どんな財布ですか?
		
		A:その財布は黒くて、長いです。
		
		
		\vspace{10pt}
		\begin{tabular}{|l|}
		\hline
		U ovoj rečenici i A i B znaju da postoji novčanik, te je zato は.\\\hline
		\end{tabular}
		\vspace{10pt}		
		
		
		\item 	
		A:こっちに面白い所があるんですよ。
		
		B:何をするんですか?
		
		A:そこは遊んだり、食べたり、色々できるんです。
	\end{mondai}

\end{document}