% !TeX document-id = {0c1c4a5c-e31a-46dd-900d-a126304b4a2f}
% !TeX program = xelatex ?me -synctex=0 -interaction=nonstopmode -aux-directory=../tex_aux -output-directory=./release
% !TeX program = xelatex

\documentclass[12pt]{article}

\usepackage{lineno,changepage,lipsum}
\usepackage[colorlinks=true,urlcolor=blue]{hyperref}
\usepackage{fontspec}
\usepackage{xeCJK}
\usepackage{tabularx}
\setCJKfamilyfont{chanto}{AozoraMinchoRegular.ttf}
\setCJKfamilyfont{tegaki}{Mushin.otf}
\usepackage[CJK,overlap]{ruby}
\usepackage{hhline}
\usepackage{multirow,array,amssymb}
\usepackage[croatian]{babel}
\usepackage{soul}
\usepackage[usenames, dvipsnames]{color}
\usepackage{wrapfig,booktabs}
\renewcommand{\rubysep}{0.1ex}
\renewcommand{\rubysize}{0.75}
\usepackage[margin=50pt]{geometry}
\modulolinenumbers[2]

\usepackage{pifont}
\newcommand{\cmark}{\ding{51}}%
\newcommand{\xmark}{\ding{55}}%

\definecolor{faded}{RGB}{100, 100, 100}

\renewcommand{\arraystretch}{1.2}

%\ruby{}{}
%$($\href{URL}{text}$)$

\newcommand{\furigana}[2]{\ruby{#1}{#2}}
\newcommand{\tegaki}[1]{
	\CJKfamily{tegaki}\CJKnospace
	#1
	\CJKfamily{chanto}\CJKnospace
}

\newcommand{\dai}[1]{
	\vspace{20pt}
	\large
	\noindent\textbf{#1}
	\normalsize
	\vspace{20pt}
}

\newcommand{\fukudai}[1]{
	\vspace{10pt}
	\noindent\textbf{#1}
	\vspace{10pt}
}

\newenvironment{bunshou}{
	\vspace{10pt}
	\begin{adjustwidth}{1cm}{3cm}
	\begin{linenumbers}
}{
	\end{linenumbers}
	\end{adjustwidth}
}

\newenvironment{reibun}{
	\vspace{10pt}
	\begin{tabular}{l l}
}{
	\end{tabular}
	\vspace{10pt}
}
\newcommand{\rei}[2]{
	#1&\textit{#2}\\
}
\newcommand{\reinagai}[2]{
	\multicolumn{2}{l}{#1}\\
	\multicolumn{2}{l}{\hspace{10pt}\textit{#2}}\\
}

\newenvironment{mondai}[1]{
	\vspace{10pt}
	#1
	
	\begin{enumerate}
		\itemsep-5pt
	}{
	\end{enumerate}
	\vspace{10pt}
}

\newenvironment{hyou}{
	\begin{itemize}
		\itemsep-5pt
	}{
	\end{itemize}
	\vspace{10pt}
}

\date{\today}

\CJKfamily{chanto}\CJKnospace
\author{Marko Miličić,Željka Ludošan}


\begin{document}
	\dai{Kondicionali}
	
	\fukudai{と}
	
1.\furigana{自然}{しぜん}\furigana{現象}{げんしょう}・\furigana{予想}{よそう}可能な\furigana{事柄}{ことがら}など

	\begin{reibun}
		\rei{春になる\underline{と}、花が咲く。}{Kad dođe proljeće, cvast će cvijeće.}
		\rei{お金を入れ\underline{と}、\furigana{切符}{きっぷ}が出ます。}{Ako ubaciš novac, izaći će karta.}
	\end{reibun}
	
	2.意外な出来事 ->\furigana{過去形}{かこけい}
	
	\begin{reibun}
		\rei{デパートに行く\underline{と}、休みだった。}{Išli smo dućan, ali nije radio.}
		\rei{うちへ帰る\underline{と}、友達が私を待っていた。}{Kad sam se vratio kući, dočekali su me prijatelji.}
		\rei{窓を開ける\underline{と}、\furigana{地面}{じめん}は雪で真っ白だった。}{Kada sam otvorio prozor,tlo je bilo bijelo bjelcato od snijega.}
	\end{reibun}


	3.\furigana{習慣的}{しゅうかんてき}なこと
	
	\begin{reibun}
		\rei{起きる\underline{と}、すぐ顔を洗う。}{Kad se dignem, odmah operem lice.}
	\end{reibun}
	
	\begin{tabular}{|l|}
		\hline
		2と3は「たら」にもあります\\\hline
	\end{tabular}

\vspace{10pt}
	
	-> \furigana{必然的}{ひつぜんてきな}\furigana{結果}{けっか}を表す
	
	->つまり、話し手の意いし、\furigana{判断}{はんだん}、\furigana{許可}{きょか}、意見、\furigana{命令}{めいれい}、\furigana{要求}{ようきゅう}などは表せません。

	\vspace{10pt}
	例:
	
	\begin{reibun}
		\rei{お金がない\underline{と}、何にも買えません。OK!}{Ako nemaš novaca, ne možeš ništa kupiti.}
		\rei{お金がない\underline{と}、働きなさい。NO!}{Ako nemaš novaca, delaj.}
	\end{reibun}
	
	\newpage		
	\fukudai{ば}
	
	1.\furigana{仮定}{かてい}
	
	\begin{reibun}
		\rei{\underline{安ければ}、買います。}{Kupit ću ako je jeftino.}
		\rei{あした\underline{晴れれば}、でかけましょう。}{Ako bude sunčano sutra, ajmo van.}
		\rei{お金が\underline{なければ}、働きなさい。}{Ako nemaš novaca, delaj.}
	\end{reibun}。

	2.必然的な結果
	
	\begin{reibun}
		\rei{\underline{春なれば}花が咲く。}{Kad dođe proljeće, cvate cvijeće.}
		\rei{お金\underline{入れれば}、切符が出ます。}{Ako ubaciš novac, izaći će karta.}
	\end{reibun}

	3.なかったことを仮定する。\furigana{後悔}{こうかい}や\furigana{残念}{ざんねん}な気持ち。->過去形
	
	\begin{reibun}
		\rei{私が鳥\underline{ならば}、あなたのところに飛んでいけたのに。}{Da sam ptica, doletio bih do tebe.}
		\rei{お金が\underline{あれば}、旅行にいけたのに。}{Da imam novaca, išao bi putovati.}
	\end{reibun}

	4.過去の習慣
	
	\begin{reibun}
		\reinagai{子供の頃、天気が\underline{良ければ}、よく外で遊んだ。}{U djetinstvu, kada je bilo lijepo vrijeme sam se često igrao vani.}
	\end{reibun}

	\begin{tabular}{|l|}
		\hline
		「ば」を使った文では、話し手は\furigana{逆}{ぎゃく}の\furigana{状況}{じょうきょう}を頭において話していることが多い。\\\hline
	\end{tabular}

	\vspace{10pt}
	例:
	
	\begin{reibun}
		\reinagai{雨が\underline{降れば}、ピクニックに行きません。(でも、晴れれば、行きます。)}{Ako će padati kiša, neću ići na piknik. (Ali, ako će biti sunčano, onda ću ići.)}
		\reinagai{もっと練習\underline{すれば}、日本語が上手になります。(しなければ、なりません。)}{Ako ću više vježbati, bit ću bolji u japanskom.(Ako neću više vježbati, neću biti bolji.)}
	\end{reibun}
	
\end{document}