% !TeX document-id = {0c1c4a5c-e31a-46dd-900d-a126304b4a2f}
% !TeX program = xelatex ?me -synctex=0 -interaction=nonstopmode -aux-directory=../tex_aux -output-directory=./release
% !TeX program = xelatex

\documentclass[12pt]{article}

\usepackage{lineno,changepage,lipsum}
\usepackage[colorlinks=true,urlcolor=blue]{hyperref}
\usepackage{fontspec}
\usepackage{xeCJK}
\usepackage{tabularx}
\setCJKfamilyfont{chanto}{AozoraMinchoRegular.ttf}
\setCJKfamilyfont{tegaki}{Mushin.otf}
\usepackage[CJK,overlap]{ruby}
\usepackage{hhline}
\usepackage{multirow,array,amssymb}
\usepackage[croatian]{babel}
\usepackage{soul}
\usepackage[usenames, dvipsnames]{color}
\usepackage{wrapfig,booktabs}
\renewcommand{\rubysep}{0.1ex}
\renewcommand{\rubysize}{0.75}
\usepackage[margin=50pt]{geometry}
\modulolinenumbers[2]

\usepackage{pifont}
\newcommand{\cmark}{\ding{51}}%
\newcommand{\xmark}{\ding{55}}%

\definecolor{faded}{RGB}{100, 100, 100}

\renewcommand{\arraystretch}{1.2}

%\ruby{}{}
%$($\href{URL}{text}$)$

\newcommand{\furigana}[2]{\ruby{#1}{#2}}
\newcommand{\tegaki}[1]{
	\CJKfamily{tegaki}\CJKnospace
	#1
	\CJKfamily{chanto}\CJKnospace
}

\newcommand{\dai}[1]{
	\vspace{20pt}
	\large
	\noindent\textbf{#1}
	\normalsize
	\vspace{20pt}
}

\newcommand{\fukudai}[1]{
	\vspace{10pt}
	\noindent\textbf{#1}
	\vspace{10pt}
}

\newenvironment{bunshou}{
	\vspace{10pt}
	\begin{adjustwidth}{1cm}{3cm}
	\begin{linenumbers}
}{
	\end{linenumbers}
	\end{adjustwidth}
}

\newenvironment{reibun}{
	\vspace{10pt}
	\begin{tabular}{l l}
}{
	\end{tabular}
	\vspace{10pt}
}
\newcommand{\rei}[2]{
	#1&\textit{#2}\\
}
\newcommand{\reinagai}[2]{
	\multicolumn{2}{l}{#1}\\
	\multicolumn{2}{l}{\hspace{10pt}\textit{#2}}\\
}

\newenvironment{mondai}[1]{
	\vspace{10pt}
	#1
	
	\begin{enumerate}
		\itemsep-5pt
	}{
	\end{enumerate}
	\vspace{10pt}
}

\newenvironment{hyou}{
	\begin{itemize}
		\itemsep-5pt
	}{
	\end{itemize}
	\vspace{10pt}
}

\date{\today}

\CJKfamily{chanto}\CJKnospace
\author{Marko Miličić,Željka Ludošan}


\begin{document}
	\dai{複合格助詞2}
	
	\fukudai{に関して・について}
	
	\furigana{関係性}{かんけいせい}
	
	\fukudai{に関して}

	\begin{reibun}
		\rei{今回の件けんに関しては以下のメールアドにお問い合わせください。}{Može se istražiti pomoću ankete.}
		\rei{「日本での生活」に関するアンケートをするつもりだ。}{Može se istražiti pomoću ankete.}
		\rei{入場にゅうじょうする方は、「入場に関しての注意事項ちゅいじこう」をよくお読み下さい}{Može se istražiti pomoću ankete.}
		\rei{今日は、パスワードの扱い方あつかいかたに関して、担当者たんとうしゃから説明があります。}{Može se istražiti pomoću ankete.}
	\end{reibun}
	
	\fukudai{について}
	
	\begin{reibun}
		\rei{昨日は、妻と今後の人生について話し合いました。}{Kuća se razrušila zbog tajfuna.}
		\rei{作文は先週の旅行について書きました。}{Kuća se razrušila zbog tajfuna.}
		\rei{ウェブサイトには、その会社についての説明が書いてあります。}{Kuća se razrušila zbog tajfuna.}
	\end{reibun}
	
	入院一泊一日について6000円の保険金ほけんきんがつく
	つく-vuči sa sobom
	について možemo ovdje zamjeniti i sa に対して
	Ne možemo koristiti ovdje に関して.
	
	に関しての=に関する, ali についての$\neq$につく
	その事件に関しての/関する/についての連絡, a ne につく
	
	\fukudai{にそって}
	意味:
	から
	にしたがって
	
	\begin{reibun}
		\rei{川にそって200メートルぐらい歩くと、右に灰色はいいろの建物が見えます。}{Način razmišljanja je drugačiji od osobe do osobe.}
		\rei{黄色きいろの線せんにそって、並んで下さい。}{Način razmišljanja je drugačiji od osobe do osobe.}
		\rei{今日はお配くばりしている資料しりょうの内容ないようにそって、発表はっぴょういたします。}{Način razmišljanja je drugačiji od osobe do osobe.}
	\end{reibun}
	
	\fukudai{とともに}
	
	意味:
	と一緒に
	と同時に
	
	\begin{reibun}
		\rei{人間は水とともに生きていると言えるだろう。}{Način razmišljanja je drugačiji od osobe do osobe.}
		\rei{ゲームでは、仲間なかまとともに敵てきを倒たおしていくのが面白い。}{Način razmišljanja je drugačiji od osobe do osobe.}
		\rei{ツイッターでは情報じょうほうを発信はっしんするとともに、様々さまざまな人と意見交換いけんこうかんをしている。}{Način razmišljanja je drugačiji od osobe do osobe.}
		\rei{このボタンを押すと、パソコンのデータが保存ほぞんされるとともに、電源でんげんが落ちます。}{Način razmišljanja je drugačiji od osobe do osobe.}
	\end{reibun}
	
	\fukudai{に基づいてもと}
	
	\furigana{発生}{はっせい}、基礎きそ
	を基礎にして、を根拠こんきょに
	
	\begin{reibun}
		\reinagai{このドラマは事実に基づいて、作られたものです。}{Ova haljina je dizajnirana od strane poznatog dizajnera.}
		\reinagai{成果せいかに基づいて、適切な給与きゅうよを支給しきゅうしている。}{Ova haljina je dizajnirana od strane poznatog dizajnera.}
		\reinagai{音読みは中国語の発音に基づいて作られました。}{Ova haljina je dizajnirana od strane poznatog dizajnera.}
	\end{reibun}
			
	\fukudai{にとって}
	
	\furigana{立場}{たちば}・\furigana{視点}{してん}
	
	
	意味:
	
	〜の立場に立って言えば
	
	〜の視点から言えば
		
	〜の立場から見れば


	\begin{reibun}
		\rei{私\underline{にとって}うちの犬は家族です。}{Što se mene tiče, moj pas je član obitelji.}
		\rei{彼\underline{にとって}、\furigana{文法}{ぶんぽう}は簡単すぎます。}{Prema njemu, ova gramatika je prelagana.}
		\rei{この絵は何をかいたのですか。}{Što si nacrtao na ovoj slici?}
		\reinagai{日本社会\underline{にとって}、\furigana{少子化}{しょうしか}は大きな問題だ。}{Prema japanskom društvu, pad nataliteta je veliki problem.}
		
	\end{reibun}
					

		私にとって\furigana{賛成}{さんせい}です。-> se ne kaže. 
		Umjesto toga se kaže: 私は賛成です。
		
		〜にとっては kada želimo naglasiti kontrast u usporedi(\furigana{強調}{きょうちょう} 、\furigana{対比}{たいひ}), u kolokvijalnom (\furigana{口語形}{こうごけい}) se može reći にとっちゃ


	\newpage
	\fukudai{として}
	
	意味:
	
	 〜という立場・\furigana{状態}{じょうたい}・\furigana{身分}{みぶん}で
					
		
	\begin{reibun}
	\reinagai{彼は生まれはアメリカだが、日本人\underline{として}\furigana{試合}{しあい}に出た。}{On je rođen u Americi, ali je izašao igrati utakmicu kao Japanac.}
	\reinagai{一度客\underline{として}この店に来たことがあるが、働くことになるとは思わなかった。
}{U ovom dućanu sam jednom bio kao kupac, ali nisam ni slutio da ću na kraju raditi ovdje.}
\rei{読み物\underline{として}も面白い。}{Zanimljivo je čak i kao štivo.}
\reinagai{社会人\underline{として}は休日よりも\furigana{残業}{ざんぎょう}{手当}{てあて}の方がありがたい。}{Kao čovjek koji radi, zadovoljniji sam s kompenzacijom za prekovremeni rad više nego sa slobodnim danima.}
	\end{reibun}
	
	\fukudai{に対して・に対する}
	
	\furigana{対象}{たいしょう}

	
	\begin{reibun}
	\rei{ご質問\underline{に対し}、できる限り\furigana{回答}{かいとう}させていただきます。}{Odgovoriti ću Vam na koliko god pitanja mogu.}
	
	\rei{\furigana{一般人}{いっぱんじん}がオタク\underline{に対して}もつ感情}{Osjećaj koji obični ljudi imau prema otakuima}
	\end{reibun}

	対比
	
	\begin{reibun}
	\reinagai{\furigana{歌舞伎町}{かぶきちょう}は夜は人がたくさんいるの\underline{に対して}、昼はがらがらだ。}{U Kabukichou-u po noći ima mnogo ljudi, u usporedbi s time, po danu je prazno.}
	\end{reibun}
	
	\begin{reibun}
	\reinagai{私の\furigana{報酬}{ほうしゅう}が30万なの\underline{に対して}、あいつは100万ももらったらしい。}{U usporedbi s mojom kompenzacijom koja je 300 000, on je izgleda dobio čak 1 000 000.}
	\end{reibun}
	
\newpage

	\fukudai{に対して・に対する}	
	
	〜に対するN 
	
	\begin{reibun}
	\reinagai{Brexit\underline{に対する}意見は人によって\furigana{異}{こと}なる。}{Mišljenje ljudi u vezi Brexita se razlikuje od čovjeka do čovjeka.}
	\reinagai{俺の彼女\underline{に対する}お前の\furigana{態度}{たいど}はどうも気に食わん。}{Ne mogu smisliti tvoje ponašanje prema mojoj curi.}
	\end{reibun}

\end{document}