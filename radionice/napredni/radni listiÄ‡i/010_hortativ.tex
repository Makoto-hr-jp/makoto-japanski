% !TeX program = xelatex

\documentclass[12pt]{article}

\usepackage[colorlinks=true,urlcolor=blue]{hyperref}
\usepackage{xeCJK}
\usepackage{tabularx}
\setCJKmainfont{AozoraMinchoRegular.ttf}
\usepackage[CJK,overlap]{ruby}
\usepackage{hhline}
\usepackage{multirow,array,amssymb}
\usepackage[croatian]{babel}
\usepackage{soul}
\renewcommand{\rubysep}{0.1ex}
\renewcommand{\rubysize}{0.55}
\usepackage[margin=0.7in]{geometry}
\pagenumbering{gobble}
\author{Tomislav Mamić}
\title{Inicijalna provjera}

%\ruby{}{}
%$($\href{URL}{text}$)$

\begin{document}
	\Large Hortativ (~よう)
	
	\large
	\vspace{20pt}
	\begin{tabular}{r l|r l|r l}
		\multicolumn{2}{c}{Nepravilni}&\multicolumn{2}{c}{一段}&\multicolumn{2}{c}{五段}\\
		\hline
		いく&いこう&\multirow{4}{20pt}{-る}&\multirow{4}{25pt}{-よう}&\multirow{4}{20pt}{-u}&\multirow{4}{30pt}{-oう}\\
		くる&\textbf{こよう}&&&\\
		する&しよう&&&\\
		ある&\textbf{あろう}&&&\\
	\end{tabular}

	\vspace{20pt}
	\large \textbf{Poziv/prijedlog}
	\vspace{10pt}
	
	\ruby{明日}{あした}、\ruby{映画}{えいが}を\ruby{見}{み}にいきましょう。
	
	\ruby{今日}{きょう}は\ruby{食堂}{しょくどう}で\ruby{食}{た}べよう。
	
	\ruby{図書館}{としょかん}で\ruby{一緒}{いっしょ}に\ruby{勉強}{べんきょう}しよう。
	
	\vspace{10pt}
	\textbf{Namjera (izravno)}
	\vspace{10pt}
	
	今日から\ruby{日記}{にっき}を\ruby{書}{か}こう。
	
	\ruby{家}{いえ}に\ruby{帰}{かえ}ったら\ruby{掃除}{そうじ}しよう。
	
	\vspace{10pt}
	\textbf{Razmatranje mogućnosti/namjera (manje izravno) s と思う}
	\vspace{10pt}
	
	\ruby{船}{ふね}を\ruby{買}{か}おうと\ruby{思}{おも}っています。
	
	\vspace{10pt}
	\textbf{Pokušaj/samo što nije s とする}
	\vspace{10pt}
	
	図書館で勉強しようとしたけど、\ruby{休}{やす}みだった。
	
	\ruby{彼}{は}\ruby{逃}{に}げようとした。
	
	\ruby{九月}{くがつ}が\ruby{始}{はじ}まろうとしていた。
	
	\ruby{西}{にし}の\ruby{空}{そら}に\ruby{日}{ひ}が\ruby{沈}{しず}もうとしています。
\end{document}