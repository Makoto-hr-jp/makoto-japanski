% !TeX document-id = {e261eb55-14f0-4659-9c75-8140cbcc1ed6}
% !TeX program = xelatex ?me -synctex=0 -interaction=nonstopmode -aux-directory=../tex_aux -output-directory=./release
% !TeX program = xelatex

\documentclass[12pt]{article}

\usepackage{lineno,changepage,lipsum}
\usepackage[colorlinks=true,urlcolor=blue]{hyperref}
\usepackage{fontspec}
\usepackage{xeCJK}
\usepackage{tabularx}
\setCJKfamilyfont{chanto}{AozoraMinchoRegular.ttf}
\setCJKfamilyfont{tegaki}{Mushin.otf}
\usepackage[CJK,overlap]{ruby}
\usepackage{hhline}
\usepackage{multirow,array,amssymb}
\usepackage[croatian]{babel}
\usepackage{soul}
\usepackage[usenames, dvipsnames]{color}
\usepackage{wrapfig,booktabs}
\renewcommand{\rubysep}{0.1ex}
\renewcommand{\rubysize}{0.75}
\usepackage[margin=50pt]{geometry}
\modulolinenumbers[2]

\usepackage{pifont}
\newcommand{\cmark}{\ding{51}}%
\newcommand{\xmark}{\ding{55}}%

\definecolor{faded}{RGB}{100, 100, 100}

\renewcommand{\arraystretch}{1.2}

%\ruby{}{}
%$($\href{URL}{text}$)$

\newcommand{\furigana}[2]{\ruby{#1}{#2}}
\newcommand{\tegaki}[1]{
	\CJKfamily{tegaki}\CJKnospace
	#1
	\CJKfamily{chanto}\CJKnospace
}

\newcommand{\dai}[1]{
	\vspace{20pt}
	\large
	\noindent\textbf{#1}
	\normalsize
	\vspace{20pt}
}

\newcommand{\fukudai}[1]{
	\vspace{10pt}
	\noindent\textbf{#1}
	\vspace{10pt}
}

\newenvironment{bunshou}{
	\vspace{10pt}
	\begin{adjustwidth}{1cm}{3cm}
	\begin{linenumbers}
}{
	\end{linenumbers}
	\end{adjustwidth}
}

\newenvironment{reibun}{
	\vspace{10pt}
	\begin{tabular}{l l}
}{
	\end{tabular}
	\vspace{10pt}
}
\newcommand{\rei}[2]{
	#1&\textit{#2}\\
}
\newcommand{\reinagai}[2]{
	\multicolumn{2}{l}{#1}\\
	\multicolumn{2}{l}{\hspace{10pt}\textit{#2}}\\
}

\newenvironment{mondai}[1]{
	\vspace{10pt}
	#1
	
	\begin{enumerate}
		\itemsep-5pt
	}{
	\end{enumerate}
	\vspace{10pt}
}

\newenvironment{hyou}{
	\begin{itemize}
		\itemsep-5pt
	}{
	\end{itemize}
	\vspace{10pt}
}

\date{\today}

\CJKfamily{chanto}\CJKnospace
\author{Marko Miličić, Tomislav Mamić}
\begin{document}
	\dai{Potpuno prijateljska anketa s točnim i netočnim odgovorima}
	
	...Nema razloga za strah, he he... he...
	
	\begin{mondai}{1. Kako sljedeći izrazi glase na japanskom?}
		\item \textit{moja mačka}
		\item \textit{ono drvo}
		\item \textit{taj kolač}
		\item \textit{prijatelj moje mlađe sestre}
		\item \textit{rep ovog psa}
	\end{mondai}

	\begin{mondai}{2. Kako bi ljudi s lijeva oslovili one s desna?}
		\item たけし(10さい)$\rightarrow$たなか(36さい、がっこうの せんせい)
		\item たなか(36さい)$\rightarrow$はなこ(11さい)
		\item はなこ(11さい)$\rightarrow$たけし(10さい)
		
	\end{mondai}

	\begin{mondai}{3. Dovršite sljedeće rečenice tako da odgovaraju prijevodu u nastavku. Parne rečenice neka budu kolokvijalne, a neparne pristojne.}
		\item あれは わたしの ぬいぐるみ(1)。\textit{Ono je moj plišanac.}
		\item リスの しっぽは ふわふわ(2)。\textit{Vjeveričin rep je bio kitnjast.}
		\item あの人は たけしくん(3)。\textit{Ono nije Takeši.}
		\item それは わたしの ともだちの じてんしゃ(4)。\textit{To nije bio bicikl mog prijatelja.}
	\end{mondai}

	\begin{mondai}{4. Dopunite rečenice tako da odgovaraju prijevodu u nastavku. Parne rečenice neka budu kolokvijalne, a neparne pristojne.}
		\item (1)くんの ねこは (2)。\textit{Takešijeva mačka je slatka.}
		\item たなかさんは (3)が (4)。\textit{Tanaka nije visok.}
		\item きのう、(5)を みた。\textit{Jučer sam vidio plavu pticu.}
		\item なくこは (6)。\textit{Djeca koja plaču nisu slatka.}
		\item あの (7)ケーキは(8)。\textit{Ona velika torta je bila slatka.}
	\end{mondai}

	\begin{mondai}{5. Dopunite prijevode tako da odgovaraju japanskom s lijeve strane.}
		\item くろい ねこを みた。\textit{(1) crnu mačku.}
		\item わるいこは にんじんを たべない。\textit{(2) djeca (3) mrkve.}
		\item あなの なかに くまが あった。\textit{U rupi (4).}
		\item たけしくんは きょうも がっこうへ こなかった。\textit{Takeši (5) nije (6) u (7).}
		\item わたしの とりは ぜんぜん うたわない。\textit{Moja ptica (8).}
		\item けんたろうくんは すうがくが わからない。\textit{Kentar\={o} (9) matematiku.}
	\end{mondai}
	
	\newpage
	\begin{mondai}{6. Ubacite čestice koje nedostaju tako da bi japanski odgovarao prijevodu u nastavku.}
		\item ゆか(1)ねた。\textit{Legao sam na pod.}
		\item ゆか(2)ねる。\textit{Spavat ću na podu.}
		\item こどもたちは こうえん(3) ボール(4) あそんだ。\textit{Djeca su se u parku igrala loptom.}
		\item えき(5) はしった。\textit{Trčao sam do stanice.}
		\item ねこは はこの中(6) とびだした。\textit{Mačka je iskočila iz kutije.}
		\item いえ(7) がっこう(8) あるく。\textit{Hodat ću od kuće do škole.}
	\end{mondai}

	\begin{mondai}{7. Izrazite količine tako da bi rečenice odgovarale prijevodima u nastavku. Količinu obavezno napišite hiraganom (ne brojem i kanji znakom) radi glasovnih promjena.}
		\item きんじょの へんな おばさんは (1) かっている。\textit{Čudna teta iz susjedstva ima 6 mačaka.}
		\item あの (2)は 何かを たくらんでいる。\textit{Ono dvoje nešto smišljaju.}
		\item あれを (3) ください。\textit{Dajte mi tri komada onog.}
	\end{mondai}

	\begin{mondai}{8. Na japanskom (punim rečenicama) odgovorite na sljedeća pitanja.}
		\item U koliko sati ste jutros ustali?
		\item Od kad do kad traju ove radionice?
		\item Prije koliko godina ste počeli učiti japanski?
	\end{mondai}

	\begin{mondai}{9. Upristojite sljedeće rečenice.}
		\item おれはナルトだってばよ。
		\item 大きい魚は小さい魚を食べた。
		\item RHCPのしんきょく、聞いた?
		\item たけしは へやを そうじした。
	\end{mondai}
	
	\begin{mondai}{10. Dopunite rečenice tako da odgovaraju prijevodima u japanskom. Parne rečenice neka budu kolokvijalne, a neparne pristojne.}
		\item この にくは (1)。\textit{Ovo meso je teško jesti.}
		\item すずきせんぱいは (2)人じゃない。\textit{Suzuki nije osoba s kojom je lako razgovarati.}
		\item (3)ケーキは 売り切れてしまいました。\textit{Torta koju sam htio pojesti se rasprodala.}
		\item おかしを (4)、おなかを こわした。\textit{Pojeo sam previše slatkiša pa mi je pozlilo.}
		\item 子供たちは (5)。\textit{Djeca su zapjevala.}
	\end{mondai}

	\begin{mondai}{11. Poredajte i spojite jednostavne rečenice u jednu dužu smislenu rečenicu.}
		\item 早くおきた。がっこにいった。いえをでた。あさごはんをたべた。
		\item 「たすけて」とさけんだ。はいきょに入った。へんな音を聞いた。ゆうれいだと思った。
	\end{mondai}

	\begin{mondai}{12. Prevedite izraze na japanski.}
		\item \textit{predug i dosadan govor} (スピーチ)
		\item \textit{čudna i tiha osoba}
		\item \textit{brzi crveni auto}
	\end{mondai}

	\newpage
	\begin{mondai}{13. Prevedite na hrvatski.}
		\item 音楽を聞いている。
		\item わさびを食べてみたい。
		\item ひまだから へやを そうじしてあげる。
		\item 先生が せつめいしてくれました。
		\item 友達にパンを買ってもらった。
	\end{mondai}

	\begin{mondai}{14. つぎの せつぞくし (veznik) を つかって、じゆうに ぶんを かきなさい。}
		\item \makebox[180pt]{\hrulefill}ながら、\makebox[180pt]{\hrulefill}。
		\vspace{5pt}
		\item \makebox[180pt]{\hrulefill}。でも、\makebox[180pt]{\hrulefill}。
		\vspace{5pt}
		\item \makebox[180pt]{\hrulefill}けど、\makebox[180pt]{\hrulefill}。
		\vspace{5pt}
		\item \makebox[180pt]{\hrulefill}。それから\makebox[180pt]{\hrulefill}。
		\vspace{5pt}
		\item \makebox[180pt]{\hrulefill}。それとも\makebox[180pt]{\hrulefill}。
		\vspace{5pt}
		\item \makebox[180pt]{\hrulefill}。だから\makebox[180pt]{\hrulefill}。
	\end{mondai}

	\begin{mondai}{15. ( )の中のどうしを てきとうな かたちに して、下の[ ]から せつぞくしを えらんで つかいなさい。
			
		[のに、けど、ので、てから]}
		\item このパソコンを なおすと(1)(いう)、やるきが でなくて なおそうとも しない。
		\item くすりを たくさん(2)(のむ)、たいちょうが いまだに よくない。
		\item まずは、じぶんで(3)(かんがえる)、先生に そうだん しましょう。
		\item 今雨が(4)(ふる)、じてんしゃで いくのが だるい。
	\end{mondai}

	\begin{mondai}{16. 下記の文に、「こと」か「の」、てきせつな ものを いれなさい。}
		\item さいごに食べた(1)は、二日前だよ。はらがへってて死にそう。
		\item 私は かえらないで、日本で はたらく(2)にしました。
		\item もちを つくる(3)に、かなりの 力が ひつようだよ。
		\item 目が わるくなったので、何かを よむときには めがねを かける(4)になった。
		\item 「たちいりきんし」という(5)は、「入ってはいけない」という(6)だ。
		\item ぼくの いもうとは バイオリンを ひく(7)ができる。
	\end{mondai}

	\begin{mondai}{17. 下記の中の ことばを つかい、じゆうに 文を 書きなさい。}
		\item とき
		\item とたん
		\item ほど
		\item ぐらい
		\item ころ
	\end{mondai}

	\newpage
	\noindent18. どうしを てきとうな かたちに しなさい。
	
	\vspace{5pt}
	\begin{tabular}{r p{100pt} p{100pt} p{100pt}}
		& どうし & 命令形 & \textasciitilde なさい\\\hline
		pr. & たべる & たべろ & たべなさい\\\hline
		1. & のぞく &  & \\\hline
		2. & かく &  & \\\hline
		3. & しゅうちゅうする &  & \\\hline
		4. & かける &  & \\\hline
		5. & 来る &  & \\\hline
		6. & すう &  & \\
	\end{tabular}

	\begin{mondai}{19. 下記の文に「思う」か「する」を てきとうな かたちで 入れなさい。}
		\item お母さんは今でかけている。スーパーに行ったと(1)。
		\item あの人は何をしようと(2)のやら、おれには分からない。
		\item 時々は おばちゃんの はかまいりに いこうとも(3)か。
		\item けいさつは いまだに はんにんを つかまえようと(4)。
		\item 私の せいに しようと(5)!かんぜんに あなたが わるいんだから。
	\end{mondai}

	\begin{mondai}{20. 下記の文から じゅどうたい(pasiv)を作りなさい。}
		\item ももこは たろうを なぐった。
		\item お母さんは 私を しかった。
		\item 犬は きゃくを かんだ。
		\item あなたを いけにえに する。
	\end{mondai}

	\begin{mondai}{21. 下記の たんごから しえきぶん(kauzativ)を作りなさい。}
		\item ぎんこういん、はんにん、おかね、よういする
		\item こども、ミルク、のむ
		\item コーチ、せんしゅ、はしる
	\end{mondai}
\end{document}