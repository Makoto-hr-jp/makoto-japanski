% !TeX program = xelatex

\documentclass[12pt]{article}

\usepackage[colorlinks=true,urlcolor=blue]{hyperref}
\usepackage{xeCJK}
\usepackage{tabularx}
\setCJKmainfont{AozoraMinchoRegular.ttf}
\usepackage[CJK,overlap]{ruby}
\usepackage{hhline}
\usepackage{multirow,array,amssymb}
\usepackage[croatian]{babel}
\usepackage{soul}
\usepackage[normalem]{ulem}
\renewcommand{\rubysep}{0.1ex}
\renewcommand{\rubysize}{0.55}
\author{Tomislav Mamić}
\title{Inicijalna provjera}

%\ruby{}{}
%$($\href{URL}{text}$)$

\begin{document}

	\subsection{Imenski predikat}
	
	\paragraph{1.} Povežite rečenice i sintagme s njihovim točnim prijevodima.
	
	\begin{tabularx}{\textwidth}{X X}
		1 Ono je mačka.&3 あれは ねこでした。\\
		2 Ono nije bila mačka.&5 これは ねこです。\\
		3 Ono je bila mačka.&4 これは ねこじゃない。\\
		4 Ovo nije mačka.&3 あれは ねこじゃなかった。\\
		5 Ovo je mačka.&1 あれは ねこだ。\\
	\end{tabularx}

	\subsection{Pridjevi}

	\paragraph{2.} Povežite rečenice i sintagme s njihovim točnim prijevodima.
	
	\begin{tabularx}{\textwidth}{X X}
		1 Ovaj auto je crven.&2 これは あかい くるまじゃない。\\
		2 Ovo nije crveni auto.&4 これは あかくなかった くるまだ。\\
		3 Ovo je bio crveni auto.&3 これは あかい くるまだった。\\
		4 Ovo je auto koji nije bio crven.&5 あかくない くるま\\
		5 auto koji nije crven&1 この くるまは あかいです。\\
		6 auto koji je bio crven&6 あかくなかった くるま\\
	\end{tabularx}

	\paragraph{3.} Posložite pridjeve u zadane sintagme na hrvatskom.
	
	\begin{tabularx}{\textwidth}{X X}
		Pridjevi:&Sintagme:\\
		きれいで あおい とり&lijepa plava ptica\\
		あおくて きれいな とり&plava lijepa ptica\\
		あかくて きらいな くだもの&crveno voće koje mrzim\\
		あおくて おいしくない くだもの&nezrelo voće koje nije fino\\
		きれいで あかくて おいしい りんご&lijepa, crvena, fina jabuka\\
		しずかで あおい うみ&tiho plavo more\\
	\end{tabularx}

	\subsection{Glagoli}
	
	\paragraph{4.} Povežite rečenice i točne prijevode.
	
	\begin{tabularx}{\textwidth}{X X}
		1 りんごを たべます。&4 (Tu) je mačka.\\
		2 とりを みた。&3 Vidjela se ptica.\\
		3 とりが みえました。&1 Pojest ću jabuku.\\
		4 ねこが いる。&2 Vidio sam pticu.\\
		5 ねこが いません。&5 Nema mačke.\\
	\end{tabularx}
	
	\newpage
	\paragraph{5.} Dopunite tablicu glagolskih oblika.
	
	* S obzirom da je glagol zadan u kolokvijalnom obliku, bilo bi ga dobro takvog i koristiti. Pristojni oblik je koristan, ali nije izvorni oblik glagola i kao takav nam ne pomaže u razumijevanju cjelokupnog sustava glagolskih oblika u japanskom jeziku. Zato se trudimo što bolje naučiti kolokvijalne oblike.
	
	\begin{tabularx}{\textwidth}{X X X X}
		Glagol:&Prošlost:&Negacija:&い oblik:\\
		かく&かいた&かかない&かき\\
		およぐ&およいだ&およがない&およぎ\\
		ころす&ころした&ころさない&ころし\\
		しぬ&しんだ&しなない&しに\\
		よむ&よんだ&よまない&よみ\\
		えらぶ&えらんだ&えらばない&えらび\\
		うたう&うたった&うたわない&うたい\\
		たつ&たった&たたない&たち\\
		はしる&はしった&はしらない&はしり\\
	\end{tabularx}
	
	\paragraph{6.} Precrtajte netočne glagolske oblike u tablici.
	
	\begin{tabularx}{\textwidth}{X X X X}
		Glagol:&て oblik:&Negacija:&い oblik:\\
		する&して&\sout{すない}&し\\
		くる&きて&\sout{きない}&き\\
		ある&\sout{あて}&\sout{あらない}&あり\\
		いく&いって&いかない&いき\\
	\end{tabularx}

	\subsection{Lokacija}
	
	\paragraph{7.} Iz zadanih dijelova sastavite 5 suvislih rečenica. Svaki dio morate iskoristiti barem jednom.
	
	* Više točnih odgovora.
	
	\begin{tabularx}{\textwidth}{X X X}
		ねこが&木の&なかに\\
		こうえんで&はこの&うえに\\
		あそんだ。&したで&いる。\\
		こどもたちは&ねた。&いえの\\
	\end{tabularx}
	
	\subsection{Vrijeme}
	
	\paragraph{8.} Precrtajte netočne dijelove u rečenicama.
	
	\begin{tabularx}{\textwidth}{X}
		きのう\sout{に} ともだちに あった。\\
		けさ6時に おきた。\\
		しごとはあさ8時\sout{まで} ごご4時\sout{から}です。\\
		おととし23さいに なった。\\
	\end{tabularx}

	\subsection{て oblik}
	
	\paragraph{9.} Dovršite rečenice pravilnim oblikom za zadani prijevod.
	
	*Ovdje su i pristojni oblici glagola OK.
	
	\begin{tabularx}{\textwidth}{X X X}
		つくえの上に本が(1)。&おいて あった。& \textit{Na stolu je (ostavljena) knjiga.}\\
		子供たちは外で(2)。&あそんで いた。& \textit{Djeca se vani igraju.}\\
		きのう友達に(4)(5)(6).&あって かえって ねた。&\textit{Jučer sam se našao s prijateljem, vratio se kući i spavao.}\\
		しずかに(7)ください。&して&\textit{Molim te budi tiho.}\\
	\end{tabularx}

	\subsection{い oblik}
	
	\paragraph{10.} Upristojite zadane rečenice.
	
	\begin{tabularx}{\textwidth}{X X}
		おれはナルトだってばよ。&わたし/ぼくは ナルトです。\\
		なっとうを たべたことは ない。&なっとうを たべたことは ありません。\\
		しけんは あしただと せんせいが いった。&しけんは あしただと せんせいが いいました。\\
		しね。&しんで ください。\\
	\end{tabularx}

	\paragraph{11.} Povežite glagole i prijevode.
	
	\begin{tabularx}{\textwidth}{X X}
		1 はしります&4 \textit{lako za pojesti}\\
		2 はしりたい&3 \textit{potrčao (sam)}\\
		3 はしりだした&6 \textit{nastaviti jesti}\\
		4 たべやすい&5 \textit{teško za pojesti}\\
		5 たべにくい&1 \textit{trčati}\\
		6 たべつづける&2 \textit{želim trčati}\\
	\end{tabularx}

	\subsection{Napredni glagolski oblici}
	
	\paragraph{12.} Prevedite na hrvatski. U slučaju dvoznačnih rečenica prihvaćam oba odgovora.
	
	\begin{tabularx}{\textwidth}{X}
		1 どこまではしれますか?\\
		2 せんせいが わたしに 「まちなさい」と いった。\\
		3 わたしは せんせいに 「まちなさい」と いわれた。\\
		4 わたしは せんせいに 「まちなさい」と いわせた。\\
		5 せんせいは わたしに 「まちなさい」と いわせられた。\\
		6 こどものころ にんじんを たべさせられた。\\
		1 Dokle možeš trčati?\\
		2 Prof mi je rekao "čekaj".\\
		3 Rečeno mi je da čekam. (prof mi je rekao)\\
		4 Dopustio(ili natjerao) sam prof da kaže "čekaj".\\
		5 Prof je bio prisiljen(ili dobio dopuštenje) da kaže "čekaj". (od mene)\\
		6 Kad sam bio mali, bio sam prisiljen jesti mrkve.\\
	\end{tabularx}

	\subsection{Rečenice složene opisom}
	
	\paragraph{13.} U stupcima su navedeni dijelovi koje valja presložiti i česticama povezati u suvisle rečenice. Jedan stupac, jedna rečenica. Ako neke dijelove ne znate iskoristiti, izostavite ih, ali svi dijelovi se daju povezati.
	
	*Jako puno mogućih odgovora.
	
	\begin{tabularx}{\textwidth}{X X X X}
		です&ある&えいが&おなじ\\
		なまえ&あそんでいた&たけしくん&すずきさん\\
		みた&こうえん&みなかった&3ねん\\
		ねこ&うしろ&しし&かいしゃ\\
		マル&えき&うごけなかった&まえ\\
		きのう&まえ&こわすぎて&みたい\\
		&よく&&おもいます\\
		&&&はなして\\
		&&&やめた\\
	\end{tabularx}

	\paragraph{14.} Prevedite rečenice iz prethodnog zadatka na hrvatski.
	
	\subsection{Rečenice složene veznicima}
	
	\paragraph{15.} Umetnite veznik koji nedostaje da bi rečenica odgovarala prijevodu s desne strane.
	
	\begin{tabularx}{\textwidth}{X X}
		べんきょうしなかった(のに)だいじょうぶだった(けど/が)つぎはどうなる(か)わからない。&Iako nisam učio, ispalo je dobro, ali ne znam kako će biti drugi put.\\
		たけしはわたしのおもちゃをぬすんだ(から/ので)おかあさんにいいつけた(けど/が)あとで(いたいめをみた)*。&Tužio sam Takešija mami jer mi je ukrao igračku, ali sam kasnije nadrapao*.
	\end{tabularx}

\flushright おつかれさまー
\end{document}