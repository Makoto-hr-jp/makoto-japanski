% !TeX program = xelatex
% !TeX program = xelatex

\documentclass[12pt]{article}

\usepackage{lineno,changepage,lipsum}
\usepackage[colorlinks=true,urlcolor=blue]{hyperref}
\usepackage{fontspec}[ Path =../../../ ]
\usepackage{xeCJK}
\usepackage{tabularx}
\usepackage{graphicx}
\setCJKfamilyfont{chanto}{AOZORAMINCHOREGULAR_0.TTF}%
\setCJKfamilyfont{tegaki}{Mushin.otf}%
\usepackage[CJK,overlap]{ruby}
\usepackage{hhline}
\usepackage{multirow,array,amssymb}
\usepackage[croatian]{babel}
\usepackage{soul}
\usepackage[usenames, dvipsnames]{color}
\usepackage{wrapfig,booktabs}
\usepackage{calc}
\renewcommand{\rubysep}{0.1ex}
\renewcommand{\rubysize}{0.75}
\usepackage[margin=50pt]{geometry}
\usepackage{hyperref}
\modulolinenumbers[2]

\date{\today}

\usepackage{fancyhdr}
\pagestyle{fancy}
\fancyhf{}
\fancyhead[LE,RO]{\thepage}
\makeatletter
\fancyhead[RE,LO]{rev. \@date 誠}
\makeatother

\usepackage{pifont}
\newcommand{\cmark}{\ding{51}}%
\newcommand{\xmark}{\ding{55}}%

\newcommand{\dosl}{{\normalfont dosl. }}%
\newcommand{\rem}[1]{{\normalfont #1 }}%

\definecolor{faded}{RGB}{100, 100, 100}

\renewcommand{\arraystretch}{1.2}

%\ruby{}{}
%$($\href{URL}{text}$)$

\newcommand{\furigana}[2]{\ruby{#1}{#2}}
\newcommand{\tegaki}[1]{
	\CJKfamily{tegaki}\CJKnospace
	#1
	\CJKfamily{chanto}\CJKnospace
}

\newcommand{\dai}[1]{
	\vspace{20pt}
	\large
	\noindent\textbf{#1}
	\normalsize
	\vspace{20pt}
}

\newcommand{\fukudai}[1]{
	\vspace{10pt}
	\noindent\textbf{#1}
	\vspace{10pt}
}

\newenvironment{bunshou}{
	\vspace{10pt}
	\begin{adjustwidth}{1cm}{3cm}
	\begin{linenumbers}
}{
	\end{linenumbers}
	\end{adjustwidth}
}

\newenvironment{reibun}[1][]{
	\vspace{10pt}
	#1
	
	\begin{tabular}{l l}
}{
	\end{tabular}
	\vspace{10pt}
}
\newcommand{\rei}[2]{
	#1&\textit{#2}\\
}
\newcommand{\reinagai}[2]{
	\multicolumn{2}{l}{#1}\\
	\multicolumn{2}{l}{\hspace{10pt}\textit{#2}}\\
}

\newenvironment{mondai}[1]{
	\vspace{10pt}
	\noindent #1
	
	\begin{enumerate}
		\itemsep-5pt
	}{
	\end{enumerate}
}

\newenvironment{hyou}{
	\begin{itemize}
		\itemsep-5pt
	}{
	\end{itemize}
	\vspace{10pt}
}

\newcommand{\juuyou}[2][20pt]{
	\vspace{5pt}
		\noindent\hspace{#1}\parbox[c]{\textwidth-#1-#1}{\centering\textit{#2}}
	\vspace{5pt}
}

\newcommand{\ten}{
	\vspace{5pt}
	\noindent\hspace{-10pt}$\bullet$
}

\CJKfamily{chanto}\CJKnospace

\frenchspacing
\author{Tomislav Mamić}

\begin{document}
	\dai{Kondicionali}
	
	U japanskom postoje 4 različita kondicionala čije je sitnije razlike u značenju jako teško opisati. U nastavku je opisana njihova tvorba, kao i najčešće situacije u kojima se prirodno koriste. Svi se u rečenici pojavljuju u obliku
	
	\juuyou[200pt]{<uvjet><posljedica>}
	
	\fukudai{Prirodna posljedica - と kondicional}

	Tvori se dodavanjem čestice と izravno na neprošli predikatni oblik.	Implicirana je uzročno posljedična veza, što znači da se prva rečenica (unatoč tome što je uvijek u neprošlom vremenu) događa prije druge (koja može biti i u prošlosti). Također, ne implicira se nesigurnost - uvijek kad se dogodi prva rečenica, dogodi se i druga.
	
	\begin{reibun}
		\rei{四月になると、\furigana{桜}{さくら}が\furigana{咲}{さ}く。}{Kad dođe Travanj, cvate sakura.}
		\rei{\furigana{家}{いえ}に\furigana{帰}{かえ}ると\furigana{母}{はは}が\furigana{待}{ま}っていた。}{Kad sam se vratio kući, majka (me) čekala.}
		&(impl. da je to nešto uobičajeno i očekivano)\\
		\rei{\furigana{速}{はや}くしないと\furigana{遅刻}{ちこく}するよ。}{Ako ne požuriš, zakasnit ćeš.}
		\rei{4に5を\furigana{足}{た}すと、9になります。}{Kad na 4 dodaš 5, dobiješ 9.}
	\end{reibun}

	\fukudai{Hipotetski \textit{ako $\rightarrow$ onda} - ば kondicional}
	
	Tvori se iz glagola i pridjeva prema sljedećoj tablici:
	
	\vspace{10pt}
	\begin{tabular}{|l|l|l|l|l|l|l|l|l|}
		\hline
		&一段&五段&い&な&くる&する&ある&だ\\\hline
		rep&\textasciitilde る&\textasciitilde u&\textasciitilde い&\textasciitilde な&&&&\\\hline
		ば kond.&\textasciitilde れば&\textasciitilde eば&\textasciitilde ければ&\textasciitilde であれば&くれば&すれば&あれば&ならば\\\hline
	\end{tabular}
	\vspace{10pt}

	Ovaj kondicional implicira hipotetski uvjet za drugi dio rečenice. Ponekad se koristi kao općeniti kondicional, ali u savjetima i prijedlozima može zvučati agresivno i inzistirajuće, pa oprezno s tim. Često stavlja naglasak na uvjet, naglašava značenje \textbf{\textit{ako}}. Vrlo je bitno da se uvjet o kojem govorimo još nije dogodio - \textit{hipotetski} je.
	
	\begin{reibun}
		\rei{どうすれば\furigana{東大}{とうだい}\footnotemark[1]に\furigana{入学}{にゅうがく}できますか。}{Što trebam raditi da upadnem na tokijsko sveučilište?}
		\rei{\furigana{勉強}{べんきょう}すれば\furigana{合格}{ごうかく}するよ。}{Ako budeš učio, proći ćeš.}
		\reinagai{あと1000円さえあれば、このコートを買えるのに。}{Da mi je još 1000 jena, mogao bih kupiti ovaj kaput.}
	\end{reibun}

	\footnotetext[1]{東大 je skraćeno od 東京大学, slično kako u Zagrebu kažemo FER, PMF ili FFZG}
	
	Kad želimo navesti minimalne uvjete za nešto, koristimo česticu さえ kao u primjeru iznad. U tom slučaju uvijek koristimo ば kondicional, a čestica さえ lagano mijenja značenje rečenice, npr:
	
	\begin{reibun}
		\rei{勉強さえすれば、合格するよ。}{Trebaš samo učiti pa ćeš proći.}
		\rei{お金さえ\furigana{払}{はら}えば、だれでも\furigana{入会}{にゅうかい}できる。}{Bilo tko se može učlaniti ako plati.}
		\reinagai{お前さえいなければ、ロドリゴは私の物になっていたのに。}{Samo da nije bilo tebe, Rodrigo bi bio moj.}
	\end{reibun}

	U iznimno kolokvijalnom govoru, ponekad se umjesto ば govori や koji se pomalo nepredvidivo stapa s glasovima ispred, npr 言えば$\rightarrow$言いや, ili すれば$\rightarrow$すりゃ.
	
	\fukudai{Općeniti kondicional - ら}
	
	Tvori se dodavanjem ら na prošlost glagola i pridjeva, a implicira da postoji velika vjerojatnost da će uvjet biti ispunjen. U hrvatskom ga zbog toga često prevodimo koristeći \textit{kad} umjesto \textit{ako}.
	
	\begin{reibun}
		\rei{\furigana{仕事}{しごと}が\furigana{終}{お}わったら\furigana{電話}{でんわ}する。}{Nazvat ću te poslije posla.}
		\rei{日本に行けたらいいな。}{Bilo bi baš dobro kad bih mogao otići u japan.}
		\rei{雨がやんだら、外に出た。}{Kad je prestala kiša, izišao sam van.}
	\end{reibun}

	Kad dođe do neočekivane posljedice, koristimo ovaj kondicional:
	
	\begin{reibun}
		\reinagai{\furigana{窓}{まど}を\furigana{開}{あ}けたら、下でたけしくんが\furigana{大声}{おおごえ}で歌っていた。}{Otvorio sam prozor, a kad tamo ispod Takeši pjeva iz petnih žila.}
		\rei{\furigana{銀行}{ぎんこう}に行ったら\furigana{友人}{ゆうじん}に会った。}{Kad sam išao u banku, sreo sam prijatelja.}
	\end{reibun}

	Zbog fleksibilnosti, ら kondicional se najčešće koristi u govoru, a vrlo je popularan među stranim govornicima jer ga je teško zloupotrijebiti. Jer odaje dojam lijenosti, u pisanju se znatno rjeđe koristi.
	
	\fukudai{Kontekstualni kondicional - なら}
	
	Dobiva se dodavanjem čestice なら na imenicu ili predikatni oblik. Za razliku od ostalih kondicionala, ne povezuje uvjet i rezultat kao uzrok i posljedicu. Zbog toga nije bitno je li i kada uvjet ispunjen u odnosu na rezultat. U ovom slučaju, bolje je \textit{uvjet} zamijeniti za \textit{kontekst}. Ovaj kondicional kaže - ako je \textit{kontekst} ovakav, onda \textit{rezultat}. Zbog toga se jako često koristi u zamolbi i davanju savjeta.
	
	\begin{reibun}
		\rei{行くなら私も\furigana{連}{つ}れて行ってください。}{Ako ideš, povedi i mene.}
		\reinagai{雨が\furigana{降}{ふ}りそうだから、出かけるなら\furigana{傘}{かさ}を\furigana{持}{も}ってね。}{Čini se da će padati kiša pa ponesi kišobran ako ideš van.}
		\reinagai{\furigana{終電}{しゅうでん}に\furigana{間}{ま}に\furigana{合}{あ}いたいなら、\furigana{速}{はや}くした方がいいよ。}{Ako hoćeš stići na zadnji vlak, bilo bi ti bolje da požuriš.}
		\rei{それなら\furigana{話}{はなし}が\furigana{早}{はや}い。}{Ako je tako, sve smo riješili.\footnotemark[2]}
	\end{reibun}
	\footnotetext[2]{Izraz 話が早い znači \textit{nema komplikacija}, \textit{jednostavno se dogovoriti} i slično.}
\end{document}