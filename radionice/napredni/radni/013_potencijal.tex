% !TeX program = xelatex

\documentclass[12pt]{article}

\usepackage[colorlinks=true,urlcolor=blue]{hyperref}
\usepackage{xeCJK}
\usepackage{tabularx}
\setCJKmainfont{AozoraMinchoRegular.ttf}
\usepackage[CJK,overlap]{ruby}
\usepackage{hhline}
\usepackage{multirow,array,amssymb}
\usepackage[croatian]{babel}
\usepackage{soul}
\usepackage[usenames, dvipsnames]{color}
\usepackage{wrapfig,lipsum,booktabs}
\renewcommand{\rubysep}{0.1ex}
\renewcommand{\rubysize}{0.55}
\usepackage[margin=50pt]{geometry}
\author{Tomislav Mamić}
\title{Inicijalna provjera}

\definecolor{faded}{RGB}{100, 100, 100}

%\ruby{}{}
%$($\href{URL}{text}$)$

\begin{document}

	\large Potencijal
		
	\begin{wraptable}[24]{l}{0pt}
	\begin{tabular}{|l|l|}
		\hline
		\multicolumn{2}{|c|}{nepravilni}\\
		\hline
		いく&いける\\
		くる&こられる\\
		&これる\footnotemark[1]\\
		ある&ありえる\footnotemark[2]\\
		する&できる\\
		\hline
		\hline
		\multicolumn{2}{|c|}{一段}\\
		\hline
		\multirow{2}{30pt}{-る}&-られる\\
		&-れる\footnotemark[1]\\
		\hline
		\hline
		\multicolumn{2}{|c|}{五段\footnotemark[3]}\\
		\hline
		く&ける\\
		ぐ&げる\\
		す&せる\\
		\hline
		ぬ&ねる\\
		む&める\\
		ぶ&べる\\
		\hline
		う&える\\
		つ&てる\\
		る&れる\\
		\hline
	\end{tabular}
	\end{wraptable}

	\vspace{20pt}
	\normalsize \textbf{Osnovno značenje}
	\vspace{20pt}
	
	Potencijal izražava praktičnu mogućnost/sposobnost subjekta da se neki glagol izvrši ili dogodi, kao u hrvatskom jeziku pom. glagol \textit{moći} + infinitiv, npr.
	
	にんじんがたべれる。 \textit{Mogu jesti mrkve.}
	
	日本語がよめる。 \textit{Mogu čitati japanski.}
	
	Za razliku od hrvatskog \textit{moći}, potencijal u japanskom \textbf{ne izražava teoretsku mogućnost}, npr.
	
	\textit{Mogao sam umrijeti.} しねた。 \textbf{iznimno krivo!}
	
	U japanskom postoji i "potencijal za lijenčine" koji se tvori nominalizacijom s こと:
	
	いける $\approx$ いくことが できる 
	
	はしれる $\approx$ はしることが できる
	
	Potencijal glagola i sam postaje 一段 glagol. Kao takav može poprimiti vrijeme, negaciju, pristojnost i spojne oblike い i て, kao i kondicional, ali semantički nema puno smisla s potencijalom, pasivom, kauzativom, imperativom ni hortativom. Čak i ako možete zamisliti situaciju gdje biste neki od tih oblika htjeli upotrijebiti, jako je neprirodno.
	
	Bitno je uočiti da potencijal nije prijelazni glagol. To znači da se događa standardno pomicanje uloga čestica - objekt (を) postaje subjekt (が), a subjekt (が) postaje tema (は):
	
	きつねは はりねずみを たべる。$\rightarrow$きつねは はりねずみが たべれる。
	
	Tema, naravno, ostaje tema.
	
	\vspace{20pt}
	\normalsize \textbf{Neki česti izrazi}
	\vspace{20pt}
	
	<reč.>かもしれない \textit{možda <reč.>}
	
	もりの なかに いる。 \textit{U šumi su.}
	
	もりの なかに いる かもしれない。 \textit{Možda su u šumi.}
	\vspace{10pt}
	
	ありえない! \textit{Nemoguće/nema šanse!}
	\vspace{10pt}
	
	しんじられない! \textit{Ne mogu vjerovati!}
	
	\vspace{20pt}
	\normalsize \textbf{Primjeri za vježbu}
	\vspace{20pt}
	
	いちねんかん れんしゅうしても かれには かてなかった。
	
	わたしの ともだちは じぶんの へやを でることさえ できない ひきこもりに なって しまいました。
	
	あと いちじかん おきていられたら さいごまで えいがが みれた かもしれない。
	
	\footnotetext[1]{Pomalo kolokvijalno, ne koristiti kad treba biti jako pristojan ili u pisanom obliku.}
	\footnotetext[2]{U rječničkom obliku može biti i ありうる, ne dajte se prevariti.}
	\footnotetext[3]{Savršeno pravilno, hiragana う red u え + る.}
	
\end{document}