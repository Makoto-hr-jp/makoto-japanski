% !TeX document-id = {be7e20a2-a756-42c8-9697-5b39ba57beca}
% !TeX program = xelatex ?me -synctex=0 -interaction=nonstopmode -aux-directory=../tex_aux -output-directory=./release
% !TeX program = xelatex

\documentclass[12pt]{article}

\usepackage{lineno,changepage,lipsum}
\usepackage[colorlinks=true,urlcolor=blue]{hyperref}
\usepackage{fontspec}[ Path =../../../ ]
\usepackage{xeCJK}
\usepackage{tabularx}
\usepackage{graphicx}
\setCJKfamilyfont{chanto}{AOZORAMINCHOREGULAR_0.TTF}%
\setCJKfamilyfont{tegaki}{Mushin.otf}%
\usepackage[CJK,overlap]{ruby}
\usepackage{hhline}
\usepackage{multirow,array,amssymb}
\usepackage[croatian]{babel}
\usepackage{soul}
\usepackage[usenames, dvipsnames]{color}
\usepackage{wrapfig,booktabs}
\usepackage{calc}
\renewcommand{\rubysep}{0.1ex}
\renewcommand{\rubysize}{0.75}
\usepackage[margin=50pt]{geometry}
\usepackage{hyperref}
\modulolinenumbers[2]

\date{\today}

\usepackage{fancyhdr}
\pagestyle{fancy}
\fancyhf{}
\fancyhead[LE,RO]{\thepage}
\makeatletter
\fancyhead[RE,LO]{rev. \@date 誠}
\makeatother

\usepackage{pifont}
\newcommand{\cmark}{\ding{51}}%
\newcommand{\xmark}{\ding{55}}%

\newcommand{\dosl}{{\normalfont dosl. }}%
\newcommand{\rem}[1]{{\normalfont #1 }}%

\definecolor{faded}{RGB}{100, 100, 100}

\renewcommand{\arraystretch}{1.2}

%\ruby{}{}
%$($\href{URL}{text}$)$

\newcommand{\furigana}[2]{\ruby{#1}{#2}}
\newcommand{\tegaki}[1]{
	\CJKfamily{tegaki}\CJKnospace
	#1
	\CJKfamily{chanto}\CJKnospace
}

\newcommand{\dai}[1]{
	\vspace{20pt}
	\large
	\noindent\textbf{#1}
	\normalsize
	\vspace{20pt}
}

\newcommand{\fukudai}[1]{
	\vspace{10pt}
	\noindent\textbf{#1}
	\vspace{10pt}
}

\newenvironment{bunshou}{
	\vspace{10pt}
	\begin{adjustwidth}{1cm}{3cm}
	\begin{linenumbers}
}{
	\end{linenumbers}
	\end{adjustwidth}
}

\newenvironment{reibun}[1][]{
	\vspace{10pt}
	#1
	
	\begin{tabular}{l l}
}{
	\end{tabular}
	\vspace{10pt}
}
\newcommand{\rei}[2]{
	#1&\textit{#2}\\
}
\newcommand{\reinagai}[2]{
	\multicolumn{2}{l}{#1}\\
	\multicolumn{2}{l}{\hspace{10pt}\textit{#2}}\\
}

\newenvironment{mondai}[1]{
	\vspace{10pt}
	\noindent #1
	
	\begin{enumerate}
		\itemsep-5pt
	}{
	\end{enumerate}
}

\newenvironment{hyou}{
	\begin{itemize}
		\itemsep-5pt
	}{
	\end{itemize}
	\vspace{10pt}
}

\newcommand{\juuyou}[2][20pt]{
	\vspace{5pt}
		\noindent\hspace{#1}\parbox[c]{\textwidth-#1-#1}{\centering\textit{#2}}
	\vspace{5pt}
}

\newcommand{\ten}{
	\vspace{5pt}
	\noindent\hspace{-10pt}$\bullet$
}

\CJKfamily{chanto}\CJKnospace

\frenchspacing
\author{Tomislav Mamić}

\begin{document}
	\dai{Napredne upotrebe い oblika I}
	
	\fukudai{Uvod}
	
	Kao ni て oblik, い oblik u sebi ne sadrži informaciju o vremenu radnje. Za razliku od て oblika koji može nositi negaciju, い oblik je nema. U tom je pogledu い najčišći oblik glagola (ne sadrži baš nikakve dodatne informacije). Zato je vrlo čest slučaj da imenice od glagola nastaju upravo iz ovog oblika (npr. 話し、考え、思い、逆立ち...), ali koliko god to bilo često, nije gramatičko pravilo na koje se možemo potpuno osloniti.
	
	Osnovnim upotrebama い oblika smatrat ćemo pristojne predikate (\textasciitilde ます) i izražavanje\\ želje (\textasciitilde たい).
	
	\fukudai{Pomoćni glagoli}
	
	Daleko najveći dio upotrebe い oblika u divljini otpada na kombinacije s pomoćnim glagolima. Njih ima pregršt (čak se i ます vodi kao pom. gl.), a značenja im mogu biti jako nepredvidiva. Iako je poklapanje ovog mehanizma s hrvatskim dosta slabo, najbliža usporedba u hrvatskom je spajanje prijedloga i glagola (npr. \textit{pogledati, nadgledati, pregledati, ugledati, izgledati...})
	
	\begin{reibun}[\ten \furigana{始}{はじ}める - \textit{početi}\textasciitilde]
		\reinagai{勉強しはじめたとたん、となりの\furigana{部屋}{へや}から\furigana{煩}{うるさ}い\furigana{音楽}{おんがく}が\furigana{聞}{き}こえてきた。}{Baš kad sam počeo učiti, iz susjedne sobe se začula glasna muzika.}
		\rei{雨が降り始めた\footnotemark[1]。}{Počela je padati kiša.}
	\end{reibun}

	\footnotetext[1]{Iako je 始める prijelazni glagol, kad radi kao pomoćni glagol na ovaj način, koristi se i za prijelazne i neprijelazne jednako.}

	Valja primijetiti da se zanemaruje prijelaznost pomoćnog glagola - koristi ga se bez čvrste veze s njegovim originalnim značenjem.

	\begin{reibun}[\ten \furigana{終}{お}わる - \textit{završiti}\textasciitilde]
		\rei{パンを食べ終わって学校へ行った。}{Pojeo sam kruh i otišao u školu.}
		\rei{\furigana{徹夜}{てつや}をしてレポートを書き終わった。}{Ostao sam budan cijelu noć i završio s pisanjem referata.}
	\end{reibun}

	Iako je često korištena čestica を, također je uobičajeno (iako krivo) reći i が. Radi se o tome da je glavni glagol prijelazni pa je を prirodan izbor, ali pomoćni nije pa nam je instinkt prebaciti を u が. Oko ovog su i izvorni govornici pomalo neodlučni. Kako to često biva, originalno značenje i gramatički okviri pomoćnog glagola se ne prenose na cijeli izraz.
	
	\newpage
	\begin{reibun}[\ten 出す - \textit{iz\textasciitilde\ / početi \textasciitilde}]
		\reinagai{たけしくんは\furigana{血}{ち}を\furigana{吐}{は}き出してくたばってしまいました。}{Takeši je ispovraćao krv i otegnuo papcima.}
		\reinagai{彼を見た山田くんは「\furigana{逃}{に}げましょう」と\furigana{平然}{へいぜん}と言い出しました。}{Vidjevši ga, Jamada je mrtav 'ladan predložio da pobjegnemo.}
		\rei{\furigana{周}{まわ}りを見て\furigana{橋}{はし}の方へ\furigana{走}{はし}りだしました。}{Pogledali smo oko sebe i potrčali prema mostu.}
	\end{reibun}

	U prvom primjeru složeni glagol je 吐き出す - \textit{ispovraćati/ispljunuti}. Ovdje je osnovno značenje pom. gl. 出す dobro sačuvano. U drugom primjeru pojavljuje se složeni glagol 言い出す - \textit{predložiti}. Postoje kombinacije glagola kao ova, kojima ukupno značenje nema gotovo nikakvu direktnu vezu s pom. glagolom. U trećem primjeru značenje glagola 走り出す je \textit{potrčati}. Ovo je donekle česta promjena značenja glagola 出す kad se koristi kao pomoćni.
	
	\begin{reibun}[\ten すぎる - \textit{pre\textasciitilde}, pretjerati s \textasciitilde]
		\reinagai{たこ\furigana{焼}{や}きを食べすぎて、タコが\furigana{嫌}{きら}いになりそうだ。}{Prejeo sam se takojakija, mislim da ću zamrziti hobotnice.}
		\reinagai{アニメを見すぎて(impl. 私の)日本語がおかしくなった。}{Gledao sam previše animea pa sad čudno pričam japanski.}
		\reinagai{勉強しすぎて、\furigana{逆}{ぎゃく}に\furigana{頭}{あたま}が\furigana{悪}{わる}くなってしまいました。}{Previše sam učio pa sam, suprotno očekivanom, poglupio.}
	\end{reibun}
	
	\begin{reibun}[\ten \furigana{続}{つづ}ける - \textit{nastaviti} \textasciitilde]
		\rei{彼が私を見続ける。}{On me nastavlja gledati.}
		\rei{彼の\furigana{視線}{しせん}を\furigana{感}{かん}じる私は\furigana{緊張}{きんちょう}し続ける。}{Osjećajući njegov pogled, nastavljam biti nezvozan.}
	\end{reibun}

	Kao i ranije, uočimo da prijelaznost pomoćnog glagola nije bitna. Zanimljivo je primijetiti da se arhaično za neprelazne glagole koristio i 続く, a danas ostaje u okamenjenim izrazima 降り続く (\textit{nastavlja padati}) i 鳴り続く (\textit{nastavlja zvoniti}).
	
	\begin{reibun}[\ten \furigana{合}{あ}う - radnja \textasciitilde\ je obostrana]
		\rei{鈴木さんともう\furigana{一度}{いちど}話し合ってみませんか。}{Hoćeš li još jednom (to) raspraviti sa Suzuki?}
		\rei{アニメでは\furigana{殺}{ころ}し合いが\furigana{多}{おお}い。}{U animeu su pokolji česti.}
	\end{reibun}
	
	\newpage
	\fukudai{Pomoćni pridjevi}
	
	\begin{reibun}[\ten やすい - \textasciitilde je lagano]
		\rei{オウムは\furigana{密林}{みつりん}でも見やすい。}{Papige je lako vidjeti čak i u džungli.}
		\rei{\furigana{古}{ふる}い\furigana{建物}{たてもの}は\furigana{壊}{こわ}れやすい。}{Stare zgrade se lako sruše.}
	\end{reibun}

	\begin{reibun}[\ten \furigana{難}{にく}い - \textasciitilde je teško (prir. samo hiragana)]
		\rei{\furigana{汚}{きたな}い字は読みにくくてしょうがない。}{Neuredna slova su teška za čitanje i tu se nema što učiniti.}
		\rei{ヤシの\furigana{実}{み}は食べにくくて\furigana{嫌}{きら}いです。}{Kokosi su gnjavaža za jesti i ne volim ih.}
	\end{reibun}

	U ovoj funkciji, pom. pridjev にくい znači nešto objektivno teško - ne nužno u smislu da je svakome teško nego za onog o kome se radi.
	
	\begin{reibun}[\ten づらい - \textasciitilde je teško, mučno, bolno]
		\reinagai{誰にでも言いづらいこと一つや二つはあると思うよ。}{Mislim da svatko ima jednu, dvije stvari o kojima mu je teško govoriti.}
		\rei{分かりづらい本は好きじゃない。}{Ne volim teške knjige.}
	\end{reibun}

	Za razliku od pridjeva iznad, づらい (osn. \furigana{辛}{つら}い) može implicirati subjektivni doživljaj nečega kao teškog i napornog. Osim ovih, u sličnom tonu se (iako jako arhaično) pojavljuju i かたい (isti 漢字 kao にくい) gotovo uvijek u \furigana{捨}{す}てがたい, kao i \furigana{苦}{くる}しい, gotovo uvijek u \furigana{見苦}{みぐる}しい.
	
	\begin{reibun}[\ten そうな - čini se kao da će \textasciitilde]
		\reinagai{お\furigana{代}{か}わりも食べそうな\furigana{勢}{いきお}いで食べ始めたけど、\furigana{結局}{けっきょく}食べきれなかった。}{Počeo je jesti kao da će tražiti još, ali na kraju nije uspio sve pojesti.}
		\rei{泣きそうな\furigana{顔}{かお}で「ごめんなさい」と言った。}{Rekao je "oprosti" lica kao da će se rasplakati.}
	\end{reibun}

	\fukudai{U kombinaciji s česticama}
	
	\begin{reibun}[\ten ながら - \textasciitilde\ ći - sadašnji glagolski prilog (p.o. vremena)]
		\reinagai{たけしくんは彼女のことを思いながら、安っぽい\furigana{詩}{し}を書いていました。}{Takeši je, razmišljajući o njoj, pisao jeftinu poeziju.}
		\rei{歩きながら食べてはいけません。}{Ne smiješ jesti hodajući.}
	\end{reibun}
\end{document}