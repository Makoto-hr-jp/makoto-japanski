% !TeX program = xelatex
% !TeX program = xelatex

\documentclass[12pt]{article}

\usepackage{lineno,changepage,lipsum}
\usepackage[colorlinks=true,urlcolor=blue]{hyperref}
\usepackage{fontspec}
\usepackage{xeCJK}
\usepackage{tabularx}
\setCJKfamilyfont{chanto}{AozoraMinchoRegular.ttf}
\setCJKfamilyfont{tegaki}{Mushin.otf}
\usepackage[CJK,overlap]{ruby}
\usepackage{hhline}
\usepackage{multirow,array,amssymb}
\usepackage[croatian]{babel}
\usepackage{soul}
\usepackage[usenames, dvipsnames]{color}
\usepackage{wrapfig,booktabs}
\renewcommand{\rubysep}{0.1ex}
\renewcommand{\rubysize}{0.75}
\usepackage[margin=50pt]{geometry}
\modulolinenumbers[2]

\usepackage{pifont}
\newcommand{\cmark}{\ding{51}}%
\newcommand{\xmark}{\ding{55}}%

\definecolor{faded}{RGB}{100, 100, 100}

\renewcommand{\arraystretch}{1.2}

%\ruby{}{}
%$($\href{URL}{text}$)$

\newcommand{\furigana}[2]{\ruby{#1}{#2}}
\newcommand{\tegaki}[1]{
	\CJKfamily{tegaki}\CJKnospace
	#1
	\CJKfamily{chanto}\CJKnospace
}

\newcommand{\dai}[1]{
	\vspace{20pt}
	\large
	\noindent\textbf{#1}
	\normalsize
	\vspace{20pt}
}

\newcommand{\fukudai}[1]{
	\vspace{10pt}
	\noindent\textbf{#1}
	\vspace{10pt}
}

\newenvironment{bunshou}{
	\vspace{10pt}
	\begin{adjustwidth}{1cm}{3cm}
	\begin{linenumbers}
}{
	\end{linenumbers}
	\end{adjustwidth}
}

\newenvironment{reibun}{
	\vspace{10pt}
	\begin{tabular}{l l}
}{
	\end{tabular}
	\vspace{10pt}
}
\newcommand{\rei}[2]{
	#1&\textit{#2}\\
}
\newcommand{\reinagai}[2]{
	\multicolumn{2}{l}{#1}\\
	\multicolumn{2}{l}{\hspace{10pt}\textit{#2}}\\
}

\newenvironment{mondai}[1]{
	\vspace{10pt}
	#1
	
	\begin{enumerate}
		\itemsep-5pt
	}{
	\end{enumerate}
	\vspace{10pt}
}

\newenvironment{hyou}{
	\begin{itemize}
		\itemsep-5pt
	}{
	\end{itemize}
	\vspace{10pt}
}

\date{\today}

\CJKfamily{chanto}\CJKnospace
\author{Tomislav Mamić}

\begin{document}
	\dai{Napredne upotrebe て oblika III}
	
	\fukudai{Teorija}
	
	Izrazi koje ćemo danas razmatrati zapravo su fiksni. Gramatički su čvrsto vezani uz て oblik, a uključuju i nestandardno korištenje čestica. Nisu komplicirani kad znamo て oblik, ali semantički su pomalo zapetljani. Pojavljuju se u obliku:
	
	\hspace{30pt} に + \furigana{動詞}{どうし}て\furigana{形}{けい}
	
	Gramatički se s desne strane svi ponašaju kao punopravni glagoli, što znači da ne moraju biti u て obliku. Ako nisu u て obliku, onda nose svoje originalno značenje.
	
	S lijeve strane se ponašaju kao čestice (označuju imenice), s tim da se u nekim slučajevima mogu koristiti i za označavanje cijelih nezavisnih rečenica.
	
	\begin{reibun}[$\bullet$ \textasciitilde について $\rightarrow$ \textit{(govoreći) o \textasciitilde}]
		\rei{\furigana{鈴木}{すずき}さんと新しい仕事について話しました。}{Sa Suzuki sam razgovarao o novom poslu.}
		\reinagai{きょうすけさんは\furigana{女性}{じょせい}について何時間も\furigana{喋}{しゃべ}れると聞いたよ。}{Čuo sam da Ky\={o}suke može satima mljeti o ženama.}
		\rei{\furigana{鯨}{くじら}について話そうよ。}{Razgovarajmo o kitovima.}
	\end{reibun}
	
	Izraz iznad veže se uz jezik, misli, znanje i informacije općenito, a ponekad se pojavljuje i kao kraj naslova članaka, knjiga itd. S desne strane uvijek označava imenicu.
	
	\begin{reibun}[$\bullet$ \textasciitilde に\furigana{従}{したが}って $\rightarrow$ \textit{u ovisnosti o \textasciitilde}, \textit{prateći \textasciitilde}]
		\rei{私の\furigana{指示}{しじ}に従ってください。}{Slušaj moje upute./Radi kako ti kažem.}
		\reinagai{日本語は\furigana{勉強}{べんきょう}が\furigana{進}{すす}むにしたがって、ややこしくなります。}{Japanski postaje to kompliciraniji što više napreduješ.}
		\rei{\furigana{湿度}{しつど}は\furigana{温度}{おんど}にしたがって変わります。}{Vlaga se mijenja u ovisnosti o temperaturi.}
	\end{reibun}

	Ovaj izraz uspostavlja odnos između dvije stvari takav da jedna stvar uvjetuje promjenu druge (ali ne nužno i obratno). U prvom primjeru korišteno je osnovno značenje glagola i zato je pisan 漢字. Osim što s lijeve strane može označiti imenicu, ovaj izraz može stajati i iza glagola u rječničkom obliku (vidi 2. primjer).
	
	\begin{reibun}[$\bullet$ \textasciitilde につれて $\rightarrow$ \textit{zajedno s \textasciitilde}]
		\rei{父は車でここにつれてきてくれた。}{Otac me ovdje doveo autom.}
		\rei{\furigana{木の葉}{このは}は日が\furigana{経}{た}つにつれて\furigana{紅葉}{こうよう}した。}{Lišće je crvenjelo kako su prolazili dani.}
		\rei{夜がふけるにつれて寒くなった。}{Kako je noć odmicala, postajalo je hladnije.}
		\rei{人口の\furigana{増加}{ぞうか}につれて\furigana{住宅}{じゅうたく}が\furigana{不足}{ふそく}してくる。}{Kako raste populacija, nedostaje sve više kuća\footnotemark[1].}
	\end{reibun}
	\footnotetext[1]{住宅 je općenita riječ za dom - nebitno radi li se o kući, stanu ili kartonskoj kutiji.}
	\newpage
	Prvi primjer je zamka, jer iako se pojavljuje izraz につれて, zapravo se radi o osnovnom značenju glagola つれる - \textit{dovesti}. Zgodna demonstracija je to da možemo jednostavno uzeti ここに i 車で i zamijeniti im mjesta bez gubitka značenja.
	
	Za razliku od にしたがって, ovdje nije implicirano da jedna stvar uzrokuje drugu nego jednostavno primjećujemo da se dvije stvari događaju zajedno, ali u praktičnim situacijama ovi izrazi se gotovo uvijek koriste kao sinonimi kad želimo izraziti istovremenost.
	
	\begin{reibun}[$\bullet$ \textasciitilde に\furigana{伴}{ともな}って $\rightarrow$ slično kao prethodna dva]
		\rei{温度が上がるに伴って湿度は下がっていく。}{Vlaga u zraku pada s porastom temperature.\footnotemark[2]}
		\rei{温度の\furigana{上昇}{じょうしょう}に伴って湿度は\furigana{低下}{ていか}していく。}{Vlaga u zraku pada s porastom temperature.}
	\end{reibun}

	Koristi se kao sinonim s prethodna dva izraza, iako sva tri glagola u osnovnom značenju imaju razlike. Valja primijetiti da, dok prethodna dva izraza često označuju glagol direktno, s に伴って to baš i nije često.
	\footnotetext[2]{Iako nije krivo, vrlo je rijetko da に伴って označava direktno glagol - najčešće se dodaje の za nominalizaciju.}
	
	\begin{reibun}[$\bullet$ \textasciitilde に\furigana{応}{おう}じて $\rightarrow$ \textit{kao odgovor/reakcija na \textasciitilde}]
		\reinagai{日本語の勉強が\furigana{難}{むずか}しくなるのに応じて、勉強時間を\furigana{増}{ふ}やすしかない。}{Kad učenje japanskog postane teško, nema druge nego povećati količinu vremena za učenje.}
		\reinagai{人間は自分の\furigana{必要}{ひつよう}に応じて\furigana{自然}{しぜん}を\furigana{変}{か}えてゆく。}{Čovjek mijenja prirodu u skladu sa svojim potrebama.}
	\end{reibun}
	
	\begin{reibun}[$\bullet$ \textasciitilde によって $\rightarrow$ \textit{od strane \textasciitilde}, \textit{ovisno o \textasciitilde}, \textit{zbog \textasciitilde}]
		\rei{場合によって逃げるかもしれない。}{Ovisno o situaciji, možda ću dati petama vjetra.}
		\rei{\furigana{例}{れい}によって彼は新聞を読みながら食べていた。}{Po običaju je jeo čitajući novine.}
		\rei{\furigana{嵐}{あらし}によって多くの人が死んだ。}{Zbog oluje je stradalo puno ljudi.}
		\rei{彼によって書かれた本は多い。}{Puno je knjiga koje je on napisao.}
	\end{reibun}

	I ovdje je kao jedan od prijevoda naveden \textit{ovisno o \textasciitilde}, ali postoje razlike u odnosu na slične izraze iznad. Osim što može označavati vršitelja radnje (pr. 4) ili krivca/uzrok (pr. 3), ovaj izraz s lijeve strane obavezno označava imenice.
	
\end{document}