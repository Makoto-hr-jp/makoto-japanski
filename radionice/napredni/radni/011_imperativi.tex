% !TeX program = xelatex

\documentclass[12pt]{article}

\usepackage[colorlinks=true,urlcolor=blue]{hyperref}
\usepackage{xeCJK}
\usepackage{tabularx}
\setCJKmainfont{AozoraMinchoRegular.ttf}
\usepackage[CJK,overlap]{ruby}
\usepackage{hhline}
\usepackage{multirow,array,amssymb}
\usepackage[croatian]{babel}
\usepackage{soul}
\usepackage[usenames, dvipsnames]{color}
\usepackage{wrapfig,lipsum,booktabs}
\renewcommand{\rubysep}{0.1ex}
\renewcommand{\rubysize}{0.55}
\usepackage[margin=50pt]{geometry}
\author{Tomislav Mamić}
\title{rl4}

\definecolor{faded}{RGB}{100, 100, 100}

%\ruby{}{}
%$($\href{URL}{text}$)$

\begin{document}

	\large Imperativi

	\vspace{20pt}
	\normalsize \textbf{Pravi imperativ}
	\vspace{20pt}
	
	\begin{wraptable}[17]{l}{0pt}
	\begin{tabular}{|l|l|}
		\hline
		\multicolumn{2}{|c|}{nepr.}\\
		\hline
		いく&いけ\\
		\hline
		くる&こい\\
		\hline
		\multirow{2}{50pt}{する}&しろ\\\cline{2-2}
		&せよ\footnotemark[1]\\
		\hline
		ある&あれ\\
		\hline
		\hline
		\multicolumn{2}{|c|}{一段}\\
		\hline
		\multirow{2}{50pt}{(i/e)る}&ろ\\\cline{2-2}
		&よ\footnotemark[1]\\
		\hline
		\hline
		\multicolumn{2}{|c|}{五段}\\
		\hline
		く&け\\
		ぐ&げ\\
		す&せ\\
		\hline
		ぬ&ね\\
		む&め\\
		ぶ&べ\\
		\hline
		う&え\\
		つ&て\\
		る&れ\\
		\hline
	\end{tabular}
	\end{wraptable}
	\vspace{20pt}

	Imperativ kao glagolski oblik je vrlo grub i izravan. U pravilu se ne koristi nikad kad pokušavamo biti imalo pristojni, ali možemo ga koristiti u citatu da bismo izrazili što je nama naređeno, čak i ako u originalu nije bio korišten pravi imperativ (dopušteno je tako parafrazirati i nije nepristojno).
	\vspace{5pt}
	
	せんせい: やまだくん、ちょっとこっちきて。
	
	やまだ: は\textasciitilde い。
	
	せんせい: きのうの しゅくだい、まだ だしていないでしょう?
	
	やまだ: うん、はい。
	
	せんせい: あしたまでには ていしゅつしないと おこりますよ。
	
	...
	
	すずき: やまだ、せんせいと なにを はなした?
	
	やまだ: あしたまでに きのうの しゅくたいを もってこいと いわれた。
	
	\vspace{20pt}
	\normalsize \textbf{Ženski imperativ}
	\vspace{20pt}
	
	Glagol u て obliku bez ikakvih dodataka predstavlja imperativ. Koriste ga žene i svi koji se tako osjećaju, a dolazi od zamolbe s ください, samo bez zamolbe.
	
	\vspace{20pt}
	\normalsize \textbf{Dječji imperativ}
	\vspace{20pt}
	
	Radi se iz い oblika glagola, dodavanjem pom. glagola なさい (なさる - する iz poštovanja). Nekad davno u povijesti japanskog jezika, glagoli su imali odvojen opisni i predikatni oblik. Opisni oblik se zvao 連体形 (れんたいけい) i bio je kao današnji kolokvijalni oblik, a predikatni se zvao 終止形 (しゅうしけい) i završavao je samoglasnikom \textit{i}. U modernom jeziku ta dva oblika su se spojila u jedan, koji izgleda kao stari 連体形, ali obavlja funkciju oba. Zato se nazivi ponekad u modernom kontekstu koriste kao sinonimi.
	
\end{document}