% !TeX program = xelatex
% !TeX program = xelatex

\documentclass[12pt]{article}

\usepackage{lineno,changepage,lipsum}
\usepackage[colorlinks=true,urlcolor=blue]{hyperref}
\usepackage{fontspec}
\usepackage{xeCJK}
\usepackage{tabularx}
\setCJKfamilyfont{chanto}{AozoraMinchoRegular.ttf}
\setCJKfamilyfont{tegaki}{Mushin.otf}
\usepackage[CJK,overlap]{ruby}
\usepackage{hhline}
\usepackage{multirow,array,amssymb}
\usepackage[croatian]{babel}
\usepackage{soul}
\usepackage[usenames, dvipsnames]{color}
\usepackage{wrapfig,booktabs}
\renewcommand{\rubysep}{0.1ex}
\renewcommand{\rubysize}{0.75}
\usepackage[margin=50pt]{geometry}
\modulolinenumbers[2]

\usepackage{pifont}
\newcommand{\cmark}{\ding{51}}%
\newcommand{\xmark}{\ding{55}}%

\definecolor{faded}{RGB}{100, 100, 100}

\renewcommand{\arraystretch}{1.2}

%\ruby{}{}
%$($\href{URL}{text}$)$

\newcommand{\furigana}[2]{\ruby{#1}{#2}}
\newcommand{\tegaki}[1]{
	\CJKfamily{tegaki}\CJKnospace
	#1
	\CJKfamily{chanto}\CJKnospace
}

\newcommand{\dai}[1]{
	\vspace{20pt}
	\large
	\noindent\textbf{#1}
	\normalsize
	\vspace{20pt}
}

\newcommand{\fukudai}[1]{
	\vspace{10pt}
	\noindent\textbf{#1}
	\vspace{10pt}
}

\newenvironment{bunshou}{
	\vspace{10pt}
	\begin{adjustwidth}{1cm}{3cm}
	\begin{linenumbers}
}{
	\end{linenumbers}
	\end{adjustwidth}
}

\newenvironment{reibun}{
	\vspace{10pt}
	\begin{tabular}{l l}
}{
	\end{tabular}
	\vspace{10pt}
}
\newcommand{\rei}[2]{
	#1&\textit{#2}\\
}
\newcommand{\reinagai}[2]{
	\multicolumn{2}{l}{#1}\\
	\multicolumn{2}{l}{\hspace{10pt}\textit{#2}}\\
}

\newenvironment{mondai}[1]{
	\vspace{10pt}
	#1
	
	\begin{enumerate}
		\itemsep-5pt
	}{
	\end{enumerate}
	\vspace{10pt}
}

\newenvironment{hyou}{
	\begin{itemize}
		\itemsep-5pt
	}{
	\end{itemize}
	\vspace{10pt}
}

\date{\today}

\CJKfamily{chanto}\CJKnospace
\author{Tomislav Mamić}

\begin{document}
	\dai{Napredne upotrebe て oblika}
	
	\fukudai{Pomoćni glagoli - pregled}
	
	Najčešća upotreba て oblika je dodavanje pomoćnog glagola. Na popisu ispod sigurno će se naći glagola koji su nam već otprije poznati, ali i nekih novih:
	
	\vspace{10pt}
	\renewcommand{\arraystretch}{1.25}
	\begin{tabular}{|l|l|l|}
		\hline
		\textbf{pom. glagol}&\textbf{značenje}&\textbf{napomena}\\
		\hline
		いる&radnja traje&označava stanje \textit{u kojem traje radnja}\\
		\hline
		ある&radnja je izvršena&označava stanje, prevodi se kao pasiv\\
		\hline
		いく&radnja se nastavlja ili eskalira&radi i u prostoru i u vremenu\\
		\hline
		くる&radnja traje do (\textasciitilde)&radi i u prostoru i u vremenu\\
		\hline
		おく&učinjeno s predumišljajem&u hrvatskom je praktički neprevodivo\\
		\hline
		あげる&subjekt čini za metu&smjer prema van\footnotemark[1]\\\cline{1-1}\cline{3-3}
		さしあげる&&pristojna verzija あげる\\
		\hline
		くれる&subjekt čini za metu&smjer prema unutra\footnotemark[1]\\\cline{1-1}\cline{3-3}
		くださる&&pristojna verzija くれる\\
		\hline
		もらう&meta čini za subjekt&neutralno na smjer\\\cline{1-1}\cline{3-3}
		いただく&&pristojna verzija もらう\\
		\hline
	\end{tabular}
	
	\fukudai{Pomoćni glagoli - primjeri}
	
	\begin{reibun}
		\rei{きょうすけさんは けっこんしている。}{Kyousuke je oženjen.}
	\end{reibun}
\end{document}