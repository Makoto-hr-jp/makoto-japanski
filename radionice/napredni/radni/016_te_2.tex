% !TeX program = xelatex
% !TeX program = xelatex

\documentclass[12pt]{article}

\usepackage{lineno,changepage,lipsum}
\usepackage[colorlinks=true,urlcolor=blue]{hyperref}
\usepackage{fontspec}[ Path =../../../ ]
\usepackage{xeCJK}
\usepackage{tabularx}
\usepackage{graphicx}
\setCJKfamilyfont{chanto}{AOZORAMINCHOREGULAR_0.TTF}%
\setCJKfamilyfont{tegaki}{Mushin.otf}%
\usepackage[CJK,overlap]{ruby}
\usepackage{hhline}
\usepackage{multirow,array,amssymb}
\usepackage[croatian]{babel}
\usepackage{soul}
\usepackage[usenames, dvipsnames]{color}
\usepackage{wrapfig,booktabs}
\usepackage{calc}
\renewcommand{\rubysep}{0.1ex}
\renewcommand{\rubysize}{0.75}
\usepackage[margin=50pt]{geometry}
\usepackage{hyperref}
\modulolinenumbers[2]

\date{\today}

\usepackage{fancyhdr}
\pagestyle{fancy}
\fancyhf{}
\fancyhead[LE,RO]{\thepage}
\makeatletter
\fancyhead[RE,LO]{rev. \@date 誠}
\makeatother

\usepackage{pifont}
\newcommand{\cmark}{\ding{51}}%
\newcommand{\xmark}{\ding{55}}%

\newcommand{\dosl}{{\normalfont dosl. }}%
\newcommand{\rem}[1]{{\normalfont #1 }}%

\definecolor{faded}{RGB}{100, 100, 100}

\renewcommand{\arraystretch}{1.2}

%\ruby{}{}
%$($\href{URL}{text}$)$

\newcommand{\furigana}[2]{\ruby{#1}{#2}}
\newcommand{\tegaki}[1]{
	\CJKfamily{tegaki}\CJKnospace
	#1
	\CJKfamily{chanto}\CJKnospace
}

\newcommand{\dai}[1]{
	\vspace{20pt}
	\large
	\noindent\textbf{#1}
	\normalsize
	\vspace{20pt}
}

\newcommand{\fukudai}[1]{
	\vspace{10pt}
	\noindent\textbf{#1}
	\vspace{10pt}
}

\newenvironment{bunshou}{
	\vspace{10pt}
	\begin{adjustwidth}{1cm}{3cm}
	\begin{linenumbers}
}{
	\end{linenumbers}
	\end{adjustwidth}
}

\newenvironment{reibun}[1][]{
	\vspace{10pt}
	#1
	
	\begin{tabular}{l l}
}{
	\end{tabular}
	\vspace{10pt}
}
\newcommand{\rei}[2]{
	#1&\textit{#2}\\
}
\newcommand{\reinagai}[2]{
	\multicolumn{2}{l}{#1}\\
	\multicolumn{2}{l}{\hspace{10pt}\textit{#2}}\\
}

\newenvironment{mondai}[1]{
	\vspace{10pt}
	\noindent #1
	
	\begin{enumerate}
		\itemsep-5pt
	}{
	\end{enumerate}
}

\newenvironment{hyou}{
	\begin{itemize}
		\itemsep-5pt
	}{
	\end{itemize}
	\vspace{10pt}
}

\newcommand{\juuyou}[2][20pt]{
	\vspace{5pt}
		\noindent\hspace{#1}\parbox[c]{\textwidth-#1-#1}{\centering\textit{#2}}
	\vspace{5pt}
}

\newcommand{\ten}{
	\vspace{5pt}
	\noindent\hspace{-10pt}$\bullet$
}

\CJKfamily{chanto}\CJKnospace

\frenchspacing
\author{Tomislav Mamić}

\begin{document}
	\dai{Napredne upotrebe て oblika}
	
	\fukudai{Pomoćni glagoli - pregled}
	
	Najčešća upotreba て oblika je dodavanje pomoćnog glagola. Na popisu ispod sigurno će se naći glagola koji su nam već otprije poznati, ali i nekih novih:
	
	\vspace{10pt}
	\renewcommand{\arraystretch}{1.25}
	\begin{tabular}{|l|l|l|}
		\hline
		\textbf{pom. glagol}&\textbf{značenje}&\textbf{napomena}\\
		\hline
		いる&radnja traje&označava stanje \textit{u kojem traje radnja}\\
		\hline
		ある&radnja je izvršena&označava stanje, prevodi se kao pasiv\\
		\hline
		いく&radnja se nastavlja ili eskalira&radi i u prostoru i u vremenu\\
		\hline
		くる&radnja traje do (\textasciitilde)&radi i u prostoru i u vremenu\\
		\hline
		おく&učinjeno s predumišljajem&u hrvatskom je praktički neprevodivo\\
		\hline
		あげる&subjekt čini za metu&smjer prema van\footnotemark[1]\\\cline{1-1}\cline{3-3}
		さしあげる&&pristojna verzija あげる\\
		\hline
		くれる&subjekt čini za metu&smjer prema unutra\footnotemark[1]\\\cline{1-1}\cline{3-3}
		くださる&&pristojna verzija くれる\\
		\hline
		もらう&meta čini za subjekt&neutralno na smjer\\\cline{1-1}\cline{3-3}
		いただく&&pristojna verzija もらう\\
		\hline
	\end{tabular}

	\footnotetext[1]{Smjer se ovdje odnosi na skupine ljudi u odnosu na govornika, npr. \textit{obitelj} $\rightarrow$ \textit{poznanici} je smjer prema van, a \textit{mušterija} $\rightarrow$ \textit{tvrtka za koju radim} je smjer prema unutra.}
	
	\fukudai{Pomoćni glagoli - primjeri}
	
	\begin{reibun}[\textasciitilde いる]
		\rei{きょうすけさんは\furigana{結婚}{けっこん}している。}{Ky\={o}suke je oženjen.}
		\rei{たけしくんは結婚していない。}{Takeši nije oženjen.}
		\rei{ドアは\furigana{開}{あ}いている。}{Vrata su otvorena.}
	\end{reibun}

	\begin{reibun}[\textasciitilde ある]
		\rei{ドアは開けてある。}{Vrata su otvorena.}
		\rei{ホテルの\furigana{予約}{よやく}はもうしてある。}{Već sam rezervirao hotel.}
		\rei{\furigana{車}{くるま}を\furigana{待}{ま}たせてありますから、こちらへどうぞ。}{Auto Vas čeka, pođite ovuda.}
	\end{reibun}

	\begin{reibun}[\textasciitilde いく]
		\rei{森はどんどん消えていく。}{Šume postepeno nestaju.}
		\rei{これ、\furigana{持}{も}っていくね。}{Odnijet ću ovo, ok?}
	\end{reibun}
	\newpage
	\begin{reibun}[\textasciitilde くる]
		\rei{今まで色々な人と\furigana{知}{し}り\furigana{合}{あ}ってきた。}{Dosad sam upoznao svakakvih ljudi.}
		\rei{ケーキ、持ってくるね。}{Donijet ću tortu, ok?}
	\end{reibun}

	\begin{reibun}[\textasciitilde おく]
		\rei{\furigana{忘}{わす}れないように\furigana{書}{か}いておこう。}{Zapisat ću da ne zaboravim.}
		\reinagai{\furigana{明日}{あした}は\furigana{遅}{おそ}く\furigana{帰}{かえ}ってくるから、今のうちに\furigana{掃除}{そうじ}しておく。}{Sutra ću se kasno vratiti pa ću sad (dok još stignem) počistiti.}
	\end{reibun}

	\begin{reibun}[\textasciitilde あげる/\textasciitilde さしあげる]
		\reinagai{\furigana{今日}{きょう}は\furigana{機嫌}{きげん}がいいから\furigana{手伝}{てつだ}ってあげてもいい。\footnotemark[2]}{Danas sam dobro raspoložen pa bih ti mogao pomoći.}
		\rei{お母さんは\furigana{子供}{こども}たちに\furigana{絵本}{えほん}を\furigana{読}{よ}んであげた。}{Mama je djeci pročitala slikovnicu.}
	\end{reibun}

	\begin{reibun}[\textasciitilde くれる/\textasciitilde くださる]
		\reinagai{きょうこちゃんが手伝ってくれたから今日は\furigana{早}{はや}く\furigana{終}{お}わった。}{Danas sam ranije završio jer mi je Ky\={o}ko pomogla.}
		\rei{\furigana{弟}{おとうと}のめんどうを見てくれてありがとう。}{Hvala ti što si se pobrinuo za mog (mlađeg) brata.}
	\end{reibun}

	\begin{reibun}[\textasciitilde もらう/\textasciitilde いただく]
		\rei{その本、\furigana{返}{かえ}してもらう。}{Vratit ćeš mi tu knjigu. (ljigo ljigava, šugo šugava)}
		\rei{その本、返してもらえませんか。}{Mogu li molim te dobiti svoju knjigu natrag?}
		\rei{\furigana{通}{とお}らせてもらいます。}{Pustit ćeš me da prođem.}
		\rei{\furigana{理解}{りかい}してもらうのが\furigana{意外}{いがい}と\furigana{難}{むず}しかった。}{Bilo je iznenađujuće teško dobiti da me razumiju.}
	\end{reibun}

	\footnotetext[2]{Ovo se u japanskom zove 上から目線(うえからめせん) i nije preporučljivo osim u šali.}
	
	\fukudai{Izrazi s česticama}
	
	Postoje fiksni izrazi oblika (\textasciitilde て) + čestica čije značenje može biti, ali često nije potpuno jasno iz same čestice. Budući se sastoje od nepromjenjivih dijelova (iako su glagoli promjenjivi, njihov て oblik više nije!), lako ih je uočiti i protumačiti u rečenici.
	
	\begin{reibun}[\textasciitilde てから $\rightarrow$ \textit{nakon što \textasciitilde}]
		\rei{たけしくんは日本に\furigana{帰}{かえ}ってから\furigana{変}{へん}です。}{Takeši je čudan otkad se vratio u Japan.}
		\rei{もんくは\furigana{自分}{じぶん}で\furigana{料理}{りょうり}してみてからいえよ。}{Prigovaraj nakon što sam probaš nešto skuhati.}
	\end{reibun}
	\newpage
	
	\begin{reibun}[\textasciitilde てまで $\rightarrow$ \textit{toliko da \textasciitilde, s takvom posljedicom da \textasciitilde\footnotemark[3]}]
		\reinagai{\furigana{親}{おや}に\furigana{家}{いえ}を出るよう言われてまで私は\furigana{彼女}{かのじょ}と\furigana{結婚}{けっこん}した。}{Oženio sam se njome pa makar su me i starci izbacili iz kuće.}
		\reinagai{\parbox{470pt}{\furigana{渋谷}{しぶや}に1時間も\furigana{並}{なら}んで\furigana{待}{ま}たなければならない人気な\furigana{店}{みせ}がある。私は並んでまでその店に入りたいと思いません。}}{\parbox{460pt}{U Šibuji postoji dućan/kafić/štogod za koji treba čekati u redu čak sat vremena. Ja ne želim ući unutra toliko da bih stao u red.}}
	\end{reibun}

	\begin{reibun}[\textasciitilde てさえ$^4$ $\rightarrow$ \textit{čak i \textasciitilde} s poz. glagolom, \textit{čak ni \textasciitilde} s neg. glagolom (slično kao も, ali jače)]
		\rei{たけしくんは自分の\furigana{将来}{しょうらい}を\furigana{考}{かんが}えてさえいない。}{Takešiju (njegova) budućnost nije ni na kraj pameti.}
		\rei{たけしくんは彼女を\furigana{愛}{あい}してさえいた。}{Zapravo, takeši je nju čak i volio.}
	\end{reibun}

	\begin{reibun}[\textasciitilde てばかり だ/いる/の $\rightarrow$ \textit{samo \textasciitilde} (i ništa drugo ne radi)]
		\reinagai{きょうこちゃんは\furigana{結婚}{けっこん}してから食べてばかりいて、このままデブになっちゃうよ。}{Otkad se udala, Ky\={o}ko samo jede, ovim tempom postat će bačvica.}
		\reinagai{たけしくんは自分の将来を考えてさえいないで\furigana{遊}{あそ}んでばかりで\furigana{親}{おや}を\furigana{困}{こま}らせている。}{Takeši ni da bi razmislio o svojoj budućnosti, samo se zabavlja i zabrinjava roditelje.}
	\end{reibun}

	\begin{reibun}[\textasciitilde ては + いけない/ならない/ダメ $\rightarrow$ \textit{ne smiješ \textasciitilde}]
		\rei{おじょうさま、そんなことを言ってはなりません。}{Mlada damo, ne smijete govoriti takve stvari.}
		\rei{\furigana{他人}{たにん}の\furigana{皿}{さら}から食べてはいけない。}{Nije u redu jesti iz tuđeg tanjura.}
	\end{reibun}

	\begin{reibun}[\textasciitilde ても + いい $\rightarrow$ \textit{smiješ \textasciitilde} uz poz. glavni glagol, \textit{ne moraš \textasciitilde} uz neg. glavni glagol]
		\rei{にんじんは食べなくてもいいよ。}{Ne moraš jesti mrkve.}
		\rei{となりに\furigana{座}{すわ}ってもいいですか。}{Smijem li sjesti pored vas?}
	\end{reibun}

	\footnotetext[3]{Ovaj je jako težak i nije bitno da ga odmah zapamtimo ili naučimo koristiti.}
	\footnotetext[4]{Postoji i čestica すら koja radi jako slično kao さえ u drugim okolnostima, ali je ne smijemo koristiti s て i い oblikom glagola kao ovdje.}
	
\end{document}