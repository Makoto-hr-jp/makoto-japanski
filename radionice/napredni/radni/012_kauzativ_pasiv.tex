% !TeX program = xelatex

\documentclass[12pt]{article}

\usepackage[colorlinks=true,urlcolor=blue]{hyperref}
\usepackage{xeCJK}
\usepackage{tabularx}
\setCJKmainfont{AozoraMinchoRegular.ttf}
\usepackage[CJK,overlap]{ruby}
\usepackage{hhline}
\usepackage{multirow,array,amssymb}
\usepackage[croatian]{babel}
\usepackage{soul}
\usepackage[usenames, dvipsnames]{color}
\usepackage{wrapfig,lipsum,booktabs}
\renewcommand{\rubysep}{0.1ex}
\renewcommand{\rubysize}{0.55}
\usepackage[margin=50pt]{geometry}
\author{Tomislav Mamić}
\title{Inicijalna provjera}

\definecolor{faded}{RGB}{100, 100, 100}

%\ruby{}{}
%$($\href{URL}{text}$)$

\begin{document}

	\large Kauzativ i pasiv
	
	\vspace{20pt}
	\normalsize \textbf{Tvorba}
	\vspace{20pt}

\begin{wraptable}[22]{l}{0pt}
	\begin{tabular}{|l|l|l|}
		\hline
		\multicolumn{3}{|c|}{nepravilni}\\
		\hline
		&kauzativ&pasiv\\
		\hline
		いく&いかせる&いかれる\\
		&いかす&\\
		くる&こさせる&こられる\\
		ある&nema\footnotemark[1]&nema\footnotemark[1]\\
		する&させる&される\\
		\hline
		\hline
		\multicolumn{3}{|c|}{一段}\\
		\hline
		&kauzativ&pasiv\\
		\hline
		る&させる&られる\footnotemark[2]\\
		\hline
		\hline
		\multicolumn{3}{|c|}{五段}\\
		\hline
		&kauzativ&pasiv\\
		\hline
		く&かせる&かれる\\
		ぐ&がせる&がれる\\
		す&させる&される\\
		\hline
		ぬ&なせる&なれる\\
		む&ませる&まれる\\
		ぶ&ばせる&ばれる\\
		\hline
		う&わせる&われる\\
		つ&たせる&たれる\\
		る&らせる&られる\\
		\hline
	\end{tabular}
\end{wraptable}

	Kako se da uočiti iz tablice desno, osnova tvorbe kauzativa i pasiva je あ stupac, kao za negaciju. Kauzativ dobijemo dodavanjem せる, a pasiv pomoću れる. Iz tablice se također da uočiti da glagol いく ima 2 oblika kauzativa. Iako u tablici nisu navedeni jer bi zauzimalo puno mjesta, a i pravilno je, zapravo svi 五段 glagoli osim す također imaju dva kauzativa. Moguće je naime umjesto せる staviti す. Bitno je da, ako odlučite koristiti す, znate da zvuči pomalo kolokvijalno i da nije dobra praksa iz kauzativa na す dalje raditi složenije oblike. Dakle u bilo kojoj situaciji gdje treba na kauzativ dodati još promjena, bolja opcija je せる.
	
	S pasivom nema nikakvih komplikacija.
	
	\vspace{20pt}
	\normalsize \textbf{Osnovno značenje}
	\vspace{20pt}
	
	Kauzativ može značiti da se neka radnja dopušta:
	
	いぬを そとに いかせた。 \textit{Pustio sam psa van.}
	
	Ili da se izvršavanje radnje prisiljava:
	
	こどもに にんじんを たべさせた。 \textit{Natjerao sam dijete da jede mrkve.}
	
	Valja primijetiti kako se događa pomak u ulogama - subjekt (は/が) postaje meta radnje (に), a novi subjekt je onaj tko dopušta ili prisiljava na radnju.
	
	\vspace{10pt}
	
	Pasiv može značiti upravo gramatički kontekst pasiva - radnja je izvršena:
	
	りんごは たべられた。 \textit{Jabuka je pojedena.}
	
	Ali nosi i potencijalno opasnu konotaciju - izriče nezadovoljstvo onog iz čije se perspektive govori učinjenim:
	
	せんせいに よく いわれた。 dosl. \textit{Prof mi je dobro rekao.}, eufemizam za \textit{Prof me sažvakao i ispljunuo.}
	
	すずきさんは りんごを たべられて ないている。 \textit{Suzuki je netko pojeo jabuku pa plače.}
	
	I kod pasiva dolazi do zamjene uloga u rečenici, ali malo složenije. Kad se radi o čistom gramatičkom pasivu, subjekt (は/が) postaje meta (に), a objekt (を) postaje subjekt:
	
	せんせいが はなこさんの ノートを みつけた。$\rightarrow$ はなこさんのノートは せんせいに 見つけられた。
	
	Ako se radi o impliciranom značenju, onaj tko nije zadovoljan izvršenom radnjom postaje tema (は), vršitelj radnje dobiva oznaku (に), a objekt (を) ostaje sačuvan:
	
	はなこさんは せんせいに ノートを 見つけられた。
	
	\footnotetext[1]{Ne možemo baš dopustiti nečemu ili natjerati nešto da biva, a bivanje je samo po sebi dosta pasivna aktivnost.}
	\footnotetext[2]{Ne pomiješati s potencijalom - \textit{potencijalno} opasno! HA ha.}
	
	\newpage
	\vspace{20pt}
	\normalsize \textbf{Pasiv kauzativa}
	\vspace{20pt}
	
	U principu znači samo kombinaciju kauzativa i pasiva - \textit{dopušteno mi je} / \textit{natjeran sam} nešto učiniti. Oblik je kompozicija kauzativa i pasiva - pasiv(kauzativ(glagol)). S obzirom da završava pasivom, može imati i onu konotaciju da govornik nije sretan situacijom, npr.
	
	こどものころ にんじんを たべさせられていた。 \textit{Kad sam bio dijete tjerali su me da jedem mrkve.} (i nisam bio sretan zbog toga)
	
	日本では さけを のませられる ことが おおい。 \textit{U japanu često budeš natjeran da piješ alkohol.}
	
	Moguće je koristiti i skraćeni kauzativ, ali ne smijemo zaboraviti da ga 一段 i 五段 す glagoli nemaju! Znači smijemo reći
	
	日本では さけを のまされる ことが おおい。
	
	\vspace{20pt}
	\normalsize \textbf{Za vježbu}
	\vspace{20pt}
	
	\vspace{5pt}
	\normalsize Lv. 1
	\vspace{5pt}
	
	せんせいに いわれた。
	
	せいとたちに そうじさせました。
	
	たけしくんは 2じかん またされた。
	
	\vspace{5pt}
	\normalsize Lv. 2
	\vspace{5pt}
	
	けさ がっこうに きた とたん、せんせいに あぶらを しぼられた。
	
	わるい せいとたちに にわを そうじさせました。
	
	たけしくんは かのじょに 2じかんも またされた。
	
	\vspace{5pt}
	\normalsize Lv. 3
	\vspace{5pt}
	
	けさ がっこうに きた とたん せんせいに あぶらを しぼられた たけしくんは なにが おきているか わからなかった。
	
	わるい せいとたちに にわの そうじ を させた せんせいは とうりょうと いっしょに それを みながら わらっていました。
	
	かのじょに 2じかんも またされて おこっていた たけしくんは けっきょく かえってしまった。
\end{document}