% !TeX document-id = {0c10369a-658e-401a-a5d4-aee0f8bb2870}
% !TeX program = xelatex ?me -synctex=0 -interaction=nonstopmode -aux-directory=../tex_aux -output-directory=./release
% !TeX program = xelatex

\documentclass[12pt]{article}

\usepackage{lineno,changepage,lipsum}
\usepackage[colorlinks=true,urlcolor=blue]{hyperref}
\usepackage{fontspec}
\usepackage{xeCJK}
\usepackage{tabularx}
\setCJKfamilyfont{chanto}{AozoraMinchoRegular.ttf}
\setCJKfamilyfont{tegaki}{Mushin.otf}
\usepackage[CJK,overlap]{ruby}
\usepackage{hhline}
\usepackage{multirow,array,amssymb}
\usepackage[croatian]{babel}
\usepackage{soul}
\usepackage[usenames, dvipsnames]{color}
\usepackage{wrapfig,booktabs}
\renewcommand{\rubysep}{0.1ex}
\renewcommand{\rubysize}{0.75}
\usepackage[margin=50pt]{geometry}
\modulolinenumbers[2]

\usepackage{pifont}
\newcommand{\cmark}{\ding{51}}%
\newcommand{\xmark}{\ding{55}}%

\definecolor{faded}{RGB}{100, 100, 100}

\renewcommand{\arraystretch}{1.2}

%\ruby{}{}
%$($\href{URL}{text}$)$

\newcommand{\furigana}[2]{\ruby{#1}{#2}}
\newcommand{\tegaki}[1]{
	\CJKfamily{tegaki}\CJKnospace
	#1
	\CJKfamily{chanto}\CJKnospace
}

\newcommand{\dai}[1]{
	\vspace{20pt}
	\large
	\noindent\textbf{#1}
	\normalsize
	\vspace{20pt}
}

\newcommand{\fukudai}[1]{
	\vspace{10pt}
	\noindent\textbf{#1}
	\vspace{10pt}
}

\newenvironment{bunshou}{
	\vspace{10pt}
	\begin{adjustwidth}{1cm}{3cm}
	\begin{linenumbers}
}{
	\end{linenumbers}
	\end{adjustwidth}
}

\newenvironment{reibun}{
	\vspace{10pt}
	\begin{tabular}{l l}
}{
	\end{tabular}
	\vspace{10pt}
}
\newcommand{\rei}[2]{
	#1&\textit{#2}\\
}
\newcommand{\reinagai}[2]{
	\multicolumn{2}{l}{#1}\\
	\multicolumn{2}{l}{\hspace{10pt}\textit{#2}}\\
}

\newenvironment{mondai}[1]{
	\vspace{10pt}
	#1
	
	\begin{enumerate}
		\itemsep-5pt
	}{
	\end{enumerate}
	\vspace{10pt}
}

\newenvironment{hyou}{
	\begin{itemize}
		\itemsep-5pt
	}{
	\end{itemize}
	\vspace{10pt}
}

\date{\today}

\CJKfamily{chanto}\CJKnospace
\author{Tomislav Mamić}

\begin{document}
	\dai{Nominalizacija predikata}
	
	\fukudai{Uvod - lažne imenice}
	
	Jedan podskup imenica se u japanskom koristi na način da umjesto svog originalnog značenja mijenja značenje svog opisa. Takve imenice vrlo rijetko shvaćamo doslovno, a prijevodi su im često nejasni i opširni. Iz perspektive hrvatskog jezika, prijevodi se mogu jako razlikovati ovisno o korištenoj imenici - gramatike se ovdje zamjetno razlikuju. (potražiti 形式名詞)
	
	\fukudai{Nominalizacija - の i こと}
	
	\juuyou[40pt]{Nominalizacija je gramatički mehanizam kojim se kao imenice ili imeničke sintagme koriste dijelovi govora koji to nisu.}
	
	Lažne imenice もの (skraćuje se u の) i こと služe za pretvaranje svog opisa u imenice. Iako im se upotreba djelomično preklapa, postoje situacije u kojima smijemo koristiti samo の ili samo こと. Pogledajmo niz primjera:
	
	\begin{reibun}
		\rei{アイスクリームが好きです。}{Volim sladoled.}
		\rei{アイスクリームを食べる。}{Jesti sladoled. (u kontekstu Pojest ću sladoled.)}
		\rei{アイスクリームを食べる人}{čovjek koji jede sladoled (sintagma)}
	\end{reibun}

	U prvom primjeru koristimo 好き kao dio imenskog predikata da opišemo アイスクリーム. To je moguće samo zato što je アイスクリーム imenica zdesna\footnotemark[1] pa može primiti česticu が. Drugi primjer je opet vrlo jednostavna rečenica koju u trećem primjeru koristimo kao opis imenice 人. U drugom primjeru, glagol たべる je predikatni oblik (završava nezavisnu rečenicu) dok je u trećem primjeru opisni oblik (završava opisnu rečenicu)\footnotemark[2]. Nominalizaciju možemo koristiti za značenja poput
	
	\begin{reibun}
		\rei{アイスクリームを食べる\underline{の}が好きです。}{Volim jesti sladoled.}
		\rei{日本語は\furigana{難}{むずか}しいと言う\underline{こと}が分かった。}{Shvatio sam da je japanski težak.}
		\rei{\furigana{彼}{かれ}が出る\underline{の}を\furigana{待}{ま}っています。}{Čekam da on iziđe.}
	\end{reibun}

	\fukudai{Razlike između の i こと}
	
	Gledajući značenja u rječniku, もの se definira kao konkretna stvar, nešto u fizičkom svijetu, dok je こと apstraktna stvar, ideja, događaj ili koncept. Međutim te definicije su u najboljem slučaju daleki podsjetnici na pravilno korištenje ovih riječi. Najbolja smjernica koja se da sročiti kao pravilo je to da se の većinom koristi za stvari koje su neposredne i/ili subjektivne, a こと za apstraktnije koncepte i situacije u kojima ne postoji direktna čvrsta veza između opisne rečenice i glavnog glagola.
	
	Slijede situacije u kojima nije dopušteno zamijeniti の i こと. Za sve situacije koje ovdje nisu pobrojane, dopušteno je (iako ne nužno uvijek prirodno) koristiti i jednu i drugu riječ.
	
	\footnotetext[1]{Po jednostavnom modelu <opis> imenica <gramatika>, \textit{imenica zdesna} označava frazu čija se desna strana s obzirom da gramatičke mogućnosti ponaša kao imenica.}
	
	\footnotetext[2]{U modernom jap. morfološka razlika između opisnog i predikatnog oblika je izgubljena za sve predikate osim imenskog s な/の pridjevima kojima je nepr. poz. predikatni oblik だ.}
	\newpage
	\fukudai{Isključivo こと}
	
	\ten Kad nominalizirani opis želimo iskoristiti kao dio imenskog predikata (uz だ/である):
	
	\begin{reibun}
		\rei{私の\furigana{趣味}{しゅみ}は\furigana{絵}{え}を\furigana{描}{か}く\underline{こと}です。}{Moj hobi je crtanje/slikanje.}
		\rei{\furigana{大事}{だいじ}なのは\furigana{生}{い}きて\furigana{帰}{かえ}る\underline{こと}です。}{Važno je vratiti se živ.}
	\end{reibun}

	Razlog zbog kojeg u ovim slučajevima obavezno koristimo こと je fiksni izraz za kraj rečenice u tonu objašnjavanja のです\footnotemark[3]. Naime, zamijenimo li こと u gornjim primjerima s の, značenje će se uvelike promijeniti i neće zvučati prirodno.
	
	\ten Kad je nominalizirani opis zapravo neupravni govor. To je gotovo uvijek slučaj s glagolima koji označavaju neki oblik misli ili komunikacije:
	
	\begin{reibun}
		\reinagai{田中さんは\furigana{間}{ま}に\furigana{合}{あ}わない\underline{こと}を\furigana{社長}{しゃちょう}に\furigana{伝}{つた}えます。}{Prenijet ću (to) da Tanaka neće stići na vrijeme šefu.}
	\end{reibun}
	
	\footnotetext[3]{Više na sljedećoj stranici.}
	
	\ten Kad je こと dio fiksnog izraza:
	
	\begin{reibun}
		\rei{そのクジラは空を飛ぶ\underline{こと}も出来る。}{Taj kit može čak i letjeti.}
		\rei{たけしくんは\furigana{転校}{てんこう}する\underline{こと}になった。}{Takeši će se prebaciti u drugu školu. (tako je ispalo)}
		\rei{たけしくんは学校を\furigana{辞}{や}める\underline{こと}にした。}{Takeši je odlučio odustati od škole. (svojevoljno)}
		\reinagai{たけしくんはクロアチアまで行った\underline{こと}があるのに、大阪に\furigana{一度}{いちど}も行った\underline{こと}がない。}{Iako je ičao čak do Hrvatske, Takeši ni jednom nije bio u Osaki.}
		\rei{\furigana{毎日}{まいにち}一キロ走る\underline{こと}にしています。}{Trudim se svaki dan trčati jedan kilometar.}
	\end{reibun}

	\fukudai{Isključivo の}
	
	\ten S glagolima percepcije:
	
	\begin{reibun}
		\rei{\furigana{隣}{となり}の\furigana{家}{いえ}で\furigana{誰}{だれ}かが\furigana{叫}{さけ}ぶ\underline{の}が\furigana{聞}{き}こえた。}{Iz susjedne kuće se čulo kako netko viče.}
		\rei{たなかくんが\furigana{逃}{に}げようとしている\underline{の}を見た。}{Vidio sam kako Tanaka pokušava uteći.}
		\rei{\furigana{指先}{ゆびさき}で何かが\furigana{動}{うご}いているのを\furigana{感}{かん}じた。}{Vrhovima prstiju sam osjetio kako se nešto miče.}
	\end{reibun}

	\newpage
	\ten Kad opis želimo koristiti kao direktni objekt nekog glagola:
	
	\begin{reibun}
		\rei{\furigana{駅}{えき}の前で彼が来る\underline{の}を待っている。}{Pred stanicom čekam da dođe.}
	\end{reibun}

	Općenito nije baš uobičajeno reći ことを izuzev primjera iznad s glagolima za komuniciranje. U nekim naprednijim situacijama ne treba slijepo gledati česticu za informaciju o ulozi riječi u rečenici, npr:
	
	\begin{reibun}
		\rei{\furigana{喧嘩}{けんか}する\underline{の}は辞めましょう。}{Samo se nemojmo svaditi. (ili tući)}
	\end{reibun}

	U primjeru iznad, čestica は zamjenjuje を i dodaje riječi 喧嘩 ulogu teme, ali je u isto vrijeme stavlja u kontrast sa svim ostalim mogućnostima (razlika između を i は što se kontrasta tiče je otprilike kao razlika između \textit{Nemojmo se svaditi} i \textit{Samo se nemojmo svaditi}), ali 喧嘩 još uvijek ostaje direktni objekt glagola 辞めましょう i zato je prirodno koristiti の.

	\ten Kao referenca na već spomenutu imenicu:
	
	\begin{reibun}
		\rei{A: どんな\furigana{車}{くるま}がほしいか。}{Kakav auto želiš?}
		\rei{B: \furigana{赤}{あか}い\underline{の}がいい。}{Neki crveni.}
	\end{reibun}

	U primjeru iznad, の se koristi umjesto ponavljanja imenice 車. Ovo je u govoru jako česta pojava, kao i u hrvatskom jeziku. Iako je mehanizam različit, nitko ne voli bespotrebno ponavljati već poznate informacije. U sljedećim primjerima ideja je ista, ali の zamjenjuje neku imenicu općenitijeg značenja (npr. とき、場所 itd.). O kojoj se točno imenici radi saznajemo u dijelu rečenice nakon の:
	
	\begin{reibun}
		\rei{たけしくんが生まれた\underline{の}は\furigana{東京}{とうきょう}です。}{Takeši se rodio u Tokiju.}
		\rei{たけしくんが生まれた\underline{の}は1991年です。}{Takeši se rodio '91.}
	\end{reibun}

	U japanskom jeziku je ovo zapravo oblik inverzije. Vrlo lako možemo preoblikovati rečenice iznad u očekivani poredak informacija:
	
	\begin{reibun}
		\rei{たけしくんは東京に生まれた。}{Takeši se rodio u Tokiju.}
		\rei{たけしくんは1991年に生まれた。}{Takeši se rodio '91.}
	\end{reibun}

	Zanimljivo, isti mehanizam možemo iskoristiti i za spoj から:
	
	\begin{reibun}
		\rei{たけしくんが行きたくなかった\underline{の}は\furigana{高}{たか}い\furigana{所}{ところ}が\furigana{怖}{こわ}いからだった。}{Takeši nije htio ići jer se boji visine.}
		\rei{たけしくんは高い所が怖いから行きたくなかった。}{Takeši nije htio ići jer se boji visine.}
	\end{reibun}

	\newpage
	\ten Kao dio fiksnog izraza za objašnjenje/uvjeravanje のだ:
	
	\begin{reibun}
		\rei{A: びしょ\furigana{濡}{ぬ}れじゃないか。}{Mokar si do kože. (wtf?)}
		\rei{B: \furigana{池}{いけ}に\furigana{落}{お}ちてしまったんだ。}{Upao sam u jezerce. (objašnjenje)}
		\reinagai{池に落ちてしまったのでびしょ濡れです。}{Mokar sam do kože jer sam upao u jezerce.}
	\end{reibun}

	Dobro je zapamtiti da spojni glagol u japanskom dolazi u kolokvijalnoj i pristojnoj inačici (だ,\linebreak です), kao i da je skraćivanje の u ん jako česta pojava, čak i u pristojnom govoru.

	Treći primjer iznad trebao bi nam biti otprije poznat kao spoj ので. Naime, upravo je ovo porijeklo tog spoja - zapravo se radi o て obliku spojnog glagola だ$\rightarrow$で.
	
\end{document}