% !TeX document-id = {bf382371-91f4-4afe-b307-d7d3d837499c}
% !TeX program = xelatex ?me -synctex=0 -interaction=nonstopmode -aux-directory=../tex_aux -output-directory=./release
% !TeX program = xelatex

\documentclass[12pt]{article}

\usepackage{lineno,changepage,lipsum}
\usepackage[colorlinks=true,urlcolor=blue]{hyperref}
\usepackage{fontspec}
\usepackage{xeCJK}
\usepackage{tabularx}
\setCJKfamilyfont{chanto}{AozoraMinchoRegular.ttf}
\setCJKfamilyfont{tegaki}{Mushin.otf}
\usepackage[CJK,overlap]{ruby}
\usepackage{hhline}
\usepackage{multirow,array,amssymb}
\usepackage[croatian]{babel}
\usepackage{soul}
\usepackage[usenames, dvipsnames]{color}
\usepackage{wrapfig,booktabs}
\renewcommand{\rubysep}{0.1ex}
\renewcommand{\rubysize}{0.75}
\usepackage[margin=50pt]{geometry}
\modulolinenumbers[2]

\usepackage{pifont}
\newcommand{\cmark}{\ding{51}}%
\newcommand{\xmark}{\ding{55}}%

\definecolor{faded}{RGB}{100, 100, 100}

\renewcommand{\arraystretch}{1.2}

%\ruby{}{}
%$($\href{URL}{text}$)$

\newcommand{\furigana}[2]{\ruby{#1}{#2}}
\newcommand{\tegaki}[1]{
	\CJKfamily{tegaki}\CJKnospace
	#1
	\CJKfamily{chanto}\CJKnospace
}

\newcommand{\dai}[1]{
	\vspace{20pt}
	\large
	\noindent\textbf{#1}
	\normalsize
	\vspace{20pt}
}

\newcommand{\fukudai}[1]{
	\vspace{10pt}
	\noindent\textbf{#1}
	\vspace{10pt}
}

\newenvironment{bunshou}{
	\vspace{10pt}
	\begin{adjustwidth}{1cm}{3cm}
	\begin{linenumbers}
}{
	\end{linenumbers}
	\end{adjustwidth}
}

\newenvironment{reibun}{
	\vspace{10pt}
	\begin{tabular}{l l}
}{
	\end{tabular}
	\vspace{10pt}
}
\newcommand{\rei}[2]{
	#1&\textit{#2}\\
}
\newcommand{\reinagai}[2]{
	\multicolumn{2}{l}{#1}\\
	\multicolumn{2}{l}{\hspace{10pt}\textit{#2}}\\
}

\newenvironment{mondai}[1]{
	\vspace{10pt}
	#1
	
	\begin{enumerate}
		\itemsep-5pt
	}{
	\end{enumerate}
	\vspace{10pt}
}

\newenvironment{hyou}{
	\begin{itemize}
		\itemsep-5pt
	}{
	\end{itemize}
	\vspace{10pt}
}

\date{\today}

\CJKfamily{chanto}\CJKnospace
\author{Tomislav Mamić}
\begin{document}
	\dai{Priložne imenice I}
	
	\fukudai{Teorija}
	
	Imenice koje ćemo proučiti u nastavku spadaju u vrlo specifičnu skupinu \textit{lažnih} ili \textit{formalnih} imenica\footnotemark[1]. Zovu se tako jer, iako ćemo ih pronaći u rječnicima, i iako će imati nerijetko mnogo raznih prijevoda, ove se imenice u praksi gotovo nikad ne koriste niti tumače kao takve - \textbf{imaju gramatičku funkciju}.
	
	\footnotetext[1]{jap. 形式名詞 - けい.しき.めい.し, dosl. \textit{formalne imenice}}
	
	\fukudai{ため - svrha, cilj, poticaj, uzrok}
	
	\noindent
	Ovu ćemo riječ naći uglavnom u jednoj od dvije upotrebe:
	
	\vspace{5pt}\noindent
	Kad označava (pozitivnu) svrhu rečenice, npr. こどものため - \textit{(ono što je dobro) za dijete}:
	
	\begin{reibun}
		\rei{こどもの ために がっこうを たてた。}{Za djecu su sagradili školu.}
		\rei{あなたの ために 言っています。}{Govorim to za tvoje dobro.}
	\end{reibun}

	\noindent\vspace{5pt}
	Isto značenje dobivamo i kad ため opišemo rečenicom:
	
	\begin{reibun}
		\reinagai{いい がっこうに 行くために 花子ちゃんは いっしょうけんめい\footnotemark[2] べんきょう した。}{Kako bi išla u dobru školu, Hanako je učila što je bolje mogla.}
		\reinagai{たけしくんは つよく なる ために まいにち しゅぎょう している。}{Takeši svaki dan trenira da bi postao jak.}
	\end{reibun}

	\footnotetext[2]{pril. izražava veliki trud subjekta u izvršenju radnje rečenice}
	
	\vspace{5pt}\noindent
	U raspoznavanju značenja uzroka u odnosu na prethodno objašnjenu upotrebu pomažu nam samo kontekst i značenje rečenica:
	
	\begin{reibun}
		\reinagai{あらしの ために しあいを えんきしました。}{Za oluju smo odgodili utakmicu. \xmark\ Zbog oluje smo odgodili utakmicu. \cmark}
	\end{reibun}

	\noindent
	Ovakva upotreba ため je vrlo formalna i u svakodnevnom se razgovoru ne pojavljuje, no često je možemo čuti u vijestima i ostalim oblicima službenog izvještavanja.
	
	\fukudai{はず - vlastito uvjerenje, pretpostavka, zaključak}
	
	Vrlo često dolazi u jednostavnom obliku <rečenica>はずだ / です, no to nije gramatičko pravilo i ima situacija gdje se koristi u složenijim rečenicama. Bitno za ovu riječ je da je uvijek iz perspektive govornika, a \textbf{ne nužno subjekta} rečenice.
	
	\begin{reibun}
		\rei{でんしゃは そろそろ つく はずです。}{Vlak bi trebao uskoro stići.}
		\rei{でんしゃは もう ついた はずです。}{Vlak je vjerojatno već stigao.}
		\rei{たけしくんが 私の りんごを 食べた はずだ。}{Sigurno mi je Takeši pojeo jabuku.}
	\end{reibun}

	\newpage\noindent
	U prethodnim rečenicama, uvjerenje govornika je trenutno važeće. Promotrimo što se dogodi sa značenjem ako glavnu rečenicu prebacimo u prošlost:
	
	\begin{reibun}
		\rei{でんしゃは そろそろ つく はずだった。}{Mislio sam da će vlak uskoro stići.}
		\rei{でんしゃは もう ついた はずだった。}{Mislio sam da je vlak već trebao stići.}
	\end{reibun}

	\noindent
	U prethodna dva primjera implicira se da je govornikovo uvjerenje promijenjeno ili poljuljano - na kraj bismo mogli dodati \textit{ali sada više tako ne mislim} ili \textit{ali nešto nije kako treba}.
	
	\vspace{5pt}\noindent
	Uz dosad pokazane upotrebe, vrlo je čest izraz \textasciitilde はずがない. Značenje tog izraza je nešto teže pogoditi iz osnovnog značenja riječi:
	
	\begin{reibun}
		\reinagai{でんしゃが もう ついた はずが ない。}{Nema šanse da je vlak već stigao.}
		\reinagai{花子ちゃんが たけしくんに すうがくを おしえる はずが ない。}{Nema šanse da će Hanako pokazati Takešiju matematiku.}
	\end{reibun}

	\noindent
	Uočimo da je u ovom slučaju simetrija s hrvatskim dosta zanimljiva - uspoređujući \textit{nema šanse} i はずが ない mogli bismo poistovjetiti hrv. \textit{šanse} i jap. はず.
	
	\fukudai{つもり - namjera}
	
	\fukudai{なか/うち - isticanje iz skupine}
\end{document}