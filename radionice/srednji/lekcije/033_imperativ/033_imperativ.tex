% !TeX program = xelatex ?me -synctex=0 -interaction=nonstopmode -aux-directory=../tex_aux -output-directory=./release
% !TeX program = xelatex

\documentclass[12pt]{article}

\usepackage{lineno,changepage,lipsum}
\usepackage[colorlinks=true,urlcolor=blue]{hyperref}
\usepackage{fontspec}[ Path =../../../ ]
\usepackage{xeCJK}
\usepackage{tabularx}
\usepackage{graphicx}
\setCJKfamilyfont{chanto}{AOZORAMINCHOREGULAR_0.TTF}%
\setCJKfamilyfont{tegaki}{Mushin.otf}%
\usepackage[CJK,overlap]{ruby}
\usepackage{hhline}
\usepackage{multirow,array,amssymb}
\usepackage[croatian]{babel}
\usepackage{soul}
\usepackage[usenames, dvipsnames]{color}
\usepackage{wrapfig,booktabs}
\usepackage{calc}
\renewcommand{\rubysep}{0.1ex}
\renewcommand{\rubysize}{0.75}
\usepackage[margin=50pt]{geometry}
\usepackage{hyperref}
\modulolinenumbers[2]

\date{\today}

\usepackage{fancyhdr}
\pagestyle{fancy}
\fancyhf{}
\fancyhead[LE,RO]{\thepage}
\makeatletter
\fancyhead[RE,LO]{rev. \@date 誠}
\makeatother

\usepackage{pifont}
\newcommand{\cmark}{\ding{51}}%
\newcommand{\xmark}{\ding{55}}%

\newcommand{\dosl}{{\normalfont dosl. }}%
\newcommand{\rem}[1]{{\normalfont #1 }}%

\definecolor{faded}{RGB}{100, 100, 100}

\renewcommand{\arraystretch}{1.2}

%\ruby{}{}
%$($\href{URL}{text}$)$

\newcommand{\furigana}[2]{\ruby{#1}{#2}}
\newcommand{\tegaki}[1]{
	\CJKfamily{tegaki}\CJKnospace
	#1
	\CJKfamily{chanto}\CJKnospace
}

\newcommand{\dai}[1]{
	\vspace{20pt}
	\large
	\noindent\textbf{#1}
	\normalsize
	\vspace{20pt}
}

\newcommand{\fukudai}[1]{
	\vspace{10pt}
	\noindent\textbf{#1}
	\vspace{10pt}
}

\newenvironment{bunshou}{
	\vspace{10pt}
	\begin{adjustwidth}{1cm}{3cm}
	\begin{linenumbers}
}{
	\end{linenumbers}
	\end{adjustwidth}
}

\newenvironment{reibun}[1][]{
	\vspace{10pt}
	#1
	
	\begin{tabular}{l l}
}{
	\end{tabular}
	\vspace{10pt}
}
\newcommand{\rei}[2]{
	#1&\textit{#2}\\
}
\newcommand{\reinagai}[2]{
	\multicolumn{2}{l}{#1}\\
	\multicolumn{2}{l}{\hspace{10pt}\textit{#2}}\\
}

\newenvironment{mondai}[1]{
	\vspace{10pt}
	\noindent #1
	
	\begin{enumerate}
		\itemsep-5pt
	}{
	\end{enumerate}
}

\newenvironment{hyou}{
	\begin{itemize}
		\itemsep-5pt
	}{
	\end{itemize}
	\vspace{10pt}
}

\newcommand{\juuyou}[2][20pt]{
	\vspace{5pt}
		\noindent\hspace{#1}\parbox[c]{\textwidth-#1-#1}{\centering\textit{#2}}
	\vspace{5pt}
}

\newcommand{\ten}{
	\vspace{5pt}
	\noindent\hspace{-10pt}$\bullet$
}

\CJKfamily{chanto}\CJKnospace

\frenchspacing
\author{Tomislav Mamić}
\begin{document}
	\dai{Imperativ}
	
	Ova glagolska forma se koristi za izražavanje naredbi ili zahtjeva. Može se koristiti među bliskim osobama (npr. roditelj-dijete) za davanje intrukcija te se također koristi u neupravnom govoru, za motiviranje i u nekim frazama.
	
	\begin{reibun}
		\rei{始めろと言われた。}{Rekli su nam da počnemo. \normalfont{(Neupravni govor)}}
		\rei{頑張れ。}{Daj sve od sebe. \normalfont{(Motivacija)}}
	\end{reibun}
	
	\fukudai{Tvorba}
	
	Tvorba ove glagolske forme se razlikuje ovisno o tipu glagola. Za いちだん glagole imperativ se tvori tako da završetak \textasciitilde る postane \textasciitilde ろ.
	
	\begin{reibun}
		\rei{道を見ろ。}{Gledaj u cestu.}
		\rei{静かに食べろ。}{Budi tiho i jedi.}
	\end{reibun}

	Za ごだん glagole slog na koji završavaju prelazi u -e oblik.
	
	\begin{reibun}
		\rei{要らない物を売れ。}{Prodaj što ti ne treba.}
		\rei{隅で立って。}{Stoj u kutu.}
	\end{reibun}

	Također imamo i nepravilne glagole koji su dani u tablici ispod zajedno s primjerima ostalih tipova glagola.

	\fukudai{Drugi oblici}
	
	Imperativ se može izraziti koristeći て formu. Dolazi od izricanja zamolbe, tipa 見てくれ ili 見てください, gdje se izostavljanjem nastavaka dobiva blaži oblik imperativa, 見て。
	
	Drugi, pristojniji oblik imperativa se dobiva koristeći dodavanjem \textasciitilde なさい na korijen glagola, kao prilikom tvorbe  \textasciitilde ます oblika.
	
	\begin{reibun}
		\rei{見なさい。}{Gledaj.}
		\rei{遊びなさい。}{Igraj.}
	\end{reibun}
	 
	\newpage		
	\fukudai{Vježba - Prevedite slijedeće rečenice:}

	\begin{mondai}{Lv. 1}
		\item 黙れ!
		\item 忘れるな!
		\item 自分を信じろ!
		\item 心配するな。
		\item 死ね!
	\end{mondai}

	\begin{mondai}{Lv. 2}
		\item もう10時だから早く寝なさいよ。
		\item みんなが待っているから早く来い。
		\item 僕の本を忘れないでくださいね。
		\item 先輩、書類の書き方を教えなさい。
		\item 早く出かける準備をしなさい。
	\end{mondai}

	\begin{mondai}{Lv. 3}
		\item 彼の言うことを信じないではいられない。
		\item この英語の問題はとてもやさしいしろものではない。
		\item 目を閉じて少しの間長椅子に横になってなさい。
		\item もうこれ以上その話を私に聞かせないでください。
		\item 助けを必要としている人には誰にでも手を貸してあげなさい。
	\end{mondai}

	\fukudai{Tablica oblika}
		\begin{table}[h]
			\centering
			\begin{tabular}{l| l l}\toprule[2pt]
				& rječnički oblik & imperativ\\
				\midrule
				いちだん glagoli & 見る & 見ろ\\
				& 食べる & 食べろ\\
				\midrule
				ごだん glagoli &泳ぐ & 泳げ\\
				& 呼ぶ  &  呼べ\\
				& 歌う  & 歌え\\
				& 待つ & 待て\\
				\midrule
				nepravilni oblici&来る & 来い\\
				& する & しろ\\
				& 下さる & 下さい\\
				\midrule
				けいご& \textasciitilde ます & \textasciitilde ませ\\
				& \textasciitilde ます & \textasciitilde なさい\\
				\bottomrule[2pt]
			\end{tabular}
		\end{table}
	
\end{document}