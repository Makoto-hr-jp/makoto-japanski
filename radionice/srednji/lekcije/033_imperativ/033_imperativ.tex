% !TeX document-id = {82dc5abd-1870-423c-ad5c-112c9df6ac63}
% !TeX program = xelatex ?me -synctex=0 -interaction=nonstopmode -aux-directory=../tex_aux -output-directory=./release
% !TeX program = xelatex

\documentclass[12pt]{article}

\usepackage{lineno,changepage,lipsum}
\usepackage[colorlinks=true,urlcolor=blue]{hyperref}
\usepackage{fontspec}
\usepackage{xeCJK}
\usepackage{tabularx}
\setCJKfamilyfont{chanto}{AozoraMinchoRegular.ttf}
\setCJKfamilyfont{tegaki}{Mushin.otf}
\usepackage[CJK,overlap]{ruby}
\usepackage{hhline}
\usepackage{multirow,array,amssymb}
\usepackage[croatian]{babel}
\usepackage{soul}
\usepackage[usenames, dvipsnames]{color}
\usepackage{wrapfig,booktabs}
\renewcommand{\rubysep}{0.1ex}
\renewcommand{\rubysize}{0.75}
\usepackage[margin=50pt]{geometry}
\modulolinenumbers[2]

\usepackage{pifont}
\newcommand{\cmark}{\ding{51}}%
\newcommand{\xmark}{\ding{55}}%

\definecolor{faded}{RGB}{100, 100, 100}

\renewcommand{\arraystretch}{1.2}

%\ruby{}{}
%$($\href{URL}{text}$)$

\newcommand{\furigana}[2]{\ruby{#1}{#2}}
\newcommand{\tegaki}[1]{
	\CJKfamily{tegaki}\CJKnospace
	#1
	\CJKfamily{chanto}\CJKnospace
}

\newcommand{\dai}[1]{
	\vspace{20pt}
	\large
	\noindent\textbf{#1}
	\normalsize
	\vspace{20pt}
}

\newcommand{\fukudai}[1]{
	\vspace{10pt}
	\noindent\textbf{#1}
	\vspace{10pt}
}

\newenvironment{bunshou}{
	\vspace{10pt}
	\begin{adjustwidth}{1cm}{3cm}
	\begin{linenumbers}
}{
	\end{linenumbers}
	\end{adjustwidth}
}

\newenvironment{reibun}{
	\vspace{10pt}
	\begin{tabular}{l l}
}{
	\end{tabular}
	\vspace{10pt}
}
\newcommand{\rei}[2]{
	#1&\textit{#2}\\
}
\newcommand{\reinagai}[2]{
	\multicolumn{2}{l}{#1}\\
	\multicolumn{2}{l}{\hspace{10pt}\textit{#2}}\\
}

\newenvironment{mondai}[1]{
	\vspace{10pt}
	#1
	
	\begin{enumerate}
		\itemsep-5pt
	}{
	\end{enumerate}
	\vspace{10pt}
}

\newenvironment{hyou}{
	\begin{itemize}
		\itemsep-5pt
	}{
	\end{itemize}
	\vspace{10pt}
}

\date{\today}

\CJKfamily{chanto}\CJKnospace
\author{Kristijan Čavić}
\begin{document}
	\dai{Imperativ}
	
	Ovaj glagolski oblik se koristi za izražavanje naredbi ili zahtjeva. Može se koristiti među bliskim osobama (npr. roditelj-dijete) za davanje instrukcija te se također koristi u neupravnom govoru\footnotemark[1], za motiviranje i u nekim frazama.
	
	\begin{reibun}
		\rei{立て。}{Ustani! \normalfont{(Naredba)}} 
		\rei{\furigana{頑張}{がんば}れ。}{Daj sve od sebe. \normalfont{(Motivacija)}}
		\rei{始めろと言われた。}{Rekli su nam da počnemo. \normalfont{(Neupravni govor)}}
	\end{reibun}
	
	\fukudai{Tvorba}
	
	Tvorba ovog glagolskog oblika se razlikuje ovisno o tipu glagola. Za いちだん glagole imperativ se tvori tako da završetak \textasciitilde る postane \textasciitilde ろ.
	
	\begin{reibun}
		\rei{ドアを\furigana{閉}{し}めろ。}{Zatvori vrata.}
		\rei{\furigana{静}{しず}かに食べろ。}{Budi tiho i jedi.}
	\end{reibun}

	Za ごだん glagole slog na koji završavaju prelazi u -e oblik.
	
	\begin{reibun}
		\rei{いらない物を売れ。}{Prodaj što ti ne treba.}
		\rei{\furigana{隅}{すみ}で立て。}{Stoj u kutu.}
	\end{reibun}

	Također postoje i nepravilni glagoli koji su dani u tablici ispod zajedno s primjerima ostalih tipova glagola.

	\fukudai{Drugi oblici}
	
	Imperativ se može izraziti koristeći て oblik. Dolazi od izricanja zamolbe, npr. 見てくれ ili 見てください, gdje se izostavljanjem nastavaka dobiva blaži oblik imperativa, 見て。
	
	Drugi, pristojniji oblik imperativa se dobiva dodavanjem \textasciitilde なさい na korijen glagola, kao prilikom tvorbe  \textasciitilde ます oblika. Koristi se u formalnim situacijama, ali i u obraćanju mlađim osobama (djeci) ili osobama nižeg statusa što, ovisno o kontekstu, se može smatrati bezobraznim.
	
	\begin{reibun}
		\rei{見なさい。}{Gledaj.}
		\rei{遊びなさい。}{Igraj.}
	\end{reibun}

	Slično て imperativu, izostavljanjem nastavaka zamolbe dobiva se negativni imperativ, npr. 見ないで. Autoritativniji oblik se dobiva dodavanjem nastavka な na rječnički oblik glagola.
	
	\begin{reibun}
		\rei{座らないで。}{Nemoj sjedit.}
		\rei{逃げるな。}{Nemoj bježat.}
	\end{reibun}
	 
	\footnotetext[1]{Neupravni govor se smatra naprednom upotrebom imperativa.} 
	\newpage		
	\fukudai{Vježba}

	\begin{mondai}{Lv. 1 - 次の文を日本語に訳しなさい:}
		\item \furigana{黙}{だま}れ!
		\item 忘れるな!
		\item 自分を信じろ!
		\item 心配するな。
		\item 死ね!
	\end{mondai}

	\begin{mondai}{Lv. 2 - 次の文を日本語に訳しなさい:}
		\item もう10時だから早く寝なさいよ。
		\item みんなが待っているから早く来い。
		\item 僕の本を忘れないでくださいね。
		\item \furigana{新人}{しん.じん}、ちゃんと\furigana{書類}{しょ.るい}を書き込みなさい。
		\item 早く出かける\furigana{準備}{じゅん.び}をしなさい。
	\end{mondai}

	\begin{mondai}{Lv. 3 - 次の文を日本語に訳しなさい:}
		\item お父さんはもう前みたいに元気じゃないということを忘れないでね。
		\item お金のことで\furigana{口論}{こう.ろん}するのはやめろ。
		\item 目を閉じて少しの間\furigana{長椅子}{なが.い.す}に\furigana{横}{よこ}になってなさい。
		\item もうこれ以上その話を私に聞かせないでください。
		\item 助けを\furigana{必要}{ひつ.よう}としている人には誰でも手を\furigana{貸}{か}してあげなさい。
	\end{mondai}

	\newpage		
	\fukudai{Tablica oblika}
	
	\noindent
	\begin{table}[h]
	\centering
	\begin{tabular}{llllll}
		\toprule[2pt]
		& 辞書形 & 命令形 & 例 & & \\
		\midrule
		一段 & 〜る & 〜ろ & 食べる&→&食べろ\\
		\midrule
		五段 & 〜く & 〜け & 行く&→&行け\\
		& 〜ぐ & 〜げ & 泳ぐ&→&泳げ\\
		& 〜す & 〜せ & 話す&→&話せ\\
		& 〜つ & 〜て & 持つ&→&持て\\
		& 〜う & 〜え & 追う&→&追え\\
		& 〜ぶ & 〜べ & 選ぶ&→&選べ\\
		& 〜む & 〜め & 読む&→&読め\\
		& 〜る & 〜れ & 切る&→&切れ\\
		& 〜ぬ & 〜ね & 死ぬ&→&死ね\\
		\midrule
		不規則	& & & 来る &→& 来い \\
		& & & する &→& しろ \\
		& & & ある &→& あれ \\
		\midrule
		否定形 & &+な & 死ぬ &→&死ぬな\\
		& & & 行く &→& 行くな \\
		\bottomrule[2pt]
	\end{tabular}
	\end{table}
\end{document}