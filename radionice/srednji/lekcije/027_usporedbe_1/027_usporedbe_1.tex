% !TeX document-id = {3d467f0e-c9db-430c-baf0-1ce20601c1ea}
% !TeX program = xelatex ?me -synctex=0 -interaction=nonstopmode -aux-directory=../tex_aux -output-directory=./release
% !TeX program = xelatex

\documentclass[12pt]{article}

\usepackage{lineno,changepage,lipsum}
\usepackage[colorlinks=true,urlcolor=blue]{hyperref}
\usepackage{fontspec}
\usepackage{xeCJK}
\usepackage{tabularx}
\setCJKfamilyfont{chanto}{AozoraMinchoRegular.ttf}
\setCJKfamilyfont{tegaki}{Mushin.otf}
\usepackage[CJK,overlap]{ruby}
\usepackage{hhline}
\usepackage{multirow,array,amssymb}
\usepackage[croatian]{babel}
\usepackage{soul}
\usepackage[usenames, dvipsnames]{color}
\usepackage{wrapfig,booktabs}
\renewcommand{\rubysep}{0.1ex}
\renewcommand{\rubysize}{0.75}
\usepackage[margin=50pt]{geometry}
\modulolinenumbers[2]

\usepackage{pifont}
\newcommand{\cmark}{\ding{51}}%
\newcommand{\xmark}{\ding{55}}%

\definecolor{faded}{RGB}{100, 100, 100}

\renewcommand{\arraystretch}{1.2}

%\ruby{}{}
%$($\href{URL}{text}$)$

\newcommand{\furigana}[2]{\ruby{#1}{#2}}
\newcommand{\tegaki}[1]{
	\CJKfamily{tegaki}\CJKnospace
	#1
	\CJKfamily{chanto}\CJKnospace
}

\newcommand{\dai}[1]{
	\vspace{20pt}
	\large
	\noindent\textbf{#1}
	\normalsize
	\vspace{20pt}
}

\newcommand{\fukudai}[1]{
	\vspace{10pt}
	\noindent\textbf{#1}
	\vspace{10pt}
}

\newenvironment{bunshou}{
	\vspace{10pt}
	\begin{adjustwidth}{1cm}{3cm}
	\begin{linenumbers}
}{
	\end{linenumbers}
	\end{adjustwidth}
}

\newenvironment{reibun}{
	\vspace{10pt}
	\begin{tabular}{l l}
}{
	\end{tabular}
	\vspace{10pt}
}
\newcommand{\rei}[2]{
	#1&\textit{#2}\\
}
\newcommand{\reinagai}[2]{
	\multicolumn{2}{l}{#1}\\
	\multicolumn{2}{l}{\hspace{10pt}\textit{#2}}\\
}

\newenvironment{mondai}[1]{
	\vspace{10pt}
	#1
	
	\begin{enumerate}
		\itemsep-5pt
	}{
	\end{enumerate}
	\vspace{10pt}
}

\newenvironment{hyou}{
	\begin{itemize}
		\itemsep-5pt
	}{
	\end{itemize}
	\vspace{10pt}
}

\date{\today}

\CJKfamily{chanto}\CJKnospace
\author{Tomislav Mamić}
\begin{document}
	\dai{Usporedbe I}
	
	\fukudai{Uvod}
	
	Iako je u japanske pridjeve upakirano više informacija nego u hrvatskom jeziku, naši pridjevi obavljaju jednu funkciju koju japanski nemaju - usporedbu. Oblici pridjeva za usporedbu zovu se \textit{komparativ} (npr. \textit{dobar} $\rightarrow$ \textit{bolji}) i superlativ (npr. \textit{dobar} $\rightarrow$ \textit{najbolji}). Uočimo kako za upotrebu komparativa trebamo dvije stvari - kažemo li da je nešto \textit{bolje}, prirodno je pitati se \textit{od čega}.
	
	\fukudai{Čestica より za usporedbu}
	
	U jednostavnom slučaju, česticom より možemo dati odgovor na pitanje \textit{od čega} iz prethodnog odlomka. Međutim, ova čestica ima nekoliko različitih upotreba pa će nam za ispravno tumačenje biti potreban i predikat uz kojeg se veže. Ako je predikat uz kojeg se čestica より veže pridjev, onda taj pridjev shvaćamo kao komparativ, a ono što je označeno česticom kao odgovor na pitanje \textit{od čega}. Pogledajmo primjere:
	
	\begin{reibun}
		\rei{猫は\underline{犬より}かわいい。}{Mačke su slatkije\footnotemark[1] \underline{od pasa}.}
		\rei{武くんは\underline{花子ちゃんより}背が高い。}{Takeši je viši \underline{od Hanako}.}
	\end{reibun}

	\footnotetext[1]{Jer \textit{slađe} ima potencijalno neugodne implikacije.}
	
	Uzevši u obzir sličnu prirodu pridjeva i priloga, nije iznenađujuće da na sličan način možemo uspoređivati i opise radnji. Na koja pitanja odgovaraju podcrtane riječi u sljedećim primjerima?
	
	\begin{reibun}
		\rei{犬は猫より\underline{たくさん}食べる。}{Psi jedu \underline{više} od mačaka.}
		\rei{花子ちゃんは武くんより数学が\underline{よく}分かる。}{Hanako razumije matematiku \underline{bolje} nego Takeši.}
	\end{reibun}

	U japanskom je uobičajeno reći da (nam) je nešto bolje od nečega, misleći u kontekstu na razna srodna značenja koja nisu povezana s onim što bismo očekivali od hrvatskog prijevoda:
	
	\begin{reibun}
		\rei{犬を買ってあげようか?}{Hoćeš li da ti kupim psa?}
		\rei{犬より猫がいい。}{Radije bih mačku nego psa. \dosl Mačka je bolja od psa.}
	\end{reibun}
	
	Međutim, ovo nije čvrsto vezano uz česticu より - često ćemo u govornom jeziku tako izražavati svoje želje i preference:
	
	\begin{reibun}
		\rei{何が食べたい?}{Što želiš jesti?}
		\rei{アイスクリームがいい。}{Može sladoled. \dosl Sladoled je dobar.}
	\end{reibun}

	Kad govorimo o dvije stvari, često je običaj na njih se referirati imenicom ほう(方). U hrvatskom sličnu funkciju obavljaju riječi \textit{jedan} i \textit{drugi}; prirodno nam je govoreći o dvije stvari reći da je \textit{jedna} ovakva, a \textit{druga} onakva. S obzirom da u usporedbi gotovo uvijek pričamo o dvije stvari, često se uz drugu od dvije stvari - (onu koja je \textit{veća}, \textit{bolja}, \textit{viša} itd.) pojavljuje ほう. Što pristojniji želimo biti, to je veća vjerojatnost da će se ほう pojaviti. Pogledajmo primjere:
	
	\begin{reibun}
		\rei{私は猫より\underline{犬のほうが}好きです。}{Meni su \underline{psi} draži od mačaka.}
		\rei{お寿司よりも\underline{アイスクリームのほうが}食べたいです。}{\underline{Sladoled} mi se jede više čak i od sušija.}
	\end{reibun}

	\fukudai{Priložna imenica くらい / ぐらい}
	
	Za potrebe razumijevanja načina na koji funkcionira, korisno je gledati na くらい i kao na prilog, i kao na priložnu imenicu. Od srednjeg vijeka ovamo, upotreba riječi se promijenila, pa se paralelno vuku i stare i nove strukture u kojima se pojavljuje. Osnovna funkcija je kao nastavak na količinu koji dodaje značenje \textit{otprilike}:
	
	\begin{reibun}
		\rei{8時に起きた。}{Ustao sam u osam.}
		\rei{八時\underline{ぐらい}に起きた。}{Ustao sam \underline{oko} osam.}
	\end{reibun}

	Originalna značenja imenice su \textit{pozicija} ili \textit{red veličine}, a još se i danas koriste. Iz značenja \textit{red veličine} izvedeno je značenje \textit{otprilike}, kao i tumačenja opisnih rečenica vezanih uz くらい. Pogledajmo primjere:
	
	\begin{reibun}
		\reinagai{犬は、猫と同じくらい好きです。}{Volim pse otprilike koliko i mačke.}
		\reinagai{夢で食べきれないくらいのお寿司を見た。}{U snu sam vidio toliko sušija da ga ne možeš svog pojesti.}
		\reinagai{鈴木さんは私と同じくらいの高さで、年も近い。}{Suzuki je visok otprilike kao ja, a i po godinama smo si tu negdje.}
	\end{reibun}

	Vrlo je bitno primijetiti da くらい koristimo kad nas (najčešće) količina ne iznenađuje. Ponekad ćemo iz stilskih razloga prekršiti to pravilo, ali implikacija da smo takvo nešto i očekivali je prisutna.
	
	\fukudai{Priložna imenica ほど}
	
	Upotrebom i povijesnim razvojem vrlo slična くらい. Za razliku od くらい, originalno značenje upućuje na raspon ili granice (najčešće) neke količine. Zbog toga ćemo upotrebu ほど preferirati kad govorimo o rasponima, krajnjim granicama ili količinama koje nas iznenađuju. Pogledajmo:
	
	\begin{reibun}
		\reinagai{八時間ほど前に日本に来ました。}{Došao sam u Japan prije otprilike 8 sati.}
		\reinagai{あの箱の中には百人が食べても食べきれないほどのリンゴがあった。}{U onoj kutiji je bilo toliko jabuka da ih ni 100 ljudi ne bi pojelo.}
		\reinagai{犬は、猫ほど好きじゃない。}{Ne volim pse toliko koliko mačke. \textnormal{(ali ih ipak volim)}}
		\reinagai{夢で食べきれないほどのお寿司を見た。}{U snu sam vidio toliko sušija da ga ne možeš svog pojesti. \textnormal{(više smo impresionirani time)}}
		\reinagai{鈴木さんは私ほどの高さで、年も近い。}{Suzuki je visok otprilike kao ja, a i po godinama smo si tu negdje. \textnormal{(a ja sam dosta visok)}}
	\end{reibun}

	\newpage
	\fukudai{Vježba}
	
	\begin{mondai}{Lv. 1}
		\item ウサギは食べたくなるほどかわいい。
		\item 死ぬほど\furigana{怖}{こわ}かった。
		\item 武くんは\furigana{学校}{が.こう}よりゲームに\furigana{興味}{きょう.み}がある。
	\end{mondai}

	\begin{mondai}{Lv. 2}
		\item \furigana{武}{たけし}くんが\furigana{家}{いえ}を出たのは7時ぐらいだった。
		\item 花子ちゃんが\furigana{大声}{おお.ごえ}を出すほど\furigana{怒}{おこ}ったので、プライドの\furigana{高}{たか}い武くんも\furigana{謝}{あやま}ることにした。
	\end{mondai}

	\begin{mondai}{Lv. 3}
		\item 武は\furigana{今回}{こん.かい}、いつも\furigana{冷静}{れい.せい}な花子ちゃんが大声を出すくらいのことをしてしまった。
	\end{mondai}
\end{document}