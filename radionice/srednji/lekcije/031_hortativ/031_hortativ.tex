% !TeX document-id = {f9d4070c-8bf0-485d-a134-28d7af22f5f7}
% !TeX program = xelatex ?me -synctex=0 -interaction=nonstopmode -aux-directory=../tex_aux -output-directory=./release
% !TeX program = xelatex

\documentclass[12pt]{article}

\usepackage{lineno,changepage,lipsum}
\usepackage[colorlinks=true,urlcolor=blue]{hyperref}
\usepackage{fontspec}
\usepackage{xeCJK}
\usepackage{tabularx}
\setCJKfamilyfont{chanto}{AozoraMinchoRegular.ttf}
\setCJKfamilyfont{tegaki}{Mushin.otf}
\usepackage[CJK,overlap]{ruby}
\usepackage{hhline}
\usepackage{multirow,array,amssymb}
\usepackage[croatian]{babel}
\usepackage{soul}
\usepackage[usenames, dvipsnames]{color}
\usepackage{wrapfig,booktabs}
\renewcommand{\rubysep}{0.1ex}
\renewcommand{\rubysize}{0.75}
\usepackage[margin=50pt]{geometry}
\modulolinenumbers[2]

\usepackage{pifont}
\newcommand{\cmark}{\ding{51}}%
\newcommand{\xmark}{\ding{55}}%

\definecolor{faded}{RGB}{100, 100, 100}

\renewcommand{\arraystretch}{1.2}

%\ruby{}{}
%$($\href{URL}{text}$)$

\newcommand{\furigana}[2]{\ruby{#1}{#2}}
\newcommand{\tegaki}[1]{
	\CJKfamily{tegaki}\CJKnospace
	#1
	\CJKfamily{chanto}\CJKnospace
}

\newcommand{\dai}[1]{
	\vspace{20pt}
	\large
	\noindent\textbf{#1}
	\normalsize
	\vspace{20pt}
}

\newcommand{\fukudai}[1]{
	\vspace{10pt}
	\noindent\textbf{#1}
	\vspace{10pt}
}

\newenvironment{bunshou}{
	\vspace{10pt}
	\begin{adjustwidth}{1cm}{3cm}
	\begin{linenumbers}
}{
	\end{linenumbers}
	\end{adjustwidth}
}

\newenvironment{reibun}{
	\vspace{10pt}
	\begin{tabular}{l l}
}{
	\end{tabular}
	\vspace{10pt}
}
\newcommand{\rei}[2]{
	#1&\textit{#2}\\
}
\newcommand{\reinagai}[2]{
	\multicolumn{2}{l}{#1}\\
	\multicolumn{2}{l}{\hspace{10pt}\textit{#2}}\\
}

\newenvironment{mondai}[1]{
	\vspace{10pt}
	#1
	
	\begin{enumerate}
		\itemsep-5pt
	}{
	\end{enumerate}
	\vspace{10pt}
}

\newenvironment{hyou}{
	\begin{itemize}
		\itemsep-5pt
	}{
	\end{itemize}
	\vspace{10pt}
}

\date{\today}

\CJKfamily{chanto}\CJKnospace
\author{Tomislav Mamić}
\begin{document}
	\dai{Hortativ}
	
	U japanskom često \furigana{意志形}{い.し.けい},\footnotemark[1] vrlo je star oblik glagola.
	Zbog svoje starosti, oblik je s promjenama u jeziku nakupio raznolika značenja i okamenio se u raznim izrazima pa je nezahvalno govoriti o jednom jedinstvenom tumačenju.
	
	\footnotetext[1]{Ili 意向形 (い.こう.けい), u eng. lingvističke izraze se preslikava na više upotreba: \textit{volitional}, \textit{hortative}, \textit{presumptive}.}
	
	\fukudai{Teorija}
	
	U starom jeziku nastavak je bio \textasciitilde む, a dodavao se na \furigana{未然形}{み.ぜん.けい}; bazu glagola koja za 五段 završava samoglasnikom \textit{a} i na koju danas dodajemo negaciju, pasiv itd. Tokom Edo perioda, nastavak se iz む promijenio u う, a kako su samoglasnici \textit{a} i \textit{u} znatno različiti po položaju i obliku usta, zadnje \textit{a} se s vremenom pomaknulo prema \textit{o}.
	
	\begin{table}[h]
		\centering
		\begin{tabular}{llllll}
			\toprule[2pt]
			& 辞書形 & 意志形 & 例 & & \\
			\midrule
			一段 & 〜る & 〜よう & 食べる&→&食べよう \\
			& & & 見る&→&見よう \\
			\midrule
			五段 & 〜く & 〜こう & 行く&→&行こう\\
			& 〜ぐ & 〜ごう & 泳ぐ&→&泳ごう\\
			& 〜す & 〜そう & 話す&→&話そう\\
			& 〜つ & 〜とう & 持つ&→&持とう\\
			& 〜う & 〜おう & 追う&→&追おう\\
			& 〜ぶ & 〜ぼう & 選ぶ&→&選ぼう\\
			& 〜む & 〜もう & 読む&→&読もう\\
			& 〜る & 〜ろう & 切る&→&切ろう\\
			& 〜ぬ & 〜のう & 死ぬ&→&死のう\\
			\midrule
			不規則 & & & ある &→& あろう \\
			& & & くる &→& こよう \\
			& & & する &→& しよう \\
			\midrule
			形容詞 & 〜い & 〜かろう & 美しい&→&美しかろう \\
			\midrule
			助動詞 & & & だ &→& だろう \\
			& & & です &→& でしょう \\
			& 〜ます & 〜ましょう & 行きます&→&行きましょう \\
			\bottomrule[2pt]
		\end{tabular}
	\end{table}

	Na našu veliku sreću, tvorba je potpuno pravilna za 五段 glagole - dovoljno je samoglasnik zadnje m\={o}re promijeniti u \textit{o} i dodati う.
	Za 一段 glagole prolazimo očekivano jeftino - samo mijenjajući る za よう.
	
	\newpage
	\fukudai{Poziv ili prijedlog}
	
	Kad rečenicu koja završava ovim oblikom uputimo sugovornicima, oni je tumače kao prijedlog ili poziv na radnju. Uočimo kako je razlika između nagađanja u starom obliku i poziva jedino u kontekstu.
	
	\begin{reibun}
		\rei{すしを食べに行こう。}{Hajmo na suši.}
		\rei{人に\furigana{指}{ゆび}を\furigana{指}{さ}すのを\furigana{辞}{や}めましょう。}{Hajmo ne pokazivati prstom na ljude.}
	\end{reibun}
	
	\fukudai{Namjera}
	
	Kažemo li iste rečenice iz prethodnog odlomka sami sebi (ne usmjereno nekom sugovorniku, iako nas sugovornik možda čuje), bit će protumačene kao namjera. Razlika je ovdje sasvim kontekstualna i u načinu na koji kažemo - možemo na takvu situaciju gledati kao da samom sebi dajemo prijedlog. Ovakvo će se značenje pojavljivati pretežno u govornom, kolokvijalnom jeziku.
	
	\begin{reibun}
		\rei{すしを食べに行こう。}{Ah, idem na suši.}
		\rei{人に指を指すのを辞めよう。}{Trebao bih prestati pokazivati prstom na ljude.}
	\end{reibun}

	Da se radi o namjeri možemo dodatno naglasiti citiranjem rečenog kao vlastite misli s \textasciitilde と思う. Ostavimo li kraj u nesvršenom obliku, možemo reći da nešto \textit{razmatramo} kao opciju.
	
	\begin{reibun}
		\rei{すしを食べに行こうと思う。}{Mislim ići na suši.}
		\rei{人に指を指すのを辞めようと思っている。}{Razmatram više ne upirati prstom u ljude.}
	\end{reibun}

	\fukudai{Izvorno značenje - nagađanje bez pokrića\footnotemark[2]}
	
	\footnotetext[2]{Bez objektivnog razloga na kojem bi govornik temeljio nagađanje - po osjećaju, intuiciji.}
	
	Ovakvo se značenje sve rjeđe sreće u svakodnevnom govoru i zvuči pomalo arhaično. Danas se za sva nagađanja ove vrste koristi だろう iza predikatnog oblika glavnog glagola. Pogledajmo primjere:
	
	\begin{reibun}
		\rei{あした、雨が\underline{ふろう}。}{Valjda će sutra kiša. \textnormal{(teatralno, arhaično)}}
		\rei{あした、雨が\underline{ふるだろう}。}{Valjda će sutra kiša.}
		\rei{武はここへ\underline{来よう}。}{Takeši će valjda doći ovdje. \textnormal{(teatralno, arhaično)}}
		\rei{武はここへ\underline{来るだろう}。}{Takeši će valjda doći ovdje.}
		\rei{\furigana{西洋人}{せい.よう.じん}の食べ物は美味しく\underline{なかろう}。}{Zapadnjačka hrana valjda nije fina. \textnormal{(vrlo arhaično)}}
		\rei{西洋人の食べ物は美味しく\underline{ないだろう}。}{Zapadnjačka hrana valjda nije fina.}
	\end{reibun}
	
	U arhaičnoj verziji ovaj oblik ima ograničenje kojeg smo u modernom jeziku oslobođeni - ne možemo ga lijepo koristiti za pretpostavke u prošlosti.
	
	\begin{reibun}
		\rei{私のケーキも食べた\underline{だろう}。}{Pojeo si i moju tortu, zar ne.}
		\rei{私のケーキも食べた\underline{ろう}。}{Pojeo si i moju tortu, zar ne. \textnormal{(kolokvijalno)\footnotemark[3]}}
	\end{reibun}
	
	\footnotetext[3]{Vrlo često kraćenje na Tokijskom govornom području, događa se \textbf{samo iza prošlog oblika} predikata da se izbjegnu dva uzastopna suglasnika \textit{t} / \textit{d}.}
	
	\newpage
	\fukudai{Pokušaj s \textasciitilde とする}
	
	Dodamo li na 意志形 i \textasciitilde とする, možemo izraziti pokušaj. Na hrvatskom \textit{pokušavamo} mnogo različitih stvari, ali u japanskom su pokušaji nešto precizniji. Za razliku od \textasciitilde てみる kojeg već vjerojatno znamo, a koji izražava \textit{pokus} više nego \textit{pokušaj},\footnotemark[4] ovdje je vršitelju radnje stalo do izvršenja radnje.
	
	\footnotetext[4]{Za potrebe razlikovanja ova dva oblika, \textit{pokus} definiramo kao radnju za koju znamo da je možemo izvršiti, ali ne znamo koji će biti krajnji rezultat, dok u \textit{pokušaj} brojimo radnju u čije izvršenje nismo sigurni.}
	
	\begin{reibun}
		\rei{花を\furigana{取}{と}ろうとした。}{Pokušao sam ubrati cvijet.}
		\rei{猫がネズミを\furigana{捕}{つか}まえようとしている。}{Mačka pokušava uloviti miša.}
	\end{reibun}

	Značenje ovog pokušaja je vrlo precizno u vremenu - uvijek imamo osjećaj da govorimo o trenutku prije nego radnja uspije ili propadne. Zato ovaj oblik (uglavnom) nećemo koristiti kad želimo pričati o nečemu na čemu već neko vrijeme dugo radimo ili što ćemo pokušati učiniti daleko u budućnosti. Uočimo kako u prošlosti nemamo takvih problema.

	\begin{reibun}
		\rei{5年後、日本へ\furigana{留学}{りゅう.がく}しようとする。}{Za pet godina ću pokušati studirati u Japanu. \xmark}
		\reinagai{いい\furigana{仕事}{し.ごと}を\furigana{確保}{かく.ほ}するため、留学しようとしていた。}{Pokušavao sam otići vani na studij kako bih si osigurao dobar posao. \cmark}
	\end{reibun}
	
	Zbog svoje uske povezanosti s vremenom, ponekad\footnotemark[5] ovakvi izrazi mogu izražavati da će se nešto upravo dogoditi - ne moraju imati nikakve veze s pokušajem! Ovo je vrlo često u pisanom jeziku i književnim djelima.
	
	\footnotetext[5]{Uvijek ako se radi o glagolima koji nemaju svjesnog vršitelja. Najčešći su takvi glagoli neprijelazni oblici parova kao 始める (svjesni vršitelj nešto \textit{započinje}) nasuprot 始まる (nešto \textit{započinje} samo od sebe).}
	
	\begin{reibun}
		\rei{\furigana{四月}{し.がつ}が\furigana{始}{はじ}まろうとする。}{Četvrti mjesec samo što nije počeo.}
		\reinagai{風が\furigana{強}{つよ}まり、\furigana{嵐}{あらし}が始まろうとしていた。}{Vjetar je jačao, oluja samo što nije bila počela. \normalfont{(pričamo o događaju u prošlosti!)}}
	\end{reibun}

	Posljedično, neke situacije mogu prenijeti značenje znatno drugačije od onog koje smo možda htjeli. Iz istog razloga vrlo je često koristiti ih kao opis trenutka za priložne oznake vremena.
	
	\begin{reibun}
		\rei{\furigana{卒業}{そつ.ぎょう}しようとしている。}{Pokušavam diplomirati. \xmark\hspace{15pt} Samo što nisam diplomirao. \cmark}
		\reinagai{電車に\furigana{乗}{の}ろうとしたとき、\furigana{後}{うし}ろから花子ちゃんの声が聞こえた。}{%
			\parbox{283pt}{Baš kad sam pokušao ući na vlak, čuo sam Hanako. \xmark\newline%
			Čuo sam Hanako taman kad sam htio ući na vlak. \cmark}
		}
	\end{reibun}
	
	\newpage
	\fukudai{Unatoč uz čestice が, と, とも}
	
	Na ovaj se oblik možemo izravno nastaviti nekom od čestica iz naslova odlomka, čime povezujemo dvije rečenice.
	Dobiveno značenje ugrubo odgovara vezniku \textit{unatoč}, a spada u nešto sofisticiraniju upotrebu jezika.
	Uočimo kako je uobičajeno ovako spojene rečenice odvojiti zarezom.
	
	\begin{reibun}
		\reinagai{たとえ\furigana{反対}{はん.たい}されようが、自分のことは自分で\furigana{決}{き}める。}{Sam ću odlučivati o sebi unatoč protivljenju. \rem{(drugih)}}
		\reinagai{\furigana{独}{ひと}りが\furigana{寂}{さび}しかろうが、猫がいれば大丈夫。}{Ako si i usamljen sam sa sobom, mačka sve popravlja.}
		\reinagai{風邪だろうと\furigana{熱}{ねつ}だろうと、明日休むわけにはいかない。}{Dobio prehladu ili temperaturu, ne mogu si priuštiti da sutra odmaram.}
	\end{reibun}
	
	\fukudai{Nemogućnost uz čestice に, にも}
	
	Nadovežemo li se na 意志形 česticom にも (ponekad skraćeno u samo に), izražavamo nemogućnost da učinimo neku radnju.
	Ovakav je oblik obično vrlo krut, tako da se isti glagol ponavlja dvaput.
	Uz izražavanje nemogućnosti ili nesposobnosti da nešto učini, govornik prenosi i svoju frustraciju time - koristimo ovaj oblik za radnje koje želimo izvršiti, ali jednostavno ne ide.
	
	\begin{reibun}
		\reinagai{\furigana{暑}{あつ}くて寝ようにも寝られない。}{Vruće je i ne mogu spavati. \rem{(i to me frustrira)}}
		\reinagai{仕事が終わらず帰ろうにも帰れない。}{Posao nikako da završi i ne mogu doma. \rem{(jako želim doma)}}
		\reinagai{この部屋を出ように\furigana{鍵}{かぎ}がどこにあるか思い出せない。}{Hoću van iz ove sobe, a ne mogu se sjetiti gdje je ključ.}
	\end{reibun}
	
	\newpage
	\fukudai{Vježba}
	
	\noindent
	Prevedimo na hrvatski:
	
	\begin{mondai}{Lv. 1}
		\item また私のケーキを食べただろう。
		\item 今度の夏休み、海へ行こう。
		\item 学生生活を一緒に楽しもう。
	\end{mondai}

	\begin{mondai}{Lv. 2}
		\item また私のケーキを食べようとしただろう。
		\item 今度の夏休み、武くんは花子ちゃんを海に誘おうと思っている。
		\item 学生生活を一緒に楽しむ気がなかろう。
	\end{mondai}

	\begin{mondai}{Lv. 3}
		\item たとえ私のケーキが食べられようとも、私はくじけたりしない。
		\item 振られるのが怖くて、花子ちゃんを海に誘おうにも誘えない。
		\item 学生生活を一緒に楽しむ気のないやつは放っておこうと思う。
	\end{mondai}

	\begin{mondai}{Lv. 4*}
		\item よかろう。
		\item 勉強しようともせずあきらめる人は多い。
		\item 武君の赤くなった巨大な鼻を目の前にした\footnotemark[6]花子ちゃんは、真面目に聞こうにも話が頭に入ってこなかった。
	\end{mondai}

	\footnotetext[6]{Izraz <nešto>を目の前にする stilski odgovara hrvatskom \textit{naći se oči u oči s} <nečim>.}
	
\end{document}