% !TeX document-id = {9fd79182-d6f3-4838-8278-4b23f44e9024}
% !TeX program = xelatex ?me -synctex=0 -interaction=nonstopmode -aux-directory=../tex_aux -output-directory=./release
% !TeX program = xelatex

\documentclass[12pt]{article}

\usepackage{lineno,changepage,lipsum}
\usepackage[colorlinks=true,urlcolor=blue]{hyperref}
\usepackage{fontspec}
\usepackage{xeCJK}
\usepackage{tabularx}
\setCJKfamilyfont{chanto}{AozoraMinchoRegular.ttf}
\setCJKfamilyfont{tegaki}{Mushin.otf}
\usepackage[CJK,overlap]{ruby}
\usepackage{hhline}
\usepackage{multirow,array,amssymb}
\usepackage[croatian]{babel}
\usepackage{soul}
\usepackage[usenames, dvipsnames]{color}
\usepackage{wrapfig,booktabs}
\renewcommand{\rubysep}{0.1ex}
\renewcommand{\rubysize}{0.75}
\usepackage[margin=50pt]{geometry}
\modulolinenumbers[2]

\usepackage{pifont}
\newcommand{\cmark}{\ding{51}}%
\newcommand{\xmark}{\ding{55}}%

\definecolor{faded}{RGB}{100, 100, 100}

\renewcommand{\arraystretch}{1.2}

%\ruby{}{}
%$($\href{URL}{text}$)$

\newcommand{\furigana}[2]{\ruby{#1}{#2}}
\newcommand{\tegaki}[1]{
	\CJKfamily{tegaki}\CJKnospace
	#1
	\CJKfamily{chanto}\CJKnospace
}

\newcommand{\dai}[1]{
	\vspace{20pt}
	\large
	\noindent\textbf{#1}
	\normalsize
	\vspace{20pt}
}

\newcommand{\fukudai}[1]{
	\vspace{10pt}
	\noindent\textbf{#1}
	\vspace{10pt}
}

\newenvironment{bunshou}{
	\vspace{10pt}
	\begin{adjustwidth}{1cm}{3cm}
	\begin{linenumbers}
}{
	\end{linenumbers}
	\end{adjustwidth}
}

\newenvironment{reibun}{
	\vspace{10pt}
	\begin{tabular}{l l}
}{
	\end{tabular}
	\vspace{10pt}
}
\newcommand{\rei}[2]{
	#1&\textit{#2}\\
}
\newcommand{\reinagai}[2]{
	\multicolumn{2}{l}{#1}\\
	\multicolumn{2}{l}{\hspace{10pt}\textit{#2}}\\
}

\newenvironment{mondai}[1]{
	\vspace{10pt}
	#1
	
	\begin{enumerate}
		\itemsep-5pt
	}{
	\end{enumerate}
	\vspace{10pt}
}

\newenvironment{hyou}{
	\begin{itemize}
		\itemsep-5pt
	}{
	\end{itemize}
	\vspace{10pt}
}

\date{\today}

\CJKfamily{chanto}\CJKnospace
\author{Tomislav Mamić}
\begin{document}
	\dai{Kauzativ}
		
	
	Ova glagolska forma se koristi u dva konteksta:
	
	1) kada se želi natjerati nekoga da nešto učini
	
	2) kada se želi dopustiti nekome da nešto učini\\
	
	Pretvorba glagola u kauzativ ovisi o njegovom tipu.
	Za いちだん glagole u rječničkom obliku se odvoji njegov korijen te se nadoda させる.
	Za ごだん glagole se zadnji slog u rječničkom obliku pretvara u -a oblik te se doda せる.
	Također postoji i kratki oblik kauzativa gdje se umjesto せる koristi す kao prefiks.
	
	\begin{reibun}
		\rei{先生が学生に宿題をたくさんさせた。}{Učitelj je natjerao učenike da pišu puno zadaće.}
		\rei{先生が質問をたくさん聞かせてくれた。}{Učitelj je dopustio (nekome) da pita puno pitanja.}
		\rei{今日は仕事を休ませてください。}{Dopustite mi da se danas odmorim od posla.}
		\rei{その部長は、よく長時間働かせる。}{Voditelj odjela često tjera (zaposlenike) da rade puno sati.}
	\end{reibun}
	
	Postoje dva izuzetka u navedenim pravilima, a to su za glagole する i くる.
	
	\begin{table}[h]
		\centering
		\begin{tabular}{l l}\toprule[2pt]
			rječnički oblik & kauzativ / kratka forma\\
			\midrule
			\textasciitilde する & \textasciitilde させる / さす\\
			\textasciitilde くる &\textasciitilde こさせる / こさす\\
			\bottomrule[2pt]
		\end{tabular}
	\end{table}
	
	\newpage
	\fukudai{Kauzativ pasiv}
	
	\fukudai{Tvorba}
		
	Glagol se iz riječničkog oblika pretvara u kauzativ pasiv tako da se glagol najprije pretvori u oblik za kauzativ, kao u gore navedenim primjerima, te se zatim prema glagolu u dobivenom obliku odnosimo kao da je u riječničkom obliku i pretvorimo u pasivnu formu. Pogledajmo to u slijedećim primjerima:
		
	\begin{table}[h]
		\centering
		\begin{tabular}{l l l}\toprule[2pt]
			rječnički oblik & kauzativ & kauzativ-pasiv\\
			\midrule
			\textasciitilde る & \textasciitilde させる & \textasciitilde させられる\\
			\textasciitilde く & \textasciitilde かせる & \textasciitilde かせられる\\
			\textasciitilde ぐ & \textasciitilde がせる & \textasciitilde がせられる\\
			\textasciitilde す & \textasciitilde させる & \textasciitilde させられる\\
			\textasciitilde ぬ & \textasciitilde なせる & \textasciitilde なせられる\\
			\textasciitilde む & \textasciitilde ませる & \textasciitilde ませられる\\
			\textasciitilde ぶ & \textasciitilde ばせる & \textasciitilde ばせられる\\
			\textasciitilde う & \textasciitilde わせる & \textasciitilde わせられる\\
			\textasciitilde つ & \textasciitilde たせる & \textasciitilde たせられる\\
			\textasciitilde る & \textasciitilde らせる & \textasciitilde らせられる\\
			\textasciitilde する & \textasciitilde させる & \textasciitilde させられる\\
			\textasciitilde くる &\textasciitilde こさせる & \textasciitilde こさせられる\\
			\bottomrule[2pt]
		\end{tabular}
	\end{table}
	
	\fukudai{Upotreba}

	Slično kao za kauzativ, postoje dva načina upotrebe ovog glagolskog oblika:
	
	1) kada se želi reći da je netko bio natjeran da nešto učini
	
	2) kada se želi reći da je nekome bilo dopušteno da nešto učini
 
	\begin{reibun}
		\rei{毎日、母に野菜を食べさせられます。}{ Svaki dan sam prisiljen jesti povrće, od strane majke. }
		\rei{私は父にスクワットを百回させられた。}{ Bio sam natjeran, od tate, da napravim 100 čučnjeva.}
		\rei{私はお母さんに宿題をさせられた。}{ Bio sam prisiljen, od mame, napraviti zadaću. }
		\rei{金曜日には家族のレストランで手伝わせられる。}{ Petkom sam prisiljen pomagati u restoranu moje obitelji.}
	\end{reibun}

	\newpage		
	\fukudai{Vježba}
		
	\begin{mondai}{Lv. 1}
		\item 書かせた。
		\item 見させる。
		\item 泳がす。
		\item させられる。
		\item 食べさせられた。
	\end{mondai}
		
	\begin{mondai}{Lv. 2}
		\item 漢字を書かせた。
		\item 映画を見させる。
		\item なんか食べさしてくれよ。\footnotemark[1]
		\item 宿題をさせられる。
		\item 朝ご飯は食べさせられた。
	\end{mondai}
	
	\begin{mondai}{Lv. 3}
		\item 漢字をたくさん書かせた。
		\item 怖い映画を見させる。
		\item お腹空いているんだから、なんか食べさしてくれよ。
		\item 親に毎日宿題をさせられる。
		\item 朝ご飯を食べたくなかったのに、食べさせられた。
	\end{mondai}

	\footnotetext[1]{tokijski dijalekt}
	
\end{document}
