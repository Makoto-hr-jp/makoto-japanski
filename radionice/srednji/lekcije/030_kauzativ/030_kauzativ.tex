% !TeX program = xelatex ?me -synctex=0 -interaction=nonstopmode -aux-directory=../tex_aux -output-directory=./release
% !TeX program = xelatex

\documentclass[12pt]{article}

\usepackage{lineno,changepage,lipsum}
\usepackage[colorlinks=true,urlcolor=blue]{hyperref}
\usepackage{fontspec}
\usepackage{xeCJK}
\usepackage{tabularx}
\setCJKfamilyfont{chanto}{AozoraMinchoRegular.ttf}
\setCJKfamilyfont{tegaki}{Mushin.otf}
\usepackage[CJK,overlap]{ruby}
\usepackage{hhline}
\usepackage{multirow,array,amssymb}
\usepackage[croatian]{babel}
\usepackage{soul}
\usepackage[usenames, dvipsnames]{color}
\usepackage{wrapfig,booktabs}
\renewcommand{\rubysep}{0.1ex}
\renewcommand{\rubysize}{0.75}
\usepackage[margin=50pt]{geometry}
\modulolinenumbers[2]

\usepackage{pifont}
\newcommand{\cmark}{\ding{51}}%
\newcommand{\xmark}{\ding{55}}%

\definecolor{faded}{RGB}{100, 100, 100}

\renewcommand{\arraystretch}{1.2}

%\ruby{}{}
%$($\href{URL}{text}$)$

\newcommand{\furigana}[2]{\ruby{#1}{#2}}
\newcommand{\tegaki}[1]{
	\CJKfamily{tegaki}\CJKnospace
	#1
	\CJKfamily{chanto}\CJKnospace
}

\newcommand{\dai}[1]{
	\vspace{20pt}
	\large
	\noindent\textbf{#1}
	\normalsize
	\vspace{20pt}
}

\newcommand{\fukudai}[1]{
	\vspace{10pt}
	\noindent\textbf{#1}
	\vspace{10pt}
}

\newenvironment{bunshou}{
	\vspace{10pt}
	\begin{adjustwidth}{1cm}{3cm}
	\begin{linenumbers}
}{
	\end{linenumbers}
	\end{adjustwidth}
}

\newenvironment{reibun}{
	\vspace{10pt}
	\begin{tabular}{l l}
}{
	\end{tabular}
	\vspace{10pt}
}
\newcommand{\rei}[2]{
	#1&\textit{#2}\\
}
\newcommand{\reinagai}[2]{
	\multicolumn{2}{l}{#1}\\
	\multicolumn{2}{l}{\hspace{10pt}\textit{#2}}\\
}

\newenvironment{mondai}[1]{
	\vspace{10pt}
	#1
	
	\begin{enumerate}
		\itemsep-5pt
	}{
	\end{enumerate}
	\vspace{10pt}
}

\newenvironment{hyou}{
	\begin{itemize}
		\itemsep-5pt
	}{
	\end{itemize}
	\vspace{10pt}
}

\date{\today}

\CJKfamily{chanto}\CJKnospace
\author{Tomislav Mamić}
\begin{document}
	\dai{Kauzativ}
		
	\fukudai{Tvorba}
	
	\begin{table}[h]
		\centering
		\begin{tabular}{l l l l}\toprule[2pt]
			\midrule
			\textasciitildeる & \textasciitildeさせる\\
			\textasciitildeく & \textasciitildeかせる\\
			\textasciitildeぐ & \textasciitildeがせる \\\vspace{5pt}
			\textasciitildeす & \textasciitildeさせる \\
			\textasciitildeぬ & \textasciitildeなせる \\
			\textasciitildeむ & \textasciitildeませる \\\vspace{5pt}
			\textasciitildeぶ & \textasciitildeばせる \\
			\textasciitildeう & \textasciitildeわせる \\
			\textasciitildeつ & \textasciitildeたせる \\
			\textasciitildeる & \textasciitildeらせる \\
			\textasciitildeする & \textasciitildeさせる
			\textasciitildeくる & \textasciitildeこさせる
			\bottomrule[2pt]
		\end{tabular}
	\end{table}
	
	
	\fukudai{Upotreba}
	
	Ova glagolska forma se koristi u dva konteksta:

	1) kada se tjera nekoga nešto učiniti
	2) kada se dopušta nešto nekome učiniti
	
	\begin{reibun}
		\rei{先生が学生に宿題をたくさんさせた。}{ Učitelj je natjerao učenike da pišu puno zadaće.}
		\rei{先生が質問をたくさん聞かせてくれた。}{Učitelj je dopustio (nekome) da pita puno pitanja.}
		\rei{今日は仕事を休ませてください。}{ Dopustite mi da se danas odmorim od posla.}
		\rei{その部長は、よく長時間働かせる。}{ Voditelj odjela često tjera (zaposlenike) da rade puno sati.}
	\end{reibun}
	
	\fukudai{Vježba}
	
	\begin{mondai}{Lv. 1}
		\item のんでみた。
		\item そうじ しておいた。
		\item よんであげた。
		\item おしえてくれた。
		\item かってもらった。
	\end{mondai}

	\begin{mondai}{Lv. 2}
		\item このおちゃを のんでみましたか。
		\item いえを そうじ しておくと きめた。
		\item 先生は 子供たちに えほんを よんであげた。
		\item ともだちが わたしに ぶつりがくを おしえてくれた。
		\item かのじょに ぎゅうにゅうを かってもらった。
	\end{mondai}

	\begin{mondai}{Lv. 3}
		\item このおちゃを のんでみたいですか。
		\item いえを そうじ しておくと きめて せんざいを かいに いった。
		\item 先生は まいにち 子供たちに えほんを よんであげていました。
		\item ぶつりがくを おしえてくれて ありがとうと ともだちに いいました。
		\item *かのじょに かいわすれた ぎゅうにゅうを かいにいってもらった。
	\end{mondai}
\end{document}
