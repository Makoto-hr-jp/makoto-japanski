% !TeX document-id = {6d8cfdcc-0400-4d6e-9c83-3bba070389e2}
% !TeX program = xelatex ?me -synctex=0 -interaction=nonstopmode -aux-directory=../tex_aux -output-directory=./release
% !TeX program = xelatex

\documentclass[12pt]{article}

\usepackage{lineno,changepage,lipsum}
\usepackage[colorlinks=true,urlcolor=blue]{hyperref}
\usepackage{fontspec}
\usepackage{xeCJK}
\usepackage{tabularx}
\setCJKfamilyfont{chanto}{AozoraMinchoRegular.ttf}
\setCJKfamilyfont{tegaki}{Mushin.otf}
\usepackage[CJK,overlap]{ruby}
\usepackage{hhline}
\usepackage{multirow,array,amssymb}
\usepackage[croatian]{babel}
\usepackage{soul}
\usepackage[usenames, dvipsnames]{color}
\usepackage{wrapfig,booktabs}
\renewcommand{\rubysep}{0.1ex}
\renewcommand{\rubysize}{0.75}
\usepackage[margin=50pt]{geometry}
\modulolinenumbers[2]

\usepackage{pifont}
\newcommand{\cmark}{\ding{51}}%
\newcommand{\xmark}{\ding{55}}%

\definecolor{faded}{RGB}{100, 100, 100}

\renewcommand{\arraystretch}{1.2}

%\ruby{}{}
%$($\href{URL}{text}$)$

\newcommand{\furigana}[2]{\ruby{#1}{#2}}
\newcommand{\tegaki}[1]{
	\CJKfamily{tegaki}\CJKnospace
	#1
	\CJKfamily{chanto}\CJKnospace
}

\newcommand{\dai}[1]{
	\vspace{20pt}
	\large
	\noindent\textbf{#1}
	\normalsize
	\vspace{20pt}
}

\newcommand{\fukudai}[1]{
	\vspace{10pt}
	\noindent\textbf{#1}
	\vspace{10pt}
}

\newenvironment{bunshou}{
	\vspace{10pt}
	\begin{adjustwidth}{1cm}{3cm}
	\begin{linenumbers}
}{
	\end{linenumbers}
	\end{adjustwidth}
}

\newenvironment{reibun}{
	\vspace{10pt}
	\begin{tabular}{l l}
}{
	\end{tabular}
	\vspace{10pt}
}
\newcommand{\rei}[2]{
	#1&\textit{#2}\\
}
\newcommand{\reinagai}[2]{
	\multicolumn{2}{l}{#1}\\
	\multicolumn{2}{l}{\hspace{10pt}\textit{#2}}\\
}

\newenvironment{mondai}[1]{
	\vspace{10pt}
	#1
	
	\begin{enumerate}
		\itemsep-5pt
	}{
	\end{enumerate}
	\vspace{10pt}
}

\newenvironment{hyou}{
	\begin{itemize}
		\itemsep-5pt
	}{
	\end{itemize}
	\vspace{10pt}
}

\date{\today}

\CJKfamily{chanto}\CJKnospace
\author{Tomislav Mamić}
\begin{document}
	\dai{Ciljevi i napomene - Nominalizacija}
	
	\fukudai{Ciljevi}
	\begin{hyou}
		\item objasniti ideju lažnih imenica (形式名詞)
		\item こと i (も)の
		\item pretvorba rečenice u imenicu "s desne strane"
		\item smjernice za こと
		\begin{hyou}
			\item uglavnom događaj, ideja, apstraktna stvar
			\item kad nominaliziranu rečenicu želimo iskoristiti kao dio imenskog predikata (uz だ/です)
			\item kad je nominalizirana rečenica neupravni govor
			\item kad se radi o fiksnom izrazu s こと (npr. ことになる, ことにする, ことができる)
		\end{hyou}
		\item smjernice za もの
		\begin{hyou}
			\item uglavnom subjektivna, konkretna stvar ili referenca na nešto poznato iz konteksta
			\item kad se radi o percepciji (npr. s 聞く, 見る, 感じる)
			\item kad se želimo referirati na već spomenutu imenicu (unatrag)
			\item kad se želimo referirati na imenicu koju ćemo naknadno spomenuti (unaprijed)
			\item kad se radi o fiksnom izrazu s の (ključan primjer \textasciitilde のだ)
		\end{hyou}
	\end{hyou}
\end{document}