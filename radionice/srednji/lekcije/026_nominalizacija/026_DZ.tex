% !TeX document-id = {98e0c3a0-933f-4217-9a07-8199bde40ad7}
% !TeX program = xelatex ?me -synctex=0 -interaction=nonstopmode -aux-directory=../tex_aux -output-directory=./release
% !TeX program = xelatex

\documentclass[12pt]{article}

\usepackage{lineno,changepage,lipsum}
\usepackage[colorlinks=true,urlcolor=blue]{hyperref}
\usepackage{fontspec}
\usepackage{xeCJK}
\usepackage{tabularx}
\setCJKfamilyfont{chanto}{AozoraMinchoRegular.ttf}
\setCJKfamilyfont{tegaki}{Mushin.otf}
\usepackage[CJK,overlap]{ruby}
\usepackage{hhline}
\usepackage{multirow,array,amssymb}
\usepackage[croatian]{babel}
\usepackage{soul}
\usepackage[usenames, dvipsnames]{color}
\usepackage{wrapfig,booktabs}
\renewcommand{\rubysep}{0.1ex}
\renewcommand{\rubysize}{0.75}
\usepackage[margin=50pt]{geometry}
\modulolinenumbers[2]

\usepackage{pifont}
\newcommand{\cmark}{\ding{51}}%
\newcommand{\xmark}{\ding{55}}%

\definecolor{faded}{RGB}{100, 100, 100}

\renewcommand{\arraystretch}{1.2}

%\ruby{}{}
%$($\href{URL}{text}$)$

\newcommand{\furigana}[2]{\ruby{#1}{#2}}
\newcommand{\tegaki}[1]{
	\CJKfamily{tegaki}\CJKnospace
	#1
	\CJKfamily{chanto}\CJKnospace
}

\newcommand{\dai}[1]{
	\vspace{20pt}
	\large
	\noindent\textbf{#1}
	\normalsize
	\vspace{20pt}
}

\newcommand{\fukudai}[1]{
	\vspace{10pt}
	\noindent\textbf{#1}
	\vspace{10pt}
}

\newenvironment{bunshou}{
	\vspace{10pt}
	\begin{adjustwidth}{1cm}{3cm}
	\begin{linenumbers}
}{
	\end{linenumbers}
	\end{adjustwidth}
}

\newenvironment{reibun}{
	\vspace{10pt}
	\begin{tabular}{l l}
}{
	\end{tabular}
	\vspace{10pt}
}
\newcommand{\rei}[2]{
	#1&\textit{#2}\\
}
\newcommand{\reinagai}[2]{
	\multicolumn{2}{l}{#1}\\
	\multicolumn{2}{l}{\hspace{10pt}\textit{#2}}\\
}

\newenvironment{mondai}[1]{
	\vspace{10pt}
	#1
	
	\begin{enumerate}
		\itemsep-5pt
	}{
	\end{enumerate}
	\vspace{10pt}
}

\newenvironment{hyou}{
	\begin{itemize}
		\itemsep-5pt
	}{
	\end{itemize}
	\vspace{10pt}
}

\date{\today}

\CJKfamily{chanto}\CJKnospace
\author{Tomislav Mamić}
\begin{document}
	\dai{Domaća zadaća - Nominalizacija}
	
	\ten\begin{mondai}{\furigana{例}{れい}と\furigana{同}{おな}じように\furigana{文}{ぶん}を書き\furigana{直}{なお}しなさい。}
		\itemsep0pt
		\item 例:アイスクリームが好きです。 $\rightarrow$ アイスクリームを食べるのが好きです。
		\item \furigana{鈴木}{すず.き}さんはお\furigana{酒}{さけ}がすきです。
		\item たけしくんの\furigana{妹}{いもうと}は\hspace{10pt}\furigana{映画}{えい.が}がすきです。
		\item 花子ちゃんは ニンジンが\furigana{嫌}{きら}いだと\furigana{思}{おも}います。
	\end{mondai}

	\vspace{20pt}
	\ten\begin{mondai}{「こと」か「の」か、\furigana{適当}{てき.とう}なものを\furigana{括弧}{かっ.こ}の中に入れなさい。}
		\itemsep0pt
		\item 例:時々\furigana{実家}{じっ.か}に\furigana{戻}{もど}る(\hspace{20pt})がある。
		\item それが\furigana{起}{お}きた(\hspace{20pt})は、私がまだ子供だったころである。
		\item\furigana{ 武}{たけし}くんはあのとき花子ちゃんに\furigana{告白}{こく.はく}しない(\hspace{20pt})にした(\hspace{20pt})を今も\furigana{後悔}{こう.かい}している。
		\item 友達に日本にいたときの(\hspace{20pt})を話した。
		\item 武くんが花子ちゃんのケーキを食べている(\hspace{20pt})を見てしまった。
		\item 今の\furigana{靴}{くつ}がいい、新しい(\hspace{20pt})は\furigana{要}{い}らない。
		\item 大事な(\hspace{20pt})は\furigana{失敗}{しっ.ぱい}から\furigana{学}{まな}ぶ(\hspace{20pt})だと、父がいつも言っていた。
		\item 武くんは\furigana{赤点}{あか.てん}を取った(\hspace{20pt})は花子ちゃんが数学を\furigana{教}{おし}えてくれなかったからだと言い\furigana{訳}{わけ}をしました。
	\end{mondai}
\end{document}