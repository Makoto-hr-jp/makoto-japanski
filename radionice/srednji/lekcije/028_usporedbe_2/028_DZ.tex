% !TeX document-id = {ab52901a-d616-4062-8026-7c03532e1da4}
% !TeX program = xelatex ?me -synctex=0 -interaction=nonstopmode -aux-directory=../tex_aux -output-directory=./release
% !TeX program = xelatex

\documentclass[12pt]{article}

\usepackage{lineno,changepage,lipsum}
\usepackage[colorlinks=true,urlcolor=blue]{hyperref}
\usepackage{fontspec}
\usepackage{xeCJK}
\usepackage{tabularx}
\setCJKfamilyfont{chanto}{AozoraMinchoRegular.ttf}
\setCJKfamilyfont{tegaki}{Mushin.otf}
\usepackage[CJK,overlap]{ruby}
\usepackage{hhline}
\usepackage{multirow,array,amssymb}
\usepackage[croatian]{babel}
\usepackage{soul}
\usepackage[usenames, dvipsnames]{color}
\usepackage{wrapfig,booktabs}
\renewcommand{\rubysep}{0.1ex}
\renewcommand{\rubysize}{0.75}
\usepackage[margin=50pt]{geometry}
\modulolinenumbers[2]

\usepackage{pifont}
\newcommand{\cmark}{\ding{51}}%
\newcommand{\xmark}{\ding{55}}%

\definecolor{faded}{RGB}{100, 100, 100}

\renewcommand{\arraystretch}{1.2}

%\ruby{}{}
%$($\href{URL}{text}$)$

\newcommand{\furigana}[2]{\ruby{#1}{#2}}
\newcommand{\tegaki}[1]{
	\CJKfamily{tegaki}\CJKnospace
	#1
	\CJKfamily{chanto}\CJKnospace
}

\newcommand{\dai}[1]{
	\vspace{20pt}
	\large
	\noindent\textbf{#1}
	\normalsize
	\vspace{20pt}
}

\newcommand{\fukudai}[1]{
	\vspace{10pt}
	\noindent\textbf{#1}
	\vspace{10pt}
}

\newenvironment{bunshou}{
	\vspace{10pt}
	\begin{adjustwidth}{1cm}{3cm}
	\begin{linenumbers}
}{
	\end{linenumbers}
	\end{adjustwidth}
}

\newenvironment{reibun}{
	\vspace{10pt}
	\begin{tabular}{l l}
}{
	\end{tabular}
	\vspace{10pt}
}
\newcommand{\rei}[2]{
	#1&\textit{#2}\\
}
\newcommand{\reinagai}[2]{
	\multicolumn{2}{l}{#1}\\
	\multicolumn{2}{l}{\hspace{10pt}\textit{#2}}\\
}

\newenvironment{mondai}[1]{
	\vspace{10pt}
	#1
	
	\begin{enumerate}
		\itemsep-5pt
	}{
	\end{enumerate}
	\vspace{10pt}
}

\newenvironment{hyou}{
	\begin{itemize}
		\itemsep-5pt
	}{
	\end{itemize}
	\vspace{10pt}
}

\date{\today}

\CJKfamily{chanto}\CJKnospace
\author{Tomislav Mamić}
\begin{document}
	\dai{Domaća zadaća - Usporedbe II}
	
	\begin{mondai}{Lv. 1}
		\Large
		\item 猫を見た\underline{ような}気がした\footnotemark[1]。
		\item \furigana{誰}{だれ}もがよく見え\underline{るように}大きな字で\furigana{黒板}{こく.ばん}に書いた。
		\item 男は\furigana{建物}{たて.もの}の\underline{ように}私たちの前にそびえたっている\footnotemark[2]。
		\item これからは\furigana{喧嘩}{けん.か}しない\underline{ようにする}。
		\item \furigana{最近}{さい.きん}は本を\furigana{読}{よ}むようになった。
	\end{mondai}

	\begin{mondai}{Lv. 2}
		\Large
		\item \furigana{箱}{はこ}の中にいた猫\underline{みたいな}生き物に\furigana{気付}{き.づ}いた\footnotemark[3]花子ちゃんは\furigana{視線}{し.せん}をそっちに\furigana{向}{む}けた。
		\item 武くんは食べすぎて歩けないパンダ\underline{みたいに}ゴロゴロしていた。
		\item 
	\end{mondai}

	\footnotetext[1]{Izraz <opis>気がする - \textit{Čini mi se} ili \textit{Imam dojam da} <opis>, npr. 分かった気がする - \textit{Mislim da sam shvatio}.}
	\footnotetext[2]{Riječ そびえたつ - \textit{stršati, biti visoko} npr. kao planina ili veliko drvo.}
	\footnotetext[3]{Riječ (\textasciitilde\ に)気付く - \textit{primijetiti} (\textasciitilde), skraćeno od izraza (\textasciitilde\ に)気が付く.}
\end{document}