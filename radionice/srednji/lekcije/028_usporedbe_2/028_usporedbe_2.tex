% !TeX document-id = {beaa175a-94a4-4793-b2c8-0503ebb5f8d3}
% !TeX program = xelatex ?me -synctex=0 -interaction=nonstopmode -aux-directory=../tex_aux -output-directory=./release
% !TeX program = xelatex

\documentclass[12pt]{article}

\usepackage{lineno,changepage,lipsum}
\usepackage[colorlinks=true,urlcolor=blue]{hyperref}
\usepackage{fontspec}
\usepackage{xeCJK}
\usepackage{tabularx}
\setCJKfamilyfont{chanto}{AozoraMinchoRegular.ttf}
\setCJKfamilyfont{tegaki}{Mushin.otf}
\usepackage[CJK,overlap]{ruby}
\usepackage{hhline}
\usepackage{multirow,array,amssymb}
\usepackage[croatian]{babel}
\usepackage{soul}
\usepackage[usenames, dvipsnames]{color}
\usepackage{wrapfig,booktabs}
\renewcommand{\rubysep}{0.1ex}
\renewcommand{\rubysize}{0.75}
\usepackage[margin=50pt]{geometry}
\modulolinenumbers[2]

\usepackage{pifont}
\newcommand{\cmark}{\ding{51}}%
\newcommand{\xmark}{\ding{55}}%

\definecolor{faded}{RGB}{100, 100, 100}

\renewcommand{\arraystretch}{1.2}

%\ruby{}{}
%$($\href{URL}{text}$)$

\newcommand{\furigana}[2]{\ruby{#1}{#2}}
\newcommand{\tegaki}[1]{
	\CJKfamily{tegaki}\CJKnospace
	#1
	\CJKfamily{chanto}\CJKnospace
}

\newcommand{\dai}[1]{
	\vspace{20pt}
	\large
	\noindent\textbf{#1}
	\normalsize
	\vspace{20pt}
}

\newcommand{\fukudai}[1]{
	\vspace{10pt}
	\noindent\textbf{#1}
	\vspace{10pt}
}

\newenvironment{bunshou}{
	\vspace{10pt}
	\begin{adjustwidth}{1cm}{3cm}
	\begin{linenumbers}
}{
	\end{linenumbers}
	\end{adjustwidth}
}

\newenvironment{reibun}{
	\vspace{10pt}
	\begin{tabular}{l l}
}{
	\end{tabular}
	\vspace{10pt}
}
\newcommand{\rei}[2]{
	#1&\textit{#2}\\
}
\newcommand{\reinagai}[2]{
	\multicolumn{2}{l}{#1}\\
	\multicolumn{2}{l}{\hspace{10pt}\textit{#2}}\\
}

\newenvironment{mondai}[1]{
	\vspace{10pt}
	#1
	
	\begin{enumerate}
		\itemsep-5pt
	}{
	\end{enumerate}
	\vspace{10pt}
}

\newenvironment{hyou}{
	\begin{itemize}
		\itemsep-5pt
	}{
	\end{itemize}
	\vspace{10pt}
}

\date{\today}

\CJKfamily{chanto}\CJKnospace
\author{Tomislav Mamić}
\begin{document}
	\dai{Usporedbe II}
	
	\fukudai{Pomoćni pridjev\footnotemark[1] よう}
	
	\footnotetext[1]{
		U gramatici standardnog japanskog, ova će riječ biti svrstana među pomoćne \textit{glagole}.
		Razlozi tome su povijesni i ne doprinose shvaćanju mehanike i uloge riječi u rečenici pa zbog lakšeg shvaćanja o njoj razmišljamo kao o な pridjevu.
		U rječnicima se zbog svrstavanja više upotreba u jednu rječ mogu naći i druge klasifikacije kao što su \textit{imenica} ili \textit{sufiks}.
	}
	
	Riječ よう u japanskom je praktično korisno promatrati kao な pridjev koji s lijeve strane opisujemo kao da je imenica. Ovisno o načinu korištenja mijenja se način na koji povezujemo opis s ostatkom rečenice.
	
	\fukudai{ような - usporedba \textit{kao}}
	
	Opišemo li imenicu pridjevom よう, opis pridružen よう shvaćamo kao stilsku figuru usporedbe (\textit{kao}). Uočimo kako sam よう u tom slučaju nema značenje - uloga mu je potpuno gramatička. Pogledajmo:
	
	\begin{reibun}
		\rei{ゆうれいを見たような\furigana{顔}{かお}}{lice kao da si vidio duha}
		\rei{ボールを食べたような\furigana{丸}{まる}い猫}{mačka okrugla kao da je pojela loptu}
		\rei{猫のような小さい犬}{mali pas sličan mački \dosl kao mačka}
	\end{reibun}

	Za opis gramatički gledano možemo iskoristiti bilo kakav opisni oblik, međutim pridjeve nećemo koristiti gotovo nikad. Vrlo slično, isto nećemo raditi ni u hrvatskom - nećemo reći da je nešto bilo \textit{kao plavo} - ili je plavo ili nije. Možemo naravno, reći da nam se čini da je nešto plavo, ali u tom ćemo slučaju koristiti drugu gramatiku\footnotemark[2].
	
	\footnotetext[2]{U konkretnom slučaju, prilog + みえる - 青くみえる}
	
	\fukudai{ように - usporedba radnje}
	
	Kao svi な pridjevi, よう ima svoj prilog koji dobijemo prebacivanjem な u に. U tom slučaju opis od よう pridružujemo radnji rečenice kao što smo i mogli očekivati:
	
	\begin{reibun}
		\rei{ゆうれいを見たように\furigana{走}{はし}った。}{Potrčao je kao da je vidio duha.}
		\rei{\furigana{風船}{ふう.せん}を食べたように\furigana{膨}{ふく}らんだ。}{Napuhao se kao da je pojeo balon.}
		\rei{猫のように\furigana{眠}{ねむ}る。}{Spavati kao mačka.}
	\end{reibun}

	\newpage
	Međutim, ovakva upotreba može nas prevariti. Naime, opis ように može osim usporedbe izreći i \textbf{cilj ili motivaciju} s kojom je radnja izvršena. U ovakvoj upotrebi, ponekad ćemo odbaciti česticu に.
	
	\begin{reibun}
		\reinagai{いい大学に入るように勉強した。}{Učio je kako bi upisao dobar faks.}
		\reinagai{花子ちゃんは\furigana{迷惑}{めい.わく}をかけないように\footnotemark[3]\furigana{外}{そと}で\furigana{待}{ま}っていました。}{Hanako je čekala vani kako ne bi smetala.}
		\reinagai{武くんは迷惑をかけるよう、中に\furigana{入}{はい}った。}{Takeši je ušao unutra kako bi smetao.}
	\end{reibun}

	\footnotetext[3]{Koristan izraz (\textasciitilde に)めいわくをかける - uzrokovati probleme (nekome).}
	
	\fukudai{ようにする - pokušaj, odluka, navika}
	
	Opisujući radnju glagola する, izričemo nečiji pokušaj, odluku ili naviku. Razlika između tri navedene stvari potječe iz vremena u kojem govorimo, ali nerijetko i iz konteksta. Ponekad je pogotovo tanka razlika između pokušaja i odluke.
	
	\begin{reibun}
		\reinagai{花子ちゃんは\furigana{笑}{わら}わないようにした。}{Hanako je odlučila da se neće smijati.}
		\reinagai{武くんも笑わないようにしたが、ダメだった。}{I Takeši se pokušao ne smijati, ali nije uspio.}
		\reinagai{日本に行くようにする。}{Potrudit ću se da odem u Japan.}
		\reinagai{\furigana{毎週}{まい.しゅう}\hspace{20pt}\furigana{実家}{じっ.か}に\furigana{電話}{でん.わ}をするようにしている。}{Trudim se svaki tjedan nazvati doma.}
	\end{reibun}

	\fukudai{ようになる - posljedica situacije}
	
	Opis pridružen glagolu なる govori nam kakva je situacija \textit{postala}. Ponekad to može biti posljedica našeg djelovanja, a ponekad samo okolnosti - ali u oba slučaja zvuči kao da između nas i onoga što je do situacije dovelo postoji određena distanca.
	
	\begin{reibun}
		\reinagai{大きくなった花子ちゃんはニンジンも食べるようになった。}{Hanako, koja je narasla, počela je jesti čak i mrkve.}
		\reinagai{大きくなった武くんは\furigana{周}{まわ}りに\furigana{迷惑}{めい.わく}をかけるようになった。}{Takeši, koji je narastao, postao je na teret svojoj neposrednoj okolini.}
		\reinagai{\furigana{毎月}{まい.つき}実家に\furigana{戻}{もど}るようになっている。}{Situacija je takva da se svaki mjesec vraćam doma.}
	\end{reibun}

	\fukudai{Kolokvijalna usporedba s みたい}
	
	Vrlo slično よう, riječ みたい možemo koristiti u nekima od ranije opisanih situacija s istim značenjem. Pri tome valja upamtiti da みたい može zvučati jako kolokvijalno - a posljedično i neprikladno.
	
	\begin{reibun}
		\rei{\furigana{人体}{じん.たい}は\furigana{機械}{き.かい}みたいなものです。}{Ljudsko tijelo je kao stroj.}
		\rei{ボールを\furigana{食}{く}ったみたいな\furigana{姿}{すがた}をしている\footnotemark[4]。}{Izgleda kao da je požderao loptu.}
		\rei{猫みたいな小さい犬}{mali pas sličan mački}
	\end{reibun}

	\footnotetext[4]{Koristan izraz (\textit{opis}) + すがた/かっこうをしている - izgleda (\textit{opis}).}
	
	S istim rezultatima možemo izmijeniti な u に da bismo opisali radnju. Ipak, za razliku od ように, opis pridružen みたいに nikad nećemo asocirati s ciljem ili motivacijom.
	
	\fukudai{Vježba}
	
	\begin{mondai}{Prevedite na japanski:}
		\item Trčao je kao da mu gore noge.
		\item Spavali su k'o zaklani.
		\item Mačka je podigla glavu kao da je čula nešto u daljini.
		\item Odlučio sam svaki dan malo čitati.
		\item Progledao je zahvaljujući modernoj medicini. (vidi おかげ)
	\end{mondai}
\end{document}
