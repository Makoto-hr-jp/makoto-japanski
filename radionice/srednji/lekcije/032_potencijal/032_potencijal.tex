% !TeX document-id = {0bd5e6a6-9b15-4a8f-9484-1e38c9fb1176}
% !TeX program = xelatex ?me -synctex=0 -interaction=nonstopmode -aux-directory=../tex_aux -output-directory=./release
% !TeX program = xelatex

\documentclass[12pt]{article}

\usepackage{lineno,changepage,lipsum}
\usepackage[colorlinks=true,urlcolor=blue]{hyperref}
\usepackage{fontspec}[ Path =../../../ ]
\usepackage{xeCJK}
\usepackage{tabularx}
\usepackage{graphicx}
\setCJKfamilyfont{chanto}{AOZORAMINCHOREGULAR_0.TTF}%
\setCJKfamilyfont{tegaki}{Mushin.otf}%
\usepackage[CJK,overlap]{ruby}
\usepackage{hhline}
\usepackage{multirow,array,amssymb}
\usepackage[croatian]{babel}
\usepackage{soul}
\usepackage[usenames, dvipsnames]{color}
\usepackage{wrapfig,booktabs}
\usepackage{calc}
\renewcommand{\rubysep}{0.1ex}
\renewcommand{\rubysize}{0.75}
\usepackage[margin=50pt]{geometry}
\usepackage{hyperref}
\modulolinenumbers[2]

\date{\today}

\usepackage{fancyhdr}
\pagestyle{fancy}
\fancyhf{}
\fancyhead[LE,RO]{\thepage}
\makeatletter
\fancyhead[RE,LO]{rev. \@date 誠}
\makeatother

\usepackage{pifont}
\newcommand{\cmark}{\ding{51}}%
\newcommand{\xmark}{\ding{55}}%

\newcommand{\dosl}{{\normalfont dosl. }}%
\newcommand{\rem}[1]{{\normalfont #1 }}%

\definecolor{faded}{RGB}{100, 100, 100}

\renewcommand{\arraystretch}{1.2}

%\ruby{}{}
%$($\href{URL}{text}$)$

\newcommand{\furigana}[2]{\ruby{#1}{#2}}
\newcommand{\tegaki}[1]{
	\CJKfamily{tegaki}\CJKnospace
	#1
	\CJKfamily{chanto}\CJKnospace
}

\newcommand{\dai}[1]{
	\vspace{20pt}
	\large
	\noindent\textbf{#1}
	\normalsize
	\vspace{20pt}
}

\newcommand{\fukudai}[1]{
	\vspace{10pt}
	\noindent\textbf{#1}
	\vspace{10pt}
}

\newenvironment{bunshou}{
	\vspace{10pt}
	\begin{adjustwidth}{1cm}{3cm}
	\begin{linenumbers}
}{
	\end{linenumbers}
	\end{adjustwidth}
}

\newenvironment{reibun}[1][]{
	\vspace{10pt}
	#1
	
	\begin{tabular}{l l}
}{
	\end{tabular}
	\vspace{10pt}
}
\newcommand{\rei}[2]{
	#1&\textit{#2}\\
}
\newcommand{\reinagai}[2]{
	\multicolumn{2}{l}{#1}\\
	\multicolumn{2}{l}{\hspace{10pt}\textit{#2}}\\
}

\newenvironment{mondai}[1]{
	\vspace{10pt}
	\noindent #1
	
	\begin{enumerate}
		\itemsep-5pt
	}{
	\end{enumerate}
}

\newenvironment{hyou}{
	\begin{itemize}
		\itemsep-5pt
	}{
	\end{itemize}
	\vspace{10pt}
}

\newcommand{\juuyou}[2][20pt]{
	\vspace{5pt}
		\noindent\hspace{#1}\parbox[c]{\textwidth-#1-#1}{\centering\textit{#2}}
	\vspace{5pt}
}

\newcommand{\ten}{
	\vspace{5pt}
	\noindent\hspace{-10pt}$\bullet$
}

\CJKfamily{chanto}\CJKnospace

\frenchspacing
\author{Tomislav Mamić}
\begin{document}
	\dai{Potencijal\footnotemark[1]}
	\footnotetext[1]{Jap. 可能形 (か.のう.けい).}
	
	Osnovna svrha ovog oblika je izražavanje mogućnosti subjekta da izvrši radnju (npr. \textit{mogu pojesti}).
	Ponekad se radi o teoretskoj mogućnosti kao u slučaju tvrdnje da \textit{zmije ne mogu letjeti}, a ponekad o praktičnoj mogućnosti ili čak volji subjekta da izvrši radnju, kao u tvrdnji \textit{ne mogu sutra ići u kino}.
	Uočimo kako u hrvatskom jeziku ovakve tvrdnje uvijek uključuju pomoćni glagol \textit{moći} i infinitiv glavnog glagola, dok se u japanskom isto značenje izražava promjenom oblika glavnog glagola.
	
	\fukudai{Tvorba}
	
	Za sve skupine glagola, potencijal preferira bazu koja završava samoglasnikom \textit{e}, a nastavak kojem se teži je \textit{e} + る.
	Tako dobiveni oblici ponašaju se kao 一段 glagoli.
	
	\begin{table}[h]
		\centering
			\begin{tabular}{l l l l l l}\toprule[2pt]
				一段 && 五段 && 不規則 &\\
				infinitiv & potencijal & infinitiv & potencijal & infinitiv & potencijal\\
				\midrule
				食べる & 食べられる & \textit{u} & \textit{e} + る & くる & こられる\\
				& 食べれる & & & & これる\footnotemark[3]\\
				見る & 見られる\footnotemark[2] & & & する & できる\\
				& 見れる & & & ある\footnotemark[4] & ありえる\\
				& & & & & ありうる\\
				\bottomrule[2pt]
			\end{tabular}
	\end{table}

	\footnotetext[2]{Upotreba potencijala glagola 見る vrlo je rijetka. Razlog tome je veliko preklapanje u značenju s glagolom 見える.}
	\footnotetext[3]{Ovakvo kraćenje potencijala od 来る dio je govornog, kolokvijalnog jezika.}
	\footnotetext[4]{Slično kao する, ali manje očito, glagol ある zapravo ne dobiva svoj potencijal dodavanjem nastavka već se radi o potpuno drugom glagolu あり得る koji ga zamjenjuje. Čitanje ありうる je arhaično i pojavljuje se samo u infinitivu, dok se oblici rade iz ありえる (npr. ありえない).}
	
	Iz tablice iznad vrijedi istaknuti nekoliko pojava.
	Potencijal 一段 glagola preklapa se s pasivom, što je u kolokvijalnom govoru za lakše razlikovanje potaknulo skraćivanje potencijala izbacivanjem m\={o}re ら.
	Potencijal 五段 glagola je savršeno pravilan.
	Isto kolokvijalno kraćenje kakvo je primijenjeno na 一段 glagole vidimo i za nepravilni glagol 来る.
	Za nepravilne glagole する i ある uočavamo da se u potencijalu koriste potpuno drugi glagoli.
	Iako zanimljivo za primijetiti, u praksi ih možemo smatrati potencijalnim oblicima する i ある.
	To je posebno točno za ありえる jer nema alternativne upotrebe, no できる se pojavljuje i s drugim značenjima od kojih je daleko najčešći izraz oblika \textit{nešto}で出来ている $\rightarrow$ \textit{napravljeno od nečega}.
	
	\fukudai{Upotreba}
	
	Osnovno značenje oblika možemo koristiti bez komplikacija, u situacijama koje se dobro preslikavaju na korištenje \textit{moći} kao pom. glagola.
	
	\begin{reibun}
		\rei{もう食べられません。}{Ne mogu više jesti.}
		\rei{もう食べれない。}{Ne mogu više jesti.}
		\rei{それはありえない。}{To nije moguće. \dosl To ne može biti. \rem{(često izraz nevjerice)}}
	\end{reibun}

	\newpage
	Slično kako bi nam stariji učitelj u školi na pitanje \textit{mogu li na WC} odgovorio s \textit{ne znam, probaj}, ni u japanskom nije ispravno potencijalom tražiti dopuštenje.
	Za ovu upotrebu koristi se neki od izravnijih oblika za traženje dopuštenja, npr. て oblik + も + いいですか (dosl. \textit{je li u redu ako...}).
	
	\begin{reibun}
		\reinagai{このリンゴ、食べられますか。}{Mogu li pojesti ovu jabuku? \xmark\ \dosl Imam li sposobnost pojesti ovu jabuku?}
		\reinagai{このリンゴを食べてもいいですか。}{Smijem li pojesti ovu jabuku? \cmark \dosl Je li u redu ako pojedem ovu jabuku?}
	\end{reibun}

	\fukudai{Teoretska mogućnost s ことができる}
	
	Želimo li izraziti vrlo čistu i doslovnu mogućnost izvršavanja neke radnje, možemo potencijal zamijeniti nešto dužim izrazom koji se oslanja na glagol できる.
	Prisjetimo se da pomoću こと možemo pretvoriti glagole u imenice:
	
	\begin{reibun}
		\rei{食べること}{jedenje}
		\rei{隠すこと}{skrivanje}
	\end{reibun}

	\noindent
	što koristimo u raznim izrazima i situacijama.
	Prisjetimo se i da こと ima apstraktan prizvuk, ne govorimo o nekoj konkretnoj situaciji gdje smo nešto učinili nego o radnji općenito.
	Zbog toga izrazi oblika
	
	\begin{reibun}
		\rei{食べることができる}{moguće je jesti \rem{(npr. bez negativnih posljedica)}}
		\rei{隠すことはできない}{nije moguće sakriti}
	\end{reibun}

	\noindent
	također poprimaju apstraktna značenja.
	Ponekad nam to može pomoći da se preciznije izrazimo, ali bitno je uočiti da je lepeza značenja koja možemo dobiti potencijalom glagola znatno šira.
	Posudivši primjere s 食べる,
	
	\begin{reibun}
		\rei{食べることができる}{moguće je jesti \rem{(bez negativnih posljedica)}}
		\rei{食べることができない}{nije moguće jesti \rem{(teoretski se ne može prožvakati)}}
		\rei{食べられる}{mogu jesti \rem{(okus mi je prihvatljiv)}}
		\rei{食べられない}{ne mogu jesti \rem{(bljak mi je)}}
		\rei{食べられない}{ne mogu jesti \rem{(već sam previše pojeo)}}
	\end{reibun}

	\noindent
	uočavamo da je potencijal glagola puno osobniji i vezan za kontekst - moramo znati ponešto o situaciji u kojoj se govornik nalazi da bismo ispravno protumačili značenje.
	
	\fukudai{Vježba}
	
\end{document}