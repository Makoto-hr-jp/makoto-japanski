% !TeX document-id = {0bd5e6a6-9b15-4a8f-9484-1e38c9fb1176}
% !TeX program = xelatex ?me -synctex=0 -interaction=nonstopmode -aux-directory=../tex_aux -output-directory=./release
% !TeX program = xelatex

\documentclass[12pt]{article}

\usepackage{lineno,changepage,lipsum}
\usepackage[colorlinks=true,urlcolor=blue]{hyperref}
\usepackage{fontspec}
\usepackage{xeCJK}
\usepackage{tabularx}
\setCJKfamilyfont{chanto}{AozoraMinchoRegular.ttf}
\setCJKfamilyfont{tegaki}{Mushin.otf}
\usepackage[CJK,overlap]{ruby}
\usepackage{hhline}
\usepackage{multirow,array,amssymb}
\usepackage[croatian]{babel}
\usepackage{soul}
\usepackage[usenames, dvipsnames]{color}
\usepackage{wrapfig,booktabs}
\renewcommand{\rubysep}{0.1ex}
\renewcommand{\rubysize}{0.75}
\usepackage[margin=50pt]{geometry}
\modulolinenumbers[2]

\usepackage{pifont}
\newcommand{\cmark}{\ding{51}}%
\newcommand{\xmark}{\ding{55}}%

\definecolor{faded}{RGB}{100, 100, 100}

\renewcommand{\arraystretch}{1.2}

%\ruby{}{}
%$($\href{URL}{text}$)$

\newcommand{\furigana}[2]{\ruby{#1}{#2}}
\newcommand{\tegaki}[1]{
	\CJKfamily{tegaki}\CJKnospace
	#1
	\CJKfamily{chanto}\CJKnospace
}

\newcommand{\dai}[1]{
	\vspace{20pt}
	\large
	\noindent\textbf{#1}
	\normalsize
	\vspace{20pt}
}

\newcommand{\fukudai}[1]{
	\vspace{10pt}
	\noindent\textbf{#1}
	\vspace{10pt}
}

\newenvironment{bunshou}{
	\vspace{10pt}
	\begin{adjustwidth}{1cm}{3cm}
	\begin{linenumbers}
}{
	\end{linenumbers}
	\end{adjustwidth}
}

\newenvironment{reibun}{
	\vspace{10pt}
	\begin{tabular}{l l}
}{
	\end{tabular}
	\vspace{10pt}
}
\newcommand{\rei}[2]{
	#1&\textit{#2}\\
}
\newcommand{\reinagai}[2]{
	\multicolumn{2}{l}{#1}\\
	\multicolumn{2}{l}{\hspace{10pt}\textit{#2}}\\
}

\newenvironment{mondai}[1]{
	\vspace{10pt}
	#1
	
	\begin{enumerate}
		\itemsep-5pt
	}{
	\end{enumerate}
	\vspace{10pt}
}

\newenvironment{hyou}{
	\begin{itemize}
		\itemsep-5pt
	}{
	\end{itemize}
	\vspace{10pt}
}

\date{\today}

\CJKfamily{chanto}\CJKnospace
\author{Tomislav Mamić}
\begin{document}
	\dai{Potencijal\footnotemark[1]}
	
	
	Osnovna svrha ovog oblika je izražavanje mogućnosti subjekta da izvrši radnju (npr. \textit{mogu pojesti}).
	Ponekad se radi o teoretskoj mogućnosti kao u slučaju tvrdnje da \textit{zmije ne mogu letjeti}, a ponekad o praktičnoj mogućnosti ili čak nedostatku volje subjekta da izvrši radnju, kao u tvrdnji \textit{ne mogu sutra ići u kino}.
	Uočimo kako u hrvatskom jeziku ovakve tvrdnje uvijek uključuju pomoćni glagol \textit{moći} i infinitiv glavnog glagola, dok se u japanskom isto značenje izražava promjenom oblika glavnog glagola.
	
	\fukudai{Tvorba}
	
	Za sve skupine glagola, potencijal preferira bazu koja završava samoglasnikom \textit{e}, a nastavak kojem se teži je \textit{e} + る.
	Tako dobiveni oblici ponašaju se kao 一段 glagoli.
	
	\begin{table}[h]
		\centering
			\begin{tabular}{l l l l l l}\toprule[2pt]
				一段 && 五段 && 不規則 &\\
				infinitiv & potencijal & infinitiv & potencijal & infinitiv & potencijal\\
				\midrule
				食べる & 食べられる & \textit{u} & \textit{e} + る & くる & こられる\\
				& 食べれる & & & & これる\footnotemark[3]\\
				見る & 見られる\footnotemark[2] & & & する & できる\\
				& 見れる & & & ある\footnotemark[4] & ありえる\\
				& & & & & ありうる\\
				\bottomrule[2pt]
			\end{tabular}
	\end{table}

	\footnotetext[2]{Upotreba potencijala glagola 見る vrlo je rijetka. Razlog tome je veliko preklapanje u značenju s glagolom 見える.}
	\footnotetext[3]{Ovakvo kraćenje potencijala od 来る dio je govornog, kolokvijalnog jezika.}
	\footnotetext[4]{Slično kao する, ali manje očito, glagol ある zapravo ne dobiva svoj potencijal dodavanjem nastavka već se radi o potpuno drugom glagolu あり得る koji ga zamjenjuje. Čitanje ありうる je arhaično i pojavljuje se samo u infinitivu, dok se oblici rade iz ありえる (npr. ありえない).}
	
	Iz tablice iznad vrijedi istaknuti nekoliko pojava.
	Potencijal 一段 glagola preklapa se s pasivom, što je u kolokvikalnom govoru za lakše razlikovanje potaknulo skraćivanje potencijala izbacivanjem m\={o}re ら.
	Potencijal 五段 glagola je savršeno pravilan.
	Isto kolokvijalno kraćenje kakvo je primijenjeno na 一段 glagole vidimo i za nepravilni glagol 来る.
	Za nepravilne glagole する i ある uočavamo da se u potencijalu koriste potpuno drugi glagoli.
	Iako zanimljivo za primijetiti, u praksi ih možemo smatrati potencijalnim oblicima する i ある.
	To je posebno točno za ありえる jer nema alternativne upotrebe, no できる se pojavljuje i s drugim značenjima od kojih je daleko najčešći izraz oblika \textit{nešto}で出来ている $\rightarrow$ \textit{napravljeno od nečega}.
	
	\fukudai{Upotreba}
	
	Osnovno značenje oblika možemo koristiti bez komplikacija, u situacijama koje se dobro preslikavaju na korištenje \textit{moći} kao pom. glagola.
	
	\begin{reibun}
		\rei{もう食べられません。}{Ne mogu više jesti.}
		\rei{もう食べれない。}{Ne mogu više jesti.}
		\rei{それはありえない。}{To nije moguće. \dosl To ne može biti. \rem{(često izraz nevjerice)}}
	\end{reibun}

	\newpage
	Slično kako bi nam stariji učitelj u školi na pitanje \textit{mogu li na WC} odgovorio s \textit{ne znam, probaj}, ni u japanskom nije ispravno potencijalom tražiti dopuštenje.
	Za ovu upotrebu koristi se neki od izravnijih oblika za traženje dopuštenja, npr. て oblik + も + いいですか (dosl. \textit{je li u redu ako...}).
	
	\begin{reibun}
		\reinagai{このリンゴ、食べられますか。}{Mogu li pojesti ovu jabuku? \xmark\ \dosl Imam li sposobnost pojesti ovu jabuku?}
		\reinagai{このリンゴを食べてもいいですか。}{Smijem li pojesti ovu jabuku? \cmark \dosl Je li u redu ako pojedem ovu jabuku?}
	\end{reibun}
	
\end{document}