% !TeX document-id = {1134c18e-7daa-45ae-9617-8851083c6388}
% !TeX program = xelatex ?me -synctex=0 -interaction=nonstopmode -aux-directory=../tex_aux -output-directory=./release
% !TeX program = xelatex

\documentclass[12pt]{article}

\usepackage{lineno,changepage,lipsum}
\usepackage[colorlinks=true,urlcolor=blue]{hyperref}
\usepackage{fontspec}
\usepackage{xeCJK}
\usepackage{tabularx}
\setCJKfamilyfont{chanto}{AozoraMinchoRegular.ttf}
\setCJKfamilyfont{tegaki}{Mushin.otf}
\usepackage[CJK,overlap]{ruby}
\usepackage{hhline}
\usepackage{multirow,array,amssymb}
\usepackage[croatian]{babel}
\usepackage{soul}
\usepackage[usenames, dvipsnames]{color}
\usepackage{wrapfig,booktabs}
\renewcommand{\rubysep}{0.1ex}
\renewcommand{\rubysize}{0.75}
\usepackage[margin=50pt]{geometry}
\modulolinenumbers[2]

\usepackage{pifont}
\newcommand{\cmark}{\ding{51}}%
\newcommand{\xmark}{\ding{55}}%

\definecolor{faded}{RGB}{100, 100, 100}

\renewcommand{\arraystretch}{1.2}

%\ruby{}{}
%$($\href{URL}{text}$)$

\newcommand{\furigana}[2]{\ruby{#1}{#2}}
\newcommand{\tegaki}[1]{
	\CJKfamily{tegaki}\CJKnospace
	#1
	\CJKfamily{chanto}\CJKnospace
}

\newcommand{\dai}[1]{
	\vspace{20pt}
	\large
	\noindent\textbf{#1}
	\normalsize
	\vspace{20pt}
}

\newcommand{\fukudai}[1]{
	\vspace{10pt}
	\noindent\textbf{#1}
	\vspace{10pt}
}

\newenvironment{bunshou}{
	\vspace{10pt}
	\begin{adjustwidth}{1cm}{3cm}
	\begin{linenumbers}
}{
	\end{linenumbers}
	\end{adjustwidth}
}

\newenvironment{reibun}{
	\vspace{10pt}
	\begin{tabular}{l l}
}{
	\end{tabular}
	\vspace{10pt}
}
\newcommand{\rei}[2]{
	#1&\textit{#2}\\
}
\newcommand{\reinagai}[2]{
	\multicolumn{2}{l}{#1}\\
	\multicolumn{2}{l}{\hspace{10pt}\textit{#2}}\\
}

\newenvironment{mondai}[1]{
	\vspace{10pt}
	#1
	
	\begin{enumerate}
		\itemsep-5pt
	}{
	\end{enumerate}
	\vspace{10pt}
}

\newenvironment{hyou}{
	\begin{itemize}
		\itemsep-5pt
	}{
	\end{itemize}
	\vspace{10pt}
}

\date{\today}

\CJKfamily{chanto}\CJKnospace
\author{Tomislav Mamić}
\begin{document}
	\dai{Opisna rečenica}
	
	\fukudai{Teorija - opis u japanskom}
	
	Dosad smo za opisivanje imenica, izuzev nekih primjera koje smo smatrali težima, koristili samo pridjeve. Naučili smo da pridjevi dolaze uglavnom u dvije vrste koje smo nazvali い i な pridjevima\footnotemark[1]. Međutim, japanski pridjevi za razliku od hrvatskih mogu nositi informacije o vremenu i negaciji, što nam je kompliciralo prijevod. Pogledajmo točno \textit{kako} se prijevod mijenja:
	
	\begin{reibun}
		\rei{くろい\furigana{猫}{ねこ}}{crna mačka}
		\rei{くろくない猫}{mačka koja nije crna}
	\end{reibun}
	
	Iako se u japanskom struktura rečenice nije značajnije promijenila, u hrvatskom smo pridjev morali zamijeniti cijelom zavisnom rečenicom \textit{koja nije crna} da bismo dodali negaciju. Međutim, pogledamo li pobliže くろくない, možemo uočiti da se radi o dva dijela koje smo već prije viđali, a koje znamo koristiti i odvojeno - くろく (prilog od くろい, \textit{crno}) i ない (negacija glagola ある, \textit{biti}). Možemo reći da je くろくない zapravo zavisna rečenica koja se sastoji od negiranog glagola ある i priloga くろく koji taj glagol opisuje.
	
	Ovakvo opisivanje predikatima jedan je od temeljnih mehanizama za građenje rečenica u japanskom jeziku.
	
	\footnotetext[1]{U jap. gramatici ove se vrste zovu 形容詞 (けい.よう.し) i 形容動詞 (けい.よう.どう.し).}
	
	\fukudai{Teorija - predikatni i opisni oblik}
	
	U prošlosti jezika, sve riječi koje su mogle poslužiti kao predikat, imale su dva različita oblika - predikatni i opisni\footnotemark[2]. Kad bi služile kao predikat glavne rečenice, dolazile bi u predikatnom obliku, a kad bi bile predikat zavisne rečenice koja opisuje neku imenicu, koristio se opisni oblik. Pogledajmo primjere:
	
	\begin{reibun}
		\rei{あかい くるま}{crveni auto \normalfont{(opis)}}
		\rei{くるまは あかい。}{Auto je crven. \normalfont{(predikat)}}
	\end{reibun}
	
	Nekad davno, pridjevi u ovim rečenicama razlikovali bi se po obliku (あかき i あかし), no u modernom jeziku い pridjevi su izgubili ove razlike. Na sličan način razlike su izgubili i glagoli pa su im danas oba oblika ista:
	
	\begin{reibun}
		\rei{りんごを 食べた。}{Pojeo sam jabuku. \normalfont{(predikat)}}
		\rei{食べた りんご}{jabuka koju sam pojeo \normalfont{(opis)}}
	\end{reibun}
	
	Jedina nama bitna skupina riječi u kojoj su se razlike zadržale do danas jesu な i の pridjevi koji u predikatnom obliku dolaze u kombinaciji sa spojnim glagolom:
	
	\begin{reibun}
		\rei{びょうきの人}{bolestan čovjek \normalfont{(opis)}}
		\rei{あの人は びょうきだ。}{Onaj čovjek je bolestan. \normalfont{(predikat)}}
		\rei{げんきな人}{zdrav / veseo čovjek \normalfont{(opis)}}
		\rei{あの人は げんきだ。}{Onaj čovjek je zdrav / veseo. \normalfont{(predikat)}}
	\end{reibun}
	
	\footnotetext[2]{U jap. gramatici 終止形 (しゅう.し.けい) i 連体形 (れん.たい.けい).}
	
	\fukudai{Teorija - strogo opisne riječi}
	
	Osim dosad spomenutih opisa, u japanskom jeziku postoji još jedna kategorija opisnih riječi za čiji naziv u hrvatskom jeziku nemamo dobar prijevod pa ćemo ih zvati 連体詞 (れん.たい.し) kao u jap. gramatici. Neke od ovih riječi već smo koristili (npr. この, その, あの), a posebne su po tome što se pojavljuju isključivo kao opis imenice i što su nepromjenjive. Najčešći zbunjujući primjeri su:
	
	\begin{reibun}
		\rei{大きな いえ}{velika kuća \cmark}
		\rei{いえは 大きだ。}{Kuća je velika. \xmark~\normalfont{(mora biti 大きい)}}
		\rei{小さな 花}{mali cvijet}
		\rei{花は 小さだ。}{Cvijet je malen. \xmark~\normalfont{(mora biti 小さい)}}
	\end{reibun}

	Iako tako izgledaju, 大きな i 小さな \textbf{nisu} な pridjevi već opisne riječi. Ovakve opisne riječi vrlo su često nastale kao okamenjeni ili malo promijenjeni oblici drugih riječi pa ih je lako zamijeniti za druge vrste i krivo upotrijebiti, no na našu sreću, nema ih puno\footnotemark[3].
	
	\footnotetext[3]{Za istraživanje potražite "\#adj-pn" na jisho.org!}
	
	\fukudai{Tvorba opisnog oblika}
	
	Zahvaljujući promjenama u modernom jeziku, tablica u nastavku postala je veoma jednostavna. Komplikacije dolaze od imenica i な pridjeva čiji predikatni oblik stvaramo pomoću spojnog glagola だ. Naime gramatički gledano, opisni oblik tog glagola trebao bi uvijek biti な, ali u praksi to nije tako i imenice gotovo uvijek opisuju druge imenice česticom の\footnotemark[4].
	
	\footnotetext[4]{Uz imenice se な pojavljuje iznimno kad opisuju imenicu の(もの) i veznike nastale od iste - ので i のに.}
	
	\begin{table}[h]
		\centering
		\begin{tabular}{l c c}\toprule[2pt]
			vrsta predikata & predikatni oblik & opisni oblik\\
			\midrule
			imenica & \textasciitilde だ & \textasciitilde の\\
			な pridjev & \textasciitilde だ & \textasciitilde な\\
			い pridjev & \multicolumn{2}{c}{nema razlike}\\
			glagol & \multicolumn{2}{c}{nema razlike}\\
			\bottomrule[2pt]
		\end{tabular}
	\end{table}

	\fukudai{Tvorba i značenje opisne rečenice}
	
	Rečenica postaje opisna kad njezin predikat prebacimo u opisni oblik (što uglavnom znači da mu ne moramo napraviti ništa!) i umetnemo je u drugu rečenicu kao opis nekoj opisivoj riječi ili izrazu. Pri tome gotovo uvijek zavisnom rečenicom opisujemo riječ koju bismo morali umetnuti u zavisnu rečenicu da bi postala samostalna. Pogledajmo primjere:
	
	\begin{reibun}
		\reinagai{たけしくん\underline{は} りんごを食べた。$\rightarrow$たけしくん\underline{が}食べた りんご}{Takeši je pojeo jabuku. $\rightarrow$ jabuka koju je Takeši pojeo}
		\reinagai{きのう、田中さんは猫を見た。$\rightarrow$きのう猫を見た田中さん}{Tanaka je jučer vidio mačku. $\rightarrow$ Tanaka koji je jučer vidio mačku}
		\reinagai{おみせでコップを かった。$\rightarrow$コップを かった おみせ}{U dućanu sam kupio šalicu. $\rightarrow$ dućan u kojem sam kupio šalicu}
	\end{reibun}

	U primjerima iznad događa se par zanimljivih stvari koje ćemo u nastavku pobliže proučiti i objasniti.
	
	\vspace{5pt}
	\textbf{Subjekt} u opisnim rečenicama nikad nije tema. Opisne rečenice dodaju informacije o jednoj riječi u nadređenoj rečenici, a temu od nje preuzimaju. Pokušamo li smisliti protuprimjere, vidjet ćemo da takvo nešto ne funkcionira po smislu. U pravilu će subjekt dobiti česticu が ili の.
	
	Ova upotreba čestice の za označavanje subjekta u zavisnim rečenicama je česta i u početku može biti zbunjujuća, no prepoznat ćemo je po tome što, za razliku od slučaja gdje izriče posjedovanje ili opis, ovdje ne spaja dvije imenice već imenicu i predikat. Tako je prvi primjer iznad mogao glasiti i たけしくん\underline{の}食べた りんご bez razlike u značenju.
	
	\vspace{5pt}
	\textbf{Čestice} uz imenicu koju vadimo iz jednostavne rečenice mogu se razlikovati. Nakon što smo imenicu tako izvadili ispred, dobili smo komad rečenice\footnotemark[5] kojem je desni kraj imenica. Na tu ćemo imenicu dodati česticu koja nam treba da je smjestimo na željeno mjesto u glavnoj rečenici.
	
	\footnotetext[5]{Ili stručnije \textit{sintagma}.}
	
	Učinivši to, izbrisali smo informaciju o tome kako se opisana imenica uklapa u pridruženu joj opisnu rečenicu! Ovdje se japanski jezik oslanja na zdrav razum i govornikovu sposobnost da iz konteksta zaključi koji je njihov međusobni odnos. Usporedimo sljedeće primjere s onima iznad:
	
	\begin{reibun}
		\rei{りんごを食べた たけしくん}{Takeši koji je pojeo jabuku}
		\rei{田中さんが きのう見た猫}{mačka koju je Tanaka jučer vidio}
		\rei{おみせで かったコップ}{šalica koju sam kupio u dućanu}
		\reinagai{きのう はなした人}{čovjek o kojem sam (ti) jučer pričao \cmark čovjek koji je jučer pričao \cmark}
	\end{reibun}

	Uočavamo da je sadržaj rečenica isti, no ovaj put smo izvukli druge imenice. Njihovo mjesto u opisnoj rečenici pretpostavit ćemo prema tome koja u njoj informacija nedostaje, što za prva tri primjera nije preteško. Međutim, u trećem primjeru 人 može pričati, ali može biti i tema razgovora pa se moramo osloniti na kontekst da odaberemo ispravno tumačenje.
	
	\fukudai{Vježba}
	
	\begin{mondai}{Prevedite rečenice u nastavku.}
		\item 猫が わたしの にわで あそんでいる。
		\item 花子さんの いもうとが猫を見た。
		\item いつも猫と いっしょに いる。
		\item 猫に えさを あげた。 (えさ - \textit{hrana za životinje})
		\item かわいくて 小さい 猫だ。
		\item 猫が木に のぼった。 (のぼる - \textit{popeti se})
	\end{mondai}

	\noindent
	Preoblikujte prethodne rečenice u sintagme koje opisuju 猫.
	
	\vspace{5pt}\noindent
	Spojite proizvoljno parove rečenica tako da jednu rečenicu odaberete kao glavnu i onda u njoj 猫 opišete neko-m drugom. Hehe. He.

\end{document}