% !TeX program = xelatex ?me -synctex=0 -interaction=nonstopmode -aux-directory=../tex_aux -output-directory=./release
% !TeX program = xelatex

\documentclass[12pt]{article}

\usepackage{lineno,changepage,lipsum}
\usepackage[colorlinks=true,urlcolor=blue]{hyperref}
\usepackage{fontspec}
\usepackage{xeCJK}
\usepackage{tabularx}
\setCJKfamilyfont{chanto}{AozoraMinchoRegular.ttf}
\setCJKfamilyfont{tegaki}{Mushin.otf}
\usepackage[CJK,overlap]{ruby}
\usepackage{hhline}
\usepackage{multirow,array,amssymb}
\usepackage[croatian]{babel}
\usepackage{soul}
\usepackage[usenames, dvipsnames]{color}
\usepackage{wrapfig,booktabs}
\renewcommand{\rubysep}{0.1ex}
\renewcommand{\rubysize}{0.75}
\usepackage[margin=50pt]{geometry}
\modulolinenumbers[2]

\usepackage{pifont}
\newcommand{\cmark}{\ding{51}}%
\newcommand{\xmark}{\ding{55}}%

\definecolor{faded}{RGB}{100, 100, 100}

\renewcommand{\arraystretch}{1.2}

%\ruby{}{}
%$($\href{URL}{text}$)$

\newcommand{\furigana}[2]{\ruby{#1}{#2}}
\newcommand{\tegaki}[1]{
	\CJKfamily{tegaki}\CJKnospace
	#1
	\CJKfamily{chanto}\CJKnospace
}

\newcommand{\dai}[1]{
	\vspace{20pt}
	\large
	\noindent\textbf{#1}
	\normalsize
	\vspace{20pt}
}

\newcommand{\fukudai}[1]{
	\vspace{10pt}
	\noindent\textbf{#1}
	\vspace{10pt}
}

\newenvironment{bunshou}{
	\vspace{10pt}
	\begin{adjustwidth}{1cm}{3cm}
	\begin{linenumbers}
}{
	\end{linenumbers}
	\end{adjustwidth}
}

\newenvironment{reibun}{
	\vspace{10pt}
	\begin{tabular}{l l}
}{
	\end{tabular}
	\vspace{10pt}
}
\newcommand{\rei}[2]{
	#1&\textit{#2}\\
}
\newcommand{\reinagai}[2]{
	\multicolumn{2}{l}{#1}\\
	\multicolumn{2}{l}{\hspace{10pt}\textit{#2}}\\
}

\newenvironment{mondai}[1]{
	\vspace{10pt}
	#1
	
	\begin{enumerate}
		\itemsep-5pt
	}{
	\end{enumerate}
	\vspace{10pt}
}

\newenvironment{hyou}{
	\begin{itemize}
		\itemsep-5pt
	}{
	\end{itemize}
	\vspace{10pt}
}

\date{\today}

\CJKfamily{chanto}\CJKnospace

\author{Katja Kržišnik}
\begin{document}
	\dai{Pasiv}
	
	\fukudai{Tvorba}
	
	Pasiv je glagolski oblik koji se tvori od 未線形(みぜんけい)-a\footnotemark[1]; tzv. あ stupca i nastavka れる. Zadnji znak hiragane iz riječničkog oblika glagola prijeđe u znak hiragane koji završava na あ, npr. くprijeđe u か. Na taj se oblik zatim doda nastavak れる.
	
	\vspace{20pt}
	
		\begin{tabular}{ccc}
		\begin{tabular}{ll}\toprule[2pt]
			一段\\
			\midrule
			Riječnički oblik & Pasiv\\
			食べる & 食べられる\\
			\addlinespace[139pt]	
			\bottomrule[0pt]	
		\end{tabular}
		&
		\begin{tabular}{ll}\toprule[2pt]
			五段\\
			\midrule
			Riječnički oblik & Pasiv\\
			使う & 使われる\\
			焼く & 焼かれる\\
			泳ぐ & 泳がれる\\
			示す & 示される\\
			待つ & 待たれる\\
			呼ぶ & 呼ばれる\\
			噛む & 噛まれる\\
			死ぬ & 死なれる\\
			走る & 走られる\\
			\bottomrule[0pt]	
		\end{tabular}
	    &
		\begin{tabular}{ll}\toprule[2pt]
			Nepravilni glagoli\\
			\midrule
			Riječnički oblik & Pasiv\\
			行く & 行かれる\\
			来る & 来られる\\
			する & される\\
			ある & /\\
			\addlinespace[87pt]	
		    \bottomrule[0pt]
		\end{tabular} 		
		\end{tabular}
	
	\vspace{10pt}
      	
	\fukudai{Korištenje i značenje}
	    	
    	Postoje okvirno \textit{tri} načina na koje se pasiv može koristiti. Ovisno o tome koji od ta tri načina se koristi značenje pasiva može se značajno promijeniti. \textit{Prvi} se zove めいわく pasiv i često izražava nezadovoljstvo subjekta onime što mu je učinjeno ili što mu se dogodilo. \textit{Drugi} je tzv. pravi ili gramatički pasiv koji se koristi za opisivanje i ne nosi sa sobom nikakve druge konotacije. \textit{Treći} način korištenja pasiva je u 敬語-u(keigu)\footnotemark[2], vrlo pristojnom jeziku koji se koristi na radnom mjestu, u politici, na televiziji, itd.\footnotemark[3]  
	 
	 \vspace{10pt}
	 
	  \fukudai{めいわく pasiv}
    	
    	Ovo nije pravi gramatički pasiv u njemu trpitelj radnje \textit{ostaje} subjekt, direktni objekt prepoznaje se po čestici を, a vršitelj radnje označava se česticom に. Međutim u slučaju da vršitelj radnje nije izrečen čestica に se ne koristi. Što se tiče značenja ovaj način često izražava \textit{nezadovoljstvo} subjekta onime što mu je napravljeno.
    	
     \footnotetext[1]{未線形(みぜんけい) je vrsta korijena glagola iz kojeg se tvori negacija, konjunktiv, kauzativ i pasiv. Detaljnije o tome možete pročitati ovdje (\href{https://jref.com/articles/mizenkei.110/}{link}) (\href{https://japaneselearningonline.blogspot.com/2015/04/5-japanese-verb-basic-conjugations.html}{link}).}
     
     \footnotetext[2]{Ako vas zanima više o keigu (\href{https://cotoacademy.com/japanese-keigo/}{link}).}
     
     \footnotetext[3]{ Međutim, potrebno je naglasiti da su ovo pdjele po načinu korištenja i cijelo vrijeme se radi o istom glagolskom obliku, 受身形 -u, Japancima je ova podjela jedan te isti pasiv. Ovdje je podjela napravljena kako bi se olakšalo razumijevanje, tako da ovo nisu službeni nazivi.}	
    	
   	
	  
	      \begin{reibun}
	  	    \rei{リンゴを食べられた。}{Netko mi je pojeo jabuku.}
	  	    \rei{(私は) お父さんに叩かれた。}{Otac me udario.}
	  	    \rei{足を椅子におられた。}{Slomio sam nogu na stolac. (dosl. Stolac mi je slomio nogu.)}
	     	\rei{土田さんは財布を盗まれた。}{Tsuchidi je netko ukrao novčanik.}
	  	    \rei{それを先生によく言われます。}{Profesor mi to često govori.}
	      \end{reibun}
	
	    
	      \fukudai{Gramatički pasiv}
    	
    	Da bi pasiv zaista gramatički bio pasiv, mora doći do zamjene perspektive subjekta i objekta tako da direktni objekt postaje subjekt (を => は/が), a subjekt dobiva česticu に kao vršitelj radnje (は/が => に). 
	  
	      \begin{reibun}
	  	    \rei{リンゴを食べた。}{リンゴが食べられた。}
	  	    \rei{Pojeo sam jabuku.}{Jabuka je pojedena.}
	    	\rei{先生は松田さんのノートを見つけた。}{松田さんのノートは先生に見つけられた。}
	    	\rei{Profesor je pronašao Matsudinu bilježnicu.}{Matsudina bilježnica je pronađena.}
	      \end{reibun}
   
   
    	Uočimo kako je u drugom primjeru u japanskom vrlo prirodno dodati informaciju o vršitelju radnje dok je na hrvatskom svaki pokušaj istog (npr. Matsudina bilježnica je pronađena od (strane) profesora) zločin protiv jezika.
    	
 	
	
	      \begin{reibun}
		      \rei{本を読んだ。}{本は読まれた。}
		      \rei{Pročitao sam knjigu.}{Knjiga je pročitana.}
		      \rei{可愛いウサギは緑いろな草を食べた。}{緑いろな草は可愛いウサギに食べられた。}
		      \rei{Slatki zeko je pojeo zelenu travu}{Zelena trava je pojedena(od strane slatkog zeke).}
		      \rei{里子ちゃんは甘いアンパン\footnotemark[4]を買った。}{甘いアンパンは里子ちゃんに買われた。}
		      \rei{Yuriko je kupila slatki anpan}{Slatki anpan je kupljen(od strane Yuriko).}
	      \end{reibun}	
	      
	      \fukudai{Pasiv u keigu}
    	
    	Pasiv se također koristi u vrlo pristojnogm jeziku 敬語 -u. U tom slučaju pasivni oblik izražava poštovanje prema vršitelju radnje glagola koje pasiviziramo.
	
	      \begin{reibun}
	  	    \rei{どちらへ行かれますか。}{Gdje idete?}
	  	  \end{reibun}
        
        \fukudai{Drugi oblici}
        
        Glagoli u pasivu ponašaju se kao 一段 glagoli.
        
        \begin{reibun}[npr.]
        	\rei{\textasciitilde れ+ない}{tvorba negacije}
        	\rei{\textasciitilde れ+ます}{tvorba pristojnog oblika}
        	\rei{\textasciitilde れ+た}{tvorba prošlosti}
        \end{reibun}
        
        \begin{reibun}
        	\rei{蚊は殺されてない。}{Komarac nije ubijen.}
        	\rei{本は書かれます。}{Knjiga je napisana.}
        	\rei{髪を切られた。}{Kosa mi je odrezana.}   	    	
        \end{reibun}
        
  \footnotetext[4]{Ovako izgleda i radi se アンパン (\href{https://www.justonecookbook.com/anpan/}{link})}
        
        
\newpage

    \fukudai{Vježba} 
    	
    	\begin{mondai}{この文を日本語に訳しなさい:}
    	   \item Mački je Tanaka stao na rep.
    	   \item Voda je popijena.
    	   \item Miš je ulovljen i stavljen u kutiju.
    	   \item Mrkva je pojedena (Matsudin sin ju je pojeo). 
        \end{mondai}{}
    
        \begin{mondai}{この文をコロアチア語に訳しなさい:}
        	\item きゅうりは切られた。
        	\item 花子ちゃんは髪を切られた。
        	\item おいしい餌は黒い猫に早く食べられた。
        	\item 髪の長いおばあさんはきれいなお兄さんに助けられた。
        \end{mondai}
    
    	\begin{mondai}{この文を pasiv にしなさい:}
    		\item 蜂は熊を刺した。
    		\item 猫は死んだネズミを土に埋めた。
    		\item 鈴木さんは古い窓をゆっくり開けた。
    		\item 花子ちゃんは墓地を見つけて、長い間眺めていた。
    	\end{mondai}
    
        \begin{mondai}{ エクストラ! この文を好きにしなさい: }
        	\item リンゴは食べられたまま机の上に輪すられた。 
        \end{mondai}
    
		
	
\end{document}