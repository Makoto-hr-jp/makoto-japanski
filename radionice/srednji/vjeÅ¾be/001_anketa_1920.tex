% !TeX document-id = {a7eb3b05-34f7-4132-9617-7d1dc681d3e9}
% !TeX program = xelatex ?me -synctex=0 -interaction=nonstopmode -aux-directory=../tex_aux -output-directory=./release
% !TeX program = xelatex

\documentclass[12pt]{article}

\usepackage{lineno,changepage,lipsum}
\usepackage[colorlinks=true,urlcolor=blue]{hyperref}
\usepackage{fontspec}
\usepackage{xeCJK}
\usepackage{tabularx}
\setCJKfamilyfont{chanto}{AozoraMinchoRegular.ttf}
\setCJKfamilyfont{tegaki}{Mushin.otf}
\usepackage[CJK,overlap]{ruby}
\usepackage{hhline}
\usepackage{multirow,array,amssymb}
\usepackage[croatian]{babel}
\usepackage{soul}
\usepackage[usenames, dvipsnames]{color}
\usepackage{wrapfig,booktabs}
\renewcommand{\rubysep}{0.1ex}
\renewcommand{\rubysize}{0.75}
\usepackage[margin=50pt]{geometry}
\modulolinenumbers[2]

\usepackage{pifont}
\newcommand{\cmark}{\ding{51}}%
\newcommand{\xmark}{\ding{55}}%

\definecolor{faded}{RGB}{100, 100, 100}

\renewcommand{\arraystretch}{1.2}

%\ruby{}{}
%$($\href{URL}{text}$)$

\newcommand{\furigana}[2]{\ruby{#1}{#2}}
\newcommand{\tegaki}[1]{
	\CJKfamily{tegaki}\CJKnospace
	#1
	\CJKfamily{chanto}\CJKnospace
}

\newcommand{\dai}[1]{
	\vspace{20pt}
	\large
	\noindent\textbf{#1}
	\normalsize
	\vspace{20pt}
}

\newcommand{\fukudai}[1]{
	\vspace{10pt}
	\noindent\textbf{#1}
	\vspace{10pt}
}

\newenvironment{bunshou}{
	\vspace{10pt}
	\begin{adjustwidth}{1cm}{3cm}
	\begin{linenumbers}
}{
	\end{linenumbers}
	\end{adjustwidth}
}

\newenvironment{reibun}{
	\vspace{10pt}
	\begin{tabular}{l l}
}{
	\end{tabular}
	\vspace{10pt}
}
\newcommand{\rei}[2]{
	#1&\textit{#2}\\
}
\newcommand{\reinagai}[2]{
	\multicolumn{2}{l}{#1}\\
	\multicolumn{2}{l}{\hspace{10pt}\textit{#2}}\\
}

\newenvironment{mondai}[1]{
	\vspace{10pt}
	#1
	
	\begin{enumerate}
		\itemsep-5pt
	}{
	\end{enumerate}
	\vspace{10pt}
}

\newenvironment{hyou}{
	\begin{itemize}
		\itemsep-5pt
	}{
	\end{itemize}
	\vspace{10pt}
}

\date{\today}

\CJKfamily{chanto}\CJKnospace
\author{Tomislav Mamić}
\begin{document}
	\dai{Poluanonimna anketa s točnim odgovorima}
	
	\begin{mondai}{Koji je, po vašem mišljenju, hrvatski prijevod sljedećih rečenica?}
		\item 木の下に ねこが いた。
		\item はこの中から ねこが 出た。
		\item でんしゃに のって うみへ 行く。
		\item ともだちが おしえてくれた みせで ラーメンを 食べに行った。
	\end{mondai}

	\begin{mondai}{U sljedećim rečenicama podcrtajte dijelove za koje smatrate da govore \textit{kad} se radnja događa.}
		\item きのう 雨が ふった。
		\item パーティは あさまで つづいた。
		\item おとといから 何も 食べていない。
		\item 6時に おきて、7時ごろには 仕事に いた。
	\end{mondai}

	\begin{mondai}{Kako biste sljedeće rečenice preveli na japanski? Količine napišite hiraganom, a ne brojem i kanđi znakom!}
		\item Gospođa Suzuki ima 24 mačke.
		\item Molim Vas 3 lista papira.
		\item Jučer sam sreo troje prijatelja.
		\item Prije tri godine bilo nas je čak pet, ali sada nas je samo troje.
	\end{mondai}

	\begin{mondai}{Na sljedeća pitanja odgovorite na japanskom, punim rečenicama.}
		\item 何さい ですか。
		\item 今、どこに いますか。
		\item 明日は 何を しますか。
		\item となりに すわっている人の かみは 何色ですか。
	\end{mondai}

	\begin{mondai}{次の文をクロアチア語に やくしなさい。}
		\item アイスクリームが食べたくなった。
		\item アイスクリームは食べたくなかった。
		\item たけしくんの もじは よみにくい。
		\item しぶやに まいしゅう おすしを 食べに くる おみせが あるの。
	\end{mondai}

	\begin{mondai}{クロアチア語訳に 合わせて (\textasciitilde)の中に てきとうな ものを 入れなさい。}
		\item Nakon što sam pojeo doručak, očistio sam sobu. = あさごはんを(1)、へやを そうじした。
		
		(食べる)
		\item Hanako mi je pokazala matematiku. = 花子ちゃんが すうがくを(2)。
		
		(おしえる)
		\item Takešiju neću pokazati matematiku. = たけしくんに すうがくを(3)。
		
		(おしえる)
		\item Pitao sam profesora da mi pokaže matematiku. = 先生に すうがくを(4)。
		
		(おしえる)
	\end{mondai}

	Nadam se da vam je bilo zabavno rješavati koliko je meni bilo zabavno smišljati zle primjere. Ako želite ispravljene ankete natrag, potpišite ih :)
\end{document}