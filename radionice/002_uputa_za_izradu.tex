% !TeX document-id = {18e9c76c-8b90-4fe0-a841-14f7886e702b}
% !TeX program = xelatex ?me -synctex=0 -interaction=nonstopmode -aux-directory=../tex_aux -output-directory=./release
% !TeX program = xelatex

\documentclass[12pt]{article}

\usepackage{lineno,changepage,lipsum}
\usepackage[colorlinks=true,urlcolor=blue]{hyperref}
\usepackage{fontspec}
\usepackage{xeCJK}
\usepackage{tabularx}
\setCJKfamilyfont{chanto}{AozoraMinchoRegular.ttf}
\setCJKfamilyfont{tegaki}{Mushin.otf}
\usepackage[CJK,overlap]{ruby}
\usepackage{hhline}
\usepackage{multirow,array,amssymb}
\usepackage[croatian]{babel}
\usepackage{soul}
\usepackage[usenames, dvipsnames]{color}
\usepackage{wrapfig,booktabs}
\renewcommand{\rubysep}{0.1ex}
\renewcommand{\rubysize}{0.75}
\usepackage[margin=50pt]{geometry}
\modulolinenumbers[2]

\usepackage{pifont}
\newcommand{\cmark}{\ding{51}}%
\newcommand{\xmark}{\ding{55}}%

\definecolor{faded}{RGB}{100, 100, 100}

\renewcommand{\arraystretch}{1.2}

%\ruby{}{}
%$($\href{URL}{text}$)$

\newcommand{\furigana}[2]{\ruby{#1}{#2}}
\newcommand{\tegaki}[1]{
	\CJKfamily{tegaki}\CJKnospace
	#1
	\CJKfamily{chanto}\CJKnospace
}

\newcommand{\dai}[1]{
	\vspace{20pt}
	\large
	\noindent\textbf{#1}
	\normalsize
	\vspace{20pt}
}

\newcommand{\fukudai}[1]{
	\vspace{10pt}
	\noindent\textbf{#1}
	\vspace{10pt}
}

\newenvironment{bunshou}{
	\vspace{10pt}
	\begin{adjustwidth}{1cm}{3cm}
	\begin{linenumbers}
}{
	\end{linenumbers}
	\end{adjustwidth}
}

\newenvironment{reibun}{
	\vspace{10pt}
	\begin{tabular}{l l}
}{
	\end{tabular}
	\vspace{10pt}
}
\newcommand{\rei}[2]{
	#1&\textit{#2}\\
}
\newcommand{\reinagai}[2]{
	\multicolumn{2}{l}{#1}\\
	\multicolumn{2}{l}{\hspace{10pt}\textit{#2}}\\
}

\newenvironment{mondai}[1]{
	\vspace{10pt}
	#1
	
	\begin{enumerate}
		\itemsep-5pt
	}{
	\end{enumerate}
	\vspace{10pt}
}

\newenvironment{hyou}{
	\begin{itemize}
		\itemsep-5pt
	}{
	\end{itemize}
	\vspace{10pt}
}

\date{\today}

\CJKfamily{chanto}\CJKnospace
\author{Tomislav Mamić}

\begin{document}
	\dai{Uputa za izradu lekcija za radionice}
	
	Svaka lekcija bi trebala biti paket sastavljen od nekoliko stvari koje pomažu i predavaču i polaznicima. Točke (1.), (3.) i (5.) mogu biti i jedna cjelina namijenjena predavačima.
	
	\fukudai{1. ciljevi lekcije}
	
	Razmisliti i napisati ukratko koji su ciljevi lekcije. To može biti novi gramatički oblik, novi vokabular, bilo što, ali dobra je ideja zapisati točno što taj tjedan pokušavamo postići. Npr:
	
	\begin{itemize}
		\itemsep-5pt
		\item objasniti što je to 漢字 i kako funkcionira
		\item ispričati o radikalima i zašto su važni
		\item pokazati neke najosnovnije radikale
		\item objasniti važnost pravilnog redoslijeda poteza
		\item objasniti kako se 漢字 čita i zašto je tako spetljano
		\item naučiti neke osnovne znakove i objasniti dobre strategije učenja
	\end{itemize}

	\fukudai{2. tekst lekcije (radni listić)}
	
	Na temelju (1.) napisati objašnjenja i primjere korištenja, kao i zadatke na kojima polaznici mogu isprobati novo znanje odmah nakon čitanja. Bilo bi dobro kad bi radni listići bili pogodni i za samostalan rad.
	
	\fukudai{3. napomene za predavanje}
	
	Popis stvari koje bi mogle biti korisne tijekom predavanja, kao primjeri, paralele u drugim jezicima, objašnjenja kulturnog i povijesnog konteksta itd. Npr:
	
	\begin{itemize}
		\itemsep-5pt
		\item 漢字 su zapravo ideogrami - predstavljaju stvari i ideje, a ne zvuk
		\item priča o uvozu 漢字 iz Kine
		\item fonetika kineskog i japanskog su potpuno različite i nekompatibilne
		\item jedan znak se kao koncept pojavljuje u više jap. riječi potpuno različitog čitanja
		\item 当て字 kad sve drugo propadne
		\item 漢字 se ne sastoji od poteza nego od radikala
		\item radikali imaju razne varijante i ponekad se mogu pisati drugim redoslijedom, ali većinom su pravilni
		\item na kraju, čitanje 漢字 znakova je potpuno proizvoljno
		\item dobri primjeri mnemotehnike:
		\begin{itemize}
			\itemsep-5pt
			\item 子 - dijete, 了 - gotovo, potpuno $\rightarrow$ dijete bez ruku nije gotovo, barem ne doslovno
			\item 立 - stajati, 木 - drvo, 見 - gledati $\rightarrow$ 親 - roditelj - s drveta špijunira što radiš
		\end{itemize}
	\end{itemize}

	\fukudai{4. zadaci za samostalni rad}
	
	Uzevši u obzir (1.) i (2.), napisati zadatke za vježbu koji postepeno postaju sve teži na takav način da je shvaćanje zadatka lakše na temelju ispravnog rješenja prethodnih.
	
	\fukudai{5. dodatni materijali}
	
	Bilo što za što pomislite da bi u toj lekciji moglo biti korisno, npr. kratki vodič za traženje 漢字 znakova po internetu.
	
\end{document}