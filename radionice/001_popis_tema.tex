% !TeX program = xelatex
% !TeX program = xelatex

\documentclass[12pt]{article}

\usepackage{lineno,changepage,lipsum}
\usepackage[colorlinks=true,urlcolor=blue]{hyperref}
\usepackage{fontspec}[ Path =../../../ ]
\usepackage{xeCJK}
\usepackage{tabularx}
\usepackage{graphicx}
\setCJKfamilyfont{chanto}{AOZORAMINCHOREGULAR_0.TTF}%
\setCJKfamilyfont{tegaki}{Mushin.otf}%
\usepackage[CJK,overlap]{ruby}
\usepackage{hhline}
\usepackage{multirow,array,amssymb}
\usepackage[croatian]{babel}
\usepackage{soul}
\usepackage[usenames, dvipsnames]{color}
\usepackage{wrapfig,booktabs}
\usepackage{calc}
\renewcommand{\rubysep}{0.1ex}
\renewcommand{\rubysize}{0.75}
\usepackage[margin=50pt]{geometry}
\usepackage{hyperref}
\modulolinenumbers[2]

\date{\today}

\usepackage{fancyhdr}
\pagestyle{fancy}
\fancyhf{}
\fancyhead[LE,RO]{\thepage}
\makeatletter
\fancyhead[RE,LO]{rev. \@date 誠}
\makeatother

\usepackage{pifont}
\newcommand{\cmark}{\ding{51}}%
\newcommand{\xmark}{\ding{55}}%

\newcommand{\dosl}{{\normalfont dosl. }}%
\newcommand{\rem}[1]{{\normalfont #1 }}%

\definecolor{faded}{RGB}{100, 100, 100}

\renewcommand{\arraystretch}{1.2}

%\ruby{}{}
%$($\href{URL}{text}$)$

\newcommand{\furigana}[2]{\ruby{#1}{#2}}
\newcommand{\tegaki}[1]{
	\CJKfamily{tegaki}\CJKnospace
	#1
	\CJKfamily{chanto}\CJKnospace
}

\newcommand{\dai}[1]{
	\vspace{20pt}
	\large
	\noindent\textbf{#1}
	\normalsize
	\vspace{20pt}
}

\newcommand{\fukudai}[1]{
	\vspace{10pt}
	\noindent\textbf{#1}
	\vspace{10pt}
}

\newenvironment{bunshou}{
	\vspace{10pt}
	\begin{adjustwidth}{1cm}{3cm}
	\begin{linenumbers}
}{
	\end{linenumbers}
	\end{adjustwidth}
}

\newenvironment{reibun}[1][]{
	\vspace{10pt}
	#1
	
	\begin{tabular}{l l}
}{
	\end{tabular}
	\vspace{10pt}
}
\newcommand{\rei}[2]{
	#1&\textit{#2}\\
}
\newcommand{\reinagai}[2]{
	\multicolumn{2}{l}{#1}\\
	\multicolumn{2}{l}{\hspace{10pt}\textit{#2}}\\
}

\newenvironment{mondai}[1]{
	\vspace{10pt}
	\noindent #1
	
	\begin{enumerate}
		\itemsep-5pt
	}{
	\end{enumerate}
}

\newenvironment{hyou}{
	\begin{itemize}
		\itemsep-5pt
	}{
	\end{itemize}
	\vspace{10pt}
}

\newcommand{\juuyou}[2][20pt]{
	\vspace{5pt}
		\noindent\hspace{#1}\parbox[c]{\textwidth-#1-#1}{\centering\textit{#2}}
	\vspace{5pt}
}

\newcommand{\ten}{
	\vspace{5pt}
	\noindent\hspace{-10pt}$\bullet$
}

\CJKfamily{chanto}\CJKnospace

\frenchspacing

\author{Tomislav Mamić}

\begin{document}
	\dai{Popis i sadržaj tema po tjednima}
	
	\noindent
	\begin{tabular}{|r|p{150pt}|p{150pt}|p{120pt}|}
		\hline
		\textbf{tjedan}&\textbf{mehanika}&\textbf{tema}&\textbf{pismo}\\
		\hline
		1&pregled jezika - gramatika, sustav pisanja i izgovor&&\\
		\hline
		2&izgovor i hiragana&&hiragana\\
		\hline
		3&imenice i pokazne zamjenice; の&priroda&hiragana\\
		\hline
		4&osobne zamjenice i oslovljavanje ljudi; や, と&obitelj&hiragana\\
		\hline
		5&spojni glagol -> im. predikat; は, が, も&stvari&katakana\\
		\hline
		6&pridjevi -> im. predikat&boje i oblici&katakana\\
		\hline
		7&opisni oblik pridjeva&dijelovi tijela&katakana\\
		\hline
		8&uvod u glagole, 一段&hrana i piće&\\
		\hline
		9&nepravilni glagoli&opisivanje gdje se nešto nalazi&\\
		\hline
		10&五段 glagoli&&\\
		\hline
		11&organizacija kanđija, radikali, metode učenja itd&&一二三人木口日目田\\
		\hline
		12&prilozi - pravi i od pridjeva&\textit{kako?}&小大白\\
		\hline
		13&priložne oznake mjesta&upute od A do B&\\
		\hline
		14&priložne oznake vremena&&今夜朝昨日明日\\
		\hline
		15&količina&broj mačaka u stanu u trenutku smrti&寺時侍日月年\\
		\hline
		16&pitanja i čestice na kraju rečenice&kviz&\\
		\hline
		17&い oblik i ます&&\\
		\hline
		18&て oblik i ている&Opisivanje što sada radiš&\\
		\hline
		19&razne upotrebe い oblika&Odlazak u grad - ići vidjeti nešto, kupiti nešto, početi raditi...&\\
		\hline
		20&razne upotrebe て oblika&&\\
		\hline
		21&spojevi s predikatnim oblikom&&\\
		\hline
		22&spojevi s opisnim oblikom&&\\
		\hline
		23&opisna rečenica&&\\
		\hline
		24&の i こと&&\\
		\hline
		25&razne priložne imenice&&\\
		\hline
		26&usporedbe s よう i みたい&&\\
		\hline
		27&hortativ (\textasciitilde よう)&&\\
		\hline
		28&potencijal&&\\
		\hline
	\end{tabular}
	\noindent
	
	\begin{tabular}{|r|p{150pt}|p{150pt}|p{120pt}|}
		\hline
		29&imperativ&&\\
		\hline
		30&pasiv&&\\
		\hline
		31&kauzativ i kombinacija s pasivom&&\\
		\hline
	\end{tabular}

\end{document}